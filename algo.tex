\paragraph{Notation.}
\label{sec:prelims}
We denote the set of arms by $A$, with the individual arms labeled $i, i=1,\ldots,K$.
We denote an arbitrary round of ClusUCB by $m$. We denote an arbitrary cluster by $s_{k}$, the subset of arms within the cluster $s_k$ by  $A_{s_{k}}$ and the set of clusters by $S$ with $|S|=p\leq K$. 
Here $p$ is a pre-specified limit for the number of clusters.
For simplicity, we assume that the optimal arm is unique and denote it by ${*}$, with $s^{*}$ denoting the corresponding cluster.
 %and the subset of arms in $s^{*}$ is denoted by $A_{s^{*}}$. 
The best arm in a cluster $s_{k}$ is denoted by $a_{max_{s_{k}}}$.  
We denote the sample mean of the rewards seen so far for arm $i$ by $\hat{r_i}$ and for the best arm within a cluster $s_k$ by $\hat{r}_{a_{\max_{s_{k}}}}$. 
%The exploration regulatory factor is denoted by $\psi$. The arm and cluster elimination parameters are denoted by $\rho_{a}$ and $\rho_{s}$ respectively. 
%$\Delta_{i}^{'}=r_{a_{\max_{s_{k}}}} - r_{i}$, such that $a_{i}\in s_{k}$.
% The variable $B_{m}$ denotes the arm set containing the arms that are not eliminated till round $m$.
%  The exploration regulatory factor is denoted by $\psi_{m}$. %\todos{\textit{``$\hat{r}_{min_{s_{i}}}\in s_{i}$ as the arm with minimum estimated payoff''}. $\hat{r}_{min_{s_{i}}}\in s_{i}$ is the minimum estimated payoff and not the arm corresponding to min est payoff. Similar writing pops up at several other places, for e.g., in the statements of the propositions. (Subho) Changed the line as saying the minimum/maximum estimated payoff}
%The parameter $\psi(m)$ is a monotonically decreasing function over the rounds, that is $\psi(m+1)\leq \psi(m)$. 
%Also, we define $w\geq 2$, a weight factor.
We assume that all arms' rewards are bounded in $[0,1]$.


%worst arm within a cluster $s_i$ by $\hat{r}_{min_{s_{k}}}\in s_{k}$

\subsubsection*{The algorithm}


%%%%%%%%%%%%%%%% alg-custom-block %%%%%%%%%%%%
\algblock{ArmElim}{EndArmElim}
\algnewcommand\algorithmicArmElim{\textbf{\em Arm Elimination}}
 \algnewcommand\algorithmicendArmElim{}
\algrenewtext{ArmElim}[1]{\algorithmicArmElim\ #1}
\algrenewtext{EndArmElim}{\algorithmicendArmElim}

\algblock{ClusElim}{EndClusElim}
\algnewcommand\algorithmicClusElim{\textbf{\em Cluster Elimination}}
 \algnewcommand\algorithmicendClusElim{}
\algrenewtext{ClusElim}[1]{\algorithmicClusElim\ #1}
\algrenewtext{EndClusElim}{\algorithmicendClusElim}
\algtext*{EndArmElim}
\algtext*{EndClusElim}

\algblock{ResParam}{EndResParam}
\algnewcommand\algorithmicResParam{\textbf{\em Reset Parameters}}
 \algnewcommand\algorithmicendResParam{}
\algrenewtext{ResParam}[1]{\algorithmicResParam\ #1}
\algrenewtext{EndResParam}{\algorithmicendResParam}

\begin{algorithm}[t]
\caption{ClusUCB}
\label{alg:clusucb}
\begin{algorithmic}
\State {\bf Input:} Number of clusters $p$, time horizon $T$, exploration parameters $\rho_a$, $\rho_s$ and $\psi$.
\State {\bf Initialization:} Set $B_{0}:=A$, $S_0 = S$ and $\epsilon_{0}:=1$.
\State Create a partition $S_0$ of the arms at random into $p$ clusters of size up to $\ell=\bigg\lceil \dfrac{K}{p} \bigg\rceil$ each.
\For{$m=0,1,..\big \lfloor \dfrac{1}{2}\log_{2} \dfrac{7T}{K}\big\rfloor$}	
\State Pull each arm in $B_m$ so that the total number of times it has been pulled is $n_{m}=\bigg\lceil\dfrac{2\log{(\psi T\epsilon_{m}^{2})}}{\epsilon_{m}}\bigg\rceil$. 
% A partition of $A$ into clusters from Algorithm \ref{alg:rua}
%\State \hspace*{2em} Calculate $w_{s_{i}}=\bigg\lceil\dfrac{1}{\ell\hat{\Delta}_{s_{i}}}\bigg\rceil$,if $\hat{\Delta}_{s_{i}}\neq 0, \forall s_{i}\in S$
%\newline\hspace*{8em}$=1$, otherwise, and $\hat{\Delta}_{s_{i}}=\max_{i\in s_{i}}{\lbrace\hat{r}_{i}\rbrace}-\min_{j\in s_{i}}{\lbrace\hat{r}_{j}\rbrace}, i\neq j$
\ArmElim
\State For each cluster $s_k \in S_{m}$, delete arm ${i}\in s_{k}$ from $B_{m}$ if
\begin{align*}
\hat{r}_{i} + \sqrt{\dfrac{\rho_{a}\log{(\psi T\epsilon_{m}^{2})}}{2 n_{m}}}  < \max_{{j}\in s_{k}}\bigg\lbrace\hat{r}_{j} -\sqrt{\dfrac{\rho_{a}\log{(\psi T\epsilon_{m}^{2})}}{2 n_{m}}} \bigg\rbrace
\end{align*}
% where $\rho_{a}=\dfrac{1}{w_{m}}$ and remove all such arms from $B_{m}$.
\EndArmElim
\ClusElim
\State Delete cluster $s_{k}\in S_{m}$ and remove all arms $i\in s_{k}$ from $B_{m}$ if 
\begin{align*}
 \max_{{i}\in s_{k}}\bigg\lbrace\hat{r}_{i} + \sqrt{\dfrac{\rho_{s}\log{(\psi T\epsilon_{m}^{2})}}{2 n_{m}}}\bigg\rbrace  \\
 < \max_{{j}\in B_{m}} \bigg\lbrace\hat{r}_{j} - \sqrt{\dfrac{\rho_{s} \log{(\psi T\epsilon_{m}^{2})}}{2 n_{m}}}\bigg\rbrace.
\end{align*}
%  and remove all such arms in the cluster $s_{k}$ from $B_{m}$ to obtain $B_{m+1}$.
\EndClusElim
\State Set $\epsilon_{m+1}:=\dfrac{\epsilon_{m}}{2}$\vspace{0.5ex}
\State Set $B_{m+1}:=B_{m}$
%\State \hspace*{2em} $\ell_{m+1}:=\min\lbrace 2\ell_{m}, K\rbrace$
%\State \hspace*{2em} $w_{m+1}:=2w_{m}$
\State Stop if $|B_{m}|=1$ and pull ${i}\in B_{m}$ till $T$ is reached.
\EndFor
\end{algorithmic}
\end{algorithm}

%\todos[inline]{Shouldn't there be a $\psi$ inside the log term on RHS of both elim conditions of Algorithm \ref{alg:clusucb}? (Subho) Addressed: $\psi$ has to be there}

As mentioned in a recent work \cite{liu2016modification}, UCB-Improved has two shortcomings: 	\\
\begin{inparaenum}[\bfseries(i)]
\item A significant number of pulls are spent in early exploration, since each round $m$ of UCB-Improved involves pulling every arm an identical $n_{m}=\bigg\lceil \dfrac{ 2\log(T\epsilon^{2}_{m})}{\epsilon^{2}_{m}} \bigg\rceil$ number of times. The quantity $\epsilon_{m}$ is initialized to $1$ and halved after every round.\\
\item In UCB-Improved, arms are eliminated conservatively, i.e, only after $\epsilon_{m}<\dfrac{\Delta_{i}}{2}$, the sub-optimal arm $i$ is discarded with high probability. This is disadvantageous when $K$ is large and the gaps are identical ($r_{1}=r_{2}=..=r_{K-1}<r^{*}$) and small.\\
\end{inparaenum}
To reduce early exploration, the number $n_m$ of times each arm is pulled per round in ClusUCB is lower than that of UCB-Improved and also that of Median-Elimination, which used $n_m=\dfrac{4}{\epsilon^{2}}\log\big(\dfrac{3}{\delta}\big)$, where $\epsilon,\delta$ are confidence parameters.
To handle the second problem mentioned above, ClusUCB partitions the larger problem into several small sub-problems using clustering and then performs local exploration aggressively to eliminate sub-optimal arms within each clusters with high probability.


As described in the pseudocode in Algorithm~\ref{alg:clusucb}, ClusUCB begins with a initial clustering of arms that is performed by random uniform allocation. The set of clusters $S$ thus obtained satisfies $|S|=p$, with individual clusters having a size that is bounded above by $\ell=\bigg\lceil \dfrac{K}{p} \bigg\rceil$.
Each round of ClusUCB involves both individual arm as well as cluster elimination conditions. These elimination conditions are inspired by UCB-Improved. Notice that, unlike UCB-Improved, there is no longer a single point of reference based on which we are eliminating arms. Instead now we have as many reference points to eliminate arms as number of clusters formed. 
%Further, the exploration factors $\rho_{a}\in (0,1]$ and $\rho_{s}\in (0,1]$ governing the arm and cluster elimination conditions, respectively, are relatively more aggressive than that in UCB-Improved. 

The exploration regulatory factor $\psi$ governing the arm and cluster elimination conditions in ClusUCB is more aggressive than that in UCB-Improved. With appropriate choice of $\psi$ and $\rho_a$ and $\rho_s$ we can achieve aggressive elimination even when the gaps $\Delta_i$ are small and $K$ is large. 

%and the gaps $\Delta_i$ are small, it is efficient to remove sub-optimal arms quickly. 

In \cite{liu2016modification}, the authors recommend incorporating a factor of $d_i$ inside the log-term of the UCB values, i.e., $\max \lbrace\hat{r}_{i}+\sqrt{\frac{d_{i}\log T{\epsilon}_{m}^{2}}{2n_{m}}}\rbrace$. 
The authors there examine the following choices for $d_i$: $\frac{T}{t_{i}}$, $\frac{\sqrt{T}}{t_{i}}$ and $\frac{\log T}{t_{i}}$, where $t_{i}$ is the number of times an arm ${i}$ has been sampled.
Unlike \cite{liu2016modification}, we employ cluster as well as arm elimination and establish from a theoretical analysis that the choice $\psi=\frac{T}{\log (K)}$ helps in achieving a better gap-dependent regret upper bound for ClusUCB as compared to UCB-Improved and MOSS (see Corollary \ref{Result:Corollary:1} in the next section). 


%Like our exploration regulatory factor $\psi$, a similar topic has already been handled in \cite{liu2016modification} where they have introduced mainly three types of regulatory factors($d_{i}$) to the term $\max \lbrace\hat{r}_{i}+\sqrt{\frac{d_{i}\log T\tilde{\Delta}_{m}^{2}}{2n_{m}}}$ which is used for selecting an arbitrary arm ${i}$ in the $t$-th timestep. This regulatory term $d_{i}$ can be of the form as $\frac{T}{t_{i}}$, $\frac{\sqrt{T}}{t_{i}}$ and $\frac{\log T}{t_{i}}$, where $t_{i}$ is the number of times an arm ${i}$ has been sampled. One has to choose a regulatory factor based on how fast the algorithm should taper its exploration in the later rounds.

%since with time, $t_{i}$ increases and only the numerator decides how fast the exploration must decrease.

%Discussion on UCB-V, not quite sure
%In this context we must also point out a similar discussion on the arm elimination parameters was handled in \cite{audibert2009exploration}(section $5.1$). There the authors proved that by introducing the factor $\rho$ in the confidence interval term of UCB1 such that $c=\sqrt{\dfrac{\rho\log T}{n_{i}}}$, where $n_{i}$ is the number of times the arm $a_{i}$ has been sampled, we can make the expected regret $\E [R_{T}]$ linearly dependent on $\rho$. So this $\rho$ in the confidence term controls the exploration and larger the $\rho$, higher is the exploration. In the original UCB1 algorithm this coefficient was set to $\rho=2$. In ClusUCB, both $\rho_{a}$ and $\rho_{s}$ are taken to be $\leq 1$. 


%For creating these clusters we rely on the parameter $\epsilon_{m}$, which is initialized at $1$ and halved after every round. 
%$\hat{\Delta}_{m}$(in which all $\hat{r}_{i}$ lies) divided by $\ell$. 
%If the range before any round is found to be $0$ then we conduct a small exploration with the alternate definition of $\epsilon_{m}=\dfrac{1}{D_{m}}$. 
%Thus, we see that as the rounds progresses, $\epsilon_{m}$ gets smaller and smaller resulting in tighter and tighter clusters with high purity level. Single link clustering tends to create large clusters with many elements in a single cluster. To avoid this behavior, mainly because we are conducting exploration inside a cluster based on its size, we bound the maximum cluster size possible in any round by $\ell=\min\lbrace 2^{m}, K \rbrace$. Hence as $\epsilon_{m}$ tends to $\Delta$ the upper limit on the cluster size gets fixed. 



%Also our total regret depends on how many arms has survived till $m$-th round and so we don't need to keep track on the number of clusters formed.

%Through cluster elimination condition we ensure that the stopping condition is reached faster. This is a much aggressive elimination condition and in the proofs we give a further analysis on why this works. The parameter $\rho\in (0,1]$ in the confidence interval actually makes the cluster elimination a faster elimination condition than the elimination condition in UCB-Improved.
	
%This is a parameter which we use to tune our exploration and we define its structure later in the proofs. There is also the weight($w_{s_{i}}$) parameter which   is calculated online and helps in eliminating arms with a higher probability which we employ to reduce the cumulative regret and this parameter is decide online specific to the cluster $s_{i}\in S$.

%\begin{algorithm}[t]
%\caption{Clustering by random uniform allocation}
%\label{alg:rua}
%\begin{algorithmic}
%\State Set $\ell=\bigg\lceil \dfrac{K}{p} \bigg\rceil$
%\State Permute arms in $A$ randomly to create $A_{random}$.
%\State Create clusters $s_{i}=\lbrace {i}\rbrace, \forall i\in A_{random}$
%\For{ $i=1$ to $K$}
%\For{$j=i+1$ to $K$}
%\State Merge $s_{i},s_{j}$ into $s_{i}$ if $\exists {i}\in s_{i} $ and $\exists {j}\in s_{j},$ such that $|s_{i}|+|s_{j}|\leq \ell$
%\EndFor
%\EndFor
%\State Rename all partitions that have been formed as $s_{1}$ to $s_{p}$, where $|S|=p$ after merging
%\end{algorithmic}
%\end{algorithm}



