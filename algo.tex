\subsection*{Algorithm}

The steps are presented in Algorithm~1.\\%\ref{alg:cucb}.
%\begin{algorithm}
%\caption{Clustered UCB($D$)}
%\label{alg:cucb}
\noindent\makebox[\linewidth]{\rule{\textwidth}{0.8pt}}\\[-0.1cm]
\textbf{Algorithm 1:} ClusUCB\\[-0.3cm]
\noindent\makebox[\linewidth]{\rule{\textwidth}{0.4pt}}\\[-0.3cm]
\begin{algorithmic}[1]
%\Procedure{\underline{Phase 1: Arm Exploration}}{}
\State Pull each arm once and calculate $\hat{r}_{i}, \forall i\in A$
\State Let $A$ be the set which contains all the arms
\State $B_{0}:=A$
%\State $\psi(m)=\dfrac{c}{m}, c>0$
\State $\ell_{0}:=2, \epsilon_{0}:=1, w_{0}:=1$
%\State For rounds $m=1,2,.. \lceil\frac{1}{2}\log T\rceil$
\State For rounds $m=0,1,..\big \lfloor \dfrac{1}{2}\log_{2} \dfrac{T}{e}\big\rfloor$
%\State \hspace*{2em} $\hat{\Delta}_{m}=\max_{i\in B_{m}}{\lbrace\hat{r}_{i}\rbrace}-\min_{j\in B_{m}}{\lbrace\hat{r}_{j}\rbrace}$
%\State \hspace*{2em} $\epsilon_{m}=\max{\bigg\lbrace\dfrac{\hat{\Delta}_{m}}{\ell_{m}}, \dfrac{2}{\sqrt{\psi{(m)T}}}\bigg\rbrace}$
%, if $\hat{\Delta}_{m}\neq 0$
%\newline\hspace*{3.7em}$=\max{\bigg\lbrace\dfrac{1}{D_{m}}, \dfrac{2}{\sqrt{\psi{(m)T}}}\bigg\rbrace}$, if $\hat{\Delta}_{m}=0$ 
%\todos{(Subho)I think that for $\hat{\Delta}_{m}$ only one condition as specified above is enough for all cases. Even if $\hat{\Delta}_{m}=0$ then $\epsilon_{m}=\dfrac{2}{\sqrt{\psi{(m)T}}}$ as $\dfrac{2}{\sqrt{\psi{(m)T}}}>0$}
%\State \hspace*{2em} Create clusters $s_{i}=\lbrace a_{i}\rbrace, \forall i\in B_{m}$
%\State \hspace*{2em} Merge $s_{i},s_{j}$ into $s_{ij}$ if $\exists a_{i}\in s_{i} $, $ a_{j}\in s_{j},$ such that $|\hat{r}_{i}-\hat{r}_{j}|\leq\epsilon_{m}$ and $|s_{i}|+|s_{j}|\leq \ell_{m}$
\State \hspace*{2em} Call SubroutineMerge
\State \hspace*{2em} Pull each arm in $s_{k}$, \newline\hspace*{2em}$n_{m}=\bigg\lceil\dfrac{2\log{(T\epsilon_{m}^{2})}}{\epsilon_{m}}\bigg\rceil$ number of times, $\forall s_{k}\in S_{m}$ 
%\State \hspace*{2em} Calculate $w_{s_{i}}=\bigg\lceil\dfrac{1}{\ell_{m}\hat{\Delta}_{s_{i}}}\bigg\rceil$,if $\hat{\Delta}_{s_{i}}\neq 0, \forall s_{i}\in S$
%\newline\hspace*{8em}$=1$, otherwise, and $\hat{\Delta}_{s_{i}}=\max_{i\in s_{i}}{\lbrace\hat{r}_{i}\rbrace}-\min_{j\in s_{i}}{\lbrace\hat{r}_{j}\rbrace}, i\neq j$
\State \hspace*{2em} \textbf{Arm Elimination:}
\newline
\hspace*{2em} Delete all arms $a_{i}\in s_{k}$ for which
\newline\hspace*{2em}$\bigg\lbrace\hat{r}_{i} + \sqrt{\dfrac{\log{(T\epsilon_{m}^{2})}}{2 n_{m}}} \bigg\rbrace < \max_{a_{j}\in s_{k}}\bigg\lbrace\hat{r}_{j} -\sqrt{\dfrac{\log{(T\epsilon_{m}^{2})}}{2 n_{m}}} \bigg\rbrace$, $\forall s_{k}\in S_{m}$ and remove all such arms from $B_{m}$.
%- \sqrt{\dfrac{\log{(4\psi(m)T\epsilon_{m}^{2})}}{2wn_{s_{i}}}}
\State \hspace*{2em} \textbf{Cluster Elimination:}
\newline
\hspace*{2em} Delete all clusters $s_{k}\in S_{m}$ for which $\max_{a_{i}\in s_{k}}\bigg\lbrace\hat{r}_{i} + \sqrt{\dfrac{\rho\log{(T\epsilon_{m}^{2})}}{2 n_{m}}}\bigg\rbrace  $\newline\hspace*{4em}$< \max_{a_{j}\in B_{m}} \bigg\lbrace\hat{r}_{j} - \sqrt{\dfrac{\rho \log{(T\epsilon_{m}^{2})}}{2 n_{m}}}\bigg\rbrace$, where $\rho=\dfrac{1}{w_{m}}$ and remove all such arms in the cluster $s_{k}$ from $B_{m}$ to obtain $B_{m+1}$.
%- \sqrt{\dfrac{(|B_{m}|)\epsilon_{m}\log{(4\psi(m)T\epsilon_{m}^{2})}}{2w\ell_{m} n_{s_{j}}}}
%\newline
%\todo{the max over all arm gives the best global arm, and we compare this with the best payoff arm from each cluster}
\State \hspace*{2em} \textbf{Reset Parameters:}
%\newline \hspace*{2em} Remove all arms from $B_{m}$ which have been eliminated either in arm elimination or cluster elimination condition to obtain $B_{m+1}$
\State \hspace*{2em} $\epsilon_{m+1}:=\dfrac{\epsilon_{m}}{2}$
\State \hspace*{2em} $\ell_{m+1}:=\min\lbrace 2\ell_{m}, K\rbrace$
\State \hspace*{2em} $w_{m+1}:=2w_{m}$
\State \hspace*{2em} \textbf{Stopping Condition:} 
\newline \hspace*{2em} Stop if $|B_{m}|=1$ and pull $\max_{a_{i}\in B_{m}}\hat{r}_{i}$ till $T$ is reached.
\end{algorithmic}
\noindent\makebox[\linewidth]{\rule{\textwidth}{0.4pt}}\\[-0.6cm]
\newline
%\noindent\makebox[\linewidth]{\rule{\textwidth}{0.4pt}}\\[-0.3cm]
\textbf{Subroutine:} SubroutineMerge\\[-0.3cm]
\begin{algorithmic}[1]
\State \hspace*{2em} Arrange all arms in $B_{m}$ in ascending order based on their $\hat{r}_{i}$
\State \hspace*{2em} Create clusters $s_{i}=\lbrace a_{i}\rbrace, \forall i\in B_{m}$
\State \hspace*{2em} For $i=1$ to $K$:
\State \hspace*{4em} For $j=i+1$ to $K$:
\State \hspace*{6em} Merge $s_{i},s_{j}$ into $s_{i}$ if $\exists a_{i}\in s_{i} $ and $\exists a_{j}\in s_{j},$ such that $|\hat{r}_{i}-\hat{r}_{j}|\leq\epsilon_{m}$ and $|s_{i}|+|s_{j}|\leq \ell_{m}$
\State \hspace*{2em} Rename all clusters that have been formed as $s_{1}$ to $s_{M}$, where $|S_{m}|=M$ after merging
\end{algorithmic}
\noindent\makebox[\linewidth]{\rule{\textwidth}{0.4pt}}\\[-0.6cm]

%\end{algorithm}



In the above algorithm we are dividing the arm set $A$ into smaller clusters which belong to the cluster set $S_{m}$, where after merging $|S_{m}|=M\leq K$. We are bounding the cluster size by $\ell_{m}$ in each round, starting from an initial value of $\ell_{0}=2$ and doubling it after every round. For creating these clusters we rely on the parameter $\epsilon_{m}$, which is initialized at $1$ and halved after every round. 
%$\hat{\Delta}_{m}$(in which all $\hat{r}_{i}$ lies) divided by $\ell_{m}$. 
%If the range before any round is found to be $0$ then we conduct a small exploration with the alternate definition of $\epsilon_{m}=\dfrac{1}{D_{m}}$. 
Thus, we see that as the rounds progresses, $\epsilon_{m}$ gets smaller and smaller resulting in tighter and tighter clusters with high purity level. Single link clustering tends to create large clusters with many elements in a single cluster. To avoid this behavior, mainly because we are conducting exploration inside a cluster based on its size, we bound the maximum cluster size possible in any round by $\ell_{m}=\min\lbrace 2^{m}, K \rbrace$. Hence as $\epsilon_{m}$ tends to $\Delta$ the upper limit on the cluster size gets fixed. We can also see that at any round $m$, the pulls $n_{m}$ increases at a lesser rate compared to the round-based variants UCB-Revisited $\bigg\lceil \dfrac{2log(T\tilde{\Delta}_{m}^{2})}{\tilde{\Delta}_{m}^{2}} \bigg\rceil$, where $\tilde{\Delta}_{m}$ is initialized at $1$ and halved after every round or Median-Elimination $\dfrac{4}{\epsilon^{2}}\log\big(\dfrac{3}{\delta}\big)$, where $\epsilon,\delta$ are the parameters for PAC guarantee.

	Next, in each of the round we eliminate arms like UCB-Revisited(\cite{auer2010ucb}) or Median Elimination(\cite{even2006action}),however, it is important to note that there is no longer a single point of reference based on which we are eliminating arms but now we have as many reference points to eliminate arms as number of clusters formed after merging that is $|S_{m}|$. We can be this aggressive because we have divided the larger problem into smaller sub-problems, doing local exploration and eliminating sub-optimal arms within each clusters with high guarantee. Hence, compared to UCB-Revisited or Median Elimination, the proposed algorithm should have a higher probability of arm deletion. Especially when $K$ is large it is efficient to remove sub-optimal arms quickly rather getting tied down in hopeless exploration. Also our total regret depends on how many arms has survived till $m$-th round and so we don't need to keep track on the number of clusters formed.

	Through cluster elimination condition we ensure that the stopping condition is reached faster. This is a much aggressive elimination condition and in the proofs we give a further analysis on why this works. The parameter $\rho\in (0,1]$ in the confidence interval actually makes the cluster elimination a faster elimination condition than the elimination condition in UCB-Revisited(\cite{auer2010ucb}).
	
%This is a parameter which we use to tune our exploration and we define its structure later in the proofs. There is also the weight($w_{s_{i}}$) parameter which   is calculated online and helps in eliminating arms with a higher probability which we employ to reduce the cumulative regret and this parameter is decide online specific to the cluster $s_{i}\in S$.
