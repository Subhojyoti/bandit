\documentclass[twoside]{article} 

\usepackage{macros}

\begin{document}

\twocolumn[

\aistatstitle{UCB with clustering and improved exploration}

\aistatsauthor{ }

\aistatsaddress{ } ]

\begin{abstract}

In this paper, we present a novel algorithm for the stochastic multi-armed bandit problem.
% , which achieves a better gap-dependent regret upper bound than UCB-Improved\cite{auer2010ucb} and MOSS\cite{audibert2009minimax} algorithms. 
% and thereby try to answer an open problem which has been raised before. 
Our proposed method, referred to as ClusUCB, partitions the arms into clusters and then follows the UCB-Improved strategy with aggressive exploration factors to eliminate sub-optimal arms as well as clusters. 
From a theoretical analysis, we establish that ClusUCB achieves a better gap-dependent regret upper bound than UCB-Improved\cite{auer2010ucb} and MOSS\cite{audibert2009minimax} algorithms. 
% We corroborate our findings both theoretically and empirically and show that by careful tuning of ClusUCB's exploration parameters, we can achieve a lower regret in comparison to MOSS and UCB-Improved. 
Further, numerical experiments on test-cases with small gaps between optimal and sub-optimal mean rewards show that ClusUCB results in lower cumulative regret than popular bandit algorithms such as UCB1\cite{auer2002finite}, UCB-Improved, MOSS, UCB-V\cite{audibert2009exploration}, Thompson Sampling\cite{agrawal2011analysis},Median Elimination\cite{even2006action}, KL-UCB\cite{garivier2011kl} and DMED\cite{honda2010asymptotically}.

%Old Abstract 27.8.2016

%In this paper, we present a novel algorithm which achieves a better upper bound on regret for the stochastic multi-armed bandit problem then some of the existing algorithms. Our proposed method clusters arms based on their average estimated payoff. This results in a more efficient elimination of arms within clusters as well as the deletion of weak clusters. We prove the regret upper bound  as 
%$\sum_{i\in B_{m}}\bigg (\max{\bigg\lbrace \bigg(\dfrac{32}{(\Delta_{i})^{3}}\bigg) ,\bigg(\dfrac{25\Delta_{i}}{(\Delta^{2})(0.16T\Delta^{2})^{2|B_{m}|^{2}\Delta/5}}\bigg)\bigg\rbrace} + \bigg(\Delta_{i}+\dfrac{32\log{(T\dfrac{\Delta_{i}^{4}}{16})}}{\Delta_{i}}\bigg)\bigg)$, $\Delta$ is the minimal gap between the means of the reward distributions of the optimal arm and a sub-optimal arm, $A$ is the set of arms, $T$ is the horizon and $\psi(m)$ is a parameter of the problem. This bound improves upon the existing algorithm of UCB-Revisited, MOSS, KL-UCB and UCB1 under certain cases. We corroborate our findings both theoretically and empirically and show that by sufficient tuning of parameter, we can achieve a lower regret than all the algorithms mentioned. In particular, in the test-cases where the arm set is dominated by small $\Delta$ and large $K$ we achieve a significantly lower cumulative regret than these algorithms.

%Old Abstract

%In this paper we achieve a lower regret bound by a novel method of using clustering of arms based on their average estimated payoff. By clustering of arms we divide the larger problem into smaller sub-problems and individually solve the sub-problems to arrive at a global solution. We prove that by this method we achieve a regret bound of the order of $O\bigg(K\log(KT\Delta^{2})\bigg)$ where $\Delta$ is the minimal gap between the means of the reward distributions of the optimal arm and a sub-optimal arm , $K$ is the number of arms and $T$ is the horizon. This improves upon UCB-Revisited having regret bound $O\bigg(K\dfrac{\log(T\Delta^{2})}{\Delta}\bigg)$ . This bound also improves upon UCB1 which has a regret of $O\bigg(\dfrac{K\log T}{\Delta}\bigg)$ and MOSS which has a minimax distribution free upper bound on the regret in the order of $O(\sqrt{TK})$ and a distribution dependent upper bound of $O\bigg(\dfrac{K\log(T\Delta^{2}/K)}{\Delta}\bigg)$. We corroborate our findings empirically and showed that by 
%sufficient tuning of some parameters we can achieve a lower regret than all the existing algorithms mentioned.
%\dots
%\keywords{Multi-Armed Bandit, UCB, Exploration-Exploitation, Clustering}
\end{abstract}

\section{Introduction}
\label{sec:intro}
In this paper, we address the stochastic multi-armed bandit problem, a classical problem in sequential decision making. Here, a learning agent is provided with a set of choices or actions, also called arms and it has to choose one arm in each timestep. After choosing an arm the agent receives a reward from the environment, which is an independent draw from a stationary distribution specific to the arm selected. The goal of the agent is to maximize the cumulative reward. The agent is oblivious to the mean of the distributions associated with each arm, denoted by $r_{i}$, including the optimal arm which will give it the best average reward, denoted by $r^{*}$. The agent records the cumulative reward it has collected for any arm divided by the number of pulls of that arm which is called the estimated mean reward of an arm denoted by $\hat{r}_{i}$. In each trial the agent faces the exploration-exploitation dilemma, that is, should the agent select the arm which has the highest observed mean reward till now (
exploitation), or should the agent explore other arms to gain more knowledge of the true mean reward of the arms and thereby avert a sub-optimal greedy decision (exploration). The objective in the bandit problem is to maximize cumulative reward which will lead to minimizing the expected cumulative regret. We define the regret$(R_{T})$ of an algorithm after $T$ trials as
%\newline
%\newline
\begin{align*}
R_{T}=r^{*}T - \sum_{i\in A} r_{i}\mathbb{E}[N_{i}]
\end{align*}
\todos{$R_T$ on LHS with an expectation on the RHS?}
%\hspace*{6em}$R_{T}=r^{*}T$ - $\sum_{i\in A} r_{i}E[N_{i}]$
%\newline
%\newline
where $N_{i}$ denotes the number of times the learning agent chooses arm $i$ within the first $T$ trials. We also define $\Delta_{i}=r^{*}-r_{i}$, that is the difference between the mean of the optimal arm and the $i$-th sub-optimal arm (for simplicity we assume that there is only one optimal arm which will give the highest payoff). The problem, gets more difficult when the $\Delta_{i}$'s are smaller and arm set is larger. Also let $\Delta=min_{i\in A}\Delta_{i}$, that is it is the minimum possible gap over all arms in $A$.
                                                                                                                                          

\subsection*{Related work}
\todos{This section is too long and lacks a qualitative comparison with Clus-UCB. If the target is one of the ML conferences, then this section has to restricted to a couple of paragraphs that compare Clus-UCB with closely related works, for e.g. UCB-revisited.}
Bandit problems have been extensively studied under various conditions. One of the first works can be seen in \cite{thompson1933likelihood}, which deals with choosing between two treatments to administer on patients who come in sequentially. Further studies by \cite{robbins1952some} and \cite{lai1985asymptotically}, established an asymptotic lower bound for the regret. Lai and Robbins proved in \cite{lai1985asymptotically} that for any allocation strategy and for any sub-optimal arm $i$, the regret is lower bounded by 
$$\lim_{T\rightarrow\infty} \inf\dfrac{\mathbb{E}[R_{T}]}{\log T}\geq
\sum_{i:r_{i}<r^{*}}\dfrac{(r^{*}-r_{i})}{D(p_{i}||p^{*})},$$
where $D(p_{i}||p^{*})$ is the Kullback-Leibler divergence over the reward density $p_{i}$ and $p^{*}$ over the arms having mean $r_{i}$ and $r^{*}$.

	Of the several algorithms mentioned in \cite{auer2002finite}, UCB1 has a regret upper bound of \\$O\bigg(\dfrac{K\log T}{\Delta}\bigg)$ whereas in the same paper the author's propose UCB2 algorithm which has a tighter regret bound than UCB1. UCB2 has a parameter $\alpha$ that needs to be tuned and its regret is upper bounded by $\sum_{i\in A}\bigg(\dfrac{(1+C_{1}(\alpha))\log T}{2\Delta_{i}}\bigg) + C_{2}(\alpha)$, where $C_{1}(\alpha)>0$ is a constant that can get arbitrarily small when $\alpha$ gets smaller but $C_{2}(\alpha)$ consequently starts increasing. Another type of strategy was first proposed by \cite{sutton1998reinforcement} called $\epsilon-$greedy strategy where the agent behaves greedily most of the time pulling the arm having highest $\hat{r}_{i},\forall i\in A$ and sometimes with a small probability $\epsilon$ it will try to explore by pulling a sub-optimal arm. In \cite{auer2002finite} they further  refined the same algorithm and proposed $\epsilon_{n}$-greedy with regret guarantee. For the 
$\epsilon_{n}$-greedy it is  proved that if the parameter $\epsilon$, is made a function over time, like $\epsilon_{t}=\dfrac{const.K}{d^{2}t}$, such that $0<\epsilon<1$, then the regret grows logarithmically $\bigg(\dfrac{K\log T}{d^{2}}\bigg)$. This algorithm performs well given that $0<d<\min_{i\in A}{\Delta_{i}}$ and for large $const$ value the result actually becomes stronger than UCB1. For further insight into various approaches in dealing with stochastic multi-armed bandit we refer the reader to \cite{bubeck2012regret}. In \cite{auer2010ucb} they prove an improved regret bounds for the algorithm UCB-Revisited in the order of $\sum_{i\in A}\bigg(const*\dfrac{K\log (T\Delta_{i}^{2})}{\Delta_{i}}\bigg)$ for very small $\Delta_{i}$ over a larger set of arms. This is a round based method, where in every round, all the arms are pulled an equal number of times, then based on certain conditions they eliminate some arms and this goes on till one arm is left. In this paper we refer to these type of algorithms 
as round based algorithms.
	
	Other prominent round based elimination algorithms include Successive Reject, Successive Elimination and Median Elimination. The Successive reject algorithm was proposed by  \cite{audibert2010best}, where they explore the pure exploration scenario in a fixed budget/horizon setup. Successive Reject tries to proceed in a phase-wise manner eliminating one arm after each phase. They try to bound the regret by defining two hardness parameter $H_{1}=\sum_{i=1}^{K}\frac{1}{\Delta_{i}^{2}}$ and $H_{2}=\max_{i\in A}{i\Delta_{i}^{-2}}$. However, knowledge of these two parameters beforehand is a difficult task so an online approach to estimate them is used in Adaptive UCB-E. The next algorithm called Successive Elimination proposed by \cite{even2006action} also proceeds by eliminating one arm after every round and the authors give PAC-guarantees for them. In PAC guarantee algorithms the learning agent comes up with an $\epsilon$-optimal arm with $\delta$ error probability. They also propose a further modification of 
Successive Elimination called Median Elimination which removes one half of the arms after every round. The sample complexity of Successive Elimination for any sub-optimal arm is bounded by $O\bigg(\sum_{\Delta_{i}>\epsilon}\dfrac{\log(\frac{K}{\delta\Delta_{i}})}{\Delta_{i}^{2}}+ \dfrac{N(\Delta,\epsilon)}{\epsilon^{2}}\log(\dfrac{N(\Delta,\epsilon)}{\delta})\bigg)$, where $N(\Delta,\epsilon)$ are the number of arms which are $\epsilon-$optimal whereas for Median Elimination the sample complexity for any sub-optimal arm is bounded by $O(\sum_{i=1}^{\log_{2}K}\dfrac{K}{\epsilon^{2}}\log(\dfrac{1}{\delta}))$, where $\delta$ and $\epsilon$ are the parameters defined before.  For further insight into various approaches in dealing with stochastic multi-armed bandit we refer the reader to \cite{bubeck2012regret}.
	
	It is also important to distinguish between two approaches in the UCB type algorithms. One approach uses explicit mean estimation using Chernoff Bounds for calculating the confidence bound whereas the other approach uses variance estimation using Bernstein or Bennett's Inequality to estimate the confidence bound. Such variance estimation is found in \cite{auer2002finite} for the UCB-Normal and UCB-Tuned algorithm and later in \cite{audibert2009exploration}  where they use Bernstein Inequality to build the confidence term for UCB-V algorithm. But UCB1, UCB2, MOSS, UCB-Revisited, KL-UCB and Median-Elimination uses no such variance estimation techniques. In our analysis also we use no variance estimation methods.
	
	Some of the more recent algorithm like MOSS and KL-UCB provides further refinements to the upper bound in the stochastic multi-armed bandit case. In \cite{audibert2009minimax} the MOSS(Minimax Optimal  Strategy in Stochastic case) algorithm achieves a distribution free upper bound on the regret as $const.\sqrt{TK}$ and the authors also propose a distribution dependent upper bound as \\$\sum_{i:\Delta_{i}>0}\dfrac{K\log(2+T\Delta_{i}^{2}/K)}{\Delta_{i}}$. In \cite{garivier2011kl} the authors propose KL-UCB which achieves an upper bound on regret as $\sum_{i:\Delta_{i}>0}\bigg(\dfrac{\Delta_{i}(1+\alpha)\log T}{D(r_{i},r^{*})}+C_{1}\log\log T+\dfrac{C_{2}(\alpha)}{T^{\beta(\alpha)}}\bigg)$ which is strictly better than UCB1 as we know from Pinsker's inequality $D(r_{i},r^{*}) > 2\Delta_{i}^{2}$. KL-UCB beats UCB1, MOSS and UCB-Tuned in various scenarios. Another algorithm that has been proposed in \cite{abbasi2011improved} called $UCB(\delta)$ creates a confidence interval which does not depend on timestep or 
on horizon T. The regret upper bound in this algorithm is given by $\sum_{i:\Delta_{i}>0}\bigg(3\Delta_{i}+\dfrac{16}{\Delta_{i}}\log\big(\dfrac{2K}{\Delta_{i}\delta}\big)\bigg)$ , where $\delta$ is the error probability as defined before. In \cite{honda2010asymptotically} the authors come up with the algorithm Deterministic Minimum Empirical Divergence also called DMED$+$(as referred by \cite{garivier2011kl}) which is first order optimal. This algorithm keeps a track of arms whose empirical mean are close to the optimal arm and takes help of large deviation ideas to find the optimal arm.
	
	Also, we mention the algorithm Exp3 in the adversarial bandit scenario. In the adversarial case, the agent is playing against an adversary who can arbitrarily set a reward on any arm. Algorithms used in adversarial case can be used in the stochastic scenario but not vice-versa. For Exp3 in \cite{auer2002nonstochastic} the authors achieve an upper bound of the order of $O(S\sqrt{KT\log KT})$, where $S$ is defined as the hardness of a problem. Finally, we mention one algorithm that involves Bayesian estimation technique called Thompson Sampling(TS) which uses a prior distribution over arms to estimate the posterior distribution and thereby converge on the optimal arm. In \cite{agrawal2011analysis} the authors come up with a regret upper bound of the order of $O([K(\dfrac{1}{\Delta^{2}})^{2}]\log T)$ for $K$-arm stochastic bandits using Thompson Sampling. 


\subsection*{Our contributions}
In this work, we present a novel method where we cluster the arms in each round based on the average estimated payoff $\hat{r}_{i}$. The purpose of this approach is to control exploration and at the same time exploit the clusters formed to create tight confidence bounds. To do this, we use hierarchical agglomerative clustering using the single-linkage clustering scheme as mentioned in \cite{friedman2001elements}. This is used to group the arms together at the start of each round. Then we deploy a two-pronged approach of exploring inside each clusters to eliminate sub-optimal arms and separately based on other conditions eliminating some weak clusters with all the arms inside it. After each round, at the beginning of the next round we again cluster arms after destroying the old cluster structures formed in the previous round. The logic behind clustering at the beginning of each round afresh is simply that at the initial rounds we have clusters formed with very bad purity level(we can imagine the purity level 
of a cluster being judged by how many arms having similar or $\epsilon_{m}-$close means $r_{i}$ getting clustered together), where in the later rounds we can have tight clusters with high purity level since now we have a better estimate of $\hat{r}_{i}, \forall i\in A$.

	The within cluster arm elimination condition and the entire cluster elimination are two complimentary strategies for speedy elimination of sub-optimal arms. Unlike UCB-Revisited from \cite{auer2010ucb} the within cluster arm elimination leads to dividing the larger problem of finding the optimal arm from the whole arm set into small sub-problems where in each such small clusters (say $M$ clusters, $M\leq K$) we have $M$ arm elimination conditions, thereby increasing the probability of deletion of a sub-optimal arm and hence reducing regret. Also by dividing it into small sub-problems the growth of our pulls in each round is always small given the optimal arm has survived which we ensure by tuning parameters appropriately.  We start with a very small cluster size limit(say 2) and double the limit after every round. We put an upper bound on this limit to be decided by the algorithm online so that the cluster size remains bounded and also since we are exploring each arm in a cluster based on its size, such 
bounds helps in controlling the exploration within a cluster because we know that single link clustering often forms large chains where the first and large elements may not at all be similar to each other. The entire cluster elimination conditions exploit the idea that if the optimal arm has survived till the later rounds it will be in a cluster of its own with no other sub-optimal arm and then we eliminate all sub-optimal clusters in a single round. This is essentially the same core concept of various round based strategies.
	
	Since, we have assumed that the horizon $T$ is known prior to the agent, hence we introduce the parameter $\psi(m)$ which can be sufficiently tuned to tide over very large or small horizons to enable a balanced exploration by the agent.
	%To achieve all the above guarantees we define three parameters in our algorithm, $D$ which is the maximum cluster size allowed, $\psi(m)$ which is a tuning parameter for exploration and $w\geq 2$ a weight factor which enables a faster elimination of sub-optimal arms.
\newline
Summarizing, the contributions of this research are listed below:

\begin{enumerate}
\item We propose a cluster based round-wise algorithm with two arm elimination conditions in each round.
%\item To our knowledge no other algorithm has proposed such UCB-type cluster based algorithm before.
\item We achieved a lower regret bound than UCB-Revisited(\cite{auer2010ucb}), UCB($\delta$) (\cite{abbasi2011improved}), UCB1(\cite{auer2002finite}), UCB2(\cite{auer2002finite}), UCB-Tuned(\cite{auer2002finite}), Median Elimination(\cite{even2006action}),  Exp3(\cite{auer2002nonstochastic}) and MOSS (\cite{audibert2009minimax} in scenarios when $\Delta$ is small and $K$ is large which is encountered frequently in web-advertising domain, which we verify empirically.
\todos{I don't see a rigourous theoretical justification for this claim anywhere in the paper. We have a regret bound in Theorem 1, but where is it shown that the bound in Thm 1 is better than UCB-Revisited(\cite{auer2010ucb}), UCB($\delta$) (\cite{abbasi2011improved}), UCB1(\cite{auer2002finite}), UCB2(\cite{auer2002finite}), UCB-Tuned(\cite{auer2002finite}), Median Elimination(\cite{even2006action}),  Exp3(\cite{auer2002nonstochastic}) and MOSS (\cite{audibert2009minimax}?}
\todos{\textit{``when $\Delta$ is small and $K$ is large which is encountered frequently in web-advertising domain...''} Can you give a reference that justifies low $\Delta$ in web-advertising?}
\item Our algorithm also compares well with DMED, DMED($+$) and KL-UCB.
\todos{Is the comparison empirical or theoretical? Please specify}
\item In the critical case when $r_{1}=r_{2}=..=r_{K-1}<r^{*}$ and $\Delta_{i}$s are small, this approach has a significant advantage over other methods.
\item We also come up with an error bound to prove that the error probability decreases exponentially after each round.
\item Unlike KL-UCB our algorithm parameter $\psi(m)$ is not distribution-specific and also our algorithm does not involve calculation of a complex, time consuming function like the divergence function of KL-UCB.
%though the authors specified in \cite{garivier2011kl} that for optimal result only one should use the divergence function specific to the type of distribution.
%\item We also provide a short discussion on what other applications our algorithm can be employed successfully.
\end{enumerate}

The paper is organized as follows, in section $5$ we present the algorithm and in section $6$ we discuss the algorithm and why it works. Section $7$ deals with all the proof including the proofs on regret bound and error bound. In section $8$ we present the experimental run of the algorithm and in section $9$ we conclude. 


\section{Clustered UCB}
\label{sec:clusucb}
\paragraph{Notation.}
\label{sec:prelims}
In this work, an arm (any arbitrary one) is denoted as $a_{i}$ and $a^{*}$ is the optimal arm. The total time horizon is $T$. Any arbitrary round is denoted by $m$. The arm set containing all the arms is $A$, while $|A|=K$. Any arbitrary cluster is denoted by $s_{k}$. $S$ denotes the set of all clusters $s_{k}\in S$ in the $m$-th round and $|S|=p\leq K$. $p$ is the number of clusters pre-specified at the start of algorithm. The variable $\ell$ is the cluster size limit.
%\todos{\textit{``$D$ is the maximum cluster size..''} I believe you mean $D_m$. Done. written $D_{m}$ (Subho)}
%The $max_{i\in s_{i}}\hat{r}_{s_{i}}$ denotes the arm $a_{i}$ which has the $\max \hat{r}_{i}, \forall i\in s_{i}$. Similarly,  $min_{j\in s_{j}}\hat{r}_{s_{j}}$ denotes the arm $a_{j}$ which has the $\min \hat{r}_{j}, \forall j\in s_{j}$. 
$\hat{r_i}$ denotes the average estimated payoff of the $i$-th arm. The variable $n_{s_{k}}$ denotes the number of times each arm $a_{i}\in s_{k}$ is pulled in each round $m$. The variable $B_{m}$ denotes the arm set containing the arms that are not eliminated till round $m$.
%The variable $\hat{\Delta}_{m}$ denotes the empirical gap between the arm having the highest empirical mean and the arm having the lowest empirical mean from $B_{m}$ in that round $m$, that is, in any round $m$, 
%$\hat{\Delta}_{m}=\max_{i\in B_{m}}{\lbrace\hat{r}_{i}\rbrace}-\min_{j\in B_{m}}{\lbrace\hat{r}_{j}\rbrace}$
%\todos{I have a few issues with this notation:\\
%1) What is $s$ in $\hat{\Delta}_{s,m}$? I don't see any $s$ on the RHS of the defn above? (removed s from all $\hat{\Delta}_{m}$, Subho)\\
%2) In CS terms, $i$ and $j$ are temporary variables for the max and min and so, I dont really get the $i \ne j$ constraint. (removed $i \ne j$, Subho)\\
%3) $\hat r_i$ is not defined (Defined $\hat r_i$, Subho)
%}
	In any cluster $s_{k}\in S$, we denote $\hat{r}_{max_{s_{k}}}\in s_{k}$ as the maximum estimated payoff and $\hat{r}_{min_{s_{k}}}\in s_{k}$ as the minimum estimated payoff. $\Delta_{i}=r^{*}-r_{i}$ and also let $\Delta=min_{a_{i}\in A}\Delta_{i}$, that is it is the minimum possible gap over all arms in $A$. The exploration regulatory factor is denoted by $\psi_{m}$. %\todos{\textit{``$\hat{r}_{min_{s_{i}}}\in s_{i}$ as the arm with minimum estimated payoff''}. $\hat{r}_{min_{s_{i}}}\in s_{i}$ is the minimum estimated payoff and not the arm corresponding to min est payoff. Similar writing pops up at several other places, for e.g., in the statements of the propositions. (Subho) Changed the line as saying the minimum/maximum estimated payoff}
%The parameter $\psi(m)$ is a monotonically decreasing function over the rounds, that is $\psi(m+1)\leq \psi(m)$. 
%Also, we define $w\geq 2$, a weight factor.
We also assume that all rewards are bounded in $[0,1]$. For simplicity we will assume that there is only one optimal arm $a^{*}$.


\subsubsection*{The algorithm}

Algorithm~\ref{alg:clusucb} presents the pseudocode of ClusUCB.

%%%%%%%%%%%%%%%% alg-custom-block %%%%%%%%%%%%
\algblock{ArmElim}{EndArmElim}
\algnewcommand\algorithmicArmElim{\textbf{\em Arm Elimination}}
 \algnewcommand\algorithmicendArmElim{}
\algrenewtext{ArmElim}[1]{\algorithmicArmElim\ #1}
\algrenewtext{EndArmElim}{\algorithmicendArmElim}

\algblock{ClusElim}{EndClusElim}
\algnewcommand\algorithmicClusElim{\textbf{\em Cluster Elimination}}
 \algnewcommand\algorithmicendClusElim{}
\algrenewtext{ClusElim}[1]{\algorithmicClusElim\ #1}
\algrenewtext{EndClusElim}{\algorithmicendClusElim}

\algtext*{EndArmElim}
\algtext*{EndClusElim}

\begin{algorithm}[t]
\caption{ClusUCB}
\label{alg:clusucb}
\begin{algorithmic}
\State {\bf Input:} A partition of $A$ into clusters from Algorithm \ref{alg:rua}, exploration parameters $\rho_a$, $\rho_s$, time horizon $T$.
\State {\bf Initialization:} Set $B_{0}:=A$, $\psi=\log T$ and $\epsilon_{0}:=1$.
%\State For rounds $m=1,2,.. \lceil\frac{1}{2}\log T\rceil$
\For{$m=0,1,..\big \lfloor \dfrac{1}{2}\log_{2} \dfrac{T}{e}\big\rfloor$}	
\State Pull each arm in $s_{k}$ $n_{m}=\bigg\lceil\dfrac{2\log{(\psi T\epsilon_{m}^{2})}}{\epsilon_{m}}\bigg\rceil$ number of times, $\forall s_{k}\in S$ 
%\State \hspace*{2em} Calculate $w_{s_{i}}=\bigg\lceil\dfrac{1}{\ell\hat{\Delta}_{s_{i}}}\bigg\rceil$,if $\hat{\Delta}_{s_{i}}\neq 0, \forall s_{i}\in S$
%\newline\hspace*{8em}$=1$, otherwise, and $\hat{\Delta}_{s_{i}}=\max_{i\in s_{i}}{\lbrace\hat{r}_{i}\rbrace}-\min_{j\in s_{i}}{\lbrace\hat{r}_{j}\rbrace}, i\neq j$
\ArmElim
\State For each cluster $s_k \in S$, delete arm $a_{i}\in s_{k}$ if
$$\hat{r}_{i} + \sqrt{\dfrac{\rho_{a}\log{(\psi T\epsilon_{m}^{2})}}{2 n_{m}}}  < \max_{a_{j}\in s_{k}}\bigg\lbrace\hat{r}_{j} -\sqrt{\dfrac{\rho_{a}\log{(T\epsilon_{m}^{2})}}{2 n_{m}}} \bigg\rbrace.$$
% where $\rho_{a}=\dfrac{1}{w_{m}}$ and remove all such arms from $B_{m}$.
\EndArmElim
\ClusElim
\State Delete cluster $s_{k}\in S$ if 
\begin{align*}
 \max_{a_{i}\in s_{k}}\bigg\lbrace\hat{r}_{i} + \sqrt{\dfrac{\rho_{s}\log{(\psi T\epsilon_{m}^{2})}}{2 n_{m}}}\bigg\rbrace  \\
 < \max_{a_{j}\in B_{m}} \bigg\lbrace\hat{r}_{j} - \sqrt{\dfrac{\rho_{s} \log{(T\epsilon_{m}^{2})}}{2 n_{m}}}\bigg\rbrace.
\end{align*}
%  and remove all such arms in the cluster $s_{k}$ from $B_{m}$ to obtain $B_{m+1}$.
\EndClusElim
\State Set $\epsilon_{m+1}:=\dfrac{\epsilon_{m}}{2}$
%\State \hspace*{2em} $\ell_{m+1}:=\min\lbrace 2\ell_{m}, K\rbrace$
%\State \hspace*{2em} $w_{m+1}:=2w_{m}$
\State Stop if $|B_{m}|=1$ and pull $\max_{a_{i}\in B_{m}}\hat{r}_{i}$ till $T$ is reached.
\EndFor
\end{algorithmic}
\end{algorithm}

ClusUCB begins with a initial clustering of arms that is performed by random uniform allocation - see Algorithm \ref{alg:rua}. The set of clusters $S$ thus obtained satisfies $|S|=p$, with individual clusters having a size that is bounded above by $\ell=\bigg\lceil \dfrac{K}{p} \bigg\rceil$.
%For creating these clusters we rely on the parameter $\epsilon_{m}$, which is initialized at $1$ and halved after every round. 
%$\hat{\Delta}_{m}$(in which all $\hat{r}_{i}$ lies) divided by $\ell$. 
%If the range before any round is found to be $0$ then we conduct a small exploration with the alternate definition of $\epsilon_{m}=\dfrac{1}{D_{m}}$. 
%Thus, we see that as the rounds progresses, $\epsilon_{m}$ gets smaller and smaller resulting in tighter and tighter clusters with high purity level. Single link clustering tends to create large clusters with many elements in a single cluster. To avoid this behavior, mainly because we are conducting exploration inside a cluster based on its size, we bound the maximum cluster size possible in any round by $\ell=\min\lbrace 2^{m}, K \rbrace$. Hence as $\epsilon_{m}$ tends to $\Delta$ the upper limit on the cluster size gets fixed. 

Each round of ClusUCB involves both individual arm as well as cluster elimination conditions that are inspired by UCB-Improved. Notice that, unlike UCB-Improved, there is no longer a single point of reference based on which we are eliminating arms. Instead now we have as many reference points to eliminate arms as number of clusters formed. 
Further, the exploration factors $\rho_{a}\in (0,1]$ and $\rho_{s}\in (0,1]$ governing the arm and cluster elimination conditions, respectively, are relatively more aggressive than that in UCB-Improved. The number $n_m$ of times each arm is pulled per round in ClusUCB is lower than that of UCB-Improved, which used $n_{m}=\bigg\lceil \dfrac{2log(T\epsilon_{m}^{2})}{\epsilon_{m}^{2}} \bigg\rceil$ and Median-Elimination, which used $n_m=\dfrac{4}{\epsilon^{2}}\log\big(\dfrac{3}{\delta}\big)$, where $\epsilon,\delta$ are confidence parameters.

We can be this aggressive because we have divided the larger problem into smaller sub-problems, doing local exploration and eliminating sub-optimal arms within each clusters with high probability. In particular, when $K$ is large and the gaps $\Delta_i$ are small, it is efficient to remove sub-optimal arms quickly rather getting tied down in hopeless exploration. 
The theoretical analysis confirms this intuition, as choosing $\rho_a=\rho_s = \max\lbrace \frac{1}{2},\frac{1}{2^m}\rbrace$ and $\psi_m=K^2T$, ClusUCB is able to achive a better gap-dependent regret upper bound as compared to UCB-Improved and MOSS. 
%Also our total regret depends on how many arms has survived till $m$-th round and so we don't need to keep track on the number of clusters formed.

%Through cluster elimination condition we ensure that the stopping condition is reached faster. This is a much aggressive elimination condition and in the proofs we give a further analysis on why this works. The parameter $\rho\in (0,1]$ in the confidence interval actually makes the cluster elimination a faster elimination condition than the elimination condition in UCB-Improved.
	
%This is a parameter which we use to tune our exploration and we define its structure later in the proofs. There is also the weight($w_{s_{i}}$) parameter which   is calculated online and helps in eliminating arms with a higher probability which we employ to reduce the cumulative regret and this parameter is decide online specific to the cluster $s_{i}\in S$.

\begin{algorithm}[t]
\caption{Clustering by random uniform allocation}
\label{alg:rua}
\begin{algorithmic}
\State Set $\ell=\bigg\lceil \dfrac{K}{p} \bigg\rceil$
\State Permute arms in $A$ randomly to create $A_{random}$.
\State Create clusters $s_{i}=\lbrace a_{i}\rbrace, \forall i\in A_{random}$
\For{ $i=1$ to $K$}
\For{$j=i+1$ to $K$}
\State Merge $s_{i},s_{j}$ into $s_{i}$ if $\exists a_{i}\in s_{i} $ and $\exists a_{j}\in s_{j},$ such that $|s_{i}|+|s_{j}|\leq \ell$
\EndFor
\EndFor
\State Rename all partitions that have been formed as $s_{1}$ to $s_{p}$, where $|S|=p$ after merging
\end{algorithmic}
\end{algorithm}





%\newpage
\section{Main results}
\label{sec:results}
	
	Here, we state the main theorem of the paper which shows the regret upper bound of ClusUCB.
	
\begin{theorem}[\textbf{\textit{Regret bound}}]
\label{Result:Theorem:1}
The regret $R_T$ of ClusUCB satisfies
\begin{align*}
&\E [R_{T}]\!\leq\! 
\sum\limits_{\substack{i\in A,\\\Delta_{i}\geq b}} \bigg\lbrace 2\Delta_{i}+
\frac{C_1(\rho_{a})T^{1-\rho_{a}}}{\Delta_{i}^{4\rho_{a}-1}} 
+ \frac{2C_1(\rho_{s})T^{1-\rho_{s}}}{\Delta_{i}^{4\rho_{s}-1}} \\
&\qquad\qquad+ \frac{32\rho_{a}\log{(\psi T\dfrac{\Delta_{i}^{4}}{16\rho_{a}^{2}})}}{\Delta_{i}}
+ \dfrac{32\rho_{s}\log{(\psi T\frac{\Delta_{i}^{4}}{16\rho_{s}^{2}})}}{\Delta_{i}}\bigg\rbrace  \\
%%%
&\qquad\qquad+
\sum\limits_{\substack{i\in A_{s^{*}},\\0\leq\Delta_{i}\leq b}}\bigg\lbrace \frac{T^{1-\rho_{a}}C_2(\rho_{a})}{\Delta_{i}^{4\rho_{a}-1}}+\frac{T^{1-\rho_{a}}C_2(\rho_a)}{b^{4\rho_{a} -1}} \bigg\rbrace + \\
%%%%%%%
&\!\sum\limits_{\substack{i\in A,\\0\leq\Delta_{i}\leq b}}\!\bigg\lbrace \! \frac{2T^{1-\rho_{s}}C_2(\rho_{s})}{\Delta_{i}^{4\rho_{s}-1}} +\frac{2T^{1-\rho_{s}}C_2(\rho_{s})}{b^{4\rho_{s} -1}} \bigg\rbrace 
\!+\! \max\limits_{i:\Delta_{i}\leq b}\Delta_{i}T, 
\end{align*}
where $b\geq \sqrt{\frac{e}{T}}$, $C_1(x) = \frac{2^{1+4x}x^{2x}}{\psi^{x}}$ and $C_2(x) = \frac{2^{2x+\frac{3}{2}}x^{2x}}{\psi^{x}}$. 
%, $\rho_{a}=\dfrac{1}{2},\rho_{s}=\dfrac{1}{2}$ and $\psi=K^{2}T$.
\end{theorem}
\begin{proof}
 A sketch of the proof is given in Section \ref{sec:proofSketch} and the complete proof is available in Appendix \ref{App:A}.
\end{proof}

	
	
%\begin{remark}
%\label{Result:Rem:1}
%By setting $b=\sqrt{\dfrac{e}{T}}$, the term $max_{i:\Delta_{i}\leq b}\Delta_{i}T$ gets trivially bounded by $\sqrt{KT}$ as mentioned in UCB-Improved.
%%which makes the logarithmic term  the main term for suitable $b$
%\end{remark}

We now specialize the result in the theorem above by substituting specific values for the exploration constants $\rho_{s}$, $\rho_{a}$ and $\psi$. 
%and how these values are reduced after every round significantly affects our regret bound. A discussion on their various definitions are given in Appendix \ref{App:E}. A discussion on exploration regulatory factor($\psi$) and its effect is deferred to Appendix \ref{App:D}. A discussion on the three approaches of using only arm elimination in Algorithm 1 or using only cluster elimination or using both is deferred to Remark \ref{App:E:Rem:1}(Appendix \ref{App:E}).

%%%% Gap dependent bound
\begin{corollary}[\textbf{\textit{Gap-dependent bound}}]
\label{Result:Corollary:1}
With $\psi=\frac{T}{\log (KT)}$, $\rho_{a}=\frac{1}{2}$, $\rho_{s}=\frac{1}{2} $ and $b\approx\sqrt{\frac{K\log K}{T}}$,  we have the following gap-dependent bound for the regret of ClusUCB:
\begin{align*}
&\E R_T \!\le\! \sum_{i\in A:\Delta_{i}\geq b}\bigg\lbrace\dfrac{12\sqrt{\log (KT)}}{\Delta_{i}}  + \dfrac{64\log{(\frac{T\Delta_{i}^{2}}{\sqrt{\log (KT)}})}}{\Delta_{i}}\bigg\rbrace \\
&+ \sum\limits_{\substack{i\in A_{s^{*}},\\0\leq\Delta_{i}\leq b}}\dfrac{5.6\sqrt{\log (KT)}}{\Delta_{i}} + \sum\limits_{\substack{i\in A,\\0\leq\Delta_{i}\leq b}}\dfrac{16\sqrt{\log (KT)}}{\Delta_{i}} 
\end{align*}
\end{corollary}
\begin{proof}
 See Appendix \ref{App:Proof:Corollary:1}.
\end{proof}

%\begin{remark}
%\label{Result:Rem:4}
The most significant term in the bound above is $\sum_{i\in A:\Delta_{i}\geq b}\frac{64\log{\big(T\frac{\Delta_{i}^{2}}{\sqrt{\log (KT)}}\big)}}{\Delta_{i}}$ and hence, the regret upper bound for ClusUCB is of the order $O\bigg(\frac{K\log \big(\frac{T\Delta^{2}}{\sqrt{\log (KT)}}\big)}{\Delta}\bigg)$. As shown in Table \ref{tab:regret-bds}, the gap-dependent bound of ClusUCB is always better than UCB1 and UCB-Improved. 
%. This can be shown by,
%
%\begin{align*}
%K*K\log\bigg(\dfrac{T\Delta^{2}}{K}\bigg) = K\log\bigg( \dfrac{T^{K-1}\Delta^{2K-2}}{K^{K}} * T\Delta^{2}\bigg)
%\end{align*}
In comparison to the gap-dependent bound of MOSS, we observe that ClusUCB will be better if 
%$K\log\bigg(\dfrac{T\Delta^{2}}{\sqrt{\log(KT)}}\bigg)$ when,
$\frac{T^{K-1}\Delta^{2K-2}}{K^{K}} > \frac{1}{\sqrt{\log(KT)}}$, which is equivalent to $\Delta > \frac{2}{\sqrt{T}(\log 2T)^{\frac{1}{4}}}$ for
$K\geq 2$. Since Corollary \ref{Result:Corollary:1} holds for all $\Delta \geq \sqrt{\frac{e}{T}} > \frac{2}{\sqrt{T}(\log 2T)^{\frac{1}{4}}}$, it can be clearly seen that for all $\sqrt{\frac{e}{T}} \leq \Delta\leq 1$ and $K\geq 2$, the gap-dependent bound is better than MOSS.
%\begin{corollary}[\textbf{\textit{Gap-independent bound}}]
%\label{Result:Corollary:2}
%With $\psi=K^{2}T$, $\rho_{a}=\dfrac{1}{4}$ ,$\rho_{s}=\dfrac{1}{4} $ and $b\approx\sqrt{\dfrac{K\log K}{T}}$, the regret of ClusUCB is bounded  by $\bigg\lbrace 11.6\sqrt{KT} + 4\dfrac{\sqrt{KT}}{p} + 64\sqrt{KT\log K} + \dfrac{32\log{(\log K)}}{\sqrt{\log K}}\bigg\rbrace$.
%\end{corollary}
%\begin{proof}
% See Appendix \ref{App:Proof:Corollary:2}.
%\end{proof}

\begin{corollary}[\textbf{\textit{Gap-independent bound}}]
\label{Result:Corollary:2}
With $\psi=K^{2}T$, $\rho_{a}=\frac{1}{4}$, $\rho_{s}=\frac{1}{2}$ and $b\approx\sqrt{\dfrac{K\log K}{T}}$, 
 we have the following gap-dependent bound for the regret of ClusUCB:
\begin{align*}
\E R_T &\!\le\! 2\sqrt{KT} + 64\sqrt{KT\log K} + \dfrac{32\log{(\log K)}}{\sqrt{\log K}} \\
&\qquad+ 4\dfrac{\sqrt{KT}}{p}  + 16\sqrt{\dfrac{T}{K\log K}}.
\end{align*}
\end{corollary}
\begin{proof}
 See Appendix \ref{App:Proof:Corollary:2}.
\end{proof}


From the above result, we observe that the order of the regret upper bound of ClusUCB is $O(\sqrt{KT\log K})$ and this matches the order of UCB-Improved. On the other hand, the current analysis used to arrive at Theorem \ref{Result:Theorem:1} for ClusUCB is not enough to obtain the order $O(\sqrt{KT})$ bound of MOSS.

Thus, we observe that clustering in conjunction with improved exploration via $\rho_{a},\rho_{s}$ helps in reducing the constant associated with the factor $\sqrt{KT}$ and the $\frac{\log\bigg(\psi\frac{T\Delta_{i}^{4}}{\rho_{s}^{2}}\bigg)}{\Delta_{i}}$ 
term and $\psi,\rho_{a},\rho_{s}$ helps in stabilizing the term $\log\bigg(\frac{\psi  T\Delta_{i}^{4}}{\rho_{s}^{2}}\bigg)$.

%\begin{corollary}[\textbf{\textit{Gap-independent bound}}]
%\label{Result:Corollary:3}
%With $\psi=K^{2}T$, $\rho_{a}=\dfrac{1}{4}$ ,$\rho_{s}=\dfrac{1}{2} $ and $b\approx\sqrt{\dfrac{K\log K}{T}}$, the regret of ClusUCB is bounded  by $\bigg\lbrace 2\sqrt{KT} + 64\sqrt{KT\log K} + \dfrac{32\log{(\log K)}}{\sqrt{\log K}} + 4\dfrac{\sqrt{KT}}{p}  + 16\sqrt{\dfrac{T}{K\log K}}\bigg\rbrace$.
%\end{corollary}
%\begin{proof}
% See Appendix \ref{App:Proof:Corollary:3}.
% 
%From the above result we see that this bound is less than than the bound in Corollary \ref{Result:Corollary:2}. Here we also define the error bound, which is the bound on the regret obtained after elimination of the optimal arm $a^{*}$. So the  error bound from Corollary \ref{Result:Corollary:2} is 
%
%\begin{align*}
%5.6\sqrt{KT} + 4\dfrac{\sqrt{KT}}{p}
%\end{align*}
%
%which is more than the error bound from Corollary \ref{Result:Corollary:3},
%
%\begin{align*}
%16\sqrt{\dfrac{T}{K\log K}} + 4\dfrac{\sqrt{KT}}{p}
%\end{align*}
%
%for $ \sqrt{\log K} \leq p\leq\frac{K}{2}$. So by taking $\rho_{a} < \rho_{s}$ we are able to reduce the error bound and this helps the algorithm to perform better in regimes where the gaps are small by keeping the optimal arm safe with high probability. 
%
%\end{proof}


\section{Proof of Theorem 1}
\label{sec:proofTheorem}
\label{sec:proofTheorem}

We sketch the proof for Theorem \ref{Result:Theorem:1} here. 
The proof involves the following steps:\\
\textbf{\textit{Step 1:}}
We analyze ClusUCB-AE, i.e., the variant of ClusUCB that uses arm elimination condition only. In other words, we bound the probability of sub-optimal arm elimination, which in turn bounds the expected regret of ClusUCB-AE (see Proposition \ref{proofSketch:Prop:1} below). 

\textbf{\textit{Step 2:}}
We analyze ClusUCB-CE, i.e., the variant of ClusUCB that uses cluster elimination condition only and pulls the best arm within the last leftover cluster.
Proposition \ref{proofSketch:Prop:2} presents the expected regret for ClusUCB-CE (see Proposition \ref{proofSketch:Prop:2} below). 

\textbf{\textit{Step 3:}}
Finally we combine the individual bounds in the steps above to get the regret upper bound in Theorem \ref{Result:Theorem:1}.  
	




\begin{proof}
The optimal cluster which contains $a^{*}$ is denoted by $s^{*}$. The subset of arms belonging to cluster $s_{k}$ is denoted by $A_{s_{k}}$ and similarly the subset of arms belonging to $s^{*}$ is denoted by $A_{s^{*}}$. Combining both the cases of Proposition \ref{proofSketch:Prop:1} and Proposition \ref{proofSketch:Prop:2} we can see that a sub-optimal arm $a_{i}$ can only be eliminated given that either $m_{i}$ or $g_{s_{k}}$s.t $\exists a_{i}\in s_{k}$ happens with $a^{*}\in s^{*}$ still surviving. In Proposition \ref{proofSketch:Prop:1} we consider only arm elimination and in Proposition \ref{proofSketch:Prop:2} we consider only cluster elimination. Also this proof we will consider $p>1$. So there will be slight modification from what we proved in proposition \ref{proofSketch:Prop:1} with $p=1$. For random uniform allocation we will assume that each cluster $s_{k},\forall s_{k}\in S$, gets such an arm such that $r_{{max_{s_{k}}}}\geq r_{a_{i}},\forall i\in s_{k}$. Again, $r_{a^{*}}\geq r_{{max_{s_{k}}}}, \forall s_{k}\in S$. Here also we take $\rho_{a},\rho_{s}\in (0,1]$ as a constant in this proof whereby in Corollary \ref{Result:Corollary:1} and \ref{Result:Corollary:2} we use the different definitions. The theoretical analysis remains same as we have always bounded the values of $\rho_{a}\in (0,1]$(see Appendix \ref{App:E}). Let $A^{'}=\lbrace i \in A,\Delta_{i}> b\rbrace$ and $A^{''}=\lbrace i \in A,0 < \Delta_{i} \leq b\rbrace$. 
%Also we cluster the arms based on $\epsilon_{m}$.
% One vital point we point out is that, $\epsilon_{m}$(in proposition $3$) = $\epsilon_{g}$(in proposition $4$).
\subsection*{Case a: \textit{Some sub-optimal arm $a_{i}$ is not eliminated in round $max(m_{i},g_{s_{k}})$ or before and the optimal arm $a^{*}\in B_{m_{i}}$}}
 
	In this case, we are looking at event of the maximum round till which atleast one of $m_{i}$ or $g_{s_{k}}$ did not happen. So, a sub-optimal arm $a_{i}$ cannot get eliminated in $4$ ways,
\begin{enumerate}
\item $a_{i}$ in $s^{*}$ and $m_{i}$ does not happen which is Proposition \ref{proofSketch:Prop:1}, case $a1$.
\item $a_{i}$ in $s_{k}$, where $r_{max_{s_{k}}}\leq r^{*}$ and $m_{i}$ does not happen. This case was not dealt in Proposition \ref{proofSketch:Prop:1} as there we took $p=1$. Since, now $p>1$ and $r_{max_{s_{k}}}\leq r^{*}$, following the same way as case $a$, Proposition \ref{proofSketch:Prop:1} we can show that the probability of $a_{i}$ not getting eliminated  cannot be worse than Proposition \ref{proofSketch:Prop:1}, case $a1$ given that $\sqrt{\rho_{a}\epsilon_{m}}< \dfrac{\Delta^{'}_{i}}{2}$ where $\Delta^{'}_{i}=r_{max_{s_{k}}} - r_{i}\geq\Delta_{a_{max_{s_{k}}}}$ such that $r_{i}\in s_{k}$. Plugging in this $\Delta^{'}_{i}$ in Proposition \ref{proofSketch:Prop:1}, case $a$ we can derive a similar bound where $\Delta^{'}_{i}\geq \Delta_{a_{max_{s_{k}}}}$ because otherwise $\sqrt{\epsilon_{m}\rho_{s}}< \dfrac{\Delta_{a_{max_{s_{k}}}}}{2}$ will happen and the cluster $s_{k}$ gets eliminated or $a_{max_{s_{k}}}$ will eliminate $a^{*}$ which is dealt later.
\item $a_{i}\in s_{k}, a^{*}\in C_{g_{s_{k}}}$ and $g_{s_{k}}$ does not happen which is Proposition \ref{proofSketch:Prop:2}, case $a1$.
\item $a_{i}\in s_{k}, a^{*}\notin C_{g_{s_{k}}}$ and $g_{s_{k}}$ does not happen which is Proposition \ref{proofSketch:Prop:2}, case $a2$.
\end{enumerate}
Taking summation of the events mentioned above($a1$-$a4$) gives us an upper bound on the regret given that the optimal arm $a^{*}$ is still surviving, 
\begin{align*}
 &\underbrace{\sum_{i\in A_{s^{*}}^{'}}\bigg(\dfrac{C_{1}(\rho_{a})T^{1-\rho_{a}}}{\Delta_{i}^{4\rho_{a}-1}}\bigg)}_{\text{case a1}} + \underbrace{\sum_{i\in A\setminus A_{s^{*}}^{'}}\bigg(\dfrac{C_{1}(\rho_{a})T^{1-\rho_{a}}}{\Delta_{i}^{4\rho_{a}-1}}\bigg)}_{\text{case a2}} \\
 & + \sum_{i\in A^{'}}\bigg\lbrace \underbrace{\bigg(\dfrac{2C_{1}(\rho_{s})T^{1-\rho_{s}}}{\psi^{\rho_{s}}\Delta_{i}^{4\rho_{s}-1}}\bigg)}_{\text{case a3+a4}}\bigg\rbrace \\
& = \sum_{i\in A^{'}}\bigg\lbrace \bigg(\dfrac{C_{1}(\rho_{a})T^{1-\rho_{a}}}{\Delta_{i}^{4\rho_{a}-1}}\bigg) + \bigg(\dfrac{2C_{1}(\rho_{s})T^{1-\rho_{s}}}{\Delta_{i}^{4\rho_{s}-1}}\bigg)\bigg\rbrace
\end{align*}

%& = \sum_{i\in A}\bigg\lbrace \bigg(\dfrac{2^{1+4\rho_{s}}\rho_{s}^{2\rho_{s}}T^{1-\rho_{s}}}{\psi^{\rho_{a}}\Delta_{i}^{4\rho_{s}-1}}\bigg) + \bigg(\dfrac{2^{2+4\rho_{s}}\rho_{s}^{2\rho_{s}}T^{1-\rho_{s}}}{\psi^{\rho_{s}}\Delta_{i}^{4\rho_{s}-1}}\bigg)\bigg\rbrace


\subsection*{Case b: \textit{Either an arm $a_{i}$ is eliminated in round $m_{i}$ or $g_{s_{k}}$ or before or else there is no optimal arm $a^{*}\in B_{m_{i}}$}} 

\subsubsection*{Case b1: \textit{If an optimal arm $a^{*}\in B_{m_{i}}$ then the maximum pull of all arms $a_{i}\in A^{'}$}} 
 
	For any sub-optimal arm still surviving given $m_{i}$ or $g_{s_{k}}:a_{i}\in s_{k}$ have not happened and $a^{*}\in s^{*}$ still surviving then they get pulled $n_{m_{i}}$ or $n_{g_{s_{k}}}$ number of times(combining the result of Proposition \ref{proofSketch:Prop:1} (case $b1$) and Proposition \ref{proofSketch:Prop:2} (case $b1)$). Hence, we can show that till an arm or a cluster is eliminated, the maximum regret suffered due to pulling of a sub-optimal arm(or a sub-optimal cluster) is no less than,
 \begin{align*}
 &\sum_{i\in A^{'}}\bigg\lbrace\bigg(\Delta_{i}+\dfrac{32\rho_{a}\log{(\psi T\dfrac{\Delta_{i}^{4}}{16\rho_{a}^{2}})}}{\Delta_{i}}\bigg) \\
 &+ \bigg(\Delta_{i}+\dfrac{32\rho_{s}\log{(\psi T\dfrac{\Delta_{i}^{4}}{16\rho_{s}^{2}})}}{\Delta_{i}}\bigg)\bigg\rbrace 
 \end{align*}

 
\subsubsection*{Case b2: \textit{Optimal arm $a^{*}$ is eliminated by a sub-optimal arm}}
  
	Here, we take into consideration the error bound, that the optimal arm $a^{*}$ or the optimal cluster $s^{*}$ gets eliminated by any sub-optimal arm or sub-optimal cluster. This, can happen in $3$ ways,
\begin{enumerate}
\item In $s^{*}$, $a^{*}$ got eliminated by other arms surviving till $m_{*}$. Let, the arms surviving till $m_{*}$ round be denoted by $A^{'}_{s^{*}}$ such that $A^{'}_{s^{*}}=\lbrace i \in A_{s^{*}},\Delta_{i}> b\rbrace$. This leaves any arm $a_{b}$ such that $\sqrt{\rho_{a}\epsilon_{m}}\geq\dfrac{\Delta_{b}}{2}$ to still survive and eliminate arm $a^{*}$ in round $m_{*}$. Let, such arms that survive $a^{*}$ belong to $A^{''}_{s^{*}}$ such that $A^{''}_{s^{*}}=\lbrace i \in A_{s^{*}},0 < \Delta_{i} \leq b\rbrace$. As proved in Proposition \ref{proofSketch:Prop:1}, case $b2$ this regret can be no more than,
 \begin{align*}
 &\sum_{i\in A^{'}_{s^{*}}}\bigg(\dfrac{C_{2}(\rho_{a})T^{1-\rho_{a}}}{\Delta_{i}^{4\rho_{a} -1}} \bigg)+\sum_{i\in A^{''}_{s^{*}}\setminus A^{'}_{s^{*}}}\bigg(\dfrac{C_{2}(\rho_{a})T^{1-\rho_{a}}}{b^{4\rho_{a} -1}} \bigg)
 \end{align*}
We also see that here, we are concerned only within $s^{*}$ because of our assumption that there is only one $a^{*}\in A$ and clusters are fixed.
\item $a^{*}\in C_{g}$ and $s^{*}$ gets eliminated by some other cluster. This is equivalent to Proposition \ref{proofSketch:Prop:2}, case $b2$
\item $a^{*}\notin C_{g}$ and $s^{*}$ gets eliminated by some other cluster. This is equivalent to Proposition \ref{proofSketch:Prop:2}, case $b3$
\end{enumerate} 


Combining cases $b21$, $b22$ and $b23$ as mentioned above we can show,
 \begin{align*}
 &\underbrace{\sum_{i\in A^{'}_{s^{*}}}\bigg(\dfrac{C_{2}(\rho_{a})T^{1-\rho_{a}}}{\Delta_{i}^{4\rho_{a} -1}} \bigg)+\sum_{i\in A^{''}_{s^{*}}\setminus A^{'}_{s^{*}}}\bigg(\dfrac{C_{2}(\rho_{a})T^{1-\rho_{a}}}{b^{4\rho_{a} -1}} \bigg)}_{\text{case b21}} \\
 & + \underbrace{\sum_{i\in A^{'}}\bigg(\dfrac{2C_{2}(\rho_{s})T^{1-\rho_{s}}}{\Delta_{i}^{4\rho_{s}-1}} \bigg)}_{\text{case b22}}+\underbrace{\sum_{i\in A^{''}\setminus A^{'}}\bigg(\dfrac{2C_{2}(\rho_{s})T^{1-\rho_{s}}}{b^{4\rho_{s} -1}} \bigg)}_{\text{case b23}}
 \end{align*}
 

Hence, the total regret by combining \textbf{case a} and \textbf{case b} is given by,
 \begin{align*}
 & R_{T}\leq \sum_{i\in A^{'}} \bigg\lbrace  \underbrace{\dfrac{C_{1}(\rho_{a})T^{1-\rho_{a}}}{\Delta_{i}^{4\rho_{a}-1}}}_{\text{case a1+a2}} + \underbrace{  \dfrac{2C_{1}(\rho_{s})T^{1-\rho_{s}}}{\Delta_{i}^{4\rho_{s}-1}}}_{\text{case a3}} \\
 & + \underbrace{2\Delta_{i}+\dfrac{32\rho_{a}\log{(\psi T\dfrac{\Delta_{i}^{4}}{16\rho_{a}^{2}})}}{\Delta_{i}} +\dfrac{32\rho_{s}\log{(\psi T\dfrac{\Delta_{i}^{4}}{16\rho_{s}^{2}})}}{\Delta_{i}}}_{\text{case b1}}\bigg\rbrace \\
 & + \underbrace{\sum_{i\in A_{s^{*}}^{'}} \dfrac{C_{2}(\rho_{a})T^{1-\rho_{a}}}{\Delta_{i}^{4\rho_{a} -1}}  + \sum_{i\in A_{s^{*}}^{''} \setminus A_{s^{*}}^{'} }\dfrac{C_{2}(\rho_{a})T^{1-\rho_{a}}}{\Delta_{i}^{4\rho_{a} -1}} }_{\text{case b2}}\\ 
 & + \underbrace{\sum_{i\in  A^{'} } \dfrac{2C_{2}(\rho_{s})T^{1-\rho_{s}}}{\Delta_{i}^{4\rho_{s}-1}}  + \sum_{i\in A^{''}\setminus A^{'} }2C_{2}(\rho_{s})\dfrac{T^{1-\rho_{s}}}{\psi^{\rho_{s}}b^{4\rho_{s} -1}} }_{\text{case b2}} \\ 
 & + max_{i:\Delta_{i}\leq b}\Delta_{i}T \\
 = & \sum_{i\in A:\Delta_{i}\geq b} \bigg\lbrace 2\Delta_{i}+\dfrac{C_{1}(\rho_{a})T^{1-\rho_{a}}}{\Delta_{i}^{4\rho_{a}-1}} + \dfrac{2C_{1}(\rho_{s})T^{1-\rho_{s}}}{\Delta_{i}^{4\rho_{s}-1}} \\
 & + \dfrac{32\rho_{a}\log{(\psi T\dfrac{\Delta_{i}^{4}}{16\rho_{a}^{2}})}}{\Delta_{i}} + \dfrac{32\rho_{s}\log{(\psi T\dfrac{\Delta_{i}^{4}}{16\rho_{s}^{2}})}}{\Delta_{i}} \bigg\rbrace \\
 & + \sum_{\substack{i\in A_{s^{*}}: \\ \Delta_{i}\geq b}} \bigg(\dfrac{C_{2}(\rho_{a})T^{1-\rho_{a}}}{\Delta_{i}^{4\rho_{a}-1}} \bigg)+ \sum_{\substack{i\in A_{s^{*}}: \\ 0\leq\Delta_{i}\leq b}}\bigg(\dfrac{C_{2}(\rho_{a})T^{1-\rho_{a}}}{b^{4\rho_{a} -1}} \bigg) \\
 & + \sum_{\substack{i\in A:\\ \Delta_{i}\geq b}} \bigg(\dfrac{2C_{2}(\rho_{s})T^{1-\rho_{s}}}{\Delta_{i}^{4\rho_{s}-1}} \bigg) + \sum_{\substack{i\in A: \\ 0\leq\Delta_{i}\leq b}}\bigg(\dfrac{2C_{2}(\rho_{s})T^{1-\rho_{s}}}{b^{4\rho_{s} -1}} \bigg) \\
 & + max_{i:\Delta_{i}\leq b}\Delta_{i}T
 \end{align*}

\end{proof}


\begin{proposition}
\label{proofSketch:Prop:1}
The regret $R_T$ for ClusUCB-AE satisfies
\begin{align*}
&\E [R_{T}]\leq \sum\limits_{i\in A:\Delta_{i}\geq b}\bigg\lbrace\frac{C_{1}(\rho_{a})T^{1-\rho_{a}}}{\Delta_{i}^{4\rho_{a}-1}} + \Delta_{i}\\
&+\frac{32\rho_{a}\log{(\dfrac{\psi  T\Delta_{i}^{4}}{16\rho_{a}^{2}})}}{\Delta_{i}}
 +  \frac{C_{2}(\rho_{a})T^{1-\rho_{a}}}{\Delta_{i}^{4\rho_{a} -1}}  \bigg \rbrace\\
&+\sum\limits_{i\in A:0\leq\Delta_{i}\leq b}\frac{C_{2}(\rho_{a})T^{1-\rho_{a}}}{b^{4\rho_{a} -1}}  + \max_{i:\Delta_{i}\leq b}\Delta_{i}T
\end{align*}
, for all $b\geq\sqrt{\dfrac{e}{T}}$, where  $C_1(x) = \frac{2^{1+4x}x^{2x}}{\psi^{x}}$,  $C_2(x) = \frac{2^{2x+\frac{3}{2}}x^{2x}}{\psi^{x}}$, $\rho_{a}=\dfrac{1}{2}$ is the arm elimination parameter, $\psi=K^{2}T$ is the exploration regulatory factor, $p$ is the number of clusters and $T$ is the horizon.
\end{proposition}
\begin{proof}
Follows in a similar fashion as the proof of Theorem $1$ in \cite{auer2010ucb}. For the sake of completeness, the proof is given in Appendix \ref{App:A}.
\end{proof}




\begin{corollary}
\label{proofSketch:Corollary:1}
For $\rho_{a}=1$ in the result of proposition $1$ for ClusUCB-AE, 
%\begin{align*}
%& \sum\limits_{i\in A:\Delta_{i}\geq b}\bigg\lbrace\frac{C_{1}(\rho_{a})T^{1-\rho_{a}}}{\Delta_{i}^{4\rho_{a}-1}} + \Delta_{i}\\
%&+\frac{32\rho_{a}\log{(\dfrac{\psi  T\Delta_{i}^{4}}{16\rho_{a}^{2}})}}{\Delta_{i}}
% +  \frac{C_{2}(\rho_{a})T^{1-\rho_{a}}}{\Delta_{i}^{4\rho_{a} -1}}  \bigg \rbrace\\
%&+\sum\limits_{i\in A:0\leq\Delta_{i}\leq b}\frac{C_{2}(\rho_{a})T^{1-\rho_{a}}}{b^{4\rho_{a} -1}}  + \max_{i:\Delta_{i}\leq b}\Delta_{i}T
%\end{align*}
 
 we get a regret bound of 
 \begin{align*}
 &\sum\limits_{i\in A:\Delta_{i}\geq b}\bigg(\Delta_{i} + \dfrac{44}{\psi(\Delta_{i})^{3}} + \dfrac{32\log{(\psi T\Delta_{i}^{4})}}{\Delta_{i}}\bigg)\\ 
 & + \sum\limits_{i\in A:0\leq\Delta_{i}\leq b}\dfrac{12}{\psi b^{3}}
 \end{align*}.
\end{corollary}

\begin{proof}
The proof of this corollary is given in Appendix \ref{App:Proof:Corollary:3}.
\end{proof}


\begin{proposition}
\label{proofSketch:Prop:2}
For $p>1$, the regret $R_T$ for ClusUCB-CE satisfies,
\begin{align*}
&\E [R_{T}]\leq \sum\limits_{i\in A:\Delta_{i}\geq b}\bigg\lbrace\bigg(\dfrac{2C_{1}(\rho_{s})T^{1-\rho_{s}}}{\Delta_{i}^{4\rho_{s}-1}}\bigg)\\
& + \bigg(\Delta_{i}+\dfrac{32\rho_{s}\log{(\psi T\dfrac{\Delta_{i}^{4}}{16\rho_{s}^{2}})}}{\Delta_{i}}\bigg) + \bigg(\dfrac{2C_{2}(\rho_{s})T^{1-\rho_{s}}}{\Delta_{i}^{4\rho_{s} -1}} \bigg)\bigg\rbrace \\
& + \sum\limits_{i\in A:0\leq\Delta_{i}\leq b}\bigg(\dfrac{2C_{2}(\rho_{s})T^{1-\rho_{s}}}{b^{4\rho_{s} -1}} \bigg)
\end{align*}
, for all $b\geq \sqrt{\dfrac{e}{T}}$, where $C_1(x) = \frac{2^{1+4x}x^{2x}}{\psi^{x}}$,  $C_2(x) = \frac{2^{2x+\frac{3}{2}}x^{2x}}{\psi^{x}}$, $\rho_{s}=\dfrac{1}{2} $ is the cluster elimination parameter, $\psi=K^{2}T$ is the exploration regulatory factor, $p$ is the number of clusters and $T$ is the horizon.
\end{proposition}
\begin{proof}
See Appendix \ref{App:B}.
\end{proof}

\section{Simulation experiments}
\label{sec:expts}
\begin{figure}
\centering
  \begin{tabular}{c}
  \begin{subfigure}{0.45\textwidth}
 \tabl{c}{\scalebox{0.8}{\begin{tikzpicture}
      \begin{axis}[
	xlabel={timestep},
	ylabel={Cumulative regret},
       clip mode=individual,grid,grid style={gray!30},
  legend style={at={(0.5,-0.2)},anchor=north,legend columns=3} ]
      % UCB
\addplot table[x index=0,y index=1,col sep=tab,each nth point={10}] {results/Expt1/clUCBcomp_subsampled.txt};
\addplot table[x index=0,y index=1,col sep=tab,each nth point={10}] {results/Expt1/DMEDcomp_subsampled.txt};
\addplot table[x index=0,y index=1,col sep=tab,each nth point={10}] {results/Expt1/KLUCBcomp_subsampled.txt};
\addplot table[x index=0,y index=1,col sep=tab,each nth point={10}] {results/Expt1/MOSScomp_subsampled.txt};
\addplot table[x index=0,y index=1,col sep=tab,each nth point={10}] {results/Expt1/UCB1comp_subsampled.txt};
\addplot table[x index=0,y index=1,col sep=tab,each nth point={10}] {results/Expt1/UCB_Vcomp_subsampled.txt};
      \legend{ClusUCB,DMED,KL-UCB,MOSS,UCB1,UCB-V}
      \end{axis}
      \end{tikzpicture}}\\}
			\caption{Experiment $1$: $20$ Bernoulli-distributed arms with $r_{i_{a_{i}\neq a^{*}}}=0.07$ and $r^{*}=0.1$.}
  \label{fig:1}
  \end{subfigure}
	\\
	%%%%%%% Expt 2
	  \begin{subfigure}{0.45\textwidth}
 \tabl{c}{\scalebox{0.8}{\begin{tikzpicture}
      \begin{axis}[
	xlabel={timestep},
	ylabel={Cumulative regret},
       clip mode=individual,grid,grid style={gray!30},
  legend style={at={(0.5,-0.2)},anchor=north,legend columns=3} ]
      % UCB
\addplot table[x index=0,y index=1,col sep=tab,each nth point={10}] {results/Expt2/clUCBCcomp_subsampled.txt};
\addplot table[x index=0,y index=1,col sep=tab,each nth point={10}] {results/Expt2/clUCBNCcomp_subsampled.txt};
\addplot table[x index=0,y index=1,col sep=tab,each nth point={10}] {results/Expt2/Med_Elimcomp_subsampled.txt};
\addplot table[x index=0,y index=1,col sep=tab,each nth point={10}] {results/Expt2/UCB_Improvedcomp_subsampled.txt};
\addplot table[x index=0,y index=1,col sep=tab,each nth point={10}] {results/Expt2/MOSScomp_subsampled.txt};
\addplot table[x index=0,y index=1,col sep=tab,each nth point={10}] {results/Expt2/UCB1comp_subsampled.txt};
      \legend{ClusUCB (p=20), ClusUCB (p=1), Med-Elim,UCB-Improved,MOSS,UCB1}
      \end{axis}
      \end{tikzpicture}}\\}
			\caption{Experiment $2$: $100$ Gaussian-distributed arms with $r_{i_{a_{i}\neq a^{*}:1-33}}=0.01$, $r_{i_{a_{i}\neq a^{*}:34-99}}=0.06$ and $r^{*}_{i=100}=0.1$.}
  \label{fig:2}
  \end{subfigure}
	\\
	%%%%%%% Expt 3
	  \begin{subfigure}{0.45\textwidth}
 \tabl{c}{\scalebox{0.8}{\begin{tikzpicture}
      \begin{axis}[
	xlabel={Arms},
	ylabel={Cumulative regret},
       clip mode=individual,grid,grid style={gray!30},
  legend style={at={(0.5,-0.2)},anchor=north,legend columns=-1} ]
      % UCB
\addplot table[x index=0,y index=1,col sep=tab] {results/Expt3/clUCB20_400.txt};
\addplot table[x index=0,y index=1,col sep=tab] {results/Expt3/MOSS20_400.txt};
      \legend{ClusUCB,MOSS}
      \end{axis}
      \end{tikzpicture}}\\}
			\caption{Experiment $3$: $20$ to $200$ Bernoulli-distributed arms with $r_{i_{a_{i}\neq a^{*}}}=0.05$ and $r^{*}=0.1$.}
  \label{fig:3}
  \end{subfigure}
  \end{tabular}
\caption{Cumulative regret for various bandit algorithms on three stochastic K-armed bandit environments. 
}
\label{fig:karmed}
\end{figure}

\begin{figure}
\centering
  \begin{tabular}{c}
  \begin{subfigure}{0.45\textwidth}
 \tabl{c}{\scalebox{0.8}{\begin{tikzpicture}
      \begin{axis}[
	xlabel={timestep},
	ylabel={Cumulative regret},
       clip mode=individual,grid,grid style={gray!30},
  legend style={at={(0.5,-0.2)},anchor=north, legend columns=2} ]
      % UCB
\addplot table[x index=0,y index=1,col sep=tab,each nth point={10}] {results/Expt4/clUCB1comp_subsampled.txt};
\addplot table[x index=0,y index=1,col sep=tab,each nth point={10}] {results/Expt4/clUCB2comp_subsampled.txt};
\addplot table[x index=0,y index=1,col sep=tab,each nth point={10}] {results/Expt4/clUCB3comp_subsampled.txt};
\addplot table[x index=0,y index=1,col sep=tab,each nth point={10}] {results/Expt4/MOSScomp_subsampled.txt};
\addplot table[x index=0,y index=1,col sep=tab,each nth point={10}] {results/Expt4/clUCB4comp_subsampled.txt};
\addplot table[x index=0,y index=1,col sep=tab,each nth point={10}] {results/Expt4/clUCB5comp_subsampled.txt};
      \legend{ClusUCB(1A),ClusUCB(4B),ClusUCB(10B),MOSS,ClusUCB(5S),ClusUCB(10S)}
      %\legend{ClusUCB (NC, p=1),ClusUCB (C, p=4),ClusUCB(C, p=10) ,MOSS, ClusUCB(C, p=5, NAE), ClusUCB(C, p=10, NAE)}
      %\legend{ClusUCB(1,A),ClusUCB(4,B),ClusUCB(10,B), MOSS,ClusUCB(5,S), ClusUCB(10,A)}
      \end{axis}
      \end{tikzpicture}}\\}
			\caption{Experiment $4$: ClusUCB for various $p$. ClusUCB(1A)= $\lbrace$ p=$1$,Only Arm Elimination $\rbrace$, ClusUCB(4B)=$\lbrace$ p=$4$, Both Arm and Cluster Elimination$\rbrace$, ClusUCB(5S)=$\lbrace$ p=$5$, Only Cluster Elimination$\rbrace$. }
  \label{Fig:variousClus}
  \end{subfigure}
  \end{tabular}

\end{figure}

%\begin{figure}[!tbp]
%\label{fig:1}
%\begin{minipage}[b]{0.5\textwidth}
%\includegraphics[width=\textwidth]{img/ClusUCB_variousAlgo.png}
%
%\caption{Experiment 1: Regret for various Algorithms. $T=60000$}
%\end{minipage}
%\end{figure}
%
%\hspace{0.1em}
%
%\begin{figure}[!tbp]
%\label{fig:2}
%\begin{minipage}[b]{0.5\textwidth}
%
%\includegraphics[width=\textwidth]{img/clusUCB_variousAlgo(expt2)_Final.png}
%\caption{Experiment 2: Regret for various Algorithms. $T=2\times 10^{6}$}
%\end{minipage}
%\end{figure}
%
%\hspace{0.1em}
%
%\begin{figure}[!tbp]
%\label{fig:3}
%\begin{minipage}[b]{0.5\textwidth}
%\includegraphics[width=\textwidth]{img/clUCB_MOSS_expt3.png}
%\caption{Experiment 3: Regret Growth for ClusUCB and MOSS . $T=10^{5} + K^{2}\times 10^{4}$ for $K=20$ to $200$}
%\end{minipage}
%\end{figure}
%
%\hspace{0.1em}
%


%In the stochastic bandit literature there are several powerful algorithms with and without proven regret bounds. Algorithms like $\epsilon$-greedy(\cite{sutton1998reinforcement}) or softmax(\cite{sutton1998reinforcement}) or UCB-Tuned(\cite{auer2002finite}) has no proven regret bounds. Again algorithms like UCB-$\delta$(\cite{abbasi2011improved}) with proven regret bound better than UCB1  falls within the realm of fixed confidence setting whereas one has to provide the probability of error $\delta$. We also make a distinction between frequentist based approach like the UCB algorithms and the Bayesian approach like the Thompson Sampling(\cite{agrawal2011analysis}). 
For the sake of performance comparison using cumulative regret as the metric, we implement the following algorithms:  KL-UCB\cite{garivier2011kl}, DMED\cite{honda2010asymptotically}, MOSS\cite{audibert2009minimax}, UCB1\cite{auer2002finite}, UCB-Improved\cite{auer2010ucb}, Median Elimination\cite{even2006action} and UCB-V\cite{audibert2009exploration}\footnote{The implementation for KL-UCB and DMED are taken from \cite{CapGarKau12}.}.

The parameters of ClusUCB algorithm are set as follows: $\psi=\log T$, $\rho_{s}=\dfrac{1}{2^{2m+1}}$ and $\rho_{a}=\dfrac{1}{2^{4m+1}}$. When $K$ is large and $p$ is small it is advantageous to run $\rho_{a} < \rho_{s}$(see Corollary \ref{Result:Corollary:2}) because this will aggressively eliminate arms within cluster while cluster elimination will be more conservative since each cluster will contain a large number of arms it should be eliminated less aggressively. 
The first experiment is conducted over a testbed of $20$ arms for the test-cases involving Bernoulli reward distribution with expected rewards of the arms $r_{i_{a_{i}\neq a^{*}}}=0.07$ and $r^{*}=0.1$. These type of cases are frequently encountered in web-advertising domain. The horizon $T$ is set to $60000$ and the number of clusters $p$ for ClusUCB is set to $4$. The regret is averaged over $100$ independent runs and is shown in Figure \ref{fig:1}. 
ClusUCB, MOSS, UCB1, UCB-V, KL-UCB and DMED are run in this experimental setup and we observe that ClusUCB performs better than all the aforementioned algorithms. Because of the short horizon $T$, we do not implement UCB-Improved and Median Elimination on this test-case.

The second experiment is conducted over a testbed of $100$ arms involving Gaussian reward distribution with expected rewards of the arms $r_{i_{a_{i}\neq a^{*}:1-33}}=0.01$, $r_{i_{a_{i}\neq a^{*}:34-99}}=0.06$ and $r^{*}_{i=100}=0.1$. The horizon $T$ is set for a large duration of $2\times 10^{6}$ and the number of clusters $p=20$. The regret is averaged over $100$ independent runs and is shown in Figure \ref{fig:2}. In this case, in addition to ClusUCB, we also show the performance of no-clustering version of ClusUCB algorithm (i.e., $p=1$).   From the results sin Figure \ref{fig:2}, we observe that ClusUCB with $p=20$ outperforms ClusUCB with $p=1$ as well as MOSS, UCB1, UCB-Improved and Median-Elimination($\epsilon=0.03,\delta=0.1$). We also observed that the ClusUCB variant that uses only  the arm elimination condition in Algorithm \ref{alg:clusucb} performs worse than the variant that employs cluster and arm elimination conditions. We also see that in this testbed UCB-Improved performs the worst and it confirms our assumption 
that it spends too much pulls in the initial exploration.

The third experiment is conducted over a testbed of $20-400$(interval of $10$) arms with Bernoulli reward distribution, where the expected rewards of the arms are $r_{i_{a_{i}\neq a^{*}}}=0.05$ and $r^{*}=0.1$. The horizon $T$ is set to $10^{5} + K^{2}\times 10^{4}$ and the number of arms are increased from $K=20$ to $200$. ClusUCB is run with $p=K/5$. The regret is averaged over $500$ independent runs and is shown in Figure \ref{fig:3}. We report the performance of MOSS and ClusUCB only over this setup. From the results in Figure \ref{fig:3}, it is evident that the growth of regret for ClusUCB is lower than MOSS. 

The fourth experiment is performed over a testbed having $20$ Bernoulli-distributed arms with $r_{i_{:{a_{i}\neq a^{*}}}}=0.06,\forall i\in A$ and $r^{*}=0.1$. In Figure \ref{Fig:variousClus}, we report the results with $T=60000$ averaged over $100$ independent runs for ClusUCB with  $p=\lbrace 1,4,10\rbrace$. In this case we see that, since $\rho_{a}$ is decreased very fast, the optimal arm $a^{*}$ gets eliminated most of the time for no clustering $p=1$. While a balance of $p,\rho_{a}$ and $\rho_{s}$ gives a much better result. ClusUCB with $p=4$ and $10$ perform better than MOSS, while $p=1$ with just arm elimination does not converge and $p=5,10$ with just Cluster elimination and no arm elimination also does not converge. As proved in Proposition \ref{proofSketch:Prop:2}, regret for using just cluster elimination is higher than using just arm elimination. A balance of cluster and arm elimination works best. For using just 
cluster elimination in ClusUCB($\sqrt{\log K}<p\leq \dfrac{K}{2}$) we stop when we are left with one cluster and output the max payoff arm of that cluster.

%We set $\psi=1$, $\rho_{s}=\frac{1}{2^{m+1}}$ and $\rho_{a}=\frac{1}{2^{2m+1}}$.

%The jumps in the graph for ClusUCB happens because of the error(eliminating optimal arm) and the margin of error(in red) is also shown in the graph. 
%\todos{I did not see this jump in the txt files shared for Expt 3}



\section{Conclusions and future work}
\label{sec:conclusions}
From a theoretical viewpoint, we conclude that the gap-dependent regret bound of ClusUCB is lower than MOSS and UCB-Improved. From the numerical experiments on settings with small gaps between optimal and sub-optimal mean rewards, we observed that ClusUCB outperforms several popular bandit algorithms. 
While we exhibited better regret bounds for ClusUCB, it would be interesting future research to improve the theoretical analysis of ClusUCB to achieve the gap-independent regret bound of MOSS and possibly also the gap-dependent bound conjectured in Section 2.4.3 of \cite{bubeck2012regret}.


\clearpage
\newpage
\bibliographystyle{plainnat}
%\vspace*{-1cm}
\bibliography{biblio}


%%%%%%%%%%%%%%%%%%%%%%%%%%%%%%%%%%%%%%%%%%%%%%%%%%%%%%%%%%%%
%%%%%%%%%%%%%%%%%%%%%%%%%%%%%%%%%%%%%%%%%%%%%%%%%%%%%%%%%%%%

\clearpage
\newpage
\onecolumn
\section*{Appendix}


%\newpage
\appendix

%	In the following appendices we will prove the bounds based on the events $\xi_{1}$,$\xi_{2}$ and $\xi_{3}$. In $\xi_{1}$, we will assume two important assumptions $i)\hat{r}^{*}<\hat{r}_{i},\forall i\in s_{i}$ and $ii)\exists a_{i}\in s_{i}$ such that $\sqrt{\dfrac{\epsilon_{m}}{w_{s_{i}}}}<\dfrac{\Delta_{i}}{5}$. For $\xi_{2}$, we will assume that $a^{*}\in s^{*}$ and $|s^{*}|=1$, $a_{i}\in s_{i} \forall a_{i}\setminus a^{*}\in B_{m}$ and $\exists a_{max_{s_{i}}}$ such that $\sqrt{\epsilon_{m}}<\dfrac{2\Delta_{s}}{5}$, where $\Delta_{s}=r^{*}-r_{max_{s_{i}}}$ and $\hat{r}_{max_{s_{i}}}>\hat{r}_{i}, \forall i\in s_{i}$. $\xi_{3}$ be the event when the optimal arm $a^{*}$ gets eliminated by a sub-optimal arm. At the start of any round $m$, we fix $\epsilon_{m}$.

%\todos{(Subho) Took $\psi(m)=1$, removed $w_{s_{i}}$ from proofs. Here, $\epsilon_{m}$ is now the $\tilde{\Delta}_{m}$ of Ucb-Revisited}


%\section{Appendix A}
%\begin{proposition}
%The probability that the optimal arm $a^{*}\in s_{i}$ will lie above $\hat{r}_{min_{s_{i}}}+ \dfrac{\hat{\Delta}_{s_{i}}}{2}$ after $\bigg\lceil\dfrac{2\log (T\epsilon_{m}^{2})}{\epsilon_{m}}\bigg\rceil$ pulls in the $m$-th round is given by $\bigg\lbrace 1- \dfrac{1}{(T\epsilon_{m}^{2})^{\ell_{m}^{2}\epsilon_{m}}} \bigg\rbrace$ where $\hat{r}_{min_{s_{i}}}$ is the minimum payoff in $s_{i}$, $\hat{\Delta}_{s_{i}}=max_{i\in s_{i}}\hat{r}_{i}-min_{j\in s_{i}}\hat{r}_{j}, i\neq j$, $\epsilon_{m}$ is halved after every round and $T$ is the horizon. 
%\end{proposition}
%
%%$\epsilon_{m}=\max{\bigg\lbrace\dfrac{\hat{\Delta}_{m}}{\ell_{m}} \dfrac{2}{\sqrt{\psi{(m)T}}}\bigg\rbrace}$
%
%\begin{proof} of Proposition 1:
%\newline
%We start by considering the worst case scenario that in the $m$-th round, in a cluster $s_{i}$, the optimal arm $a^{*}$ has performed worst, such that $\hat{r}^{*}<\hat{r}_{i},\forall a_{i} \in s_{i}$. Let, $\hat{\Delta}_{s_{i}}=max_{i\in s_{i}}\hat{r}_{i}-min_{j\in s_{i}}\hat{r}_{j}$ where $i\neq j$. Also, let $|s_{i}|=k_{s_{i}}$ and $\hat{r}^{*}=\hat{r}_{min_{s_{i}}}\leq\hat{r}_{i},\forall i\in s_{i}$ also $\hat{r}_{max_{s_{i}}}\geq\hat{r}_{i},\forall i\in s_{i}$.
%\newline
%%Now, we have to bound the $\mathbb{P}\lbrace\hat{r}^{*}\geq\hat{r}_{max_{s_{i}}} - \hat{\Delta}_{s_{i}}\rbrace$
%%%\leq U_{m}$, where $U_{m}$ is an upper bound.
%%\newline
%Again, given that there are $k_{s_{i}}$ number of arms in $s_{i}$, and for each $a_{i},a_{j}\in s_{i}$ since $|\hat{r}_{i}-\hat{r}_{j}|\leq\epsilon_{m}$, the longest possible gap is $(k_{s_{i}}-1)\epsilon_{m}$ which is greater than the actual estimated gap $\hat{\Delta}_{s_{i}}$.
%\newline
%So, $\hat{\Delta}_{s_{i}}\leq (k_{s_{i}}-1)\epsilon_{m}$, as $|\hat{r}_{i}-\hat{r}_{j}|\leq\epsilon_{m}, \forall i,j \in s_{i}$
%\newline\hspace*{3.5em}$\leq \ell_{m}\epsilon_{m}$, as $k_{s_{i}}\leq \ell_{m}$
%%\newline Again in $\xi_{1}$, $\hat{\Delta}_{s_{i}}\geq \epsilon_{m}$, for sufficiently large $\ell_{m}$, as $\epsilon_{m}=\dfrac{\hat{\Delta}_{s,m}}{\ell_{m}}$, where $\hat{\Delta}_{s,m}=\max_{i\in B_{m}}{\hat{r}_{i}}-\min_{j\in B_{m}}{\hat{r}_{j}},i\neq j$ and $\ell_{m}$ is doubled after every round.
%%$\mathbb{P}\lbrace\hat{r}^{*}\geq\hat{r}_{m}+ \dfrac{\hat{\Delta}_{s_{i}}}{2}\rbrace=
%%\newline\hspace*{0em} $\mathbb{P}\lbrace\hat{r}^{*}\geq\hat{r}_{max_{s_{i}}} - \hat{\Delta}_{s_{i}}\rbrace\Rightarrow\mathbb{P}\lbrace\hat{r}^{*}+\dfrac{\hat{\Delta}_{s_{i}}}{2}\geq\hat{r}_{max_{s_{i}}} - \dfrac{\hat{\Delta}_{s_{i}}}{2}\rbrace$
%%\leq U_{m}$ .
%%\newline But we know that $r^{*}>r_{s}$ and we know that in $\xi_{1}$, $a_{s}$ has performed such that $\hat{r}_{s} \leq r^{*}$.
%\newline Now, applying Chernoff-Hoeffding bound and considering independence of events,
%\newline $\mathbb{P}\lbrace\hat{r}^{*}\leq{r}^{*} + \dfrac{\hat{\Delta}_{s_{i}}}{2}\rbrace\Rightarrow \mathbb{P}\lbrace\hat{r}^{*}\leq{r}^{*} + \dfrac{\ell_{m}\epsilon_{m}}{2}\rbrace $
%%exp(-2 \dfrac{\hat{\Delta}_{s_{i}}^{2}}{4}n^{*})$
%%\newline\hspace*{8em}
%$\leq exp(-2\dfrac{(\ell_{m}\epsilon_{m})^{2}}{4} n^{*})$
%%as $\hat{\Delta}_{s_{i}}\geq \epsilon$
%%\newline For simplicity we will take $\psi(m)=1$
%\newline Now, putting $n_{m}=n^{*}=\dfrac{2\log (T\epsilon_{m}^{2})}{\epsilon_{m}}$
%\newline$\mathbb{P}\lbrace\hat{r}^{*}\leq{r}^{*} +  \dfrac{\ell_{m}\epsilon_{m}}{2}\rbrace\leq exp(-\ell_{m}^{2}\epsilon_{m} \log(T\epsilon_{m}^{2}))$
%%\leq exp(-\hat{\Delta}_{s_{i}} \log(\psi(m)T\epsilon_{m}^{2}))
%%\newline Now, $w_{s_{i}}=k_{s_{i}}D\leq$
%%\newline\hspace*{8em}$\leq exp(-\ell_{m}^{2}\epsilon \log(4\psi(m)T\epsilon_{m}^{2}))$
%%as $\dfrac{1}{w\ell_{m}}< \hat{\Delta}_{s_{i}}, \forall m\in {1,2,..,\lceil\log T\rceil}$
%\newline $\mathbb{P}\lbrace\hat{r}^{*}\leq{r}^{*} +  \dfrac{\ell_{m}\epsilon_{m}}{2}\rbrace\leq \dfrac{1}{(T\epsilon_{m}^{2})^{\ell_{m}^{2}\epsilon_{m}}}$
%%\leq \dfrac{1}{(4\psi(m)T\epsilon_{m}^{2})^{\ell_{m}^{2}\Delta}}$, as $\forall m, \epsilon_{m}\geq \Delta$
%\newline
%%Similarly, $\mathbb{P}\lbrace\hat{r}_{max_{s_{i}}}\geq{r}_{max_{s_{i}}} -  \dfrac{\ell_{m}\epsilon_{m}}{2}\rbrace\leq \dfrac{1}{(\psi(m)T\epsilon_{m}^{2})^{\ell_{m}^{2}\epsilon_{m}}}$
%%as $\forall m, \epsilon_{m}\geq \Delta$
%%\newline
%Hence, the probability that the optimal arm $a^{*}$ after $n_{m}$ pulls going above $\hat{r}_{min_{s_{i}}}+\dfrac{\hat{\Delta}_{s_{i}}}{2}$ is $\bigg\lbrace 1- \dfrac{1}{(T\epsilon_{m}^{2})^{\ell_{m}^{2}\epsilon_{m}}} \bigg\rbrace$
%
%\end{proof}

%\begin{figure}[!tbp]
%\centering
%\includegraphics[scale=0.4]{img/diag1.jpg}
%\caption{Steps 9-11}
%\end{figure}
%\begin{figure}[!tbp]
%\includegraphics[scale=0.4]{img/diag2.jpg}
%\caption{Steps 12-14}
%\end{figure}
%\begin{figure}[!tbp]
%\includegraphics[scale=0.25]{img/diag3.jpg}
%\caption{Different scenarios of Cluster Elimination}
%\end{figure}




\section{Appendix A}
\begin{proposition}
Considering only the arm elimination condition, the total regret till $T$ is upper bounded by $R_{T}\leq \sum_{i\in A:\Delta_{i}\geq b}\bigg \lbrace \bigg(\dfrac{44}{(\Delta_{i})^{3}}\bigg) + \bigg(\Delta_{i}+\dfrac{32\log{(T\dfrac{\Delta_{i}^{4}}{16})}}{\Delta_{i}}\bigg)\bigg\rbrace + \sum_{i\in A:0\leq\Delta_{i}\leq b}\dfrac{12}{b^{3}} + max_{i:\Delta\leq b}\Delta_{i}T$, where $T$ is the horizon.
\end{proposition}

\begin{proof} of Proposition 5:

Let, $\rho_{a}=1$ and for each sub-optimal arm $a_{i}$, $m_{i}=\min{\lbrace m|\sqrt{\epsilon_{m}}\leq \dfrac{\Delta_{i}}{2} \rbrace}$ be the first round when $\sqrt{\epsilon_{m}}\leq \dfrac{\Delta_{i}}{2}$.

\subsection{Case a:} 
Some sub-optimal arm $a_{i}$ is not eliminated in round $m_{i}$ or before and the optimal arm $a^{*}\in B_{m_{i}}$
\newline In arm elimination condition, given the choice of confidence interval $c_{m}$, we want to bound the event $\hat{r}_{i}+c_{m_{i}}\leq \hat{r}^{*}-c_{m_{i}}$. 
%In this proof we will consider $w_{s_{i}}=1$ and $\psi(m)=1$. 
%Later, we will discuss how different values of $w_{s_{i}}$ actually effects the regret bound.
%with a more tighter event of $ \hat{r}_{i} + \sqrt{w_{s_{i}}}c_{m} \leq \hat{r}^{*} - \sqrt{w_{s_{i}}}c_{m}$ which will result in faster elimination of arms within a cluster, given the choice of $c_{m}$ and $w_{s_{i}}$.
\newline Now, $c_{m_{i}}=\sqrt{\dfrac{\log (T\epsilon_{m_{i}}^{2})}{2 n_{m_{i}}}}$.
\newline Putting the value of $n_{m_{i}}=\dfrac{2\log{(T\epsilon_{m_{i}}^{2})}}{\epsilon_{m_{i}}}$ in $c_{m_{i}}$,
\newline $c_{m_{i}}=\sqrt{\dfrac{\epsilon_{m_{i}}\log (T\epsilon_{m_{i}}^{2})}{2*2 \log(T\epsilon_{m_{i}}^{2})}}=\dfrac{\sqrt{\epsilon_{m_{i}}}}{2} = \sqrt{\epsilon_{m_{i}+1}} < \dfrac{\Delta_{i}}{4} $
%\leq\dfrac{\epsilon_{m}\sqrt{\ell_{m}}}{2\sqrt{w\ell_{m}}}\leq \dfrac{\epsilon_{m}}{2\sqrt{w}}$.
%\newline But in $\xi_{1}$, $\ell_{m}=2^{m}$.
%\newline Hence, $c\leq \dfrac{\epsilon_{m} 2^{m/2}}{4}$.
\newline Again, $\exists a_{i} \in s_{i}$ such that, 
$\hat{r}_{i} + c_{m_{i}}\leq r_{i} + 2c_{m_{i}} $
%\newline\hspace*{14em}$= \hat{r}_{i}-\sqrt{\epsilon_{m}} + 2c_{m} +\sqrt{\epsilon_{m}}$
\newline\hspace*{14em}$= \hat{r}_{i} + 4c_{m_{i}} - 2c_{m_{i}} $
\newline\hspace*{14em}$\leq r_{i} + \Delta_{i} - 2c_{m_{i}}$
\newline\hspace*{14em}$< r^{*} -2c_{m_{i}} $
\newline\hspace*{14em}$\leq \hat{r}^{*} - c_{m_{i}}$
%\newline\hspace*{14em}$= r_{i} + 5\dfrac{\sqrt{\epsilon_{m}}}{\sqrt{w_{s_{i}}}} - 2\sqrt{w_{s_{i}}}c_{m}$
%\newline \hspace*{4em}
%\newline But, $\epsilon_{m}=\dfrac{\hat{\Delta}_{s,m}}{\ell_{m}}$, where $\hat{\Delta}_{s,m}=\max_{i\in B_{m}}{\hat{r}_{i}}-\min_{j\in B_{m}}{\hat{r}_{j}},i\neq j$ and $\ell_{m}$ is increased after every round.
%\newline \hspace*{4em}$\leq \hat{r}_{i} + \epsilon_{m} 2^{(m-4)/2} - 2c$
\newline Hence, we get that as soon as $\sqrt{\epsilon_{m_{i}}}<\dfrac{\Delta_{i}}{2}$, $\exists a_{i}$ which gets eliminated.
%\newline So, $\hat{r}_{i}+c_{m}\leq \hat{r}_{i}+2c_{m}\leq r_{i} + \Delta_{i} - 2\sqrt{w_{s_{i}}}c_{m}\leq r^{*} - 2\sqrt{w_{s_{i}}}c_{m}$
%\leq \hat{r}^{*} - \sqrt{w}c_{m}
%\newline $\Rightarrow\hat{r}_{i}+c_{m}\leq \hat{r}_{i} - \sqrt{w}c_{m}  \leq r^{*} - \sqrt{w}c_{m}$
%\newline $\Rightarrow \hat{r}_{i} \leq {r}^{*} - 2\sqrt{w_{s_{i}}}c_{m} - 2c_{m} \leq \hat{r}^{*} - 2\sqrt{w_{s_{i}}}c_{m}$
\newline So, we need to bound the event of $\hat{r}_{i}+c_{m_{i}}\leq \hat{r}^{*}-c_{m_{i}}$ given that $\sqrt{\epsilon_{m_{i}}}<\dfrac{\Delta_{i}}{2}$ becomes true for some arm $a_{i}$ after the $m$-th round and $c_{m}=\sqrt{\dfrac{\log (T\epsilon_{m_{i}}^{2})}{2 n_{m_{i}}}}$.
%\newline $\Rightarrow \hat{r}_{i}+2c_{m}\leq \hat{r}_{i} + 2\sqrt{w}c_{m} \leq \hat{r}^{*}$
%\newline $\Rightarrow \hat{r}_{i} + \sqrt{w}c_{m} \leq \hat{r}^{*} - \sqrt{w}c_{m}$
%\newline $\Rightarrow\hat{r}_{i}+c_{m}\sqrt{\dfrac{w\ell_{m}}{\epsilon_{m}}} \leq \hat{r}^{*}-c_{m}\sqrt{\dfrac{w\ell_{m}}{\epsilon_{m}}} $, as $c_{m}\sqrt{\dfrac{w\ell_{m}}{\epsilon_{m}}} > 0$

%\begin{proof} of Proposition 2:
%\newline
%Now, we can bound $\hat{r}_{i}+c_{m}\leq \hat{r}^{*}-c_{m}$ given that $\sqrt{\epsilon_{m}}<\dfrac{\Delta_{i}}{2}$ for some arm $a_{i}\in s_{i}$. 
	So, we need to bound the probability,
\newline\hspace*{4em} $\mathbb{P}\lbrace\hat{r}^{*}\leq r^{*} - c_{m_{i}}\rbrace\leq U_{m}$, where $U_{m}$ is an  arbitrary upper bound.
%, for a fixed $n_{s_{i}}$.
%\mathbb{P}\lbrace\hat{r}^{*}\leq r^{*} - c_{m}\rbrace\leq
\newline
%Here, we guarantee that only if $\hat{r}^{*}\leq r^{*} - c_{m}\sqrt{\dfrac{w\ell_{m}}{\epsilon_{m}}}$ or $\hat{r}_{i}\geq r_{i} + c_{m}\sqrt{\dfrac{w\ell_{m}}{\epsilon_{m}}}$ then only arm will not be deleted. This is a more aggressive arm elimination condition than simply looking at $\hat{r}^{*}\leq r^{*} - c_{m}$ or $\hat{r}_{i}\geq r_{i} + c_{m}$ because we are exploring much carefully by dividing the larger problem into sub-problems.
%\newline
Applying Chernoff-Hoeffding bound and considering independence of events,
\newline
\newline\hspace*{0em} $\mathbb{P}\lbrace\hat{r}^{*}\leq r^{*} - c_{m_{i}}\rbrace\leq exp(-2c_{m_{i}}^{2}n_{m_{i}})$
\newline\hspace*{8em} $\leq exp(-2 * \dfrac{\log (T\epsilon_{m_{i}}^{2})}{2 n_{m_{i}}} *n_{m_{i}})$
\newline\hspace*{8em} $\leq \dfrac{1}{T\epsilon_{m_{i}}^{2}}$
%$\leq \bigg(\dfrac{1}{4\psi(m)T\epsilon_{m}^{2}}\bigg)^{D}$, as $\ell_{m}-1\leq D$
%\newline\hspace*{2em}
\newline
Similarly, $\mathbb{P}\lbrace\hat{r}_{i}\geq r_{i} + c_{m_{i}}\rbrace\leq \dfrac{1}{T\epsilon_{m_{i}}^{2}}$
\newline
Summing, the two up, the probability that a sub-optimal arm $a_{i}$ is not eliminated in $m_{i}$-th round is  $\bigg(\dfrac{2}{T\epsilon_{m_{i}}^{2}}\bigg)$. 
\newline
Summing up over all arms in $A$ and bounding trivially by $T\Delta_{i}$,
%\sum_{i\in A}\bigg(\dfrac{2}{T\epsilon_{m_{i}}^{2}}\bigg)\leq
\newline\hspace*{4em} $\sum_{i\in A}\bigg(\dfrac{2T\Delta_{i}}{T\epsilon_{m_{i}}\dfrac{\Delta_{i}}{2}^{4}}\bigg)\leq \sum_{i\in A}\bigg(\dfrac{8}{\epsilon_{m_{i}}\Delta_{i}}\bigg)\leq \sum_{i\in A}\bigg(\dfrac{32}{\Delta_{i}^{3}}\bigg)$


\subsection{Case b1:} 
Either an arm $a_{i}$ is eliminated in round $m_{i}$ or before or else there is no optimal arm $a^{*}\in B_{m_{i}}$.
\newline
Also, since we are eliminating a sub-optimal arm $a_{i}$ on or before round $m_{i}$, it is pulled no longer than,
\newline
\hspace*{4em}$n_{m_{i}}=\bigg\lceil\dfrac{2\log{(T\epsilon_{m_{i}}^{2})}}{\epsilon_{m_{i}}}\bigg\rceil$
\newline
%$\sqrt{\dfrac{\epsilon_{m}}{w}}<\dfrac{\Delta_{i}}{5}\Rightarrow \sqrt{\dfrac{\epsilon_{m}}{\ell_{m}^{2}}}<\dfrac{\Delta_{i}}{5}$, as $w\geq \ell_{m}^{2}$
%\newline
%$\Rightarrow \sqrt{\dfrac{\epsilon_{m}}{\ell_{m}^{2}}}<\dfrac{\Delta_{i}}{5}$
%\newline
So, the total contribution of $a_{i}$  till round $m_{i}$ is given by,
\newline
\hspace*{4em}$\Delta_{i}\bigg\lceil\dfrac{2\log{(T\epsilon_{m_{i}}^{2})}}{\epsilon_{m_{i}}}\bigg\rceil$
%\newline
%\hspace*{4em}
$\leq\Delta_{i}\bigg\lceil\dfrac{2\log{(T(\dfrac{\Delta_{i}}{2})^{4})}}{(\dfrac{\Delta_{i}}{2})^{2}}\bigg\rceil$, since $\sqrt{\epsilon_{m_{i}}}\leq\dfrac{\Delta_{i}}{2}$
\newline
\hspace*{12em}
$\leq\Delta_{i}\bigg(1+\dfrac{32\log{(T(\dfrac{\Delta_{i}}{2})^{4})}}{\Delta_{i}^{2}}\bigg)$
\newline
\hspace*{12em}
$\leq\Delta_{i}\bigg(1+\dfrac{32\log{(T\dfrac{\Delta_{i}^{4}}{16})}}{\Delta_{i}^{2}}\bigg)$
\newline
Summing over all arms,
\newline
\hspace*{4em}$\leq\sum_{i\in A}\Delta_{i}\bigg(1+\dfrac{32\log{(T\dfrac{\Delta_{i}^{4}}{16}})}{\Delta_{i}^{2}}\bigg)$
%\newline
%\hspace*{4em}$\leq\sum_{i\in B_{m}}\bigg(\Delta_{i}+\dfrac{27\log{(\psi(m)T\dfrac{\Delta_{i}^{\frac{8}{5}}}{12})}}{\Delta_{i}^{\frac{3}{5}}}\bigg)$
%\newline
%\hspace*{4em}$\leq\sum_{i\in B_{m}}\bigg(\Delta_{i}+\dfrac{12.5\log{(\psi(m)T\Delta_{i}^{4})}}{\Delta_{i}}\bigg)$
	%Thus, we see that the growth of the $n_{s_{i}}$ is always linear and not quadratic as in UCB-Revisited(\cite{auer2010ucb}). Also once $2l_{m}=D$, then $n_{s_{i}}$ will remain constant for the next rounds till stopping condition is met. Thus we have a more controlled exploration than UCB-Revisited(\cite{auer2010ucb}) and Median Elimination(\cite{even2006action}). Hence, 
%\newline
%\hspace*{4em}$\leq\sum_{i\in B_{m}}\bigg(\Delta_{i}+\dfrac{54\log{(\psi(m)T\dfrac{\Delta_{i}^{\frac{8}{5}}}{12})}}{\Delta_{i}^{\frac{3}{5}}}\bigg)$
%\newline
%But, $\psi(m)\leq c/m$ for any $c>0$ and so the regret upper bound comes off as 
%But, $\psi(m) = 1$ the regret upper bound comes off as 
%\newline
%$R_{T}\leq \sum_{i\in A}\bigg (\max{\bigg\lbrace \bigg(\dfrac{32}{(\Delta_{i})^{3}}\bigg) ,\bigg(\dfrac{25\Delta_{i}}{(\Delta^{2})(0.16T\Delta^{2})^{2|B_{m}|^{2}\Delta/5}}\bigg)\bigg\rbrace} + \bigg(\Delta_{i}+\dfrac{32\log{(T\dfrac{\Delta_{i}^{4}}{16})}}{\Delta_{i}}\bigg)\bigg)$. 

\subsection{case b2:} 
In this case we will consider that the optimal arm $a^{*}$ was eliminated by a sub-optimal arm. Firstly, if conditions of case b1 holds then the optimal arm $a^{*}$ will not be eliminated in round $m=m_{*}$ or it will lead to the contradiction that $r_{i}>r^{*}$. In any round $m_{*}$, if the optimal arm $a^{*}$ gets eliminated then for any round from $1$ to $m_{j}$ all arms $a_{j}$ such that $\sqrt{\epsilon_{m}}<\dfrac{\Delta_{j}}{2}$ were eliminated according to assumption in case b1. Let, the arms surviving till $m_{*}$ round be denoted by $A^{'}$. This leaves any arm $a_{b}$ such that $\sqrt{\epsilon_{m}}\geq\dfrac{\Delta_{b}}{2}$ to still survive and eliminate arm $a^{*}$ in round $m_{*}$. Let, such arms that survive $a^{*}$ belong to $A^{''}$. Also maximal regret per step after eliminating $a^{*}$ is the maximal $\Delta_{j}$ among the remaining arms $a_{j}$ with $m_{j}\geq m_{*}$.  Let $m_{b}$ be the round when $\sqrt{\epsilon_{m}}<\dfrac{\Delta_{b}}{2}$ that is $m_{b}=min\lbrace m|\sqrt{\epsilon_{m}}<\dfrac{\Delta_{b}}{2}\rbrace$. Hence, the maximal regret after eliminating the arm $a^{*}$ is upper bounded by, 
\newline
$\sum_{m_{*}=0}^{max_{j\in A^{'}}m_{j}}\sum_{i\in A^{''}:m_{i}>m_{*}}\bigg(\dfrac{2}{T\epsilon_{m}^{2}} \bigg).T\max_{j\in A^{''}:m_{j}\geq m_{*}}{\Delta}_{j}$
\newline
\hspace*{0em}$\leq\sum_{m_{*}=0}^{max_{j\in A^{'}}m_{j}}\sum_{i\in A^{''}:m_{i}>m_{*}}\bigg(\dfrac{2}{T\epsilon_{m}^{2}} \bigg).T.2\sqrt{\epsilon_{m}}$, since $\sqrt{\epsilon_{m}}<\dfrac{\Delta_{i}}{2}$
\newline
\hspace*{0em}$\leq\sum_{m_{*}=0}^{max_{j\in A^{'}}m_{j}}\sum_{i\in A^{''}:m_{i}>m_{*}}4\bigg(\dfrac{1}{\epsilon_{m}^{3/2}} \bigg) $
%\newline
%\hspace*{0em}$\leq\sum_{m_{*}=0}^{max_{j\in A^{'}}m_{j}}\sum_{i\in A^{''}:m_{i}>m_{*}}\bigg(\dfrac{4.4^{3/2}}{\Delta_{i}^{3}} \bigg) $, as $\sqrt{\epsilon_{m}}\leq\dfrac{\Delta_{i}}{2}$
\newline
\hspace*{0em}$\leq\sum_{i\in A^{''}:m_{i}>m_{*}}\sum_{m_{*}=0}^{\min{\lbrace m_{i},m_{b}\rbrace}}\bigg(\dfrac{4}{2^{-(3/2)m_{*}}} \bigg) $
\newline
\hspace*{0em}$\leq\sum_{i\in A^{'}}\bigg(\dfrac{4}{2^{-(3/2)m_{*}}} \bigg)+\sum_{i\in A^{''}\setminus A^{'}}\bigg(\dfrac{4}{2^{-(3/2)m_{b}}} \bigg)$
%\newline
%\hspace*{0em}$<\sum_{i\in A^{'}}\bigg(4*2^{-(3/2)m_{*}} \bigg)+\sum_{i\in A^{''}\setminus A^{'}}\bigg(4*2^{-(3/2)m_{b}} \bigg)$
\newline
\hspace*{0em}$\leq\sum_{i\in A^{'}}\bigg(\dfrac{4*2^{3/2}}{\Delta_{i}^{3}} \bigg)+\sum_{i\in A^{''}\setminus A^{'}}\bigg(\dfrac{4*2^{3/2}}{b^{3}} \bigg)$
\newline
\hspace*{0em}$\leq\sum_{i\in A^{'}}\bigg(\dfrac{12}{\Delta_{i}^{3}} \bigg)+\sum_{i\in A^{''}\setminus A^{'}}\bigg(\dfrac{12}{b^{3}} \bigg)$
\newline
Summing up \textbf{Case a}, \textbf{Case b1} and \textbf{Case b2}, the total regret till round $m$ is given by,
\newline $R_{T}\leq \sum_{i\in A:\Delta_{i}\geq b}\bigg \lbrace \bigg(\dfrac{44}{(\Delta_{i})^{3}}\bigg) + \bigg(\Delta_{i}+\dfrac{32\log{(T\dfrac{\Delta_{i}^{4}}{16})}}{\Delta_{i}}\bigg)\bigg\rbrace + \sum_{i\in A:0\leq\Delta_{i}\leq b}\dfrac{12}{b^{3}} + max_{i:\Delta\leq b}\Delta_{i}T$
\end{proof}

\begin{remark}
In this proof for simplicity we take $\rho_{a}=1$. After proving proposition $4$ we give an alternate regret bound for arm elimination, which closely follows from the proof of both proposition $1$ and $2$.  
\end{remark}


\section{Appendix B}

An illustrative diagram explaining Cluster Elimination is given in Figure 6.

\begin{figure}
\includegraphics[scale=0.3]{img/diagCluster.jpg}
\caption{Cluster Elimination}
\end{figure}

\begin{proposition}
Considering only the cluster elimination condition, the total regret till $T$ is upper bounded by $R_{T}\leq \sum_{i\in A:\Delta_{i}\geq b}\bigg(\dfrac{2^{1+4\rho_{s}}\rho_{s}^{2\rho_{s}}T^{1-\rho_{s}}}{\Delta_{i}^{4\rho_{s}-1}}\bigg) + \bigg(\Delta_{i}+\dfrac{32(K+p)\rho_{s}\log{(T\dfrac{\Delta_{i}^{4}}{16\rho_{s}^{2}})}}{p\Delta_{i}}\bigg)  +  \bigg(\dfrac{T^{1-\rho_{s}}2^{2\rho_{s}+\frac{3}{2}}}{\Delta_{i}^{4\rho_{s} -1}} \bigg) \bigg \rbrace+\sum_{i\in A:0\leq\Delta_{i}\leq b}\bigg(\dfrac{T^{1-\rho_{s}}2^{2\rho_{s}+\frac{3}{2}}}{b^{4\rho_{s} -1}} \bigg) + max_{i:\Delta\leq b}\Delta_{i}T$, where $\rho_{s}\in (0,1)$, $p$ is the number of clusters and $T$ is the horizon.
\end{proposition}


\begin{remark} A sketch of the proof is given below,
\newline
\begin{itemize}
\item Define $g$-th round(in proposition 6) as the same way as the $m$-th round signifying the first/minimum round when a cluster gets eliminated.
\item Let in the $g$-th round $C_{g}$ be the set which contains all the max payoff arms from each cluster.
\item For regret bound proof according to proposition $5$, consider only this $C_{g}$ and proof following the same way as in proposition $5$ with some minor change. Hence, $C_{g}$ actually behaves like the set $B_{m}$ in proposition $5$ but contains just the max payoff arm from each cluster.
\item Introduce a bounded parameter $\rho_{s}\in (0,1]$ for cluster elimination to mimic the arm elimination condition in proposition $5$, but with the condition that whenever $\sqrt{\rho_{s}\epsilon_{g}}\leq \dfrac{\Delta_{i}}{2}, a_{i}\in C_{g}$, then the cluster $s_{k}$(where $\hat{r}_{i}$ is the max payoff) gets eliminated. Because of the algorithm, we are guaranteed that the size of the cluster in the $g$-th round is $\ell=\bigg\lceil \dfrac{K}{p}\bigg\rceil$.
\end{itemize}
\end{remark}

%\begin{remark} We also point out a few other observations:
%\newline
%\begin{itemize}
%\item The maximum number of clusters that can be formed in any round is $\big\lceil \dfrac{K}{2}\big\rceil$, since we start with a minimum cluster size of $\ell_{g}=2$.
%\item Also as $\epsilon_{g}$ decreases after every round and as $\epsilon_{g}\to \Delta, |S_{g}|\to D $, where $D$ is the true number of clusters based on the underlying distribution of $r_{i}, \forall i\in A$.
%\item On the implementation side, the subroutine SubroutineCluster(p) is implemented in such a way, that after arranging the arms in ascending order based on their $\hat{r}_{i}$, a sweep from left to right will put the arms in their respective clusters in such a way that $|\hat{r}_{i}-\hat{r}_{j}|,\forall i,j \in B_{g}$ is at most $\epsilon_{g}$. 
%\end{itemize}
%\end{remark}

\begin{proof} of Proposition 6:

Let $C_{g}=\lbrace \hat{r}_{max_{s_{i}}}| \forall s_{i}\in S \rbrace$, that is let $C_{g}$ be the set of all arms which has the maximum estimated payoff in their respective clusters in the $g$-th round.

Let, for each sub-optimal arm $a_{i}\in C_{g}$, $g_{i}=\min{\lbrace g|\sqrt{\rho_{s}\epsilon_{g}}\leq \dfrac{\Delta_{i}}{2} \rbrace}$. So, $g_{i}$ be the first round when $\sqrt{\rho_{s}\epsilon_{g}}\leq \dfrac{\Delta_{i}}{2}$ where $a_{i}\in C_{g}$ is the maximum payoff arm in cluster $s_{k}$. We will also consider that $\max \hat{r}_{i}\in C_{g}$ is $a^{*}$ and it has not still been eliminated. The parameter $\rho_{s}$ is introduced just to make sure that the cluster elimination is a more aggressive elimination than arm elimination.

So, for cluster elimination we will only be considering the arms in $C_{g}$ and following the proof of proposition $5$,

\subsection{Case a:} 
Some sub-optimal arm $a_{i}$ is not eliminated in round $g_{i}$ or before and the optimal arm $a^{*}\in B_{g_{i}} \subset B_{m_{i}}$
\newline In arm elimination condition, given the choice of confidence interval $c_{g}$, we want to bound the event $\hat{r}_{i}+c_{g_{i}}\leq \hat{r}^{*}-c_{g_{i}}$.
%with a more tighter event of $ \hat{r}_{i} + \sqrt{w_{s_{i}}}c_{m} \leq \hat{r}^{*} - \sqrt{w_{s_{i}}}c_{m}$ which will result in faster elimination of arms within a cluster, given the choice of $c_{m}$ and $w_{s_{i}}$.
\newline Now, $c_{g_{i}}=\sqrt{\dfrac{\rho_{s} \log (T\epsilon_{g_{i}}^{2})}{2 n_{g_{i}}}}$, where $0 < \rho_{s}\leq 1$
\newline Putting the value of $n_{g_{i}}=\dfrac{2\log{(T\epsilon_{g_{i}}^{2})}}{\epsilon_{g_{i}}}$ in $c_{g_{i}}$,
\newline $c_{g_{i}}=\sqrt{\dfrac{\rho_{s}*\epsilon_{g_{i}}\log (T\epsilon_{g_{i}}^{2})}{2*2 \log(T\epsilon_{g_{i}}^{2})}}=\sqrt{\dfrac{\rho_{s}\epsilon_{g_{i}}}{2}} = \sqrt{\rho_{s}\epsilon_{g_{i}+1}} < \dfrac{\sqrt{\rho_{s}}\Delta_{i}}{4} < \dfrac{\Delta_{i}}{4} $
%\leq\dfrac{\epsilon_{m}\sqrt{\ell_{m}}}{2\sqrt{w\ell_{m}}}\leq \dfrac{\epsilon_{m}}{2\sqrt{w}}$.
%\newline But in $\xi_{1}$, $\ell_{m}=2^{m}$.
%\newline Hence, $c\leq \dfrac{\epsilon_{m} 2^{m/2}}{4}$.
\newline Again, $\exists a_{i} \in C_{g}$ such that, 
$\hat{r}_{i} + c_{g_{i}}\leq r_{i} + 2c_{g_{i}} $
%\newline\hspace*{14em}$= \hat{r}_{i}-\sqrt{\epsilon_{m}} + 2c_{m} +\sqrt{\epsilon_{m}}$
\newline\hspace*{14em}$= \hat{r}_{i} + 4c_{g_{i}} - 2c_{g_{i}} $
\newline\hspace*{14em}$\leq r_{i} + \Delta_{i} - 2c_{g_{i}}$
\newline\hspace*{14em}$< r^{*} -2c_{g_{i}} $
\newline\hspace*{14em}$\leq \hat{r}^{*} - c_{g_{i}}$
%\newline\hspace*{14em}$= r_{i} + 5\dfrac{\sqrt{\epsilon_{m}}}{\sqrt{w_{s_{i}}}} - 2\sqrt{w_{s_{i}}}c_{m}$
%\newline \hspace*{4em}
%\newline But, $\epsilon_{m}=\dfrac{\hat{\Delta}_{s,m}}{\ell_{m}}$, where $\hat{\Delta}_{s,m}=\max_{i\in B_{m}}{\hat{r}_{i}}-\min_{j\in B_{m}}{\hat{r}_{j}},i\neq j$ and $\ell_{m}$ is increased after every round.
%\newline \hspace*{4em}$\leq \hat{r}_{i} + \epsilon_{m} 2^{(m-4)/2} - 2c$
\newline Hence, we get that as soon as $\sqrt{\rho_{s}\epsilon_{g_{i}}}<\dfrac{\Delta_{i}}{2}$, $\exists a_{i}\in C_{g}$ which gets eliminated.
%\newline So, $\hat{r}_{i}+c_{m}\leq \hat{r}_{i}+2c_{m}\leq r_{i} + \Delta_{i} - 2\sqrt{w_{s_{i}}}c_{m}\leq r^{*} - 2\sqrt{w_{s_{i}}}c_{m}$
%\leq \hat{r}^{*} - \sqrt{w}c_{m}
%\newline $\Rightarrow\hat{r}_{i}+c_{m}\leq \hat{r}_{i} - \sqrt{w}c_{m}  \leq r^{*} - \sqrt{w}c_{m}$
%\newline $\Rightarrow \hat{r}_{i} \leq {r}^{*} - 2\sqrt{w_{s_{i}}}c_{m} - 2c_{m} \leq \hat{r}^{*} - 2\sqrt{w_{s_{i}}}c_{m}$
\newline So, we need to bound the event of $\hat{r}_{i}+c_{g_{i}}\leq \hat{r}^{*}-c_{g_{i}}$ given that $\sqrt{\rho_{s}\epsilon_{g_{i}}}<\dfrac{\Delta_{i}}{2}$ becomes true for some arm $a_{i}\in C_{g}$ after the $g$-th round and $c_{g_{i}}=\sqrt{\dfrac{\rho_{s} \log (T\epsilon_{g_{i}}^{2})}{2 n_{g_{i}}}}$.
%\newline $\Rightarrow \hat{r}_{i}+2c_{m}\leq \hat{r}_{i} + 2\sqrt{w}c_{m} \leq \hat{r}^{*}$
%\newline $\Rightarrow \hat{r}_{i} + \sqrt{w}c_{m} \leq \hat{r}^{*} - \sqrt{w}c_{m}$
%\newline $\Rightarrow\hat{r}_{i}+c_{m}\sqrt{\dfrac{w\ell_{m}}{\epsilon_{m}}} \leq \hat{r}^{*}-c_{m}\sqrt{\dfrac{w\ell_{m}}{\epsilon_{m}}} $, as $c_{m}\sqrt{\dfrac{w\ell_{m}}{\epsilon_{m}}} > 0$

%\begin{proof} of Proposition 2:
%\newline
%Now, we can bound $\hat{r}_{i}+c_{m}\leq \hat{r}^{*}-c_{m}$ given that $\sqrt{\epsilon_{m}}<\dfrac{\Delta_{i}}{2}$ for some arm $a_{i}\in s_{i}$. 
	So, we need to bound the probability,
\newline\hspace*{4em} $\mathbb{P}\lbrace\hat{r}^{*}\leq r^{*} - c_{g_{i}}\rbrace\leq U_{g}$, where $U_{g}$ is an  arbitrary upper bound.
%, for a fixed $n_{s_{i}}$.
%\mathbb{P}\lbrace\hat{r}^{*}\leq r^{*} - c_{m}\rbrace\leq
\newline
%Here, we guarantee that only if $\hat{r}^{*}\leq r^{*} - c_{m}\sqrt{\dfrac{w\ell_{m}}{\epsilon_{m}}}$ or $\hat{r}_{i}\geq r_{i} + c_{m}\sqrt{\dfrac{w\ell_{m}}{\epsilon_{m}}}$ then only arm will not be deleted. This is a more aggressive arm elimination condition than simply looking at $\hat{r}^{*}\leq r^{*} - c_{m}$ or $\hat{r}_{i}\geq r_{i} + c_{m}$ because we are exploring much carefully by dividing the larger problem into sub-problems.
%\newline
Applying Chernoff-Hoeffding bound and considering independence of events,
\newline
\newline\hspace*{0em} $\mathbb{P}\lbrace\hat{r}^{*}\leq r^{*} - c_{g_{i}}\rbrace\leq exp(-2c_{g_{i}}^{2}n_{g_{i}})$
\newline\hspace*{8em} $\leq exp(-2 * \dfrac{\rho_{s}\log ( T\epsilon_{g_{i}}^{2})}{2 n_{g_{i}}} *n_{g_{i}})$
\newline\hspace*{8em} $\leq \dfrac{1}{(T\epsilon_{g_{i}}^{2})^{\rho_{s}}}$
%$\leq \bigg(\dfrac{1}{4\psi(m)T\epsilon_{m}^{2}}\bigg)^{D}$, as $\ell_{m}-1\leq D$
%\newline\hspace*{2em}
\newline
Similarly, $\mathbb{P}\lbrace\hat{r}_{i}\geq r_{i} + c_{g_{i}}\rbrace\leq \dfrac{1}{(T\epsilon_{g_{i}}^{2})^{\rho_{s}}}$
\newline
Summing, the two up, the probability that a sub-optimal arm $a_{i}\in C_{g}$ is not eliminated in $g_{i}$-th round is  $\bigg(\dfrac{2}{(T\epsilon_{g_{i}}^{2})^{\rho_{s}}}\bigg)$. 
\newline Now, for each round $g$, all the elements of $C_{g}$ are the respective max payoff arms of their cluster $s_{k}, \forall s_{k}\in S_{g}$, that is all the other arms in their respective clusters have performed worse than them. Hence, since $A\supset C_{g}$ and we can bound the max probability that a sub-optimal arm $a_{j}\in A$ is not eliminated in the $g$-th round by the same probability of $\bigg(\dfrac{2}{(T\epsilon_{g_{i}}^{2})^{\rho_{s}}}\bigg)$. 
\newline
Summing up over all arms in $A$ and bounding trivially by $T\Delta_{i}$,
\newline\hspace*{4em} $\sum_{i\in A}\bigg(\dfrac{2T\Delta_{i}}{(T\dfrac{\Delta_{i}}{16\rho_{s}^{2}}^{4})^{\rho_{s}}}\bigg)\leq \sum_{i\in A}\bigg(\dfrac{2^{1+4\rho_{s}}T^{1-\rho_{s}}\rho_{s}^{2\rho_{s}}\Delta_{i}}{\Delta_{i}^{4\rho_{s}}}\bigg)$
\newline\hspace*{12em}
$\leq \sum_{i\in A}\bigg(\dfrac{2^{1+4\rho_{s}}\rho_{s}^{2\rho_{s}}T^{1-\rho_{s}}}{\Delta_{i}^{4\rho_{s}-1}}\bigg)$
\newline Thus, we see that putting a value of $\rho_{s}=1$, nicely brings out the result of case b1 of proposition $5$.


\subsection{Case b1:} 
Either an arm $a_{i}\in C_{g}$ is eliminated along with all the arms in that cluster in round $g_{i}$(or before) or else there is no optimal arm $a^{*}\in C_{g_{i}}$. Again, in the $g$-th round, the maximum total elements in the cluster can be no more than $\ell=\bigg\lceil \dfrac{K}{p}\bigg\rceil$.
\newline
Also, since we are eliminating a sub-optimal arm $a_{i}\in C_{g_{i}}$ on or before round $g_{i}$, it is pulled (along with all the other arms in that cluster) no longer than,
\newline
\hspace*{4em}$n_{g_{i}}=\bigg\lceil\dfrac{2\log{(T\epsilon_{g_{i}}^{2})}}{\epsilon_{g_{i}}}\bigg\rceil$
\newline
%$\sqrt{\dfrac{\epsilon_{m}}{w}}<\dfrac{\Delta_{i}}{5}\Rightarrow \sqrt{\dfrac{\epsilon_{m}}{\ell_{m}^{2}}}<\dfrac{\Delta_{i}}{5}$, as $w\geq \ell_{m}^{2}$
%\newline
%$\Rightarrow \sqrt{\dfrac{\epsilon_{m}}{\ell_{m}^{2}}}<\dfrac{\Delta_{i}}{5}$
%\newline
So, the total contribution of $a_{i}$  along with all the other arms in the cluster till round $g_{i}$ is given by,
\newline
\hspace*{4em}$\Delta_{i}\bigg\lceil\dfrac{2\ell\log{(T\epsilon_{g_{i}}^{2})}}{\epsilon_{g_{i}}}\bigg\rceil$
%\newline
%\hspace*{4em}
$\leq\Delta_{i}\bigg\lceil\dfrac{2\ell\log{(T(\dfrac{\Delta_{i}}{2\sqrt{\rho_{s}}})^{4})}}{(\dfrac{\Delta_{i}}{2\sqrt{\rho_{s}}})^{2}}\bigg\rceil$, since $\sqrt{\rho_{s}\epsilon_{g_{i}}}\leq\dfrac{\Delta_{i}}{2}$
\newline
\hspace*{12em}
$\leq\Delta_{i}\bigg(1+\dfrac{32*(\frac{K}{p}+1)*\rho_{s}*\log{(T(\dfrac{\Delta_{i}}{2\sqrt{\rho_{s}}})^{4})}}{\Delta_{i}^{2}}\bigg)$, since in the $g$-th round the maximum cluster size is bounded by $\ell=\bigg\lceil \dfrac{K}{p} \bigg\rceil$.
\newline
\hspace*{12em}$\leq\Delta_{i}\bigg(1+\dfrac{32*(K+p)*\rho_{s}*\log{(T(\dfrac{\Delta_{i}}{2\sqrt{\rho_{s}}})^{4})}}{p\Delta_{i}^{2}}\bigg)$
\newline
\hspace*{12em}
$\leq\Delta_{i}\bigg(1+\dfrac{32(K+p)\rho_{s}\log{(T\dfrac{\Delta_{i}^{4}}{16\rho_{s}^{2}})}}{p\Delta_{i}^{2}}\bigg)$
\newline
Summing over all arms in $A \supset C_{g}$,
\newline
\hspace*{4em}$\leq\sum_{i\in A}\Delta_{i}\bigg(1+\dfrac{32\rho_{s}(K+p)\log{(T\dfrac{\Delta_{i}^{4}}{16\rho_{s}^{2}}})}{p\Delta_{i}^{2}}\bigg)$
\newline
%\hspace*{4em}$\leq\sum_{i\in B_{m}}\bigg(\Delta_{i}+\dfrac{27\log{(\psi(m)T\dfrac{\Delta_{i}^{\frac{8}{5}}}{12})}}{\Delta_{i}^{\frac{3}{5}}}\bigg)$
%\newline
%\hspace*{4em}$\leq\sum_{i\in B_{m}}\bigg(\Delta_{i}+\dfrac{12.5\log{(\psi(m)T\Delta_{i}^{4})}}{\Delta_{i}}\bigg)$
	%Thus, we see that the growth of the $n_{s_{i}}$ is always linear and not quadratic as in UCB-Revisited(\cite{auer2010ucb}). Also once $2l_{m}=D$, then $n_{s_{i}}$ will remain constant for the next rounds till stopping condition is met. Thus we have a more controlled exploration than UCB-Revisited(\cite{auer2010ucb}) and Median Elimination(\cite{even2006action}). Hence, 
%\newline
%\hspace*{4em}$\leq\sum_{i\in B_{m}}\bigg(\Delta_{i}+\dfrac{54\log{(\psi(m)T\dfrac{\Delta_{i}^{\frac{8}{5}}}{12})}}{\Delta_{i}^{\frac{3}{5}}}\bigg)$
%\newline
%But, $\psi(m)\leq c/m$ for any $c>0$ and so the regret upper bound comes off as 
%But, $\psi(m) = 1$ the regret upper bound comes off as 
%\newline
%$R_{T}\leq \sum_{i\in A}\bigg (\max{\bigg\lbrace \bigg(\dfrac{32}{(\Delta_{i})^{3}}\bigg) ,\bigg(\dfrac{25\Delta_{i}}{(\Delta^{2})(0.16T\Delta^{2})^{2|B_{m}|^{2}\Delta/5}}\bigg)\bigg\rbrace} + \bigg(\Delta_{i}+\dfrac{32\log{(T\dfrac{\Delta_{i}^{4}}{16})}}{\Delta_{i}}\bigg)\bigg)$. 
%\todo{will prove case b2 once you verify that this approach is correct}

\subsection{Case b2:} 
In this case we will consider that the cluster containing the optimal arm $a^{*}$ was eliminated by a sub-optimal cluster. Firstly, if conditions of case b1 holds then the optimal arm $a^{*}\in C_{g}$ will not be eliminated in round $g=g_{*}$ or it will lead to the contradiction that $r_{i}>r^{*}$ where $a_{i},a^{*}\in C_{g}$. In any round $g_{*}$, if the optimal arm $a^{*}$ gets eliminated then for any round from $1$ to $g_{j}$ all arms $a_{j}\in C_{g}$ such that $\sqrt{\rho_{s}\epsilon_{g}}<\dfrac{\Delta_{j}}{2}$ were eliminated according to assumption in case b1. Let, the arms surviving till $g_{*}$ round be denoted by $C_{g}^{'}$. This leaves any arm $a_{b}$ such that $\sqrt{\rho_{s}\epsilon_{g}}\geq\dfrac{\Delta_{b}}{2}$ to still survive and eliminate arm $a^{*}$ in round $g_{*}$. Let, such arms that survive $a^{*}$ belong to $C_{g}^{''}$. Also maximal regret per step after eliminating $a^{*}$ is the maximal $\Delta_{j}$ among the remaining arms $a_{j}$ with $g_{j}\geq g_{*}$.  Let $g_{b}$ be the round when $\sqrt{\rho_{s}\epsilon_{g}}<\dfrac{\Delta_{b}}{2}$ that is $g_{b}=min\lbrace g|\sqrt{\rho_{s}\epsilon_{g}}<\dfrac{\Delta_{b}}{2}\rbrace$. Hence, the maximal regret after eliminating the arm $a^{*}$ is upper bounded by, 
\newline
$\sum_{g_{*}=0}^{max_{j\in C_{g}^{'}}g_{j}}\sum_{i\in C_{g}^{''}:g_{i}>g_{*}}\bigg(\dfrac{2}{(T\epsilon_{g}^{2})^{\rho_{s}}} \bigg).T\max_{j\in C_{g}^{''}:g_{j}\geq g_{*}}{\Delta}_{j}$
\newline
But, we know that for any round $g$, elements of $C_{g}$ are the best performers in their respective clusters. So, taking that into account and $A'\supset C_{g}^{'}$ and $A''\supset C_{g}^{''}$
\newline
$\leq\sum_{g_{*}=0}^{max_{j\in A^{'}}g_{j}}\sum_{i\in A^{''}:g_{i}>g_{*}}\bigg(\dfrac{2}{(T\epsilon_{g}^{2})^{\rho_{s}}} \bigg).T\max_{j\in A^{''}:g_{j}\geq g_{*}}{\Delta}_{j}$
\newline
%$=\sum_{g_{*}=0}^{max_{j\in A^{'}}g_{j}}\sum_{i\in A^{''}:g_{i}>g_{*}}\bigg(\dfrac{2}{T\epsilon_{g}^{2}} \bigg).T\max_{j\in A^{''}:g_{j}\geq g_{*}}{\Delta}_{j}$
%\newline
\hspace*{0em}$\leq\sum_{g_{*}=0}^{max_{j\in A^{'}}g_{j}}\sum_{i\in A^{''}:g_{i}>g_{*}}\bigg(\dfrac{2}{(T\epsilon_{g}^{2})^{\rho_{s}}} \bigg).T.2\sqrt{\rho_{s}\epsilon_{g}}$, since $\sqrt{\rho_{s}\epsilon_{g}}<\dfrac{\Delta_{i}}{2}$
\newline
\hspace*{0em}$\leq\sum_{g_{*}=0}^{max_{j\in A^{'}}g_{j}}\sum_{i\in A^{''}:g_{i}>g_{*}}\bigg(\dfrac{4 T^{1-\rho_{s}}}{\epsilon_{g}^{2\rho_{s} - \frac{1}{2}}} \bigg) $
%\newline
%\hspace*{0em}$\leq\sum_{m_{*}=0}^{max_{j\in A^{'}}m_{j}}\sum_{i\in A^{''}:m_{i}>m_{*}}\bigg(\dfrac{4.4^{3/2}}{\Delta_{i}^{3}} \bigg) $, as $\sqrt{\epsilon_{m}}\leq\dfrac{\Delta_{i}}{2}$
\newline
\hspace*{0em}$\leq\sum_{i\in A^{''}:g_{i}>g_{*}}\sum_{g_{*}=0}^{\min{\lbrace g_{i},g_{b}\rbrace}}\bigg(\dfrac{4T^{1-\rho_{s}}}{2^{({2\rho_{s} - \frac{1}{2}})g_{*}}} \bigg) $
\newline
\hspace*{0em}$\leq\sum_{i\in A^{'}}\bigg(\dfrac{4T^{1-\rho_{s}}}{2^{({2\rho_{s} - \frac{1}{2}})g_{*}}} \bigg)+\sum_{i\in A^{''}\setminus A^{'}}\bigg(\dfrac{4T^{1-\rho_{s}}}{2^{({2\rho_{s} - \frac{1}{2}})g_{b}}} \bigg)$
%\newline
%\hspace*{0em}$<\sum_{i\in A^{'}}\bigg(4*2^{-(3/2)m_{*}} \bigg)+\sum_{i\in A^{''}\setminus A^{'}}\bigg(4*2^{-(3/2)m_{b}} \bigg)$
\newline
\hspace*{0em}$\leq\sum_{i\in A^{'}}\bigg(\dfrac{4T^{1-\rho_{s}}*2^{2\rho_{s}-\frac{1}{2}}}{\Delta_{i}^{4\rho_{s}-1}} \bigg)+\sum_{i\in A^{''}\setminus A^{'}}\bigg(\dfrac{4T^{1-\rho_{s}}*2^{2\rho_{s}-\frac{1}{2}}}{b^{4\rho_{s}-1}} \bigg)$
\newline
\hspace*{0em}$\leq\sum_{i\in A^{'}}\bigg(\dfrac{T^{1-\rho_{s}}2^{2\rho_{s}+\frac{3}{2}}}{\Delta_{i}^{4\rho_{s}-1}} \bigg)+\sum_{i\in A^{''}\setminus A^{'}}\bigg(\dfrac{T^{1-\rho_{s}}2^{2\rho_{s}+\frac{3}{2}}}{b^{4\rho_{s}-1}} \bigg)$
\newline
Summing up \textbf{Case a} and \textbf{Case b1} and \textbf{Case b2}, the total regret till round $g$ is given by,
\newline $R_{T}\leq \sum_{i\in A:\Delta_{i}\geq b}\bigg(\dfrac{2^{1+4\rho_{s}}\rho_{s}^{2\rho_{s}}T^{1-\rho_{s}}}{\Delta_{i}^{4\rho_{s}-1}}\bigg) + \bigg(\Delta_{i}+\dfrac{32(K+p)\rho_{s}\log{(T\dfrac{\Delta_{i}^{4}}{16\rho_{s}^{2}})}}{p\Delta_{i}}\bigg)  +  \bigg(\dfrac{T^{1-\rho_{s}}2^{2\rho_{s}+\frac{3}{2}}}{\Delta_{i}^{4\rho_{s} -1}} \bigg) \bigg \rbrace+\sum_{i\in A:0\leq\Delta_{i}\leq b}\bigg(\dfrac{T^{1-\rho_{s}}2^{2\rho_{s}+\frac{3}{2}}}{b^{4\rho_{s} -1}} \bigg) + max_{i:\Delta\leq b}\Delta_{i}T$


\end{proof}
\par Again, we see that $\rho_{s}=1$ and $p=K$(that is each arm is in a cluster of its own and so $\bigg\lceil \dfrac{K}{p} \bigg\rceil=1$) brings out a near equivalent result as the result of proposition $5$. That is, for $\rho_{s}=1,p=K$,
\newline $R_{T}\leq \sum_{i\in A:\Delta_{i}\geq b}\bigg \lbrace \bigg(\dfrac{44}{(\Delta_{i})^{3}}\bigg) + \bigg(\Delta_{i}+\dfrac{32\log{(T\dfrac{\Delta_{i}^{4}}{16})}}{\Delta_{i}}\bigg)\bigg\rbrace + \sum_{i\in A:0\leq\Delta_{i}\leq b}\dfrac{12}{b^{3}} + max_{i:\Delta\leq b}\Delta_{i}T$
\newline So, the most significant term for cluster elimination that is $\dfrac{32(K+p)\log{(T\dfrac{\Delta_{i}^{4}}{16})}}{p\Delta_{i}}$ is slightly more than arm elimination most significant term $\dfrac{32\log{(T\dfrac{\Delta_{i}^{4}}{16})}}{\Delta_{i}}$ because of the factor $\dfrac{K+p}{p}$, but its order is nearly same as arm elimination regret bound.
\par The principal takeaway from this result is that a lower value of $\rho_{s}\in (0,1]$ makes the algorithm risky(by increasing the error bound), but reduces expected regret whereas a higher value of $\rho_{s} > 1$ actually makes the algorithm less risky but at a cost of higher expected regret. The risk stems from the fact that the probability of sub-optimal arm elimination increases without proper exploration. So, there is always this trade-off between exploration and risk. So, in our algorithm we decrease the $\rho_{s}$ in a graded fashion halving after every round but $\rho_{s}$ is always bounded that is, $\rho \in (0,1)$. \

\begin{remark}
Considering, $\rho_{a}\in (0,1]$ interval, the regret upper bound for arm elimination can be modified in the same way giving us a bound of
\newline
$R_{T}\leq \sum_{i\in A:\Delta_{i}\geq b}\bigg(\dfrac{2^{1+4\rho_{a}}\rho_{a}^{2\rho_{a}}T^{1-\rho_{a}}}{\Delta_{i}^{4\rho_{a}-1}}\bigg) + \bigg(\Delta_{i}+\dfrac{32\rho_{a}\log{(T\dfrac{\Delta_{i}^{4}}{16\rho_{a}^{2}})}}{\Delta_{i}}\bigg)  +  \bigg(\dfrac{T^{1-\rho_{a}}2^{2\rho_{a}+\frac{3}{2}}}{\Delta_{i}^{4\rho_{a} -1}} \bigg) \bigg \rbrace+$\newline$\sum_{i\in A:0\leq\Delta_{i}\leq b}\bigg(\dfrac{T^{1-\rho_{a}}2^{2\rho_{a}+\frac{3}{2}}}{b^{4\rho_{a} -1}} \bigg) + max_{i:\Delta\leq b}\Delta_{i}T$. 
\newline
Putting $\rho_{a}=1$ gives us directly the result of proposition $3$.
\end{remark}

\section{Appendix C}

\begin{theorem}
Considering both the arm elimination and cluster elimination condition, the total regret till $T$ is upper bounded by $R_{T}\leq \sum_{i\in A:\Delta\geq b} \bigg\lbrace 2\Delta_{i}+ \bigg(\dfrac{44}{\Delta_{i}^{3}}\bigg) + \bigg(\dfrac{2^{1+4\rho_{s}}\rho_{s}^{2\rho_{s}}T^{1-\rho_{s}}}{\Delta_{i}^{4\rho_{s}-1}}\bigg) + \bigg(\dfrac{32\log{(T\dfrac{\Delta_{i}^{4}}{16})}}{\Delta_{i}}\bigg) + \bigg(\dfrac{32(K+p)\rho_{s}\log{(T\dfrac{\Delta_{i}^{4}}{16\rho_{s}^{2}})}}{p\Delta_{i}}\bigg)\bigg\rbrace + \sum_{i\in A:0\leq\Delta_{i}\leq b}\bigg\lbrace \bigg(\dfrac{12}{b^{3}} \bigg) + \bigg(\dfrac{T^{1-\rho_{s}}2^{2\rho_{s}+\frac{3}{2}}}{\Delta_{i}^{3}} \bigg)+\bigg(\dfrac{T^{1-\rho_{s}}2^{2\rho_{s}+\frac{3}{2}}}{b^{4\rho_{s} -1}} \bigg) \bigg\rbrace + max_{i:\Delta\leq b}\Delta_{i}T $, where $\rho_{s}\in (0,1]$, $p$ is the number of clusters and $T$ is the horizon.
\end{theorem}

\begin{proof}
Combining both the cases of Proposition $5$ and Proposition $6$ we can see that a sub-optimal arm $a_{i}$ can only be eliminated given that either $m_{i}$ or $g_{i}$ happens. In Proposition $5$ we consider only arm elimination and in Proposition $6$ we consider only cluster elimination. One vital point we point out is that, $\epsilon_{m}$(in proposition $5$) = $\epsilon_{g}$(in proposition $6$. Also we cluster the arms based on $\epsilon_{m}$.
\subsection{Case a:} 
So, we take the summation of the two events mentioned in Proposition $5$(case a) and Proposition $6$(case a)
which gives us an upper bound on the regret given that the optimal arm $a^{*}$ is still surviving, 
\newline
$\leq \sum_{i\in A}\bigg\lbrace\bigg(\dfrac{32}{\Delta_{i}^{3}}\bigg) + \bigg(\dfrac{2^{1+4\rho_{s}}\rho_{s}^{2\rho_{s}}T^{1-\rho_{s}}}{\Delta_{i}^{4\rho_{s}-1}}\bigg)\bigg\rbrace$ 
\newline
before a sub-optimal arm is eliminated by arm elimination or cluster elimination condition.
\subsection{Case b1:} 
Again, combining Proposition $5$(case b1) and Proposition $6$(case b1), we can show that till an arm or a cluster is eliminated, the maximum regret suffered due to pulling of a sub-optimal arm(or a sub-optimal cluster) is no less than,
\newline
$\sum_{i\in A}\bigg\lbrace\bigg(\Delta_{i}+\dfrac{32\log{(T\dfrac{\Delta_{i}^{4}}{16})}}{\Delta_{i}}\bigg) + \bigg(\Delta_{i}+\dfrac{32(K+p)\rho_{s}\log{(T\dfrac{\Delta_{i}^{4}}{16\rho_{s}^{2}})}}{p\Delta_{i}}\bigg)\bigg\rbrace $
\newline
\subsection{Case b2:} 
Lastly we have to take into consideration the error bound, that the optimal arm $a^{*}$ or the optimal cluster(that is the cluster containing the arm $a^{*}$) gets eliminated. Combining Proposition $5$(case b2) and Proposition $6$(case b2), we can show,
\newline
$\leq\sum_{i\in A^{'}}\bigg(\dfrac{12}{\Delta_{i}^{3}} \bigg)+\sum_{i\in A^{''}\setminus A^{'}}\bigg(\dfrac{12}{b^{3}} \bigg) + \sum_{i\in A^{'}}\bigg(\dfrac{T^{1-\rho_{s}}2^{2\rho_{s}+\frac{3}{2}}}{\Delta_{i}^{3}} \bigg)+\sum_{i\in A^{''}\setminus A^{'}}\bigg(\dfrac{T^{1-\rho_{s}}2^{2\rho_{s}+\frac{3}{2}}}{b^{4\rho_{s} -1}} \bigg)$
\newline
%Summing over all arms in $A$,
%\newline
%$\leq \sum_{i\in A} \bigg\lbrace\bigg(\Delta_{i}+\dfrac{32\log{(T\dfrac{\Delta_{i}^{4}}{16})}}{\Delta_{i}}\bigg) + \bigg(\Delta_{i}+\dfrac{512\rho_{s}\log{(T\dfrac{\Delta_{i}^{4}}{16\rho_{s}^{2}})}}{\Delta_{i}}\bigg)\bigg\rbrace $
%\newline
Hence, the total regret by combining \textbf{case a}, \textbf{case b1} and \textbf{case b2} is given by,
\newline
$R_{T}\leq \sum_{i\in A:\Delta\geq b} \bigg\lbrace \bigg(\dfrac{32}{\Delta_{i}^{3}}\bigg) + \bigg(\dfrac{2^{1+4\rho_{s}}\rho_{s}^{2\rho_{s}}T^{1-\rho_{s}}}{\Delta_{i}^{4\rho_{s}-1}}\bigg) + \bigg(\Delta_{i}+\dfrac{32\log{(T\dfrac{\Delta_{i}^{4}}{16})}}{\Delta_{i}}\bigg) + \bigg(\Delta_{i}+\dfrac{32(K+p)\rho_{s}\log{(T\dfrac{\Delta_{i}^{4}}{16\rho_{s}^{2}})}}{p\Delta_{i}}\bigg)\bigg\rbrace + \sum_{i\in A:0\leq\Delta_{i}\leq b}\bigg\lbrace \bigg(\dfrac{12}{\Delta_{i}^{3}} \bigg) + \bigg(\dfrac{12}{b^{3}} \bigg) + \bigg(\dfrac{T^{1-\rho_{s}}2^{2\rho_{s}+\frac{3}{2}}}{\Delta_{i}^{3}} \bigg)+\bigg(\dfrac{T^{1-\rho_{s}}2^{2\rho_{s}+\frac{3}{2}}}{b^{4\rho_{s} -1}} \bigg) \bigg\rbrace + max_{i:\Delta\leq b}\Delta_{i}T $
\newline
$R_{T}= \sum_{i\in A:\Delta\geq b} \bigg\lbrace 2\Delta_{i}+ \bigg(\dfrac{44}{\Delta_{i}^{3}}\bigg) + \bigg(\dfrac{2^{1+4\rho_{s}}\rho_{s}^{2\rho_{s}}T^{1-\rho_{s}}}{\Delta_{i}^{4\rho_{s}-1}}\bigg) + \bigg(\dfrac{32\log{(T\dfrac{\Delta_{i}^{4}}{16})}}{\Delta_{i}}\bigg) + \bigg(\dfrac{32(K+p)\rho_{s}\log{(T\dfrac{\Delta_{i}^{4}}{16\rho_{s}^{2}})}}{p\Delta_{i}}\bigg)\bigg\rbrace + \sum_{i\in A:0\leq\Delta_{i}\leq b}\bigg\lbrace \bigg(\dfrac{12}{b^{3}} \bigg) + \bigg(\dfrac{T^{1-\rho_{s}}2^{2\rho_{s}+\frac{3}{2}}}{\Delta_{i}^{3}} \bigg)+\bigg(\dfrac{T^{1-\rho_{s}}2^{2\rho_{s}+\frac{3}{2}}}{b^{4\rho_{s} -1}} \bigg) \bigg\rbrace + max_{i:\Delta\leq b}\Delta_{i}T $
%\newline
%Considering the case that all the arms are pulled once at the start,
%\newline
%$R_{T}\leq \sum_{i\in A:\Delta\geq b} \bigg\lbrace 3\Delta_{i}+ \bigg(\dfrac{44}{\Delta_{i}^{3}}\bigg) + \bigg(\dfrac{2^{1+4\rho_{s}}\rho_{s}^{2\rho_{s}}T^{1-\rho_{s}}}{\Delta_{i}^{4\rho_{s}-1}}\bigg) + \bigg(\dfrac{32\log{(T\dfrac{\Delta_{i}^{4}}{16})}}{\Delta_{i}}\bigg) + \bigg(\dfrac{32(K+p)\rho_{s}\log{(T\dfrac{\Delta_{i}^{4}}{16\rho_{s}^{2}})}}{p\Delta_{i}}\bigg)\bigg\rbrace + \sum_{i\in A:0\leq\Delta_{i}\leq b}\bigg\lbrace \bigg(\dfrac{12}{b^{3}} \bigg) + \bigg(\dfrac{T^{1-\rho_{s}}2^{2\rho_{s}+\frac{3}{2}}}{\Delta_{i}^{3}} \bigg)+\bigg(\dfrac{T^{1-\rho_{s}}2^{2\rho_{s}+\frac{3}{2}}}{b^{4\rho_{s} -1}} \bigg) \bigg\rbrace + max_{i:\Delta\leq b}\Delta_{i}T $
%\newline
%$R_{T}\leq \sum_{i\in A} \bigg\lbrace 2\Delta_{i}+\bigg(\dfrac{32}{\Delta_{i}^{3}}\bigg) + \bigg((2)^{8-4\rho_{s}}(\Delta_{i})^{1-4\rho_{s}}\rho_{s}^{2\rho_{s}}(T)^{1-\rho_{s}}\bigg) + \bigg(\dfrac{32\log{(T\dfrac{\Delta_{i}^{4}}{16})}}{\Delta_{i}}\bigg) + \bigg(\dfrac{512\rho_{s}\log{(T\dfrac{\Delta_{i}^{4}}{16\rho_{s}^{2}})}}{\Delta_{i}}\bigg)\bigg\rbrace $
\end{proof}

\begin{remark}
Considering the case when $\rho_{a}\in (0,1]$, the total regret bound can be rewritten as $R_{T}\leq \sum_{i\in A:\Delta\geq b} \bigg\lbrace \bigg(\bigg(\dfrac{2^{1+4\rho_{a}}\rho_{a}^{2\rho_{a}}T^{1-\rho_{a}}}{\Delta_{i}^{4\rho_{a}-1}}\bigg) + \bigg(\dfrac{2^{1+4\rho_{s}}\rho_{s}^{2\rho_{s}}T^{1-\rho_{s}}}{\Delta_{i}^{4\rho_{s}-1}}\bigg) + \bigg(\Delta_{i}+\dfrac{32\rho_{a}\log{(T\dfrac{\Delta_{i}^{4}}{16\rho_{a}^{2}})}}{\Delta_{i}}\bigg) +$ \newline $ + \bigg(\Delta_{i}+\dfrac{32(K+p)\rho_{s}\log{(T\dfrac{\Delta_{i}^{4}}{16\rho_{s}^{2}})}}{p\Delta_{i}}\bigg)\bigg\rbrace + \sum_{i\in A:0\leq\Delta_{i}\leq b}\bigg\lbrace \bigg(\dfrac{12}{b^{3}} \bigg) + \bigg(\dfrac{T^{1-\rho_{a}}2^{2\rho_{a}+\frac{3}{2}}}{\Delta_{i}^{3}} \bigg)+\bigg(\dfrac{T^{1-\rho_{a}}2^{2\rho_{a}+\frac{3}{2}}}{b^{4\rho_{a} -1}} \bigg) + \bigg(\dfrac{12}{b^{3}} \bigg) + \bigg(\dfrac{T^{1-\rho_{s}}2^{2\rho_{s}+\frac{3}{2}}}{\Delta_{i}^{3}} \bigg)+\bigg(\dfrac{T^{1-\rho_{s}}2^{2\rho_{s}+\frac{3}{2}}}{b^{4\rho_{s} -1}} \bigg) \bigg\rbrace + max_{i:\Delta\leq b}\Delta_{i}T   =    \sum_{i\in A:\Delta\geq b} \bigg\lbrace \bigg(\bigg(\dfrac{2^{1+4\rho_{a}}\rho_{a}^{2\rho_{a}}T^{1-\rho_{a}}}{\Delta_{i}^{4\rho_{a}-1}}\bigg) + \bigg(\dfrac{2^{1+4\rho_{s}}\rho_{s}^{2\rho_{s}}T^{1-\rho_{s}}}{\Delta_{i}^{4\rho_{s}-1}}\bigg) + \bigg(\Delta_{i}+\dfrac{32\rho_{a}\log{(T\dfrac{\Delta_{i}^{4}}{16\rho_{a}^{2}})}}{\Delta_{i}}\bigg) +$ \newline $ + \bigg(\Delta_{i}+\dfrac{32(K+p)\rho_{s}\log{(T\dfrac{\Delta_{i}^{4}}{16\rho_{s}^{2}})}}{p\Delta_{i}}\bigg)\bigg\rbrace + \sum_{i\in A:0\leq\Delta_{i}\leq b}\bigg\lbrace \bigg(\dfrac{24}{b^{3}} \bigg) + \bigg(\dfrac{T^{1-\rho_{a}}2^{2\rho_{a}+\frac{3}{2}}}{\Delta_{i}^{3}} \bigg)+\bigg(\dfrac{T^{1-\rho_{a}}2^{2\rho_{a}+\frac{3}{2}}}{b^{4\rho_{a} -1}} \bigg) + \bigg(\dfrac{T^{1-\rho_{s}}2^{2\rho_{s}+\frac{3}{2}}}{\Delta_{i}^{3}} \bigg)+\bigg(\dfrac{T^{1-\rho_{s}}2^{2\rho_{s}+\frac{3}{2}}}{b^{4\rho_{s} -1}} \bigg) \bigg\rbrace + max_{i:\Delta\leq b}\Delta_{i}T $ which follows directly from Proposition 1,2 and Theorem 1.

	So we see that the most significant term in the regret is the $\bigg(K\dfrac{32(K+p)\rho_{s}\log{(T\dfrac{\Delta_{i}^{4}}{16\rho_{s}^{2}})}}{p\Delta_{i}}\bigg)$. Also, its evident from the result that different values of $\rho_{a}$ and $\rho_{s}$ affects the regret bound differently. When $K$ is large and $p$ is small it is advantageous to run $\rho_{a}< \rho_{s}$ for any round $m$. This will aggressively eliminate arms within cluster while cluster elimination will be more conservative since each cluster will contain a large number of arms it is a good idea to eliminate clusters in the later rounds when sufficient exploration has been done. In the algorithm $\rho_{a}$ and $\rho_{s}$ are decreased in a graded manner so that the risk is low in the initial rounds when sufficient exploration has not been done.
	
\end{remark}

\section{Appendix D}

In this section we discuss about the various exploration regulatory function that we can use to control exploration. A similar topic has already been handled in \cite{liu2016modification} where they have introduced mainly three types of regulatory factors to the term $\bigg\lbrace\hat{r}_{i}+\sqrt{\dfrac{d_{i}\log T\tilde{\Delta}_{m}^{2}}{2n_{m}}}\bigg\rbrace$ which is used for selecting an arbitrary arm $a_{i}$ in the $t$-th timestep which maximizes the above term. This regulatory term $d_{i}$ can be of the form as $\dfrac{T}{t_{i}}$, $\dfrac{\sqrt{T}}{t_{i}}$ and $\dfrac{\log T}{t_{i}}$, where $t_{i}$ is the number of times an arm $a_{i}$ is sampled. One has to choose a regulatory factor based on how fast the algorithm should taper its exploration in the later rounds since with time, $t_{i}$ increases and only the numerator decides how fast the exploration must decrease.

	We also deploy a similar exploration regulatory factor called $\psi(m)=\dfrac{\sqrt{\log T}}{m+1}$ or $\dfrac{\log T}{m+1}$, where $m$ is the round. Depending on how fast the exploration must reduce we might choose any one of them. So $\psi(m)$ is dependent on round as opposed to each timestep or the number of times an individual arm is sampled as done in \cite{liu2016modification}. Also as $m$ increases, $\psi(m)$ decreases and we have a tapered exploration. We also point out that now our number of pulls becomes $n_{m}=\bigg\lceil\dfrac{2\log{(\psi(m)T\epsilon_{m}^{2})}}{\epsilon_{m}}\bigg\rceil$ and both the arm elimination and cluster elimination confidence interval becomes $\sqrt{\dfrac{\rho_{a}\log{(\psi(m)T\epsilon_{m}^{2})}}{2 n_{m}}}$ and $\sqrt{\dfrac{\rho_{s} \log{(\psi(m)T\epsilon_{m}^{2})}}{2 n_{m}}}$  respectively. The rest of the theoretical analysis remains unchanged. 
	
	But this $\psi(m)=\dfrac{\log T}{m+1}$ significantly affects two factors, the regret suffered for arm and cluster elimination and the error bound. In the first case, it becomes $ \sum_{i\in A}\bigg\lbrace\bigg(\dfrac{32}{(log T)\Delta_{i}^{3}}\bigg) + \bigg(\dfrac{2^{1+4\rho}\rho^{2\rho}T^{1-\rho}}{(\log T)\Delta_{i}^{4\rho-1}}\bigg)\bigg\rbrace$ from case a) in Theorem 1 and $\sum_{i\in A^{'}}\bigg(\dfrac{12}{(\log T)\Delta_{i}^{3}} \bigg)+\sum_{i\in A^{''}\setminus A^{'}}\bigg(\dfrac{12}{(\log T) b^{3}} \bigg) + \sum_{i\in A^{'}}\bigg(\dfrac{T^{1-\rho}2^{2\rho+\frac{3}{2}}}{(\log T) \Delta_{i}^{3}} \bigg)+\sum_{i\in A^{''}\setminus A^{'}}\bigg(\dfrac{T^{1-\rho}2^{2\rho+\frac{3}{2}}}{(\log T) b^{4\rho -1}} \bigg)$ from case b2) in Theorem 1. So, in both the cases we see that the exploratory regulatory factor actually decreases our risk by decreasing the error bound and at the same time increases the probability of arm or cluster elimination. 
	
	For the main regret contributing term $\bigg(\dfrac{32(K+p)\rho\log{(\psi(m)T\dfrac{\Delta_{i}^{4}}{16\rho^{2}})}}{p\Delta_{i}}\bigg) \leq \bigg(\dfrac{32(K+p)\rho\log{(T\dfrac{\Delta_{i}^{4}}{16\rho^{2}}\log T)}}{p\Delta_{i}}\bigg)$ , the $\log \log T$ term is nearly negligible and only slightly increase the expected regret.


\section{Appendix E}
%\begin{table}
%\caption{Regret Upper Bound of Algorithms}
\label{sample-table}
\begin{center}
\begin{tabular}{ll}
\multicolumn{1}{c}{\bf Algorithm}  &\multicolumn{1}{c}{\bf Cumulative Regret Upper Bound} \\
\hline \\
UCB1         &\hspace*{5em}$\min\bigg\lbrace O(\sqrt{KT\log T}) ,O\bigg(\dfrac{K\log T}{\Delta}\bigg)\bigg\rbrace$ \\
UCB2         &\hspace*{5em}$O\bigg(K\bigg(\dfrac{(1 + \epsilon(\alpha)) log(T)}{2\Delta} + C(\alpha)\bigg)\bigg)$, $0<\alpha<1$ \\
$\epsilon_{n}$-greedy         &\hspace*{5em}$O\bigg(\dfrac{K\Delta\log T}{d^{2}}\bigg)$, $0<d<\Delta$ \\
EXP3             &\hspace*{5em}$O\bigg(S \sqrt{KT \log(KT)}\bigg)$, where $S$ is the hardness of the problem \\
UCB($\delta$)	&\hspace*{5em}$O\bigg(\dfrac{16K}{\Delta}\log\big(\dfrac{2K}{\Delta\delta}\big)\bigg)$ , where $\delta$ is the error probability\\
UCB-Revisited             &\hspace*{5em}$\min\bigg\lbrace O(\sqrt{KT}\dfrac{log(K\log K)}{\sqrt{K}}), O\bigg(\dfrac{K\log (T\Delta^{2})}{\Delta}\bigg)\bigg\rbrace$ \\
MOSS				&\hspace*{5em}$\min\bigg\lbrace O\bigg(\sqrt{KT}\bigg), O\bigg(\dfrac{K\log(T\Delta^{2}/K)}{\Delta}\bigg) \bigg \rbrace$\\
KL-UCB         &\hspace*{5em}$O\bigg(K\bigg(\dfrac{\Delta \log(T)(1 + \epsilon)}{d(r_{i}, r^{*} )} + \log(\log(T)) + \dfrac{(\epsilon)}{T^{\beta(\epsilon)}}\bigg)\bigg)$, where $\epsilon > 0$ and $d(r_{i}, r^{*})>2\Delta_{i}^{2}$\\
UCB-Clustered             &\hspace*{5em}$\min\bigg\lbrace\dfrac{\sqrt{T}\log{(\dfrac{K^{4}(\log K)^{2}}{T})}}{\sqrt{K\log K}},O\bigg(\dfrac{(K+p)K\log (T\Delta^{4})}{p\Delta}\bigg)\bigg\rbrace$, \\& \hspace*{5em} where $p$ is the number of clusters \\
%Clustered-UCB         &$O\bigg(K\bigg (\max{\bigg\lbrace \bigg(\dfrac{1}{\psi(m)(\Delta)^{\frac{3}{5}}}\bigg) ,\bigg(\dfrac{\Delta}{(\psi(m)\Delta^{2})(\psi(m)T\Delta^{2})^{2K^{2}\Delta/5}}\bigg)\bigg\rbrace}$\\&$ + \bigg(\Delta +\dfrac{\log{(\psi(m)T\Delta^{\frac{8}{5}})}}{\Delta^{\frac{3}{5}}}\bigg)\bigg)\bigg)$, let $\psi(m)=\dfrac{c}{m},c>0$ \\
\end{tabular}
\end{center}
%\end{table}






\end{document}
