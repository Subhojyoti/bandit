
%\newpage
\appendix
The Appendix is organized as follows. First we prove some technical lemmas in Appendix \ref{App:Lemma1} and Appendix \ref{App:Lemma2}. Next we prove the main theorem in Appendix \ref{sec:proofTheorem}. In Appendix \ref{App:Proof:Corollary:1} we prove Corollary \ref{Result:Corollary:1}. In Appendix \ref{App:A} we prove Proposition \ref{proofTheorem:Prop:1}. 

% and in Appendix \ref{App:Proof:Corollary:2} we prove Corollary \ref{Result:Corollary:2}. Appendix \ref{App:E} deals with the idea of why we do clustering. The simple regret bound of EClusUCB and its associated Corollary is proved in \ref{App:SR_EClusUCB}. Algorithm \ref{alg:aclusucb}, Adaptive Clustered UCB is shown in Appendix \ref{App:AClusUCB}. More experiments are shown in Appendix \ref{App:MoreExp}.

%\section{Regret Bound Table}
%\label{App:Table}
%\begin{table}[!h]
%\caption{Gap-dependent regret bounds for different bandit algorithms}
%\label{tab:regret-bds}
%\begin{center}
%\begin{tabular}{|c|c|}
%\toprule
%Algorithm  & Upper bound \\
%\midrule
%UCB1         &$O\left(\frac{K\log T}{\Delta}\right)$ \\\midrule
%UCB-Improved &$O\left(\frac{K\log (T\Delta^{2})}{\Delta}\right)$ \\\midrule
%MOSS	     &$O\left(\frac{K^{2}\log\left(T\Delta^{2}/K\right)}{\Delta}\right)$\\\midrule
%EClusUCB      &$O\left(\frac{K\log\left(\frac{T\Delta^{2}}{\sqrt{\log (K)}}\right)}{\Delta}\right)$\\\bottomrule
%\end{tabular}
%\end{center}
%\vspace*{-2em}
%\end{table}

\section{Proof of Lemma 1}
\label{App:Lemma1}

\begin{lemma}
\label{proofTheorem:Lemma:1}
If $T\geq K^{2.4}$, $\psi=\dfrac{T}{ K^2}$, $\rho_a =\dfrac{1}{2}$ and $m\leq \dfrac{1}{2} \log_2\left(\dfrac{T}{e}\right) $, then,
\begin{align*}
\dfrac{\rho_a m \log(2)}{\log(\psi T) - 2m\log( 2)} \leq \frac{3}{2}
\end{align*}
\end{lemma}

\begin{proof}
The proof is based on contradiction. Suppose
\begin{eqnarray*}
\dfrac{\rho_a m \log(2)}{\log(\psi T) - 2m\log( 2)} > \frac{3}{2}.
\end{eqnarray*}
Then, with $\psi=\dfrac{T}{ K^2}$ and $\rho_a =\dfrac{1}{2}$, we obtain
\begin{align*}
&\dfrac{\rho_a m \log(2)}{\log(\frac{T^2}{K^2}) - 2m\log( 2)} > \frac{3}{2}\\
&\Rightarrow 2\rho_a m \log(2) > 6\log(\frac{T}{K}) - 6m\log( 2)
\end{align*}
This can be further reduced to,
\begin{eqnarray*}
6\log(K) 
&>& 6\log(T) - 7m\log(2) \\
&\overset{(a)}{\ge}& 6\log(T) - \frac{7}{2} \log_2\left(\frac{T}{e}\right) \log(2) \\
&=& 2.5\log(T) + 3.5 \log_2(e)\log(2)  \\
&\overset{(b)}{=}& 2.5\log(T) +3.5
\end{eqnarray*}
where $(a)$ is obtained using $m\leq \dfrac{1}{2} \log_2\left(\dfrac{T}{e}\right)$, while $(b)$ follows from the identity $\log_2(e)\log(2) =1$. Finally, for $T\ge K^{2.4}$ we obtain, $6\log(K)>6\log(K)+3.5$, which is a contradiction. Hence, for $T\geq K^{2.4}$, $\psi=\dfrac{T}{ K^2}$, $\rho=\dfrac{1}{2}$ and $m \leq \dfrac{1}{2} \log_2\left(\dfrac{T}{e}\right) $ we have, 
\begin{align*}
\dfrac{\rho m \log(2)}{\log(\psi T) - 2m\log( 2)} \leq \frac{3}{2}
\end{align*}
\end{proof}

\section{Proof of Lemma 2}
\label{App:Lemma2}
\begin{lemma}
\label{proofTheorem:Lemma:2}
If $T\geq K^{2.4}$, $\psi=\dfrac{T}{ K^2}$, $\rho_a =\dfrac{1}{2}$, $m_i = min\lbrace m|\sqrt{2\epsilon_{m} } < \dfrac{\Delta_i}{4} \rbrace $ and $c_{m_i} =\sqrt{\frac{\rho_{a}\log (\psi T\epsilon_{m_{i}})}{2 n_{m_i}}}$, then, $c_{m_i} < \dfrac{\Delta_i}{4}$.
\end{lemma}

\begin{proof}
In the $m_i$-th round $c_{m_i} =\sqrt{\frac{\rho_{a}\log (\psi T\epsilon_{m_{i}})}{2 n_{m_i}}}$. Substituting the value of $n_{m_i}=\dfrac{\log{(\psi T\epsilon_{m_{i}}^{2})}}{2\epsilon_{m_{i}}}$ in $c_{m_i}$ we get,
\begin{align*}
	c_{m_i} &\leq \sqrt{\dfrac{\rho_a \epsilon_{m_{i}}\log (\psi T\epsilon_{m_{i}})}{\log(\psi T\epsilon_{m_{i}}^{2})}} \leq \sqrt{\dfrac{\rho_a \epsilon_{m_{i}}\log (\frac{\psi T\epsilon_{m_{i}}^{2}}{\epsilon_{m_{i}}})}{\log(\psi T\epsilon_{m_{i}}^{2})}} \\
	%%%%%%%%%%%%%%%%%%%%%%%%%%
	& = \sqrt{\dfrac{\rho_a\epsilon_{m_{i}}\log (\psi T\epsilon_{m_{i}}^{2}) - \rho_a\epsilon_{m_{i}}\log (\epsilon_{m_{i}})}{\log(\psi T\epsilon_{m_{i}}^{2})}} 
	\leq  \sqrt{\rho_a\epsilon_{m_{i}} - \dfrac{\rho_a\epsilon_{m_i}\log(\frac{1}{2^{m_i}})}{\log(\psi T \frac{1}{2^{2m_i}})}} \\
	%%%%%%%%%%%%%%%%%%%%%%%%%%
	&\leq \sqrt{\rho_a\epsilon_{m_{i}} + \dfrac{\rho_a\epsilon_{m_i}\log(2^{m_i})}{\log(\psi T) - \log( 2^{2m_i})}}  \leq \sqrt{\rho_a\epsilon_{m_{i}} + \dfrac{\rho_a\epsilon_{m_i}m_i \log(2)}{\log(\psi T) - 2m_i\log( 2)}} \\ 
	%%%%%%%%%%%%%%%%%%%%%%%%%%
	 & \overset{(a)}{\leq} \sqrt{\rho_a\epsilon_{m_{i}} + \frac{3}{2}\epsilon_{m_i}} 
	  < \sqrt{2\epsilon_{m_i}} 
	  < \dfrac{\Delta_{i}}{4} 
	\end{align*}
In the above simplification, $(a)$ is obtained using Lemma~\ref{proofTheorem:Lemma:1}. 
\end{proof}



\section{Proof of Theorem 1}
\label{sec:proofTheorem}
%\label{sec:proofTheorem}

%We sketch the proof for Theorem \ref{Result:Theorem:1} here. 
%The proof involves the following steps:\\
%\textbf{\textit{Step 1:}}
%We analyze ClusUCB-AE, i.e., the variant of ClusUCB that uses arm elimination condition only. In other words, we bound the probability of sub-optimal arm elimination, which in turn bounds the expected regret of ClusUCB-AE (see Proposition \ref{proofSketch:Prop:1} below). 
%
%\textbf{\textit{Step 2:}}
%We analyze ClusUCB-CE, i.e., the variant of ClusUCB that uses cluster elimination condition only and pulls the best arm within the last leftover cluster.
%Proposition \ref{proofSketch:Prop:2} presents the expected regret for ClusUCB-CE (see Proposition \ref{proofSketch:Prop:2} below). 
%
%\textbf{\textit{Step 3:}}
%Finally we combine the individual bounds in the steps above to get the regret upper bound in Theorem \ref{Result:Theorem:1}.  

%\pagebreak
%\subsection*{New proof sketch}
%
%Suppose we consider the following cases and analyze the contribution to expected regret for each of them:
%
%\textbf{Case a:} \textit{Some sub-optimal arm $a_{i}$ is not eliminated in round $\max(m_{i},g_{s_{k}})$ or before, with the optimal arm $a^{*}\in C_{\max(m_{i},g_{s_{k}})}$.}\\
%Note that in this case, we assume that optimal arm $a^*$ is in the set of cluster-heads in round $\max(m_{i},g_{s_{k}})$.
%	
%\textbf{Case a1:} \textit{In round $\max(m_{i},g_{s_{k}})$, $a_{i} \in s^{*}$.} \\
%For this case, the analysis is as in case a1 before. 
%
%\textbf{Case a2:} \textit{In round $\max(m_{i},g_{s_{k}})$, $a_{i} \in s_k$ for some $s_k \ne s^{*}$.} \\
%Since $a^*$ is in the set of cluster-heads $ C_{\max(m_{i},g_{s_{k}})}$, the analysis for this case is the same as that for case a3 earlier. In other words, $a_i$ belongs to a sub-optimal cluster $s_k$ and we bound the probability that the cluster $s_k$ does not get eliminated via the cluster elimination condition.
%
%\textbf{Case b:} \textit{For each arm $a_i$, either $a_{i}$ is eliminated in round $\max (m_{i},g_{s_{k}})$ or before or there is no optimal arm $a^{*}$ in $C_{\max(m_{i},g_{s_{k}})}$.} \\
%The case caption above is quite similar to that earlier, except that we have $C_{\max(m_{i},g_{s_{k}})}$ instead of $B_{\max(m_{i},g_{s_{k}})}$.
%
%\textbf{Case b1:} \textit{$a^{*}\in C_{\max(m_{i},g_{s_{k}})}$ and each $a_{i}\in A^{'}$ is  eliminated on or before $\max (m_{i},g_{s_{k}})$.} 
%
%\textbf{Case b2:} \textit{Optimal arm $a^{*}$ is eliminated by a sub-optimal arm.}\\
%This case should be like earlier case b2, though I have not checked the details of b22,b23 cases.
%
%\textbf{Case b3:} \textit{Optimal arm $a^{*}$ is not in $C_{\max(m_{i},g_{s_{k}})}$.}\\
%This is a new case and we should attempt to analyze the regret contribution for this.
%
%\hrulefill

\begin{proof}

%$\Delta_{i}^{'}=r_{a_{\max_{s_{k}}}} - r_{i}$ such that $a_{i}\in s_{k}$,
% m_{i}^{'}=\min{\lbrace m|\sqrt{\rho_{a}\epsilon_{m}} < \frac{\Delta_{i}^{'}}{2} \rbrace}
Let $A^{'}=\lbrace i \in A,\Delta_{i}> b\rbrace$,  $A^{''}=\lbrace i \in A,0 < \Delta_{i} \leq b\rbrace$, $A^{'}_{s_{k}}=\lbrace i \in A_{s_{k}},\Delta_{i}> b\rbrace$ and $A^{''}_{s_{k}}=\lbrace i \in A_{s_{k}},0 < \Delta_{i} \leq b\rbrace$. $C_{g}$ is the cluster set containing max payoff arm from each cluster in $g$-th round. The arm having the highest payoff in a cluster $s_{k}$ is denote by $a_{\max_{s_{k}}}$. Let for each sub-optimal arm $a_{i}\in A$, $m_{i}=\min{\lbrace m|\sqrt{\rho_{a}\epsilon_{m}} < \frac{\Delta_{i}}{2} \rbrace}$ and let for each cluster $s_{k}\in S$, $g_{s_{k}}=\min{\lbrace g|\sqrt{\rho_{s}\epsilon_{g}} < \frac{\Delta_{a_{\max_{s_{k}}}}}{2} \rbrace}$. 
%and $A^{'''}_{s_{k}}=\lbrace i \in A_{s_{k}}, b < \Delta_{a_{\max_{s_{k}}}} < \Delta_{i}^{'}  \rbrace$
%\todos[inline]{define $g_{s_k}$ for each cluster $s_k \in S$}
%\todos[inline]{$a_{max_{s_{k}}}$ is never defined. In the notation of Sec 2 $r_{max_{s_{k}}}$ is defined as the best arm within the cluster $s_k$}
%\todos[inline]{Change max to the operator $\max$ everywhere}
%\todos[inline]{critical error: $m_i$ defintion should be with $\sqrt{\rho_{a}\epsilon_{m}}< \frac{\Delta_{i}}{2}$. Same for $g$} 

%be the first round when $\sqrt{\rho_{a}\epsilon_{m}}\leq \dfrac{\Delta_{i}}{2}$ and for each sub-optimal cluster arm $a_{max_{s_{k}}}\in C_{g_{s_{k}}}$,
%So, $g_{s_{k}}$ be the first round when $\sqrt{\rho_{s}\epsilon_{g_{s_{k}}}}\leq \dfrac{\Delta_{a_{max_{s_{k}}}}}{2}$ where $a_{max_{s_{k}}}\in C_{g_{s_{k}}}$ is the maximum payoff arm in cluster $s_{k}$ and then $s_{k}$ gets eliminated
%The theoretical analysis remains same as we have always bounded the values of $\rho_{a}\in (0,1]$(see Appendix \ref{App:E}).
%Also we cluster the arms based on $\epsilon_{m}$.
% One vital point we point out is that, $\epsilon_{m}$(in proposition $3$) = $\epsilon_{g}$(in proposition $4$).
The analysis proceeds by considering the contribution to the regret in each of the following cases:

\textbf{Case a:} \textit{Some sub-optimal arm $a_{i}$ is not eliminated in round $\max(m_{i},g_{s_{k}})$ or before, with the optimal arm $a^{*}\in C_{\max(m_{i},g_{s_{k}})}$.}

%\todos[inline]{The stmt ``In this case, we are looking at event of the maximum round till which atleast one of $m_{i}$ or $g_{s_{k}}$ did not happen.'' is unnecessary give the case caption above} 
%	In this case, we are looking at event of the maximum round till which atleast one of $m_{i}$ or $g_{s_{k}}$ did not happen. 
We consider an arbitrary sub-optimal arm $a_{i}$ and analyze the contribution to the regret when $a_i$ is not eliminated in the following exhaustive sub-cases:\\
%\todos[inline]{Get rid of enumerate to save space. You could just have the case labels, say case a1 and such}
\textbf{Case a1:} \textit{In round $\max(m_{i},g_{s_{k}})$, $a_{i} \in s^{*}$.}

Similar to case (a) of \cite{auer2010ucb}, observe that when the following two conditions hold, arm $a_i$ gets eliminated:
\begin{align}
\hat{r}_{i}  \le r_{i} + c_{m_{i}} \text{ and } 
 \hat{r}^{*}\geq  r^{*} - c_{m_{i}}, \label{eq:armelim-casea}
\end{align}
where  $c_{m_{i}}=\sqrt{\frac{\rho_{a}\log (\psi T\epsilon_{m_{i}}^{2})}{2 n_{m_{i}}}}$.
The arm $a_i$ gets eliminated because 
  \begin{align*}
\hat{r}_{i} + c_{m_{i}}&\leq r_{i} + 2c_{m_{i}} < r_{i} + \Delta_{i} - 2c_{m_{i}} = r^{*} -2c_{m_{i}} \\
 &\leq \hat{r}^{*} - c_{m_{i}}.
  \end{align*}
%\todos[inline]{The stmt ``bound the probability of the event $\hat{r}_{i}+c_{m_{i}}\leq \hat{r}^{*}-c_{m_{i}}$'' is wrong. We bound the complementary event using Hoeffding} 
%\todos[inline]{The stmt ``$m_{i}$ does not happen'' makes no sense given that we are in round $\max(m_i,g_i)$}
%\todos[inline]{What is $c_m$?}
  %Now, $c_{m_{i}}=\sqrt{\frac{\rho_{a}\log (\psi T\epsilon_{m_{i}}^{2})}{2 n_{m_{i}}}}$.
In the above, we have used the fact that \\ $c_{m_{i}} = \sqrt{\rho_{a}\epsilon_{m_{i}+1}} < \frac{\Delta_{i}}{4}$, since $n_{m_{i}}=\frac{2\log{(\psi T\epsilon_{m_{i}}^{2})}}{\epsilon_{m_{i}}}$ and $\rho_{a}\in (0,1]$.

From the foregoing, we have to bound the events complementary to that in \eqref{eq:armelim-casea} for an arm $a_i$ to not get eliminated. This is done as follows:
%\todos[inline]{In the following, $\exists a_i$ is spurious given that we are talking about arm $a_i$ all through in this case} 
%  %Again, $\exists a_{i} \in A_{s^{*}}^{'}$ such that, 
%\todos[inline]{$r_{i} + 2c_{m_{i}} 
% < r_{i} + \Delta_{i} - 2c_{m_{i}}$}
%\todos[inline]{The final inequality above does not hold unless you assume $\hat{r}^{*}\geq r^{*} - c_{m_{i}}$ and this is never mentioned?}
  %Hence, we get that when $\sqrt{\rho_{a}\epsilon_{m_{i}}}<\frac{\Delta_{i}}{2}$, $a_{i}$ gets eliminated. Applying Chernoff-Hoeffding bound and considering independence of events,
  \begin{align*}
&\mathbb{P}\left(\hat{r}^{*}\leq r^{*} - c_{m_{i}}\right)\leq \exp(-2c_{m_{i}}^{2}n_{m_{i}})\\
&\leq \exp(-2 * \frac{\rho_{a}\log (\psi T\epsilon_{m_{i}}^{2})}{2 n_{m_{i}}} *n_{m_{i}})\\
&\leq \frac{1}{(\psi T\epsilon_{m_{i}}^{2})^{\rho_{a}}}   
  \end{align*}
Along similar lines, we have 
$\mathbb{P}\left(\hat{r}_{i}\geq r_{i} + c_{m_{i}}\right)\leq \frac{1}{(\psi  T\epsilon_{m_{i}}^{2})^{\rho_{a}}}.$
Thus, the probability that a sub-optimal arm $a_{i}$ is not eliminated in any round on or before $m_{i}$ is bounded above by  $\bigg(\frac{2}{(\psi T\epsilon_{m_{i}}^{2})^{\rho_{a}}}\bigg)$. 
 Summing up over all arms in $A_{s^{*}}^{'}$ in conjunction with a simple bound of $T\Delta_{i}$ for each arm, we obtain
   \begin{align*}
&\sum_{i\in A_{s^{*}}^{'}}\bigg(\dfrac{2T\Delta_{i}}{(\psi T\epsilon_{m_{i}}^{2})^{\rho_{a}}}\bigg)
\leq\sum_{i\in A_{s^{*}}^{'}}\bigg(\frac{2T\Delta_{i}}{(\psi T\dfrac{\Delta_{i}^{4}}{16\rho_{a}^{2}})^{\rho_{a}}}\bigg)\\
%&\leq \sum_{i\in A_{s^{*}}^{'}}\bigg(\frac{2^{1+4\rho_{a}}T^{1-\rho_{a}}\rho_{a}^{2\rho_{a}}\Delta_{i}}{\psi^{\rho_{a}}\Delta_{i}^{4\rho_{a}}}\bigg)\\
& =\sum_{i\in A_{s^{*}}^{'}}\bigg(\frac{C_{1}(\rho_{a})T^{1-\rho_{a}}}{\Delta_{i}^{4\rho_{a}-1}}\bigg) \text{, where } C_1(x) = \frac{2^{1+4x}x^{2x}}{\psi^{x}}
   \end{align*}

%%%%%%%%%%%%%%%%%%%%%%%%%%%%%%%%%%%%%%%%%%%%%%%%%%%%%%%%%%%%%%%%%%%%%%%%%%%%%%%%%%%%%%%%%%%%%%%   
%\textbf{Case a2:} \textit{In round $m_{i}^{'}$, $a_{i} \in s_{k}$ for some $s_k \ne s^*$} % where $r_{\max_{s_{k}}}\leq r^{*}$ 
%
%%\todos[inline]{The description in the text below for this case doesnt make sense to me. The final bound arrived at uses $\Delta_i$, while it doesnt figure here in the argument here at all. }
%%We can show that the probability of $a_{i}$ not getting eliminated cannot be worse than Case $a1$ given that $m_{i}^{'}< g_{s_{k}}$ or else $g_{s_{k}}$ will happen and the cluster $s_{k}$ will get eliminated or $a_{\max_{s_{k}}}$ will eliminate $a^{*}$ which are dealt later. 
%
%Approaching the same way as above we define $\Delta_{i}^{'}=r_{a_{\max_{s_{k}}}} - r_{i}$, for $a_{i}\in s_{k}$, $m_{i}^{'}=\min{\lbrace m|\sqrt{\rho_{a}\epsilon_{m}} < \frac{\Delta_{i}^{'}}{2} \rbrace}$.Then plugging in $\Delta^{'}_{i}$ in Case $a1$ and bounding the complementary events mentioned in \ref{eq:armelim-casea} by using $r_{i}$ and $r_{a_{\max_{s_{k}}}}$, we can show that for an arm $a_{i}\in A_{s_{k}}^{'}$ the maximum probability of not getting eliminated on or before $m_{i}^{'}$ is  $\bigg(\dfrac{2}{(\psi T\epsilon_{m_{i}^{'}}^{2})^{\rho_{a}}}\bigg)$. So bounding trivially over $T\Delta_{i}^{'}$ the regret is bounded by,
%
%\begin{align*}
%& \sum_{i\in A_{s_{k}}^{'}}\frac{C_{1}(\rho_{a})T^{1-\rho_{a}}}{\Delta_{i}^{'^{{4\rho_{a}-1}}}} 
%   \end{align*}
%   %\leq \sum_{i\in A_{s_{k}}^{'''}} \frac{C_{1}(\rho_{a})T^{1-\rho_{a}}}{\Delta_{a_{\max_{s_{k}}}}^{4\rho_{a}-1}}
%   %\leq \sum_{\substack{i\in A_{s_{k}}^{'}: \\ \Delta_{i}^{'}\geq \Delta_{a_{\max_{s_{k}}}} }}\frac{C_{1}(\rho_{a})T^{1-\rho_{a}}}{\Delta_{a_{\max_{s_{k}}}}^{4\rho_{a}-1}}
%   %and considering $ \frac{1}{4} \leq \rho_{a} \leq 1 $
%Summing over all $p-1$ clusters excluding $s^{*}$ the regret is,
%\begin{align*}
%& \sum_{k=1}^{p-1}\sum_{i\in A_{s_{k}}^{'}\setminus A_{s^{*}}^{'}} \frac{C_{1}(\rho_{a})T^{1-\rho_{a}}}{\Delta_{i}^{'^{^{4\rho_{a}-1}}}} \leq \sum_{i\in A^{'}\setminus A^{'}_{s^{*}}}\frac{C_{1}(\rho_{a})T^{1-\rho_{a}}}{\Delta_{i}^{'^{4\rho_{a}-1}}} 
%   \end{align*}
%   
%%   For any round $m_{i}^{'} > g_{s_{k}}$ and $a_{i}\neq a_{max_{s_{k}}}$ means that $\hat{r}_{i} - c_{m_{i}} > \hat{r_{a_{max_{s_{k}}}}} + c_{m_{i}}$ and also $\hat{r_{a_{max_{s_{k}}}}}  - c_{m_{i}} > \hat{r}^{*} + c_{m_{i}}$ which leads to the violation of the condition that 




%%%%%%%%%%%%%%%%%%%%%%%%%%%%%%%%%%%%%%%%%%%%%%%%%%%%%%%%%%%%%%%%%%%%%%%%%%%%%%%%%%%%%%%%%%%%%%%   
\textbf{Case a2:} \textit{In round $\max(m_{i},g_{s_{k}})$, $a_{i} \in s_k$ for some $s_k \ne s^{*}$.}

%\todos[inline]{Fix this case analysis to read as well as case a1. The first part until the Hoeffding bounds can be shorter than case a1, as the analysis to arrive at Hoeffding events follows using parallel arguments.} 
%then in cluster elimination condition, given the choice of confidence interval $c_{g_{s_{k}}}=\sqrt{\frac{\rho_{s} \log (\psi T\epsilon_{g_{s_{k}}}^{2})}{2 n_{g_{s_{k}}}}}$, we want to bound the probability of the event $\hat{r}_{s_{k}}+c_{g_{s_{k}}}\geq \hat{r}^{*}-c_{g_{s_{k}}}$.
%
%
%  Putting the value of $n_{g_{s_{k}}}=\frac{2\log{(\psi T\epsilon_{g_{s_{k}}}^{2})}}{\epsilon_{g_{s_{k}}}}$ in $c_{g_{s_{k}}}$, we get $c_{g_{s_{k}}} =\sqrt{\rho_{s}\epsilon_{g_{s_{k}}+1}} < \frac{\sqrt{\rho_{s}}\Delta_{a_{\max_{s_{k}}}}}{4} < \frac{\Delta_{a_{\max_{s_{k}}}}}{4}$.
%
%  
%  \begin{align*}
%  \hat{r}_{a_{\max_{s_{k}}}} + c_{g_{s_{k}}}&\leq r_{a_{\max_{s_{k}}}} + 2c_{g_{s_{k}}} = r_{a_{\max_{s_{k}}}} + 4c_{g_{k}} - 2c_{g_{s_{k}}}\\
%  &< r_{a_{\max_{s_{k}}}} + \Delta_{a_{\max_{s_{k}}}} - 2c_{g_{s_{k}}} = r^{*} -2c_{g_{s_{k}}}\\
%  &\leq \hat{r}^{*} - c_{g_{s_{k}}} \text{, as } \hat{r}^{*}\geq r^{*} - c_{g_{s_{k}}}
%  \end{align*}
%   
% 	Hence, we get that when $\sqrt{\rho_{s}\epsilon_{g_{s_{k}}}}<\frac{\Delta_{a_{\max_{s_{k}}}}}{2}$, $a_{\max_{s_{k}}}\in C_{g_{s_{k}}}$ gets eliminated leading to elimination of $s_{k}$. Applying Chernoff-Hoeffding bound and considering independence of events,
% 
% 
% \begin{align*}
% \mathbb{P}\bigg\lbrace\hat{r}^{*} &\leq r^{*} - c_{g_{s_{k}}}\bigg\rbrace \leq exp(-2c_{g_{s_{k}}}^{2}n_{g_{s_{k}}})
% \leq \dfrac{1}{(\psi T\epsilon_{g_{k}}^{2})^{\rho_{s}}}
% \end{align*}
%
%Similarly, $\mathbb{P}\bigg\lbrace\hat{r}_{a_{\max_{s_{k}}}}\geq r_{a_{\max_{s_{k}}}} + c_{g_{s_{k}}}\bigg\rbrace\leq \dfrac{1}{(\psi T\epsilon_{g_{s_{k}}}^{2})^{\rho_{s}}}$

Following a parallel argument like in Case $a1$, we have to bound the following two events of arm $a_{\max_{s_{k}}}$ not getting eliminated on or before $g_{s_{k}}$-th round,
\begin{align*}
  \hat{r}_{a_{\max_{s_{k}}}} \geq r_{a_{\max_{s_{k}}}} +c_{g_{s_{k}}} \text{ and } \hat{r}^{*} \leq r^{*} -c_{g_{s_{k}}}
\end{align*} 

We can prove using Chernoff-Hoeffding bounds and considering independence of events that for $c_{g_{s_{k}}}=\sqrt{\frac{\rho_{s} \log (\psi T\epsilon_{g_{s_{k}}}^{2})}{2 n_{g_{s_{k}}}}}$ and  $n_{g_{s_{k}}}=\frac{2\log{(\psi T\epsilon_{g_{s_{k}}}^{2})}}{\epsilon_{g_{s_{k}}}}$ the probability of the above two events is bounded by $\bigg(\dfrac{2}{(\psi  T\epsilon_{g_{s_{k}}}^{2})^{\rho_{s}}}\bigg)$.
%Summing, the two up, the probability that a sub-optimal cluster arm $a_{\max_{s_{k}}}\in C_{g_{s_{k}}}$ is not eliminated
  Now, for any round $g_{s_{k}}$, all the elements of $C_{g_{s_{k}}}$ are the respective maximum payoff arms of their cluster $s_{k}, \forall s_{k}\in S$, and since all the surviving arms are pulled equally in each round and since clusters are fixed so we can bound the maximum probability that a sub-optimal arm $a_{i}\in A^{'}$  and $a_{i}\in s_{k}$ such that $a_{\max_{s_{k}}}\in C_{g_{s_{k}}}$ is not eliminated on or before the $g_{s_{k}}$-th round by the same probability as above. 

%\begin{align*}
%\bigg(\frac{2}{(\psi T\epsilon_{g_{s_{k}}}^{2})^{\rho_{s}}}\bigg)
%\end{align*}
 
%Summing up over all arms in $s_{k}$ and bounding trivially by $T\Delta_{i}$,
%\begin{align*}
%\sum_{i\in A_{s_{k}}}\bigg(\frac{2T\Delta_{i}}{(\psi T\epsilon_{g_{s_{k}}}^{2})^{\rho_{s}}}\bigg)
%\end{align*}

Summing up over all $p$ clusters and bounding trivially by $T\Delta_{i}$,
 \begin{align*}
 &\sum_{k=1}^{p}\sum_{i\in A_{s_{k}}^{'}}\bigg(\frac{2T\Delta_{i}}{(\psi T\frac{\Delta_{i}^{4}}{16\rho_{s}^{2}})^{\rho_{s}}}\bigg) = \sum_{i\in A^{'}}\bigg(\frac{2T\Delta_{i}}{(\psi  T\frac{\Delta_{i}^{4}}{16\rho_{s}^{2}})^{\rho_{s}}}\bigg) \\
 &\leq \sum_{i\in A^{'}}\bigg(\frac{2^{1+4\rho_{s}}\rho_{s}^{2\rho_{s}}T^{1-\rho_{s}}}{\psi^{\rho_{s}}\Delta_{i}^{4\rho_{s}-1}}\bigg) = \sum_{i\in A^{'}}\frac{C_{1}(\rho_{s})T^{1-\rho_{s}}}{\Delta_{i}^{4\rho_{s}-1}}
 \end{align*}
% &= \sum_{i\in A^{'}}\bigg(\frac{C_{1}(\rho_{s})T^{1-\rho_{s}}}{\Delta_{i}^{4\rho_{s}-1}}\bigg) \text{, where } C_1(x) = \frac{2^{1+4x}x^{2x}}{\psi^{x}}
%&\leq \sum_{i\in A}\bigg(\frac{2^{1+4\rho_{s}}T^{1-\rho_{s}}\rho_{s}^{2\rho_{s}}\Delta_{i}}{\psi^{\rho_{s}}\Delta_{i}^{4\rho_{s}}}\bigg)\\

%%%%%%%%%%%%%%%%%%%%%%%%%%%%%%%%%%%%%%%%%%%%%%%%%%%%%%%%%%%%%%%%%%%%%%%%%%%%%%%%%%%%%%%%%%%%%%%   
%\textbf{Case a4:} \textit{In round $g_{s_{k}}$, $a_{i}\in s_{k}, a^{*}\notin C_{g_{s_{k}}}$, but $a_{\max_{s^{*}}}\in C_{g_{s_{k}}}$, where $a_{\max_{s^{*}}}$ satisfies $\hat{r}_{a_{\max_{s^{*}}}}> \hat{r}^{*}$ and $a_{max_{s^{*}}} \in s^*$ } 
%
%In this case for some sub-optimal arm $a_{\max_{s_{k}}}\in C_{g_{s_{k}}}$, we have to bound the events
%	\begin{align*}
%  \hat{r}_{a_{\max_{s_{k}}}} \geq r_{a_{\max_{s_{k}}}} +c_{g_{s_{k}}} \text{ and } \hat{r}_{a_{\max_{s^{*}}}} \leq r_{a_{\max_{s^{*}}}} -c_{g_{s_{k}}}
%\end{align*}
%%	\begin{align*}
%%	&\mathbb{P}\bigg\lbrace\hat{r}_{a_{\max_{s_{k}}}}+c_{g_{s_{k}}}\bigg\rbrace \geq \mathbb{P}\bigg\lbrace\hat{r}_{a_{\max_{s^{*}}}}-c_{g_{s_{k}}}\bigg\rbrace
%%	\end{align*}		 
%	 
%	 
%	 But, this probability can be no worse than Case $a3$ since $r_{\max_{s^{*}}} < r^{*}$ and all arms get pulled $n_{g_{s_{k}}}$ number of times on or before the $g_{s_{k}}$-th round. After summing over all arms in $A^{'}$ and bounding 
%trivially by $T\Delta_{i}$ we get the same result as above and can show that the regret can be no more than,
% \begin{align*}
% &\sum_{i\in A^{'}}\bigg(\frac{2^{1+4\rho_{s}}\rho_{s}^{2\rho_{s}}T^{1-\rho_{s}}}{\psi^{\rho_{s}}\Delta_{i}^{4\rho_{s}-1}}\bigg)=\sum_{i\in A^{'}}\bigg(\frac{C_{1}(\rho_{s})T^{1-\rho_{s}}}{\Delta_{i}^{4\rho_{s}-1}}\bigg)
% \end{align*}
% 
%Summing the bounds in Cases a1--a4 and observing that the bounds in the aforementioned cases hold for any round $\max(m_i,g_{s_k})$, we obtain the following contribution to the expected regret from case a:
%   %Taking summation of the events mentioned above($a1$-$a4$) gives us an upper bound on the regret given that the optimal arm $a^{*}$ is still surviving, 
%\begin{align*}
%&\sum_{i\in A_{s^*}} \frac{C_{1}(\rho_{a})T^{1-\rho_{a}}}{\Delta_{i}^{4\rho_{a}-1}} + \sum_{i\in A^{'} \setminus  A_{s^*} } \frac{C_{1}(\rho_{a})T^{1-\rho_{a}}}{\Delta_{i}^{'^{4\rho_{a}-1}}} \\
% & + \sum_{i\in A^{'}}\bigg(\frac{2C_{1}(\rho_{s})T^{1-\rho_{s}}}{\Delta_{i}^{4\rho_{s}-1}}\bigg)
%\end{align*}


Summing the bounds in Cases $a1-a2$ and observing that the bounds in the aforementioned cases hold for any round $C_{\max \lbrace m_i,g_{s_k}\rbrace}$, we obtain the following contribution to the expected regret from case a:
   %Taking summation of the events mentioned above($a1$-$a4$) gives us an upper bound on the regret given that the optimal arm $a^{*}$ is still surviving, 
\begin{align*}
&\sum_{i\in A_{s^*}} \frac{C_{1}(\rho_{a})T^{1-\rho_{a}}}{\Delta_{i}^{4\rho_{a}-1}} + \sum_{i\in A^{'}}\bigg(\frac{C_{1}(\rho_{s})T^{1-\rho_{s}}}{\Delta_{i}^{4\rho_{s}-1}}\bigg)
\end{align*}

%So the regret for not eliminating a sub-optimal cluster even when $a^{*}\notin C_{g_{s_{k}}}$(but still surviving in $s^{*}$) can be no worse than,
%	 \begin{align*} 
%	 \bigg(\frac{2}{(T\epsilon_{g_{s_{k}}}^{2})^{\rho_{s}}}\bigg) 
%	 \end{align*}
%&\underbrace{\sum_{i\in A_{s^{*}}^{'}}\bigg(\dfrac{C_{1}(\rho_{a})T^{1-\rho_{a}}}{\Delta_{i}^{4\rho_{a}-1}}\bigg)}_{\text{case a1}} + \underbrace{\sum_{i\in A\setminus A_{s^{*}}^{'}}\bigg(\dfrac{C_{1}(\rho_{a})T^{1-\rho_{a}}}{\Delta_{i}^{4\rho_{a}-1}}\bigg)}_{\text{case a2}} \\
% & + \sum_{i\in A^{'}}\bigg\lbrace \underbrace{\bigg(\dfrac{2C_{1}(\rho_{s})T^{1-\rho_{s}}}{\psi^{\rho_{s}}\Delta_{i}^{4\rho_{s}-1}}\bigg)}_{\text{case a3+a4}}\bigg\rbrace \\
%& =

%%%%%%%%%%%%%%%%%%%%%%%%%%%%%%%%%%%%%%%%%%%%%%%%%%%%%%%%%%%%%%%%%%%%%%%%%%%%%%%%%%%%%%%%%%%%
\textbf{Case b:} \textit{For each arm $a_i$, either $a_{i}$ is eliminated in round $\max (m_{i},g_{s_{k}})$ or before or there is no optimal arm $a^{*}$ in $C_{\max(m_{i},g_{s_{k}})}$.} \\

\textbf{Case b1:} \textit{$a^{*}\in C_{\max(m_{i},g_{s_{k}})}$ and each $a_{i}\in A^{'}$ is  eliminated on or before $\max (m_{i},g_{s_{k}})$.} 

%\todos[inline]{Case b1 caption is highly unclear}

%\todos[inline]{Fix margin overflows}

%\todos[inline]{max should be $\max$ and exp should be $\exp$: throughout the paper}

	For any sub-optimal arm $a_{i}\in s_{k}$ is eliminated on or before $\max\lbrace m_{i}, g_{s_{k}} \rbrace$ with $a^{*}\in s^{*}$ still surviving then they get pulled not more than $n_{m_{i}}$ or $n_{g_{s_{k}}}$ number of times. Hence, the maximum regret suffered due to pulling of a sub-optimal arm(or a sub-optimal cluster) is no less than,
 \begin{align*}
 &\sum_{i\in A^{'}}\Delta_{i}\bigg\lceil\dfrac{2\log{(\psi T\epsilon_{m_{i}}^{2})}}{\epsilon_{m_{i}}}\bigg\rceil + \sum_{k=1}^{p}\sum_{i\in A_{s_{k}}^{'}}\Delta_{i}\bigg\lceil\dfrac{2\log{(\psi T\epsilon_{g_{s_{k}}}^{2})}}{\epsilon_{g_{s_{k}}}}\bigg\rceil \\
&\leq\sum_{i\in A^{'}}\Delta_{i}\bigg(1+\dfrac{32\rho_{a}\log{(\psi T(\dfrac{\Delta_{i}}{2\sqrt{\rho_{a}}})^{4})}}{\Delta_{i}^{2}}\bigg) \\
&+ \sum_{i\in A^{'}}\Delta_{i}\bigg(1+\dfrac{32\rho_{s}\log{(\psi T(\dfrac{\Delta_{i}}{2\sqrt{\rho_{s}}})^{4})}}{\Delta_{i}^{2}}\bigg)
\\
&\text{ since } \sqrt{\rho_{a}\epsilon_{m_{i}}}\leq\frac{\Delta_{i}}{2} \text{ \& } \sqrt{\rho_{s}\epsilon_{n_{g_{s_{k}}}}}\leq\frac{\Delta_{i}}{2}
\\
 &\leq \sum_{i\in A^{'}}\bigg\lbrace 2\Delta_{i}+\dfrac{32\rho_{a}\log{(\psi T\dfrac{\Delta_{i}^{4}}{16\rho_{a}^{2}})}}{\Delta_{i}} +\dfrac{32\rho_{s}\log{(\psi T\dfrac{\Delta_{i}^{4}}{16\rho_{s}^{2}})}}{\Delta_{i}}\bigg\rbrace 
 \end{align*}

%&\leq\Delta_{i}\bigg(1+\dfrac{32\rho_{a}\log{(\psi T\dfrac{\Delta_{i}^{4}}{16\rho_{a}^{2}})}}{\Delta_{i}^{2}}\bigg)\\
%&\leq\Delta_{i}\bigg\lceil\dfrac{2\log{(\psi T(\dfrac{\Delta_{i}}{2\sqrt{\rho_{a}})^{4})}}}{(\dfrac{\Delta_{i}}{2\sqrt{\rho_{a}}})^{2}}\bigg\rceil \\
%\text{, since } \sqrt{\rho_{a}\epsilon_{m_{i}}}\leq\dfrac{\Delta_{i}}{2}
 
%%%%%%%%%%%%%%%%%%%%%%%%%%%%%%%%%%%%%%%%%%%%%%%%%%%%%%%%%%%%%%%%%%%%%%%%%%%%%%%%%%%%%%%%%%%%%%%   
\textbf{Case b2:} \textit{Optimal arm $a^{*}$ is eliminated by a sub-optimal arm.}\\
  
	This, can happen in $3$ ways,
\newline
\textbf{Case b21:} \textit{In $s^{*}$, $a^{*}$ got eliminated by other arms surviving till $m_{*}$} 

Firstly, if conditions of  Case $a$ holds then the optimal arm $a^{*}$ will not be eliminated in round $m=m_{*}$ or it will lead to the contradiction that $r_{i}>r^{*}$ for any $i\in A^{'}$. In any round $m_{*}$, if the optimal arm $a^{*}$ gets eliminated then for any round from $1$ to $m_{j}$ all arms $a_{j}\in s^{*}$ such that $m_{j} < m_{*}$ were eliminated according to assumption in Case $a$. Let, the arms surviving till $m_{*}$ round be denoted by $A^{'}$. This leaves any arm $a_{b}$ such that $m_{b}\geq m_{*} $ to still survive and eliminate arm $a^{*}$ in round $m_{*}$. Let, such arms that survive $a^{*}$ belong to $A^{''}$. Also maximal regret per step after eliminating $a^{*}$ is the maximal $\Delta_{j}$ among the remaining arms in $A^{''}$ with $m_{j}\geq m_{*}$.  Let $m_{b}=\min\lbrace m|\sqrt{\rho_{a}\epsilon_{m}}<\frac{\Delta_{b}}{2}\rbrace$. Let $C_2(x) = \frac{2^{2x+\frac{3}{2}}x^{2x}}{\psi^{x}}$. Hence, the maximal regret after eliminating the arm $a^{*}$ is upper bounded by, 
\begin{align*}
&\sum_{m_{*}=0}^{max_{j\in A^{'}_{s^{*}}}m_{j}}\sum_{\substack{i\in A^{''}_{s^{*}}: \\ m_{i}\geq m_{*}}}\bigg(\dfrac{2}{(\psi  T\epsilon_{m_{*}}^{2})^{\rho_{a}}} \bigg).T\max_{\substack{j\in A^{''}_{s^{*}}: \\ m_{j}\geq m_{*}}}{\Delta}_{j}\\
&\leq\sum_{m_{*}=0}^{max_{j\in A^{'}_{s^{*}}}m_{j}}\sum_{i\in A^{''}_{s^{*}}:m_{i} \geq m_{*}}\bigg(\dfrac{2}{(\psi  T\epsilon_{m_{*}}^{2})^{\rho_{a}}} \bigg).T.2\sqrt{\rho_{a}\epsilon_{m_{*}}} \\
&\leq\sum_{m_{*}=0}^{max_{j\in A^{'}_{s^{*}}}m_{j}}\sum_{i\in A^{''}_{s^{*}}:m_{i} \geq m_{*}}4\bigg(\dfrac{T^{1-\rho_{a}}}{\psi^{\rho_{a}}\epsilon_{m_{*}}^{2\rho_{a}-\frac{1}{2}}} \bigg)\\
&\leq\sum_{i\in A^{''}_{s^{*}}:m_{i} \geq m_{*}}\sum_{m_{*}=0}^{\min{\lbrace m_{i},m_{b}\rbrace}}\bigg(\dfrac{4T^{1-\rho_{a}}}{\psi^{\rho_{a}}2^{-(2\rho_{a}-\frac{1}{2})m_{*}}} \bigg)\\
&\leq\sum_{i\in A^{'}_{s^{*}}}\dfrac{4T^{1-\rho_{a}}}{\psi^{\rho_{a}}2^{-(2\rho_{a}-\frac{1}{2})m_{*}}} +\sum_{i\in A^{''}_{s^{*}}\setminus A^{'}_{s^{*}}}\dfrac{4T^{1-\rho_{a}}}{\psi^{\rho_{a}}2^{-(2\rho_{a}-\frac{1}{2})m_{b}}} \\
&\leq\sum_{i\in A^{'}_{s^{*}}}\dfrac{T^{1-\rho_{a}}\rho_{a}^{2\rho_{a}}2^{2\rho_{a}+\frac{3}{2}}}{\psi^{\rho_{a}}\Delta_{i}^{4\rho_{a}-1}} +\sum_{i\in A^{''}_{s^{*}}\setminus A^{'}_{s^{*}}}\dfrac{T^{1-\rho_{a}}\rho_{a}^{2\rho_{a}}2^{2\rho_{a}+\frac{3}{2}}}{\psi^{\rho_{a}}b^{4\rho_{a}-1}} \\
& = \sum_{i\in A^{'}_{s^{*}}}\dfrac{ C_{2}(\rho_{a}) T^{1-\rho_{a}}}{\Delta_{i}^{4\rho_{a}-1}} +\sum_{i\in A^{''}_{s^{*}}\setminus A^{'}_{s^{*}}}\dfrac{C_{2(\rho_{a})}T^{1-\rho_{a}}}{b^{4\rho_{a}-1}}
\end{align*}

%&\text{ since } \sqrt{\rho_{a}\epsilon_{m}}<\dfrac{\Delta_{i}}{2}\\
%&\leq\sum_{i\in A^{'}}\dfrac{4\rho_{a}^{2\rho_{a}}T^{1-\rho_{a}}*2^{2\rho_{a}-\frac{1}{2}}}{\psi^{\rho_{a}}\Delta_{i}^{4\rho_{a}-1}} +\sum_{i\in A^{''}\setminus A^{'}}\dfrac{4\rho_{a}^{2\rho_{a}}T^{1-\rho_{a}}*2^{2\rho_{a}-\frac{1}{2}}}{\psi^{\rho_{a}}b^{4\rho_{a}-1}} \\

% \begin{align*}
% &\sum_{i\in A^{'}_{s^{*}}}\bigg(\dfrac{C_{2}(\rho_{a})T^{1-\rho_{a}}}{\Delta_{i}^{4\rho_{a} -1}} \bigg)+\sum_{i\in A^{''}_{s^{*}}\setminus A^{'}_{s^{*}}}\bigg(\dfrac{C_{2}(\rho_{a})T^{1-\rho_{a}}}{b^{4\rho_{a} -1}} \bigg)
% \end{align*}
%We also see that here, we are concerned only within $s^{*}$ because of our assumption that there is only one $a^{*}\in A$ and clusters are fixed.


%%%%%%%%%%%%%%%%%%%%%%%%%%%%%%%%%%%%%%%%%%%%%%%%%%%%%%%%%%%%%%%%%%%%%%%%%%%%%%%%%%%%%%%%%%%%%%%   
\textbf{Case b22:} \textit{$a^{*}\in C_{\max \lbrace m_{i},g_{s_{k}}\rbrace}$ and $s^{*}$ gets eliminated by another sub-optimal cluster arm} 
%\newline
%Firstly, if conditions of case $b1$ holds then the optimal arm $a^{*}\in C_{g_{s_{k}}}$ will not be eliminated in round $g_{s_{k}}=g_{*}$ or it will lead to the contradiction that $r_{a_{max_{s_{k}}}}>r^{*}$ where $a_{max_{s_{k}}},a^{*}\in C_{g_{s_{k}}}$. In any round $g_{*}$, if the optimal arm $a^{*}$ gets eliminated then for any round from $1$ to $g_{s_{j}}$ all arms $a_{s_{j}}\in C_{g_{s_{k}}},\forall s_{j}\neq s^{*}$ such that $\sqrt{\rho_{s}\epsilon_{g_{s_{k}}}}<\dfrac{\Delta_{a_{s_{j}}}}{2}$ were eliminated according to assumption in case $b1$. Let, the arms surviving till $g_{*}$ round be denoted by $C_{g}^{'}$. This leaves any arm $a_{s_{b}}$ such that $\sqrt{\rho_{s}\epsilon_{g_{s_{b}}}}\geq\dfrac{\Delta_{a_{s_{b}}}}{2}$ to still survive and eliminate arm $a^{*}$ in round $g_{*}$. Let, such arms that survive $a^{*}$ belong to $C_{g}^{''}$. Also maximal regret per step after eliminating $a^{*}$ is the maximal $\Delta_{j}$ among the remaining arms $a_{j}\in B_{m}$ with $g_{s_{j}}\geq g_{
%*}$.  Let $g_{s_{b}}$ be the round when $\sqrt{\rho_{s}\epsilon_{g_{s_{b}}}}<\dfrac{\Delta_{s_{b}}}{2}$ that is $g_{b}=min\lbrace g|\sqrt{\rho_{s}\epsilon_{g_{s_{b}}}}<\dfrac{\Delta_{b}}{2}\rbrace$ and the cluster $s_{b}$ gets eliminated. Hence, the maximal regret after eliminating the arm $a^{*}$ is upper bounded by, 
% \begin{align*}
% &\sum_{g_{*}=0}^{max_{j\in C_{g}^{'}}g_{s_{j}}}\sum_{i\in C_{g}^{''}:g_{s_{k}}>g_{*}}\bigg(\dfrac{2}{(\psi T\epsilon_{g_{s_{k}}}^{2})^{\rho_{s}}} \bigg).T\max_{j\in C_{g}^{''}:g_{s_{j}}\geq g_{*}}{\Delta}_{a_{s_{j}}}
% \end{align*}

Following the same way as Case $b21$, in any round $g_{*}$, if the optimal arm $a^{*}$ gets eliminated then for any round from $1$ to $g_{s_{j}}$ all arms $a_{\max_{s_{j}}}$ such that $g_{s_{j}}< g_{*}$ were eliminated according to assumption in Case $a$. Let, the arms surviving till $g_{*}$ round in $C_{\max \lbrace m_{i},g_{s^{*}}\rbrace}$ be denoted by $C_{g}^{'}$. This leaves any arm $a_{\max_{s_{b}}}$ such that $g_{s_{b}}\geq g_{*}$ to still survive and eliminate arm $a^{*}$ in round $g_{*}$. Let, such arms that survive $a^{*}$ belong to $C_{g}^{''}$. Hence, maximal regret after eliminating $a^{*}$ is,
 \begin{align*}
 &\sum_{g_{*}=0}^{max_{j\in C_{g}^{'}}g_{s_{j}}}\sum_{\substack{i\in C_{g}^{''}: \\ g_{s_{k}} \geq g_{*}}}\bigg(\dfrac{2}{(\psi T\epsilon_{g_{s^{*}}}^{2})^{\rho_{s}}} \bigg).T\max_{\substack{j\in C_{g}^{''}: \\ g_{s_{j}}\geq g_{*}}}{\Delta}_{a_{\max_{s_{j}}}}
 \end{align*}
But, we know that for any round $g$, elements of $C_{g}$ are the best performers in their respective clusters. So, taking that into account and $A'\supset C_{g}^{'}$ and $A''\supset C_{g}^{''}$ where $A^{'}$ is the set of all the arms across clusters surviving till $g_{*}$ round and $A^{''}$ be the set of all arms across clusters to still survive and eliminate arm $a^{*}$ in round $g_{*}$ respectively.
\begin{align*}
 & \sum_{g_{*}=0}^{max_{j\in A^{'}}g_{s_{j}}}\sum_{\substack{i\in A^{''}: \\ g_{s_{k}}\geq g_{*}}}\bigg(\dfrac{2}{(\psi T\epsilon_{g_{s^{*}}}^{2})^{\rho_{s}}} \bigg).T\max_{\substack{j\in A^{''}: \\ g_{s_{j}}\geq g_{*}}}{\Delta}_{a_{\max_{s_{j}}}}
\end{align*}
Like Case $b21$, we can bound the regret as,
\begin{align*}
 &\sum_{i\in A^{'}}\dfrac{T^{1-\rho_{s}}\rho_{s}^{2\rho_{s}}2^{2\rho_{s}+\frac{3}{2}}}{\psi^{\rho_{s}}\Delta_{i}^{4\rho_{s}-1}} +\sum_{i\in A^{''}\setminus A^{'}}\dfrac{T^{1-\rho_{s}}\rho_{s}^{2\rho_{s}}2^{2\rho_{s}+\frac{3}{2}}}{\psi^{\rho_{s}}b^{4\rho_{s}-1}} \\
 & = \sum_{i\in A^{'}}\dfrac{C_{2}(\rho_{s})T^{1-\rho_{s}}}{\Delta_{i}^{4\rho_{s}-1}} +\sum_{i\in A^{''}\setminus A^{'}}\dfrac{C_{2}(\rho_{s})T^{1-\rho_{s}}}{b^{4\rho_{s}-1}} 
\end{align*}

%\\ & \text{ where } C_2(x) = \frac{2^{2x+\frac{3}{2}}x^{2x}}{\psi^{x}}
%&\leq\sum_{g_{*}=0}^{max_{j\in A^{'}}g_{s_{j}}}\sum_{i\in A^{''}:g_{s_{k}}>g_{*}}\bigg(\dfrac{2}{(\psi T\epsilon_{g_{s_{k}}}^{2})^{\rho_{s}}} \bigg).T.2\sqrt{\rho_{s}\epsilon_{g_{s_{j}}}} \text{, since }\sqrt{\rho_{s}\epsilon_{g_{s_{j}}}}\leq\dfrac{\Delta_{a_{s_{j}}}}{2}\leq  \dfrac{\Delta_{j}}{2}\text{, as }{r}_{a_{s_{j}}}>{r}_{j},\forall j\in s_{j}\\ 
%&\leq\sum_{g_{*}=0}^{max_{j\in A^{'}}g_{s_{j}}}\sum_{i\in A^{''}:g_{s_{k}}>g_{*}}\bigg(\dfrac{4T^{1-\rho_{s}}}{\psi^{\rho_{s}}\epsilon_{g_{s_{k}}}^{2\rho_{s} - \frac{1}{2}}} \bigg)\\
% &\leq\sum_{i\in A^{''}:g_{s_{k}}>g_{*}}\sum_{g_{*}=0}^{\min{\lbrace g_{s_{k}},g_{s_{b}}\rbrace}}\bigg(\dfrac{4T^{1-\rho_{s}}}{\psi^{\rho_{s}}2^{({2\rho_{s} - \frac{1}{2}})g_{*}}} \bigg) \\
% &\leq\sum_{i\in A^{'}}\bigg(\dfrac{4T^{1-\rho_{s}}}{\psi^{\rho_{s}}2^{({2\rho_{s} - \frac{1}{2}})g_{*}}} \bigg)+\sum_{i\in A^{''}\setminus A^{'}}\bigg(\dfrac{4T^{1-\rho_{s}}}{\psi^{\rho_{s}}2^{({2\rho_{s} - \frac{1}{2}})g_{s_{b}}}} \bigg)\\ 
% &\leq\sum_{i\in A^{'}}\bigg(\dfrac{4\rho_{s}^{2\rho_{s}}T^{1-\rho_{s}}*2^{2\rho_{s}-\frac{1}{2}}}{\psi^{\rho_{s}}\Delta_{i}^{4\rho_{s}-1}} \bigg)+\sum_{i\in A^{''}\setminus A^{'}}\bigg(\dfrac{4\rho_{s}^{2\rho_{s}}T^{1-\rho_{s}}*2^{2\rho_{s}-\frac{1}{2}}}{\psi^{\rho_{s}}b^{4\rho_{s}-1}} \bigg)\\

%%%%%%%%%%%%%%%%%%%%%%%%%%%%%%%%%%%%%%%%%%%%%%%%%%%%%%%%%%%%%%%%%%%%%%%%%%%%%%%%%%%%%%%%%%%%%%%   
\textbf{Case b23:} \textit{$a^{*}\notin C_{\max \lbrace m_{i},g_{s_{k}} \rbrace}$ and $s^{*}$ gets eliminated by another sub-optimal cluster arm} 

This will be the mirror case of Case $b22$ with $a^{*}\notin C_{g}$. So, let $a_{\max_{s^{*}}}$ satisfies $\hat{r}_{a_{\max_{s^{*}}}}> \hat{r}^{*}$ in round $C_{\max \lbrace m_{i},g_{s^{*}} \rbrace}$. Following the same way as Case $a2$, we can bound the events
\begin{align*}
  \hat{r}_{a_{\max_{s_{k}}}} \geq r_{a_{\max_{s_{k}}}} +c_{g_{s_{k}}} \text{ and } \hat{r}_{a_{\max_{s^{*}}}} \leq r_{a_{\max_{s^{*}}}} -c_{g_{s_{k}}}
\end{align*}
 
by Chernoff-Hoeffding bound and considering independence of events and show that it cannot be worse than $\bigg(\dfrac{2}{(\psi  T\epsilon_{g_{s^{*}}}^{2})^{\rho_{s}}}\bigg)$ for any $g_{s_{k}}=g_{*}$.
%In this case for some sub-optimal arm $a_{\max_{s_{k}}}\in C_{g_{s_{k}}}$, we have to bound the events
%	\begin{align*}
%  \hat{r}_{a_{\max_{s_{k}}}} \geq r_{a_{\max_{s_{k}}}} +c_{g_{s_{k}}} \text{ and } \hat{r}_{a_{\max_{s^{*}}}} \leq r_{a_{\max_{s^{*}}}} -c_{g_{s_{k}}}
%\end{align*} 
So $s^{*}$ gets eliminated by $a_{\max_{s_{b}}}$ such that $g_{s_{b}}\geq g_{*}$. Also maximal regret per step after eliminating $a^{*}$ is the maximal $\Delta_{j}$ among the remaining arms $a_{j}\in A^{''}$ with $g_{s_{j}}\geq g_{*}$. Following the same way above we can bound the regret as,
%In this case we will consider that the cluster $s^{*}$ containing the optimal arm $a^{*}$ was eliminated by another sub-optimal cluster and $a^{*}\notin C_{g}$. 
\begin{align*}
&\sum_{i\in A^{'}}\dfrac{T^{1-\rho_{s}}\rho_{s}^{2\rho_{s}}2^{2\rho_{s}+\frac{3}{2}}}{\psi^{\rho_{s}}\Delta_{i}^{4\rho_{s}-1}} +\sum_{i\in A^{''}\setminus A^{'}}\dfrac{T^{1-\rho_{s}}\rho_{s}^{2\rho_{s}}2^{2\rho_{s}+\frac{3}{2}}}{\psi^{\rho_{s}}b^{4\rho_{s}-1}} \\
& = \sum_{i\in A^{'}}\dfrac{C_{2}(\rho_{s})T^{1-\rho_{s}}}{\Delta_{i}^{4\rho_{s}-1}} +\sum_{i\in A^{''}\setminus A^{'}}\dfrac{C_{2}(\rho_{s})T^{1-\rho_{s}}}{b^{4\rho_{s}-1}}
\end{align*}
\newline
Combining Cases $b21$, $b22$ and $b23$ as mentioned above we can show,
 \begin{align*}
 &\underbrace{\sum_{i\in A^{'}_{s^{*}}}\bigg(\dfrac{C_{2}(\rho_{a})T^{1-\rho_{a}}}{\Delta_{i}^{4\rho_{a} -1}} \bigg)+\sum_{i\in A^{''}_{s^{*}}\setminus A^{'}_{s^{*}}}\bigg(\dfrac{C_{2}(\rho_{a})T^{1-\rho_{a}}}{b^{4\rho_{a} -1}} \bigg)}_{\text{case b21}} \\
 & + \underbrace{\sum_{i\in A^{'}}\bigg(\dfrac{2C_{2}(\rho_{s})T^{1-\rho_{s}}}{\Delta_{i}^{4\rho_{s}-1}} \bigg)}_{\text{case b22}}+\underbrace{\sum_{i\in A^{''}\setminus A^{'}}\bigg(\dfrac{2C_{2}(\rho_{s})T^{1-\rho_{s}}}{b^{4\rho_{s} -1}} \bigg)}_{\text{case b23}}
 \end{align*}
 
\textbf{Case b3:} \textit{Optimal arm $a^{*}$ is not in $C_{\max(m_{i},g_{s_{k}})}$.}

Since, the optimal arm $a^{*}\in s^{*}$ is not eliminated, also $s^{*}$ is not eliminated in this case and for all sub-optimal arms $a_i$ in $A'$ which have not been eliminated on or before $\max \lbrace m_{i},g_{s_{k}} \rbrace$ will get pulled no less than,

\begin{align*}
 &\sum_{i\in A^{'}}\Delta_{i}\bigg\lceil\dfrac{2\log{(\psi T\epsilon_{m_{i}}^{2})}}{\epsilon_{m_{i}}}\bigg\rceil
\end{align*}
Since $a^{*}$ is definitely in $B_{m_{i}}$ then following the same way as Case $b1$ we can show that can be no worse than,
\begin{align*}
 &\sum_{i\in A^{'}}\bigg\lbrace \Delta_{i}+\dfrac{32\rho_{a}\log{(\psi T\dfrac{\Delta_{i}^{4}}{16\rho_{a}^{2}})}}{\Delta_{i}} \bigg\rbrace 
\end{align*} 



The main claim follows by summing the contributions to the expected regret from each of the cases above.

\end{proof}


\begin{proposition}
\label{proofTheorem:Prop:1}
The regret $R_T$ for ClusUCB-AE satisfies
\begin{align*}
&\E [R_{T}]\leq \sum\limits_{i\in A:\Delta_{i} > b}\bigg\lbrace\frac{C_{1}(\rho_{a})T^{1-\rho_{a}}}{\Delta_{i}^{4\rho_{a}-1}} + \Delta_{i}\\
&+\frac{32\rho_{a}\log{(\dfrac{\psi  T\Delta_{i}^{4}}{16\rho_{a}^{2}})}}{\Delta_{i}}
 +  \frac{C_{2}(\rho_{a})T^{1-\rho_{a}}}{\Delta_{i}^{4\rho_{a} -1}}  \bigg \rbrace\\
&+\sum\limits_{i\in A:0\leq\Delta_{i}\leq b}\frac{C_{2}(\rho_{a})T^{1-\rho_{a}}}{b^{4\rho_{a} -1}}  + \max_{i:\Delta_{i}\leq b}\Delta_{i}T
\end{align*}
, for all $b\geq\sqrt{\frac{e}{T}}$, where  $C_1(x) = \frac{2^{1+4x}x^{2x}}{\psi^{x}}$,  $C_2(x) = \frac{2^{2x+\frac{3}{2}}x^{2x}}{\psi^{x}}$, $\rho_{a}=\frac{1}{2}$ is the arm elimination parameter, $\psi=K^{2}T$ is the exploration regulatory factor, $p$ is the number of clusters and $T$ is the horizon.
\end{proposition}
\begin{proof}
%Follows in a similar fashion as the proof of Theorem $1$ in \cite{auer2010ucb}. For the sake of completeness, the proof is given in Appendix \ref{App:A}.
See Appendix \ref{App:A}.
\end{proof}

\begin{proposition}
\label{proofTheorem:Prop:2}
The regret $R_T$ for ClusUCB-CE satisfies,
\begin{align*}
&\E [R_{T}]\leq \sum\limits_{i\in A:\Delta_{i} > b}\bigg\lbrace\bigg(\dfrac{2C_{1}(\rho_{s})T^{1-\rho_{s}}}{\Delta_{i}^{4\rho_{s}-1}}\bigg)\\
& + \bigg(\Delta_{i}+\dfrac{32\rho_{s}\log{(\psi T\dfrac{\Delta_{i}^{4}}{16\rho_{s}^{2}})}}{\Delta_{i}}\bigg) + \bigg(\dfrac{2C_{2}(\rho_{s})T^{1-\rho_{s}}}{\Delta_{i}^{4\rho_{s} -1}} \bigg)\bigg\rbrace \\
& + \sum\limits_{i\in A:0\leq\Delta_{i}\leq b}\bigg(\dfrac{2C_{2}(\rho_{s})T^{1-\rho_{s}}}{b^{4\rho_{s} -1}} \bigg) + \max_{i:\Delta_{i}\leq b}\Delta_{i}T
\end{align*}
, for all $b\geq \sqrt{\frac{e}{T}}$, where $C_1(x) = \frac{2^{1+4x}x^{2x}}{\psi^{x}}$,  $C_2(x) = \frac{2^{2x+\frac{3}{2}}x^{2x}}{\psi^{x}}$, $\rho_{s}=\frac{1}{2} $ is the cluster elimination parameter, $\psi=K^{2}T$ is the exploration regulatory factor, $p$ is the number of clusters and $T$ is the horizon.
\end{proposition}
\begin{proof}
See Appendix \ref{App:B}.
\end{proof}


\section{Proof of Corollary 1}
\label{App:Proof:Corollary:1}
\begin{proof}
%As stated in \cite{auer2010ucb}, the regret bound can be of the order of $\sqrt{KT\log K}$ in non-stochastic MAB setting. This is shown in Exp4\cite{auer2002nonstochastic} algorithm. 
First we recall the definition of Theorem \ref{Result:Theorem:1} below,
\begin{align*}
&\E [R_{T}]\leq 
\sum\limits_{\substack{i\in A_{s^{*}},\\\Delta_{i} > b}}\bigg\lbrace \Delta_{i} + 12K
+ \frac{32\log{(\frac{T\Delta_i^2}{K})}}{\Delta_{i}} \bigg\rbrace
 + \! \! \sum\limits_{\substack{i\in A,\\\Delta_{i} > b}} \bigg\lbrace 2\Delta_{i} +
12K + \frac{64\log{(\frac{T\Delta_i^2}{K})}}{\Delta_{i}} \bigg\rbrace \\
%%%%%%%%%%%%%%%%%
&+ \sum\limits_{\substack{i\in A_{s^{*}},\\ \Delta_{i} > b}} 
16K+\sum\limits_{\substack{i\in A_{s^{*}},\\0 < \Delta_{i}\leq b}} 16K + \sum_{\substack{i\in A\setminus A_{s^*}:\\\Delta_{i}> b}}32K +\sum_{\substack{i\in A \setminus A_{s^*}:\\ 0 < \Delta_{i} \leq b}}32K 
 \!+\! \max\limits_{i:\Delta_{i}\leq b}\Delta_{i}T
\end{align*}

Now we know from \cite{bubeck2011pure} that the function $x\in [0,1]\mapsto x\exp(-Cx^2)$ is  decreasing on $\left[\dfrac{1}{\sqrt{2C}},1\right ]$ for any $C>0$. So, taking $C=\left\lfloor \dfrac{T}{e}\right\rfloor$ and by choosing  $\Delta_{i}=\Delta=\sqrt{\dfrac{K\log K}{T}}>\sqrt{\dfrac{e}{T}}$ for all ${i:i\neq *}\in A$ and substituting $p=\left\lceil \dfrac{K}{\log K}\right\rceil $ in the bound of ClusUCB we get,

	\begin{align*}
	\sum_{i\in A_{s^{*}}:\Delta_{i} > b} 12K =12\dfrac{K^2}{p}
	\end{align*}		
	 Similarly, for the term, 
	 \begin{align*}
	 \sum_{i\in A:\Delta_{i} > b} 12K = 12 K^2
	 \end{align*}
	 
	
	For the term regarding number of pulls,
	\begin{align*}
	\sum_{i\in A:\Delta_{i} > b}\dfrac{64\log{(\frac{T\Delta_i^2}{K})}}{\Delta_{i}} &\leq  \dfrac{64K\sqrt{T}\log{(T\dfrac{K\log K}{T K})}}{\sqrt{K\log K}} \leq  \dfrac{64\sqrt{KT}\log{(\log K)}}{\sqrt{\log K}}\\
	%%%%%%%%%%%%%%%%%%%%%%%
	&\overset{(a)}{\leq} 64\sqrt{KT}
	\end{align*}		
	
	Here $(a)$ is obtained by the identity $\dfrac{\log\log K}{\sqrt{\log K}} < 1$ for $K\geq 2$. Lastly we can bound the error terms as, 
	\begin{align*}
	\sum\limits_{i\in A_{s^{*}}:0\leq\Delta_{i}\leq b} 16K =\dfrac{16K^2}{p} \overset{(a)}{\leq} 16K\log K
	\end{align*}	 	
 	Here we obtain $(a)$ by substituting the value of $p$. Similarly for the term,
 	\begin{align*}
 	\sum_{i\in A\setminus A_{s^*}: \Delta_{i} > b} 16K =\dfrac{16K^2}{p} < 16K\log K
	\end{align*} 	
	Also, for all $b\geq \sqrt{\dfrac{e}{T}}$,
	\begin{align*}
 	\sum_{i\in A\setminus A_{s^*}: 0 < \Delta_{i} \leq b} 32K = \left(K-\dfrac{K}{p}\right) 32K
	\end{align*} 	
	
	Now, $K-\dfrac{K}{p}= K\left( \dfrac{p-1}{p} \right) < K\left(  \dfrac{\frac{K}{\log K}+1-1}{\frac{K}{\log K}+1 }\right) < \dfrac{K^2}{K+\log K}$. So, after substituting the value of $p=\left\lceil \dfrac{K}{\log K} \right\rceil$, we get,
	
	\begin{align*}
 	\sum_{i\in A\setminus A_{s^*}: 0 < \Delta_{i} \leq b} 32K = \left(K-\dfrac{K}{p}\right)32K < \dfrac{32 K^3}{K+\log K}
	\end{align*} 	
	
	Summing up all the contribution from the individual cases as shown above, the total gap-independent regret is given by,	
	
	\begin{align*}
	\E[R_{T}]\leq & 12K\log K + 32\sqrt{KT} + 12K^2 + 64\sqrt{KT} + 32K\log K + \dfrac{64 K^3}{K+\log K}
	\end{align*}
 	
	So, the total bound for using both arm and cluster elimination cannot be worse than,
	
	\begin{align*}
	\E[R_{T}]\leq 96\sqrt{KT} + 12K^2 + 44K\log K + \dfrac{64 K^3}{K+\log K}\\ 
	\end{align*}		
\end{proof}


\section{Proof of Proposition 1}
\label{App:A}

\begin{proof}
Let $p=1$ such that all the arms in $A$ belongs to a single cluster. Hence, in ClusUCB-AE there is only arm elimination and no cluster elimination. Let, for each sub-optimal arm ${i}$, $m_{i}=\min{\lbrace m|\sqrt{\epsilon_{m}} < \dfrac{\Delta_{i}}{2} \rbrace}$. Also $\rho_{a}=\frac{1}{2}$ is a constant in this proof. Let $A^{'}=\lbrace i\in A: \Delta_{i} > b \rbrace$ and $A^{''}=\lbrace i\in A: \Delta_{i} > 0 \rbrace$. 

%Also $z_{i}$ denotes total number of times an arm $i$ has been pulled. In the $m$-th round, $n_{m}$ denotes the number of pulls allocated to the surviving arms in $B_{m}$. 
%The theoretical analysis remains same as we have always bounded the values of $\rho_{a}\in (0,1]$.

\subsection*{Case $a$: \textit{Some sub-optimal arm ${i}$ is not eliminated in round $m_{i}$ or before and the optimal arm ${*}\in B_{m_{i}}$}}
  
	Following the steps of Theorem \ref{Result:Theorem:1} Case $a1$, an arbitrary sub-optimal arm ${i}\in A^{'}$ can get eliminated only when the event,
	\begin{align}
	\hat{r}_{i}  \le r_{i} + c_{m_i} \text{ and } \label{eq:appA:armelim-casea}
 	\hat{r}^{*}\geq  r^{*} - c_{m_i}
	\end{align}
	
	takes place. So to bound the regret we need to bound the probability of the complementary event of these two conditions. Note that  $c_{m_{i}} = \sqrt{\frac{\rho_{a}\log (\psi T\epsilon_{m_{i}})}{2 n_{m_i}}}$. A sub-optimal arm $i$ will get eliminated in the $m_i$-th round because $n_{m_{i}}=\dfrac{\log{(\psi T\epsilon_{m_{i}}^{2})}}{2\epsilon_{m_{i}}}$ and substituting this in $c_{m_i}$ and applying Lemma \ref{proofTheorem:Lemma:2} we get, $c_{m_i} < \dfrac{\Delta_{i}}{4} $. Hence, for a sub-optimal arm ${i} \in A^{'}$, 
  \begin{align*}
\hat{r}_{i} + c_{m_i}&\leq r_{i} + 2c_{m_i} 
%&= \hat{r}_{i} + 4c_{m_{i}} - 2c_{m_{i}} \\
 < r_{i} + \Delta_{i} - 2c_{m_i}
 \leq r^{*} -2c_{m_i} 
 \leq \hat{r}^{*} - c_{m_i}
  \end{align*}

	Applying Chernoff-Hoeffding bound and considering independence of complementary of the two events in \ref{eq:appA:armelim-casea},
  \begin{align*}
\mathbb{P}\lbrace\hat{r}_{i}&\geq r_{i} + c_{m_i}\rbrace\leq \exp(-2c_{m_i}^{2}n_{m_{i}})
\leq \exp(-2 * \dfrac{\rho_{a}\log (\psi T\epsilon_{m_{i}})}{2 n_{m_{i}}} *n_{m_{i}})
\leq \dfrac{1}{(\psi T\epsilon_{m_{i}})^{\rho_{a}}}   
  \end{align*}
 
%$\leq \bigg(\dfrac{1}{4\psi T\epsilon_{m}^{2}}\bigg)^{D}$, as $\ell_{m}-1\leq D$
% \hspace*{2em}
 
Similarly, $\mathbb{P}\lbrace\hat{r}^{*}\leq r^{*} - c_{m_i}\rbrace\leq \dfrac{1}{(\psi  T\epsilon_{m_{i}})^{\rho_{a}}}$. Summing the two up, the probability that a sub-optimal arm ${i}$ is not eliminated on or before $m_{i}$-th round is  $\bigg(\dfrac{2}{(\psi T\epsilon_{m_{i}})^{\rho_{a}}}\bigg)$. 
 
Summing up over all arms in $A^{'}$ and bounding the regret for each arm $i\in A^{'}$ trivially by $T\Delta_{i}$, we obtain
   \begin{align*}
\sum_{i\in A^{'}}\bigg(\dfrac{2T\Delta_{i}}{(\psi T\epsilon_{m_{i}})^{\rho_{a}}}\bigg)
\leq\sum_{i\in A^{'}}\bigg(\dfrac{2T\Delta_{i}}{(\psi T\dfrac{\Delta_{i}^{2}}{32})^{\rho_{a}}}\bigg)
&\leq \sum_{i\in A^{'}}\bigg(\dfrac{2^{1+5\rho_{a}}T^{1-\rho_{a}}\Delta_{i}}{\psi^{\rho_{a}}\Delta_{i}^{2\rho_{a}}}\bigg)
\leq \sum_{i\in A^{'}}\bigg(\dfrac{2^{1+5\rho_{a}}T^{1-\rho_{a}}}{\psi^{\rho_{a}}\Delta_{i}^{2\rho_{a}-1}}\bigg)\\  
%%%%%%%%%%%%%%%%%% 
& \overset{(a)}{\leq}\sum_{i\in A^{'}}\leq 8\sqrt{2} K
   \end{align*}

Here, $(a)$ is obtained by substituting the values of $\psi$ and $\rho_a$.

\subsection*{Case $b$: \textit{Either an arm ${i}$ is eliminated in round $m_{i}$ or before or else there is no optimal arm ${*}\in B_{m_{i}}$ }}

\subsubsection*{Case $b1$: \textit{${*}\in B_{m_{i}}$ and each ${i}\in A^{'}$ is  eliminated on or before $m_{i}$ } }

 Since we are eliminating a sub-optimal arm ${i}$ on or before round $m_{i}$, it is pulled no longer than, 
 \begin{align*}
  \bigg\lceil\dfrac{\log{(\psi T\epsilon_{m_{i}}^{2})}}{2\epsilon_{m_{i}}}\bigg\rceil
 \end{align*}

So, the total contribution of ${i}$  till round $m_{i}$ is given by, 
\begin{align*}
&\Delta_{i}\bigg\lceil\dfrac{\log{(\psi T\epsilon_{m_{i}}^{2})}}{2\epsilon_{m_{i}}}\bigg\rceil
\leq\Delta_{i}\bigg\lceil\dfrac{\log{(\psi T(\dfrac{\Delta_{i}}{4\sqrt{2}})^{4})}}{(\dfrac{\Delta_{i}}{4\sqrt{2}})^{2}}\bigg\rceil \text{, since } \sqrt{2\epsilon_{m_{i}}} < \dfrac{\Delta_{i}}{4}\\
&\overset{(a)}{\leq}\Delta_{i}\bigg(1+\dfrac{32\log{(\frac{T}{K^2} T(\Delta_{i})^{4})}}{\Delta_{i}^{2}}\bigg)
\leq\Delta_{i}\bigg(1+\dfrac{32\log{( \frac{T\Delta_i^2}{K})}}{\Delta_{i}^{2}}\bigg)
\end{align*} 
 
In the above case, $(a)$ is obtained by substituting the values of $\psi$ and $\rho_a$. Summing over all arms in $A^{'}$ the total regret is given by, 
\begin{align*}
\sum_{i\in A^{'}}\Delta_{i}\bigg(1+\dfrac{32\log{( \frac{T\Delta_i^2}{K})}}{\Delta_{i}^{2}}\bigg)
\end{align*}

\subsubsection*{Case $b2$: \textit{Optimal arm ${*}$ is eliminated by a sub-optimal arm  }}


	Firstly, if conditions of Case $a$ holds then the optimal arm ${*}$ will not be eliminated in round $m=m_{*}$ or it will lead to the contradiction that $r_{i}>r^{*}$. In any round $m_{*}$, if the optimal arm ${*}$ gets eliminated then for any round from $1$ to $m_{j}$ all arms ${j}$ such that $m_{j}< m_{*}$ were eliminated according to assumption in Case $a$. Let the arms surviving till $m_{*}$ round be denoted by $A^{'}$. This leaves any arm $a_{b}$ such that $m_{b}\geq m_{*}$ to still survive and eliminate arm ${*}$ in round $m_{*}$. Let such arms that survive ${*}$ belong to $A^{''}$. Also maximal regret per step after eliminating ${*}$ is the maximal $\Delta_{j}$ among the remaining arms ${j}$ with $m_{j}\geq m_{*}$.  Let $m_{b}=\min\lbrace m|\sqrt{2\epsilon_{m}}<\dfrac{\Delta_{b}}{4}\rbrace$. Hence, the maximal regret after eliminating the arm ${*}$ is upper bounded by, 
\begin{align*}
&\sum_{m_{*}=0}^{max_{j\in A^{'}}m_{j}}\sum_{i\in A^{''}:m_{i}>m_{*}}\bigg(\dfrac{2}{(\psi  T\epsilon_{m_{*}})^{\rho_{a}}} \bigg).T\max_{j\in A^{''}:m_{j}\geq m_{*}}{\Delta}_{j}\\
%%%%%%%%%%%%%%%%%%%%%%%%%%%
&\leq\sum_{m_{*}=0}^{max_{j\in A^{'}}m_{j}}\sum_{i\in A^{''}:m_{i}>m_{*}}\bigg(\dfrac{2}{(\psi  T\epsilon_{m_{*}})^{\rho_{a}}} \bigg).T.4\sqrt{2}\sqrt{\epsilon_{m_{*}}}\\
%%%%%%%%%%%%%%%%%%%%%%%%%%
&\leq\sum_{m_{*}=0}^{max_{j\in A^{'}}m_{j}}\sum_{i\in A^{''}:m_{i}>m_{*}}8\sqrt{2}\bigg(\dfrac{T^{1-\rho_{a}}}{\psi^{\rho_{a}}\epsilon_{m_{*}}^{\rho_{a}-\frac{1}{2}}} \bigg)\\
%%%%%%%%%%%%%%%%%%%%%%%%%%
&\leq\sum_{i\in A^{''}:m_{i}>m_{*}}\sum_{m_{*}=0}^{\min{\lbrace m_{i},m_{b}\rbrace}}\bigg(\dfrac{8\sqrt{2}T^{1-\rho_{a}}}{\psi^{\rho_{a}}2^{-(\rho_{a}-\frac{1}{2})m_{*}}} \bigg)\\
%%%%%%%%%%%%%%%%%%%%%%%%%%
&\leq\sum_{i\in A^{'}}\bigg(\dfrac{8\sqrt{2}T^{1-\rho_{a}}}{\psi^{\rho_{a}}2^{-(\rho_{a}-\frac{1}{2})m_{*}}} \bigg)+\sum_{i\in A^{''}\setminus A^{'}}\bigg(\dfrac{8\sqrt{2}T^{1-\rho_{a}}}{\psi^{\rho_{a}}2^{-(\rho_{a}-\frac{1}{2})m_{b}}} \bigg)\\
%%%%%%%%%%%%%%%%%%%%%%%%%%
&\leq\sum_{i\in A^{'}}\bigg(\dfrac{4T^{1-\rho_{a}}*2^{\rho_{a}-\frac{1}{2}}}{\psi^{\rho_{a}}\Delta_{i}^{8\sqrt{2}\rho_{a}-1}} \bigg)+\sum_{i\in A^{''}\setminus A^{'}}\bigg(\dfrac{8\sqrt{2}T^{1-\rho_{a}}*2^{\rho_{a}-\frac{1}{2}}}{\psi^{\rho_{a}}b^{2\rho_{a}-1}} \bigg)\\
%%%%%%%%%%%%%%%%%%%%%%%%%%
&\leq\sum_{i\in A^{'}}\bigg(\dfrac{T^{1-\rho_{a}}2^{\rho_{a}+\frac{7}{2}}}{\psi^{\rho_{a}}\Delta_{i}^{2\rho_{a}-1}} \bigg)+\sum_{i\in A^{''}\setminus A^{'}}\bigg(\dfrac{T^{1-\rho_{a}}2^{\rho_{a}+\frac{7}{2}}}{\psi^{\rho_{a}}b^{2\rho_{a}-1}} \bigg)\\
%%%%%%%%%%%%%%%%%%%%%%%%%%
&\overset{(a)}{\leq}\sum_{i\in A^{'}}16K +\sum_{i\in A^{''}\setminus A^{'}} 16K\\
%& = \sum_{i\in A^{'}}\bigg(\dfrac{ C_{2}(\rho_{a}) T^{1-\rho_{a}}}{\Delta_{i}^{4\rho_{a}-1}} \bigg)+\sum_{i\in A^{''}\setminus A^{'}}\bigg(\dfrac{C_{2(\rho_{a})}T^{1-\rho_{a}}}{b^{4\rho_{a}-1}} \bigg) \text{, where } C_2(x) = \frac{2^{2x+\frac{3}{2}}}{\psi^{x}}
\end{align*}

Again $(a)$ is obtained by substituting the values of $\psi$ and $\rho_a$. Summing up \textbf{Case a} and \textbf{Case b}, the total regret is given by,
\begin{align*}
 \E[R_{T}] \leq &\sum\limits_{i\in A:\Delta_{i} > b} \left\lbrace 12K + \bigg(\Delta_{i}+\dfrac{32\log{(\frac{T\Delta_i^2}{K})}}{\Delta_{i}}\bigg) + 16K\right\rbrace +\sum\limits_{i\in A:0 < \Delta_{i}\leq b} 16K + \max_{i\in A:\Delta_{i}\leq b}\Delta_{i}T
\end{align*}
\end{proof}



%\section{Why Clustering?}
%\label{App:E}
%
%In this section we want to specify the apparent use of clustering. The error bounds are shown in Table \ref{App:E:table:3}.
%
%\begin{table}[!h]
%\caption{Error Bound}
%\label{App:E:table:3}
%\begin{center}
%\begin{tabular}{p{1.4cm}p{10.3cm}p{3.5cm}}
%\multicolumn{1}{c}{\bf Elim Type} &\multicolumn{1}{c}{\bf Error Bound} &\multicolumn{1}{c}{\bf Remarks} \\
%\hline \\
%Only Arm Elimination (ClusUCB-AE)	& \begin{align*}\underbrace{\sum_{i\in A:\Delta_{i} > b}\bigg(\dfrac{C_{2}(\rho_{a})T^{1-\rho_{a}}}{\Delta_{i}^{4\rho_{a} -1}} \bigg)}_{\text{Case b2, Proposition \ref{proofTheorem:Prop:1}}} + \underbrace{\sum_{i\in A:0 < \Delta_{i}\leq b}\bigg( \dfrac{C_{2}(\rho_{a})T^{1-\rho_{a}}}{b^{4\rho_{a} -1}} \bigg)}_{\text{Case b2, Proposition \ref{proofTheorem:Prop:1}}}\end{align*}  & With $\rho_{a}=\frac{1}{2},$ and $\psi=\frac{T}{196 \log K}$ this gives $300\sqrt{KT}+300\sqrt{KT\log K}$. Hence, this has an order of $O(\sqrt{KT\log K})$.\\
%\hline\\
%%%%%%%%%%%%%%%%%%%%%%%%%%%%%%%%%%%%%%%%%%%%%%%%%%%%%%%%%%%%%%%%%%%%%%%%%%%
%%%%%%%%%%%%%%%%%%%%%%%%%%%%%%%%%%%%%%%%%%%%%%%%%%%%%%%%%%%%%%%%%%%%%%%%%%%
%Arm \& Cluster Elimination (ClusUCB) 	& \begin{align*}  \underbrace{\sum_{i\in A_{s^{*}}:\Delta_{i} > b}\bigg(\dfrac{C_{2}(\rho_{a})T^{1-\rho_{a}}}{\Delta_{i}^{4\rho_{a}-1}} \bigg)+ \sum_{i\in A_{s^{*}}:0\leq\Delta_{i}\leq b}\bigg(\dfrac{C_{2}(\rho_{a})T^{1-\rho_{a}}}{b^{4\rho_{a} -1}} \bigg)}_{\text{Case b2, Arm Elim, Theorem \ref{Result:Theorem:1}}}\\   
% + \underbrace{\sum_{i\in A\setminus A_{s^*}:\Delta_{i} > b}\bigg(\dfrac{2C_{2}(\rho_{s})T^{1-\rho_{s}}}{\Delta_{i}^{4\rho_{s}-1}} \bigg)+ \sum_{i\in A\setminus A_{s^*}:0\leq\Delta_{i}\leq b}\bigg(\dfrac{2C_{2}(\rho_{s})T^{1-\rho_{s}}}{b^{4\rho_{s} -1}} \bigg)}_{\text{Case b3+b4, Clus Elim, Theorem \ref{Result:Theorem:1}}} \end{align*} & With $\rho_{a}=\frac{1}{2}$, $\rho_{s}=\frac{1}{2}, p=\lceil \frac{K}{\log K}\rceil$ and $\psi=\frac{T}{196 \log K}$ this gives $\frac{300 \sqrt{T}\log K^{\frac{3}{2}} }{\sqrt{K}} + \frac{300 \sqrt{T}\log K}{\sqrt{K}} + 600 \frac{K}{K+\log K}\sqrt{KT\log K} + 600 \frac{K}{K+\log K}\sqrt{KT}$. So we can reduce the error bound to $O(\frac{K}{K+\log K}\sqrt{KT\log K})$.\\
%\hline
%\end{tabular}
%\end{center}	
%\end{table}
%
%While looking at the error terms in Table~\ref{App:E:table:3}, we see that using just arm elimination (ClusUCB-AE) the elimination error bound is more than using both arm and cluster  elimination simultaneously (ClusUCB). 

\section{Proof of Lemma 3}
\label{App:Lemma3}
\begin{lemma}
\label{proofTheorem:Lemma:3}
If $m_i = min\lbrace m|\sqrt{2\epsilon_{m} } < \dfrac{\Delta_i}{4} \rbrace $, $c_{i}^{'} =\sqrt{\frac{\log (\psi T\epsilon_{m_{i}})}{z_{i}}}$ and $n_{m_i}=\dfrac{\log{(\psi T\epsilon_{m_{i}}^{2})}}{2\epsilon_{m_{i}}}$ then, 
\begin{align*}
\Pb\lbrace c^{*'} > c_i^{'} \rbrace \leq \dfrac{2}{T(\psi \epsilon_{m_i})^2}.
\end{align*}
\end{lemma}

\begin{proof}
From the definition of $c_i^{'}$ we know that $c_i^{'}\propto \frac{1}{z_i}$ as $\psi$ and $T$ are constants. Therefore in the $m_i$-th round,
\begin{align*}
\Pb\lbrace c^{*'} > c_i^{'} \rbrace \leq  \Pb\lbrace  z^* < z_i  \rbrace \leq \sum_{m=0}^{m_i}\sum_{z_i =1}^{n_{m_i}}\sum_{z^* =1}^{n_{m_i}}\bigg(\Pb\lbrace \hat{r}^* < r^* - c^{*'}\rbrace + \Pb\lbrace \hat{r}_i > r_i + c_i^{'}\rbrace\bigg)
\end{align*}

Now, applying Chernoff-Hoeffding bound we can show that,
\begin{align*}
&\Pb\lbrace \hat{r}^* < r^* - c^{*'}\rbrace \leq \exp(- 2(c^{*'})^2 n^*)\leq \frac{1}{(\psi T\epsilon_{m_i})^2} \\ 
&\Pb\lbrace \hat{r}_i > r_i + c_i^{'}\rbrace \leq \exp(- 2(c_i^{'})^2 n_i)\leq \frac{1}{(\psi T\epsilon_{m_i})^2}
\end{align*}

Hence, summing everything up, 
\begin{align*}
\Pb\lbrace c^{*'} > c_i^{'} \rbrace \leq \sum_{m=0}^{m_i}\sum_{z_i =1}^{n_{m_i}}\sum_{z^* =1}^{n_{m_i}} \dfrac{2}{(\psi T\epsilon_{m_i})^2} \leq \dfrac{2T}{(\psi T\epsilon_{m_i})^2} \leq \dfrac{2}{T(\psi\epsilon_{m_i})^2}.
\end{align*}

\end{proof}


\section{Proof of Lemma 4}
\label{App:Lemma4}
\begin{lemma}
\label{proofTheorem:Lemma:4}
If $m_i = min\lbrace m|\sqrt{2\epsilon_{m} } < \dfrac{\Delta_i}{4} \rbrace $, $c_{i}^{'} =\sqrt{\frac{\log (\psi T\epsilon_{m_{i}})}{ z_{i}}}$ and $n_{m_i}=\dfrac{\log{(\psi T\epsilon_{m_{i}}^{2})}}{2\epsilon_{m_{i}}}$ then, 
\begin{align*}
\Pb\lbrace z_i > n_{m_i} \rbrace \leq \dfrac{2}{T(\psi \epsilon_{m_i})^2}.
\end{align*}
\end{lemma}

\begin{proof}
Following a similar argument as in Lemma \ref{proofTheorem:Lemma:3}, we can show that in the $m_i$-th round,
\begin{align*}
\Pb\lbrace z_i > n_{m_i} \rbrace &\leq \sum_{m=0}^{m_i}\sum_{z_i =1}^{n_{m_i}}\sum_{z^* =1}^{n_{m_i}}\bigg(\Pb\lbrace \hat{r}^* < r^* - c^{*'}\rbrace + \Pb\lbrace \hat{r}_i > r_i + c_i^{'}\rbrace\bigg) \\
%%%%%%%%%%%%%%%%%%%%%%%%%%%%%%%%%%%%
&\leq \sum_{m=0}^{m_i}\sum_{z_i =1}^{n_{m_i}}\sum_{z^* =1}^{n_{m_i}} \dfrac{2}{(T\psi \epsilon_{m_i})^2} \leq \dfrac{2 T}{(T\psi \epsilon_{m_i})^2}\leq  \dfrac{2}{T(\psi \epsilon_{m_i})^2}.
\end{align*}
\end{proof}

\section{Proof of Lemma 5}
\label{App:Lemma5}
\begin{lemma}
\label{proofTheorem:Lemma:5}
If $T\geq K^{2.4}$ and $	\Delta_{i} > \sqrt{\frac{e}{T}}$ then, $\dfrac{K^4}{T^2 \Delta_i^3} \leq K^{2.8}$.
\end{lemma}

\begin{proof}
Substituting the value of $\Delta_{i}=\sqrt{\frac{e}{T}}$ in the above expression we get,
\begin{align*}
\dfrac{K^4 T^{\frac{3}{2}}}{T^2 (e)^{\frac{3}{2}}} \leq \frac{K^{4}}{\sqrt{T}} \overset{(a)}{\leq}  K^{2.8}
\end{align*}  
\end{proof}

Here, $(a)$ happens because $T\geq K^{2.4}$.


\section{Proof of Theorem 2}
\label{sec:proofTheorem2}
\begin{proof}
We follow the same steps as in Therorem \ref{Result:Theorem:1}. We again recall the definition of some of our notations. Let $A^{'}=\lbrace i \in A,\Delta_{i}> b\rbrace$,  $A^{''}=\lbrace i \in A, \Delta_{i} > 0\rbrace$, $A^{'}_{s_{k}}=\lbrace i \in A_{s_{k}},\Delta_{i}> b\rbrace$ and $A^{''}_{s_{k}}=\lbrace i \in A_{s_{k}}, \Delta_{i} > 0 \rbrace$. $C_{g}$ is the cluster set containing max payoff arm from each cluster in $g$-th round. The arm having the true highest payoff in a cluster $s_{k}$ is denote by $a_{\max_{s_{k}}}$. Let for each sub-optimal arm ${i}\in A$, $m_{i}=\min{\lbrace m|\sqrt{2\epsilon_{m}} < \frac{\Delta_{i}}{4} \rbrace}$ and let for each cluster $s_{k}\in S$, $g_{s_{k}}=\min{\lbrace g|\sqrt{2\epsilon_{g}} < \frac{\Delta_{a_{\max_{s_{k}}}}}{4} \rbrace}$. Let $\check{A}=\lbrace {i}\in A^{'} | {i}\in s_{k} , \forall s_{k}\in S \rbrace$. Moreover we define $z_i$ as the total number of times an arm has been pulled. The analysis proceeds by considering the contribution to the regret in each of the following cases:

\textbf{Case a:} \textit{Some sub-optimal arm ${i}$ is not eliminated in round $\max(m_{i},g_{s_{k}})$ or before, with the optimal arm ${*}\in C_{\max(m_{i},g_{s_{k}})}$.}
We consider an arbitrary sub-optimal arm ${i}$ and analyze the contribution to the regret when $i$ is not eliminated in the following exhaustive sub-cases:\\
\textbf{Case a1:} \textit{In round $\max(m_{i},g_{s_{k}})$, ${i} \in s^{*}$.}

Note that when the following four conditions hold, arm $i$ gets eliminated:
\begin{align}
\hat{r}_{i}  \le r_{i} + c_{i} \text{ and } 
 \hat{r}^{*}\geq  r^{*} - c^{*} \text{ and } c^{*'} \leq c_i^{'} \text{ and } z_i \geq n_{m_i}, \label{eq:armelim-casea2}
\end{align}
where  $c_{i} = \sqrt{\frac{\rho_{a}\log (\psi T\epsilon_{m_{i}})}{2 z_{i}}}$ and $c_{i}^{'} = \sqrt{\frac{\log (\psi T\epsilon_{m_{i}})}{ z_{i}}}$. The arm $i$ gets eliminated because 
  \begin{align*}
\hat{r}_{i} +c_{i} & \leq r_{i} + 2c_{i} < r_{i} + \Delta_{i} - 2c_{i}\\
 &\leq r^{*} -2c_{i} \leq r^{*} - 2c^* \leq \hat{r}^{*} - c^{*}  .
  \end{align*}
In the above, we have used the fact that since $z_i\geq n_{m_i}$, so $ c_{i} = \sqrt{\epsilon_{m_{i}+1}} < \frac{\Delta_{i}}{4}$, from Lemma~\ref{proofTheorem:Lemma:2}. From the foregoing, we have to bound the events complementary to that in \eqref{eq:armelim-casea2} for an arm $i$ to not get eliminated. Considering Chernoff-Hoeffding bound this is done as follows:
  \begin{align*}
&\mathbb{P}\left(\hat{r}_{i}\geq r_{i} + c_{i}\right)\leq \exp(-2c_{i}^{2}z_{i})\\
&\leq \exp(-2 * \frac{\rho_{a}\log (\psi T\epsilon_{m_{i}})}{2 z_{i}}*z_{i})
\leq \frac{1}{(\psi T\epsilon_{m_{i}})^{\rho_{a}}}   
  \end{align*}
Similarly, we have $\mathbb{P}\left(\hat{r}^{*}\leq r^{*} - c^{*}\right)\leq \dfrac{1}{(\psi  T\epsilon_{m_{i}})^{\rho_{a}}}$. Again, applying Lemma \ref{proofTheorem:Lemma:3} we can show that in the $m_i$-th round $\Pb\lbrace c^{*'} > c_i^{'} \rbrace \leq \dfrac{2}{T(\psi \epsilon_{m_i})^2}$. Similarly, applying Lemma \ref{proofTheorem:Lemma:4} we can show that $\Pb\lbrace z_i > n_{m_i} \rbrace \leq \dfrac{2}{T(\psi \epsilon_{m_i})^2}$. Thus, the probability that a sub-optimal arm $i$ is not eliminated in any round on or before $m_{i}$ is bounded above by $\left(\dfrac{2}{(\psi T\epsilon_{m_{i}})^{\rho_{a}}} + \dfrac{4}{T(\psi \epsilon_{m_i})^2}\right)$. Summing up over all arms in $A_{s^{*}}^{'}$ in conjunction with a simple bound of $T\Delta_{i}$ for each arm we obtain,
   \begin{align*}
&\sum_{i\in A_{s^{*}}^{'}}\left(\dfrac{2T\Delta_{i}}{(\psi T\epsilon_{m_{i}})^{\rho_{a}}} + \dfrac{4}{T(\psi \epsilon_{m_i})^2}\right)
\leq\sum_{i\in A_{s^{*}}^{'}}\left(\frac{2T\Delta_{i}}{(\psi T\dfrac{\Delta_{i}^{2}}{32})^{\rho_{a}}} +  \dfrac{4T\Delta_{i}}{T(\psi \epsilon_{m_i})^2}\right)\\
%%%%%%%%%%%%%%%%%%%%%%%%%%%%%%
&\overset{(a)}{\leq} \sum_{i\in A_{s^{*}}^{'}}\left(\frac{2T\Delta_{i}}{(\dfrac{T^2}{K^2}\dfrac{\Delta_{i}^{2}}{32})^{\frac{1}{2}}} + \dfrac{4}{(\dfrac{T}{K^2} \dfrac{\Delta_{i}^{3}}{32})^2}\right)  \leq \sum_{i\in A_{s^{*}}^{'}}\left(8\sqrt{2} K + \dfrac{4096 K^4}{T^2 \Delta_i^3}\right)\\
%%%%%%%%%%%%%%%%%%%%%%%%%%%%%%
&\overset{(b)}{\leq}  \sum_{i\in A_{s^{*}}^{'}}\left(8\sqrt{2} K + 4096 K^{2.8}\right)
%\leq \sum_{i\in A_{s^{*}}^{'}}\bigg(\frac{C_{1}(\rho_{a})T^{1-\rho_{a}}}{\Delta_{i}^{4\rho_{a}-1}}\bigg) %\text{, where } C_1(x) = \frac{2^{1+4x}}{\psi^{x}}
   \end{align*}
   
  Here, in $(a)$ we substituted the value $\rho_a$ and $\psi$ and $(b)$ is obtained by applying Lemma  \ref{proofTheorem:Lemma:5}.

%%%%%%%%%%%%%%%%%%%%%%%%%%%%%%%%%%%%%%%%%%%%%%%%%%%%%%%%%%%%%%%%%%%%%%%%%%%%%%%%%%%%%%%%%%%%%%%   



%%%%%%%%%%%%%%%%%%%%%%%%%%%%%%%%%%%%%%%%%%%%%%%%%%%%%%%%%%%%%%%%%%%%%%%%%%%%%%%%%%%%%%%%%%%%%%%   
\textbf{Case a2:} \textit{In round $\max(m_{i},g_{s_{k}})$, ${i} \in s_k$ for some $s_k \ne s^{*}$.}

Following a parallel argument like in Case $a1$, we have to bound the following two events of arm $a_{\max_{s_{k}}}$ not getting eliminated on or before $g_{s_{k}}$-th round,
\begin{align*}
  \hat{r}_{a_{\max_{s_{k}}}} \geq r_{a_{\max_{s_{k}}}} +c_{a_{\max_{s_{k}}}} \text{ and } \hat{r}^{*} \leq r^{*} - c^{*} 
\end{align*} 

We can prove using Chernoff-Hoeffding bounds and considering independence of events mentioned above, that for $c_{a_{\max_{s_{k}}}}=\sqrt{\frac{\rho_{s} \log (\psi T\epsilon_{g_{s_{k}}})}{2 z_{a_{\max_{s_{k}}}}}}$ and  $z_{a_{\max_{s_{k}}}} > n_{g_{s_{k}}}=\frac{\log{(\psi T\epsilon_{g_{s_{k}}}^{2})}}{2\epsilon_{g_{s_{k}}}}$ the probability of the above two events is bounded by $\left(\dfrac{2}{(\psi T\epsilon_{g_{s_{k}}})^{\rho_{s}}} + \dfrac{4}{T(\psi \epsilon_{g_{s_k}})^2}\right)$.

  Now, for any round $g_{s_{k}}$, all the elements of $C_{\max(m_{i},g_{s_{k}})}$ are the respective maximum payoff arms of their cluster $s_{k}, \forall s_{k}\in S$, and since clusters are fixed so we can bound the maximum probability that a sub-optimal arm ${i}\in A^{'}$  and ${i}\in s_{k}$ such that $a_{\max_{s_{k}}}\in C_{g_{s_{k}}}$ is not eliminated on or before the $g_{s_{k}}$-th round by the same probability as above. Summing up over all $p$ clusters and bounding the regret for each arm $i\in A_{s_{k}}^{'}$ trivially by $T\Delta_{i}$,
 \begin{align*}
 &\sum_{k=1}^{p}\sum_{i\in A_{s_{k}}^{'}}\left(\frac{2T\Delta_{i}}{(\psi T\frac{\Delta_{i}^{2}}{32})^{\rho_{s}}} + \dfrac{4T\Delta_{i}}{T(\psi \frac{\Delta_{i}^{2}}{32})^2}\right)\leq \sum_{i\in A^{'}}\left(8\sqrt{2} K + 4096 K^{2.8}\right)
 %= \sum_{i\in A^{'}}\bigg( \frac{2T\Delta_{i}}{(\psi  T\frac{\Delta_{i}^{2}}{32})^{\rho_{s}}} \bigg) \\
 %& \overset{(a)}{\leq} \sum_{i\in A^{'}}\left(\frac{2T\Delta_{i}}{(\frac{T^2}{K^2}\frac{\Delta_{i}^{2}}{32})^{\frac{1}{2}}}\right) = \sum_{i\in A^{'}} \left( 8\sqrt{2}K \right)
% &\leq \sum_{i\in A^{'}}\bigg(\frac{2^{1+4\rho_{s}}T^{1-\rho_{s}}}{\psi^{\rho_{s}}\Delta_{i}^{4\rho_{s}-1}}\bigg) = \sum_{i\in A^{'}}\frac{C_{1}(\rho_{s})T^{1-\rho_{s}}}{\Delta_{i}^{4\rho_{s}-1}}
 \end{align*}

%Again we obtain $(a)$ by substituting the value of $\rho_s$ and $\psi$.

Summing the bounds in Cases $a1-a2$ and observing that the bounds in the aforementioned cases hold for any round $C_{\max \lbrace m_i,g_{s_k}\rbrace}$, we obtain the following contribution to the expected regret from case a:
   %Taking summation of the events mentioned above($a1$-$a4$) gives us an upper bound on the regret given that the optimal arm $a^{*}$ is still surviving, 
\begin{align*}
 \sum_{i\in A_{s^*}^{'}} 4108 K^{2.8} + \sum_{i\in A^{'}} 4108 K^{2.8}
%&\sum_{i\in A_{s^*}} \frac{C_{1}(\rho_{a})T^{1-\rho_{a}}}{\Delta_{i}^{4\rho_{a}-1}} + \sum_{i\in A^{'}}\bigg(\frac{C_{1}(\rho_{s})T^{1-\rho_{s}}}{\Delta_{i}^{4\rho_{s}-1}}\bigg)
\end{align*}

%%%%%%%%%%%%%%%%%%%%%%%%%%%%%%%%%%%%%%%%%%%%%%%%%%%%%%%%%%%%%%%%%%%%%%%%%%%%%%%%%%%%%%%%%%%%
\textbf{Case b:} \textit{For each arm $i$, either ${i}$ is eliminated in round $\max (m_{i},g_{s_{k}})$ or before or there is no optimal arm ${*}$ in $C_{\max(m_{i},g_{s_{k}})}$.} \\
\textbf{Case b1:} \textit{${*}\in C_{\max(m_{i},g_{s_{k}})}$ for each arm $i \in A'$ and cluster $s_k \in \check A$.} 
The condition in the case description above implies the following: \\
\begin{inparaenum}[\bfseries (i)]
\item each sub-optimal arm ${i}\in A^{'}$ is  eliminated on or before $\max (m_{i},g_{s_{k}})$ and hence  pulled not more than $z_i \leq n_{m_{i}}$ number of times.\\
\item each sub-optimal cluster $s_k \in \check A$ is  eliminated on or before $\max (m_{i},g_{s_{k}})$ and hence  pulled not more than $ z_{a_{\max_{s_{k}}}} \leq n_{g_{s_{k}}}$ number of times.
\end{inparaenum}

Hence, proceeding similarly to case $b1$ in Therorem \ref{Result:Theorem:1}, we can show that the maximum regret suffered due to pulling of a sub-optimal arm or a sub-optimal cluster is no more than the following:
 \begin{align*}
 &\sum_{i\in A^{'}}\Delta_{i}\bigg\lceil\dfrac{\log{(\psi T\epsilon_{m_{i}}^{2})}}{2\epsilon_{m_{i}}}\bigg\rceil 
\!+\! \sum_{k=1}^{p}\sum_{i\in A_{s_{k}}^{'}}\Delta_{i}\bigg\lceil\dfrac{\log{(\psi T\epsilon_{g_{s_{k}}}^{2})}}{2\epsilon_{g_{s_{k}}}}\bigg\rceil \\
%%%%%%%%%%%%%%%%%%%%
&\overset{a}{\leq}\sum_{i\in A^{'}}\Delta_{i}\bigg(1+\dfrac{16\log{\left(\psi T\left(\frac{\Delta_{i}}{4\sqrt{2}}\right)^{4}\right)}}{\Delta_{i}^{2}}\bigg) 
\quad+ \sum_{i\in A^{'}}\Delta_{i}\bigg(1+\dfrac{16\log{\left(\psi T\left(\frac{\Delta_{i}}{4\sqrt{2}}\right)^{4}\right)}}{\Delta_{i}^{2}}\bigg)
\\
%%%%%%%%%%%%%%%%%%%%
 &\overset{b}{\leq} \sum_{i\in A^{'}}\!\bigg[ 2\Delta_{i}+\dfrac{16(\log{(\frac{T^2}{K^2}\frac{\Delta_{i}^{4}}{1024})} + \log{(\frac{T^2}{K^2}\frac{\Delta_{i}^{4}}{1024})})}{\Delta_{i}} \bigg] \leq \sum_{i\in A^{'}}\!\bigg[ 2\Delta_{i}+\dfrac{32\left(\log{(\frac{T\Delta_{i}^2}{K})} + \log{(\frac{T\Delta_{i}^2}{K})}\right)}{\Delta_{i}} \bigg]
%  \\
% & \qquad \qquad +\dfrac{32\rho_{s}\log{(\psi T\dfrac{\Delta_{i}^{4}}{16\rho_{s}^{2}})}}{\Delta_{i}}\bigg\rbrace 
 \end{align*}
In the above, the $(a)$ follows since $\sqrt{2\epsilon_{m_{i}}} < \frac{\Delta_{i}}{4}$ and $\sqrt{2\epsilon_{n_{g_{s_{k}}}}} < \frac{\Delta_{a_{\max_{s_{k}}}}}{4}$ and $(b)$ is obtained by substituting the values of $\rho_a,\rho_s$ and $\psi$.

%&\leq\Delta_{i}\bigg(1+\dfrac{32\rho_{a}\log{(\psi T\dfrac{\Delta_{i}^{4}}{16\rho_{a}^{2}})}}{\Delta_{i}^{2}}\bigg)\\
%&\leq\Delta_{i}\bigg\lceil\dfrac{2\log{(\psi T(\dfrac{\Delta_{i}}{2\sqrt{\rho_{a}})^{4})}}}{(\dfrac{\Delta_{i}}{2\sqrt{\rho_{a}}})^{2}}\bigg\rceil \\
%\text{, since } \sqrt{\rho_{a}\epsilon_{m_{i}}}\leq\dfrac{\Delta_{i}}{2}
 
%%%%%%%%%%%%%%%%%%%%%%%%%%%%%%%%%%%%%%%%%%%%%%%%%%%%%%%%%%%%%%%%%%%%%%%%%%%%%%%%%%%%%%%%%%%%%%%   
%\textbf{Case b2:} \textit{Optimal arm $a^{*}$ is eliminated by a sub-optimal arm.}\\
  %
	%This, can happen in $3$ ways,
%\newline
\textbf{Case b2:} \textit{${*}$ is eliminated by some sub-optimal arm in $s^*$} \\
%In this case, we are concerned with the arm elimination condition only. 
Optimal arm $*$ can get eliminated by some sub-optimal arm $i$ only if arm elimination condition holds, i.e., 
\begin{align*}
\hat r_{i} - c_{i} > \hat{r}^{*}+ c^{*},
\end{align*}
where, as mentioned before, $c_{i}  =\sqrt{\frac{\rho_{a}\log (\psi T\epsilon_{m_{i}})}{2 n_{i}}}$.
From analysis in Case $a1$, notice that, if \eqref{eq:armelim-casea2} holds in conjunction with the above, arm $i$ gets eliminated. Also, recall from Case $a1$ that the events complementary to \eqref{eq:armelim-casea2} have low-probability and can be upper bounded by $\left(\dfrac{2}{(\psi  T\epsilon_{m_{*}})^{\rho_{a}}} + \dfrac{4}{T(\psi\epsilon_{m_{*}})^2}\right)$. Moreover, a sub-optimal arm that eliminates $*$ has to survive until round $m_*$. In other words, all arms ${j}\in s^{*}$ such that $m_{j} < m_{*}$ are eliminated on or before $m_*$ (this corresponds to case b1). Let, the arms surviving till $m_{*}$ round be denoted by $A^{'}_{s^{*}}$. This leaves any arm $a_{b}$ such that $m_{b}\geq m_{*} $ to still survive and eliminate arm ${*}$ in round $m_{*}$. Let, such arms that survive ${*}$ belong to $A^{''}_{s^{*}}$. Also maximal regret per step after eliminating ${*}$ is the maximal $\Delta_{j}$ among the remaining arms in $A^{''}_{s^{*}}$ with $m_{j}\geq m_{*}$.  Let $m_{b}=\min\lbrace m|\sqrt{2\epsilon_{m}}<\frac{\Delta_{b}}{4}\rbrace$. Hence, the maximal regret after eliminating the arm ${*}$ is upper bounded by, 
%Let $C_2(x) = \frac{2^{2x+\frac{3}{2}}}{\psi^{x}}$.
\begin{align*}
&\sum_{m_{*}=0}^{max_{j\in A^{'}_{s^{*}}}m_{j}}\sum_{\substack{i\in A^{''}_{s^{*}}: \\ m_{i}\geq m_{*}}}\bigg(\dfrac{2}{(\psi  T\epsilon_{m_{*}})^{\rho_{a}}} + \dfrac{4}{T(\psi\epsilon_{m_{*}})^2}\bigg).T\max_{\substack{j\in A^{''}_{s^{*}}: \\ m_{j}\geq m_{*}}}{\Delta}_{j}\\
%%%%%%%%%%%%%%%
&\leq\sum_{m_{*}=0}^{max_{j\in A^{'}_{s^{*}}}m_{j}}\sum_{i\in A^{''}_{s^{*}}:m_{i} \geq m_{*}}\bigg(\dfrac{2}{(\psi  T\epsilon_{m_{*}})^{\rho_{a}}} + \dfrac{4}{T(\psi\epsilon_{m_{*}})^2} \bigg).T.4\sqrt{2\epsilon_{m_{*}}} \\
%%%%%%%%%%%%%%%
&\leq\sum_{m_{*}=0}^{max_{j\in A^{'}_{s^{*}}}m_{j}}\sum_{i\in A^{''}_{s^{*}}:m_{i} \geq m_{*}}8\sqrt{2}\bigg(\dfrac{T^{1-\rho_{a}}}{\psi^{\rho_{a}}\epsilon_{m_{*}}^{\rho_{a}-\frac{1}{2}}} + \dfrac{2}{\psi^2 \epsilon_{m_{*}}^{2-\frac{1}{2}}}\bigg)\\
%%%%%%%%%%%%%%%
&\leq\sum_{i\in A^{''}_{s^{*}}:m_{i} \geq m_{*}}\sum_{m_{*}=0}^{\min{\lbrace m_{i},m_{b}\rbrace}}\bigg(\dfrac{8\sqrt{2} T^{1-\rho_{a}}}{\psi^{\rho_{a}}2^{-(\rho_{a}-\frac{1}{2})m_{*}}} + \dfrac{16\sqrt{2}}{\psi^2 2^{-\frac{3}{2}m_{*}}}\bigg)\\
%%%%%%%%%%%%%%
&\!\leq\!\!\sum_{i\in A^{'}_{s^{*}}}\frac{8\sqrt{2} T^{1-\rho_{a}}}{\psi^{\rho_{a}}2^{-(\rho_{a}-\frac{1}{2})m_{*}}}\! +\!\!\!\sum_{i\in A^{''}_{s^{*}}\setminus A^{'}_{s^{*}}}\!\frac{8\sqrt{2} T^{1-\rho_{a}}}{\psi^{\rho_{a}}2^{-(\rho_{a}-\frac{1}{2})m_{b}}} + \sum_{i\in A^{'}_{s^{*}}}\dfrac{16\sqrt{2}}{\psi^2 2^{-\frac{3}{2}m_{*}}} + \sum_{i\in A^{''}_{s^{*}}\setminus A^{'}_{s^{*}}}\dfrac{16\sqrt{2}}{\psi^2 2^{-\frac{3}{2}m_{b}}} \\
%%%%%%%%%%%%%%
&\!\leq\!\!\sum_{i\in A^{'}_{s^{*}}}\frac{T^{1-\rho_{a}}2^{\rho_{a}+\frac{7}{2}}}{\psi^{\rho_{a}}\Delta_{i}^{2\rho_{a}-1}} \!+\!\!\!\sum_{i\in A^{''}_{s^{*}}\setminus A^{'}_{s^{*}}}\!\!\frac{T^{1-\rho_{a}}2^{\rho_{a}+\frac{7}{2}}}{\psi^{\rho_{a}}b^{2\rho_{a}-1}} +  \sum_{i\in A^{'}_{s^{*}}}\dfrac{46}{\psi^2 \Delta_i^{\frac{3}{2}}} + \sum_{i\in A^{''}_{s^{*}}\setminus A^{'}_{s^{*}}}\dfrac{46}{\psi^2 b^{\frac{3}{2}}}\\
%%%%%%%%%%%%%%
& \overset{(a)}{\leq} \sum_{i\in A^{'}_{s^{*}}} 16K \!+\!\!\!\sum_{i\in A^{''}_{s^{*}}\setminus A^{'}_{s^{*}}}\!\! 16K + \sum_{i\in A^{'}_{s^{*}}}\dfrac{46 K^4 T^{\frac{3}{2}}}{T^2 e^{\frac{3}{2}}} + \sum_{i\in A^{''}_{s^{*}}\setminus A^{'}_{s^{*}}}\dfrac{46 K^4 T^{\frac{3}{2}}}{T^2 e^{\frac{3}{2}}}\\
%%%%%%%%%%%%%%
& \overset{(b)}{\leq} \sum_{i\in A^{'}_{s^{*}}} 16K \!+\!\!\!\sum_{i\in A^{''}_{s^{*}}\setminus A^{'}_{s^{*}}}\!\! 16K + \sum_{i\in A^{'}_{s^{*}}} 46K^{2.8} \!+\!\!\!\sum_{i\in A^{''}_{s^{*}}\setminus A^{'}_{s^{*}}}\!\! 46K^{2.8} \leq \sum_{i\in A^{'}_{s^{*}}} 62K^{2.8} \!+\!\!\!\sum_{i\in A^{''}_{s^{*}}\setminus A^{'}_{s^{*}}}\!\! 62K^{2.8} 
%& = \sum_{i\in A^{'}_{s^{*}}}\dfrac{ C_{2}(\rho_{a}) T^{1-\rho_{a}}}{\Delta_{i}^{4\rho_{a}-1}} +\sum_{i\in A^{''}_{s^{*}}\setminus A^{'}_{s^{*}}}\dfrac{C_{2(\rho_{a})}T^{1-\rho_{a}}}{b^{4\rho_{a}-1}}.
\end{align*}

Here, in $(a)$ we substitute $\sqrt{\frac{e}{T}}\leq b <\Delta_i$ and $(b)$ is obtained by applying Lemma \ref{proofTheorem:Lemma:5}. 

%%%%%%%%%%%%%%%%%%%%%%%%%%%%%%%%%%%%%%%%%%%%%%%%%%%%%%%%%%%%%%%%%%%%%%%%%%%%%%%%%%%%%%%%%%%%%%%

%%%%%%%%%%%%%%%%%%%%%%%%%%%%%%%%%%%%%%%%%%%%%%%%%%%%%%%%%%%%%%%%%%%%%%%%%%%%%%%%%%%%%%%%%%%%%%%   
\textbf{Case b3:} \textit{$s^{*}$ is eliminated by some sub-optimal cluster.} 
Let $C_{g}^{'}=\lbrace a_{max_{s_{k}}}\in A^{'}|\forall s_{k}\in S \rbrace$ and $C_{g}^{''}=\lbrace a_{max_{s_{k}}}\in A^{''}|\forall s_{k}\in S \rbrace$. A sub-optimal cluster $s_k$ will eliminate $s^*$ in round $g_*$ only if the cluster elimination condition of Algorithm \ref{alg:clusucb} holds, which is the following when ${*}\in C_{g_{*}}$:
\begin{align}
\hat r_{a_{\max_{s_k}}} - c_{a_{\max_{s_k}}} > \hat{r}^{*}+ c^{*}.
\label{eq:caseb3-cluselim1}
\end{align}
Notice that when ${*}\notin C_{g_{*}}$, since $r_{a_{max_{s_{k}}}}>r^{*}$, the inequality in \eqref{eq:caseb3-cluselim1} has to hold for cluster $s_k$ to eliminate $s^*$.
As in case $b2$, the probability that a given sub-optimal cluster $s_k$ eliminates $s^*$ is upper bounded by  $\left(\dfrac{2}{(\psi T\epsilon_{g_{s^{*}}})^{\rho_{s}}} +  \dfrac{4}{T(\psi\epsilon_{g_{s^*}})^2}\right)$ and all sub-optimal clusters with $g_{s_{j}}< g_{*}$ are eliminated before round $g_*$. This leaves any arm $a_{\max_{s_{b}}}$ such that $g_{s_{b}}\geq g_{*}$ to still survive and eliminate arm ${*}$ in round $g_{*}$. Let, such arms that survive ${*}$ belong to $C_{g}^{''}$. Hence, following the same way as case $b2$,  the maximal regret after eliminating ${*}$ is,
 \begin{align*}
 \!\!\sum_{g_{*}=0}^{\max\limits_{a_{\max_{s_{j}}}\in C_{g}^{'}}g_{s_{j}}}\!\!\!\!\!\sum_{\substack{\scriptsize a_{\max_{s_{k}}}\in C_{g}^{''}: \\ g_{s_{k}} \geq g_{*}}}\bigg(\dfrac{2}{(\psi T\epsilon_{g_{s^{*}}})^{\rho_{s}}} + \dfrac{4}{T(\psi\epsilon_{g_{s^*}})^2}\bigg)T\max_{\substack{a_{\max_{s_{j}}}\in C_{g}^{''}: \\ g_{s_{j}}\geq g_{*}}}{\Delta}_{a_{\max_{s_{j}}}}
 \end{align*}
Using $A'\supset C_{g}^{'}$ and $A''\supset C_{g}^{''}$, we can bound the regret contribution from this case in a similar manner as Case $b2$ as follows:
\begin{align*}
% &\!\!\sum_{i\in A^{'}\setminus A_{s^*}^{'}}\frac{T^{1-\rho_{s}}2^{\rho_{s}+\frac{5}{2}}}{\psi^{\rho_{s}}\Delta_{i}^{2\rho_{s}-1}} 
% \!+\!\!\!\sum_{i\in A^{''}\setminus A^{'}\cup A_{s^*}^{'}}\!\!\!\!\frac{T^{1-\rho_{s}}2^{\rho_{s}+\frac{5}{2}}}{\psi^{\rho_{s}}b^{2\rho_{s}-1}} \\
 & \sum_{i\in A^{'}\setminus A_{s^*}^{'}} 62K^{2.8} +\sum_{i\in A^{''}\setminus A^{'}\cup A_{s^*}^{'}} 62K^{2.8}
% & = \sum_{i\in A^{'}\setminus A_{s^*}^{'}}\frac{C_{2}(\rho_{s})T^{1-\rho_{s}}}{\Delta_{i}^{4\rho_{s}-1}} +\sum_{i\in A^{''}\setminus A^{'}\cup A_{s^*}^{'}}\frac{C_{2}(\rho_{s})T^{1-\rho_{s}}}{b^{4\rho_{s}-1}} 
\end{align*}

%%%%%%%%%%%%%%%%%%%%%%%%%%%%%%%%%%%%%%%%%%%%%%%%%%%%%%%%%%%%%%%%%%%%%%%%%%%%%%%%%%%%%%%%%
\textbf{Case b4:} \textit{${*}$ is not in $C_{\max(m_{i},g_{s_{k}})}$, but belongs to $B_{\max(m_{i},g_{s_{k}})}$.}

In this case the optimal arm ${*}\in s^{*}$ is not eliminated, also $s^{*}$ is not eliminated. So, for all sub-optimal arms $i$ in $A_{s^*}^{'}$ which gets eliminated on or before $\max \lbrace m_{i},g_{s_{k}} \rbrace$ will get pulled no more than $ \left\lceil\dfrac{\log{(\psi T\epsilon_{m_{i}}^{2})}}{2\epsilon_{m_{i}}}\right\rceil$ number of times, which leads to the following bound the contribution to the expected regret, as in Case $b1$:
\begin{align*}
 &\sum_{i\in A_{s^*}^{'}}\bigg\lbrace \Delta_{i}+\dfrac{32\log{(\frac{T\Delta_i^2}{K})}}{\Delta_{i}} \bigg\rbrace 
\end{align*} 

For arms $a_i \notin s^*$, the contribution to the regret cannot be greater than that in Case $b3$. So the regret is bounded by,

\begin{align*}
\sum_{i\in A^{'}\setminus A_{s^*}^{'}} 62K^{2.8} +\sum_{i\in A^{''}\setminus A^{'} \cup A_{s^*}^{'}} 62K^{2.8}
%\sum_{i\in A^{'}\setminus A_{s^*}^{'}}\dfrac{C_{2}(\rho_{s})T^{1-\rho_{s}}}{\Delta_{i}^{4\rho_{s}-1}} +\sum_{i\in A^{''}\setminus A^{'} \cup A_{s^*}^{'}}\dfrac{C_{2}(\rho_{s})T^{1-\rho_{s}}}{b^{4\rho_{s}-1}}
\end{align*}
The main claim follows by summing the contributions to the expected regret from each of the cases above.
\end{proof}
