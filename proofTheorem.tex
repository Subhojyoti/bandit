%\label{sec:proofTheorem}

%We sketch the proof for Theorem \ref{Result:Theorem:1} here. 
%The proof involves the following steps:\\
%\textbf{\textit{Step 1:}}
%We analyze ClusUCB-AE, i.e., the variant of ClusUCB that uses arm elimination condition only. In other words, we bound the probability of sub-optimal arm elimination, which in turn bounds the expected regret of ClusUCB-AE (see Proposition \ref{proofSketch:Prop:1} below). 
%
%\textbf{\textit{Step 2:}}
%We analyze ClusUCB-CE, i.e., the variant of ClusUCB that uses cluster elimination condition only and pulls the best arm within the last leftover cluster.
%Proposition \ref{proofSketch:Prop:2} presents the expected regret for ClusUCB-CE (see Proposition \ref{proofSketch:Prop:2} below). 
%
%\textbf{\textit{Step 3:}}
%Finally we combine the individual bounds in the steps above to get the regret upper bound in Theorem \ref{Result:Theorem:1}.  
	




\begin{proof}
%The optimal cluster which contains $a^{*}$ is denoted by $s^{*}$. The subset of arms belonging to cluster $s_{k}$ is denoted by $A_{s_{k}}$ and similarly the subset of arms belonging to $s^{*}$ is denoted by $A_{s^{*}}$.
\todos[inline]{What is there to assume in the stmt: ``For random uniform allocation we will ...''} 
For random uniform allocation we will assume that each cluster $s_{k},\forall s_{k}\in S$, gets such an arm such that $r_{{max_{s_{k}}}}\geq r_{a_{i}},\forall i\in s_{k}$. Again, $r_{a^{*}}\geq r_{{max_{s_{k}}}}, \forall s_{k}\in S$. 
\todos[inline]{Algorithm \ref{alg:clusucb} has $\rho_{a},\rho_{s}\in (0,1]$ as constants and the stmt ``Here we take..'' is needless}
Here also we take $\rho_{a},\rho_{s}\in (0,1]$ as a constant in this proof whereby in Corollary \ref{Result:Corollary:1} and \ref{Result:Corollary:2} we use the different definitions. Let $A^{'}=\lbrace i \in A,\Delta_{i}> b\rbrace$,  $A^{''}=\lbrace i \in A,0 < \Delta_{i} \leq b\rbrace$, $A^{'}_{s_{k}}=\lbrace i \in A_{s_{k}},\Delta_{i}> b\rbrace$ and  $A^{''}_{s_{k}}=\lbrace i \in A_{s_{k}},0 < \Delta_{i} \leq b\rbrace$. $C_{g}$ is the cluster set containing max payoff arm from each cluster in $g$-th round. Let for each sub-optimal arm $a_{i}$, $m_{i}=\min{\lbrace m|\sqrt{\rho_{a}\epsilon_{m}}\leq \dfrac{\Delta_{i}}{2} \rbrace}$ and $g_{s_{k}}=\min{\lbrace g|\sqrt{\rho_{s}\epsilon_{g}}\leq \dfrac{\Delta_{a_{max_{s_{k}}}}}{2} \rbrace}$. 
\todos[inline]{define $g_{s_k}$ for each cluster $s_k \in S$}
\todos[inline]{$a_{max_{s_{k}}}$ is never defined. In the notation of Sec 2 $r_{max_{s_{k}}}$ is defined as the best arm within the cluster $s_k$}
\todos[inline]{Change max to the operator $\max$ everywhere}
\todos[inline]{critical error: $m_i$ defintion should be with $\sqrt{\rho_{a}\epsilon_{m}}< \frac{\Delta_{i}}{2}$. Same for $g$} 
%be the first round when $\sqrt{\rho_{a}\epsilon_{m}}\leq \dfrac{\Delta_{i}}{2}$ and for each sub-optimal cluster arm $a_{max_{s_{k}}}\in C_{g_{s_{k}}}$,
%So, $g_{s_{k}}$ be the first round when $\sqrt{\rho_{s}\epsilon_{g_{s_{k}}}}\leq \dfrac{\Delta_{a_{max_{s_{k}}}}}{2}$ where $a_{max_{s_{k}}}\in C_{g_{s_{k}}}$ is the maximum payoff arm in cluster $s_{k}$ and then $s_{k}$ gets eliminated
%The theoretical analysis remains same as we have always bounded the values of $\rho_{a}\in (0,1]$(see Appendix \ref{App:E}).
%Also we cluster the arms based on $\epsilon_{m}$.
% One vital point we point out is that, $\epsilon_{m}$(in proposition $3$) = $\epsilon_{g}$(in proposition $4$).
\subsection*{Case a: \textit{Some sub-optimal arm $a_{i}$ is not eliminated in round $max(m_{i},g_{s_{k}})$ or before and the optimal arm $a^{*}\in B_{m_{i}}$}}

\todos[inline]{The stmt ``In this case, we are looking at event of the maximum round till which atleast one of $m_{i}$ or $g_{s_{k}}$ did not happen.'' is unnecessary give the case caption above} 
	In this case, we are looking at event of the maximum round till which atleast one of $m_{i}$ or $g_{s_{k}}$ did not happen. So, a sub-optimal arm $a_{i}$ cannot get eliminated in $4$ ways,
	\todos[inline]{Get rid of enumerate to save space. You could just have the case labels, say case a1 and such}
\begin{enumerate}
\item $a_{i}$ in $s^{*}$ and $m_{i}$ does not happen. In arm elimination condition, given the choice of confidence interval $c_{m}$, we want to bound the probability of the event $\hat{r}_{i}+c_{m_{i}}\leq \hat{r}^{*}-c_{m_{i}}$.
\todos[inline]{The stmt ``bound the probability of the event $\hat{r}_{i}+c_{m_{i}}\leq \hat{r}^{*}-c_{m_{i}}$'' is wrong. We bound the complementary event using Hoeffding} 
\todos[inline]{The stmt ``$m_{i}$ does not happen'' makes no sense given that we are in round $\max(m_i,g_i)$}
\todos[inline]{What is $c_m$?}

  Now, $c_{m_{i}}=\sqrt{\frac{\rho_{a}\log (\psi T\epsilon_{m_{i}}^{2})}{2 n_{m_{i}}}}$.
  Putting the value of $n_{m_{i}}=\frac{2\log{(\psi T\epsilon_{m_{i}}^{2})}}{\epsilon_{m_{i}}}$ in $c_{m_{i}}$ we get,
  $c_{m_{i}} = \sqrt{\rho_{a}\epsilon_{m_{i}+1}} < \frac{\Delta_{i}}{4} $, as $\rho_{a}\in (0,1]$

\todos[inline]{In the following, $\exists a_i$ is spurious given that we are talking about arm $a_i$ all through in this case} 
  Again, $\exists a_{i} \in A_{s^{*}}^{'}$ such that, 
  \begin{align*}
\hat{r}_{i} + c_{m_{i}}&\leq r_{i} + 2c_{m_{i}} 
 \leq r_{i} + \Delta_{i} - 2c_{m_{i}}
 \leq \hat{r}^{*} - c_{m_{i}}   
  \end{align*}
\todos[inline]{$r_{i} + 2c_{m_{i}} 
 < r_{i} + \Delta_{i} - 2c_{m_{i}}$}
\todos[inline]{The final inequality above does not hold unless you assume $\hat{r}^{*}\geq r^{*} - c_{m_{i}}$ and this is never mentioned?}
  Hence, we get that as soon as $\sqrt{\rho_{a}\epsilon_{m_{i}}}<\frac{\Delta_{i}}{2}$, $\exists a_{i}$ which gets eliminated. Applying Chernoff-Hoeffding bound and considering independence of events,
  \begin{align*}
\mathbb{P}\lbrace\hat{r}^{*}&\leq r^{*} - c_{m_{i}}\rbrace\leq exp(-2c_{m_{i}}^{2}n_{m_{i}})\\
&\leq exp(-2 * \frac{\rho_{a}\log (\psi T\epsilon_{m_{i}}^{2})}{2 n_{m_{i}}} *n_{m_{i}})\\
&\leq \frac{1}{(\psi T\epsilon_{m_{i}}^{2})^{\rho_{a}}}   
  \end{align*}

Similarly, $\mathbb{P}\lbrace\hat{r}_{i}\geq r_{i} + c_{m_{i}}\rbrace\leq \dfrac{1}{(\psi  T\epsilon_{m_{i}}^{2})^{\rho_{a}}}$
 
Summing, the two up, the probability that a sub-optimal arm $a_{i}$ is not eliminated in $m_{i}$-th round is  $\bigg(\dfrac{2}{(\psi T\epsilon_{m_{i}}^{2})^{\rho_{a}}}\bigg)$. 
 
Summing up over all arms in $A_{s^{*}}^{'}$ and bounding trivially by $T\Delta_{i}$, we obtain
   \begin{align*}
&\sum_{i\in A_{s^{*}}^{'}}\bigg(\dfrac{2T\Delta_{i}}{(\psi T\epsilon_{m_{i}}^{2})^{\rho_{a}}}\bigg)
\leq\sum_{i\in A_{s^{*}}^{'}}\bigg(\frac{2T\Delta_{i}}{(\psi T\dfrac{\Delta_{i}^{4}}{16\rho_{a}^{2}})^{\rho_{a}}}\bigg)\\
&\leq \sum_{i\in A_{s^{*}}^{'}}\bigg(\frac{2^{1+4\rho_{a}}T^{1-\rho_{a}}\rho_{a}^{2\rho_{a}}\Delta_{i}}{\psi^{\rho_{a}}\Delta_{i}^{4\rho_{a}}}\bigg)\\
& =\sum_{i\in A_{s^{*}}^{'}}\bigg(\frac{C_{1}(\rho_{a})T^{1-\rho_{a}}}{\Delta_{i}^{4\rho_{a}-1}}\bigg) \text{, where } C_1(x) = \frac{2^{1+4x}x^{2x}}{\psi^{x}}
   \end{align*}
   
\item $a_{i}$ in $s_{k}$, where $r_{max_{s_{k}}}\leq r^{*}$ and $m_{i}$ does not happen. We can show that the probability of $a_{i}$ not getting eliminated  cannot be worse than case $a1$ given that $\sqrt{\rho_{a}\epsilon_{m}}< \frac{\Delta^{'}_{i}}{2}$ where $\Delta^{'}_{i}=r_{max_{s_{k}}} - r_{i}\geq\Delta_{a_{max_{s_{k}}}}$ such that $r_{i}\in s_{k}$. Plugging in this $\Delta^{'}_{i}$ in case $a1$ we can derive a similar bound where $\Delta^{'}_{i}\geq \Delta_{a_{max_{s_{k}}}}$ because otherwise $\sqrt{\epsilon_{m}\rho_{s}}< \frac{\Delta_{a_{max_{s_{k}}}}}{2}$ will happen and the cluster $s_{k}$ gets eliminated or $a_{max_{s_{k}}}$ will eliminate $a^{*}$ which is dealt later.

\item $a_{i}\in s_{k}, a^{*}\in C_{g_{s_{k}}}$ and $g_{s_{k}}$ does not happen. In arm elimination condition, given the choice of confidence interval $c_{g_{s_{k}}}$, we want to bound the probability of the event $\hat{r}_{s_{k}}+c_{g_{s_{k}}}\leq \hat{r}^{*}-c_{g_{s_{k}}}$.

  Now, $c_{g_{s_{k}}}=\sqrt{\frac{\rho_{s} \log (\psi T\epsilon_{g_{s_{k}}}^{2})}{2 n_{g_{s_{k}}}}}$, where $0 < \rho_{s}\leq 1$.
  Putting the value of $n_{g_{s_{k}}}=\frac{2\log{(\psi T\epsilon_{g_{s_{k}}}^{2})}}{\epsilon_{g_{s_{k}}}}$ in $c_{g_{s_{k}}}$, we get
  \begin{align*}
  c_{g_{s_{k}}} =\sqrt{\rho_{s}\epsilon_{g_{s_{k}}+1}} < \frac{\sqrt{\rho_{s}}\Delta_{a_{max_{s_{k}}}}}{4} < \frac{\Delta_{a_{max_{s_{k}}}}}{4}
  \end{align*}

  Again, $\exists a_{max_{s_{k}}} \in C_{g_{s_{k}}}$ such that, 
  \begin{align*}
  \hat{r}_{a_{max_{s_{k}}}} &+ c_{g_{k}} \leq r_{a_{max_{s_{k}}}} + 2c_{g_{k}} \\
  &\leq r_{a_{max_{s_{k}}}} + \Delta_{a_{max_{s_{k}}}} - 2c_{g_{k}} \leq \hat{r}^{*} - c_{g_{k}}
  \end{align*}
   
 	Hence, we get that as soon as $\sqrt{\rho_{s}\epsilon_{g_{s_{k}}}}<\frac{\Delta_{a_{max_{s_{k}}}}}{2}$, $\exists a_{max_{s_{k}}}\in C_{g_{s_{k}}}$ which gets eliminated. Applying Chernoff-Hoeffding bound and considering independence of events,
 	
%So, we need to bound the probability,

%So, we need to bound the event of $\hat{r}_{a_{max_{s_{k}}}}+c_{g_{s_{k}}}\leq \hat{r}^{*}-c_{g_{s_{k}}}$ given that $\sqrt{\rho_{s}\epsilon_{g_{s_{k}}}}<\dfrac{\Delta_{a_{max_{s_{k}}}}}{2}$ becomes true for some arm $a_{max_{s_{k}}}\in C_{g_{s_{k}}}$ after the $g$-th round and $c_{g_{s_{k}}}=\sqrt{\dfrac{\rho_{s} \log (\psi T\epsilon_{g_{s_{k}}}^{2})}{2 n_{g_{s_{k}}}}}$.

%	\begin{align*}
%	&\mathbb{P}\lbrace\hat{r}^{*}\leq r^{*} - c_{g_{s_{k}}}\rbrace\leq U_{g}\text{, where $U_{g}$ is an  arbitrary upper bound.}
%	\end{align*}
 

 
 \begin{align*}
 \mathbb{P}\lbrace\hat{r}^{*} &\leq r^{*} - c_{g_{s_{k}}}\rbrace \leq exp(-2c_{g_{s_{k}}}^{2}n_{g_{s_{k}}})
 \leq \dfrac{1}{(\psi T\epsilon_{g_{k}}^{2})^{\rho_{s}}}
 \end{align*}
%\\&\leq exp(-2 * \dfrac{\rho_{s}\log ( \psi T\epsilon_{g_{s_{k}}}^{2})}{2 n_{g_{s_{k}}}} *n_{g_{s_{k}}})&
Similarly, $\mathbb{P}\lbrace\hat{r}_{a_{max_{s_{k}}}}\geq r_{a_{max_{s_{k}}}} + c_{g_{s_{k}}}\rbrace\leq \dfrac{1}{(\psi T\epsilon_{g_{s_{k}}}^{2})^{\rho_{s}}}$
 
Summing, the two up, the probability that a sub-optimal cluster arm $a_{max_{s_{k}}}\in C_{g_{s_{k}}}$ is not eliminated in $g_{s_{k}}$-th round is  $\bigg(\dfrac{2}{(\psi  T\epsilon_{g_{s_{k}}}^{2})^{\rho_{s}}}\bigg)$. 
  Now, for each round $g$, all the elements of $C_{g_{s_{k}}}$ are the respective max payoff arms of their cluster $s_{k}, \forall s_{k}\in S$, and since all the surviving arms are pulled equally in each round and since clusters are fixed so we can bound the maximum probability that a sub-optimal arm $a_{j}\in A^{'}$  and $a_{j}\in s_{k}$ such that $a_{max_{s_{k}}}\in C_{g_{s_{k}}}$ is not eliminated in the $g$-th round by the same probability as above. 
  
%\begin{align*}
%\bigg(\frac{2}{(\psi T\epsilon_{g_{s_{k}}}^{2})^{\rho_{s}}}\bigg)
%\end{align*}
 
%Summing up over all arms in $s_{k}$ and bounding trivially by $T\Delta_{i}$,
%\begin{align*}
%\sum_{i\in A_{s_{k}}}\bigg(\frac{2T\Delta_{i}}{(\psi T\epsilon_{g_{s_{k}}}^{2})^{\rho_{s}}}\bigg)
%\end{align*}

Summing up over all $p$ clusters and bounding trivially by $T\Delta_{i}$,
 \begin{align*}
 &\sum_{k=1}^{p}\sum_{i\in A_{s_{k}}^{'}}\bigg(\frac{2T\Delta_{i}}{(\psi T\frac{\Delta_{i}^{4}}{16\rho_{s}^{2}})^{\rho_{s}}}\bigg) = \sum_{i\in A^{'}}\bigg(\frac{2T\Delta_{i}}{(\psi  T\frac{\Delta_{i}^{4}}{16\rho_{s}^{2}})^{\rho_{s}}}\bigg) \\
 &\leq \sum_{i\in A^{'}}\bigg(\frac{2^{1+4\rho_{s}}\rho_{s}^{2\rho_{s}}T^{1-\rho_{s}}}{\psi^{\rho_{s}}\Delta_{i}^{4\rho_{s}-1}}\bigg) = \sum_{i\in A^{'}}\frac{C_{1}(\rho_{s})T^{1-\rho_{s}}}{\Delta_{i}^{4\rho_{s}-1}}
 \end{align*}
% &= \sum_{i\in A^{'}}\bigg(\frac{C_{1}(\rho_{s})T^{1-\rho_{s}}}{\Delta_{i}^{4\rho_{s}-1}}\bigg) \text{, where } C_1(x) = \frac{2^{1+4x}x^{2x}}{\psi^{x}}
%&\leq \sum_{i\in A}\bigg(\frac{2^{1+4\rho_{s}}T^{1-\rho_{s}}\rho_{s}^{2\rho_{s}}\Delta_{i}}{\psi^{\rho_{s}}\Delta_{i}^{4\rho_{s}}}\bigg)\\

\item $a_{i}\in s_{k}, a^{*}\notin C_{g_{s_{k}}}$ and $g_{s_{k}}$ does not happen. 
%\newline
%In the above case we considered that $a^{*}\in C_{g_{s_{k}}}$ being the max-payoff arm from optimal cluster $s^{*}$.
Now, if $a^{*}\notin C_{g_{s_{k}}}$ and $\exists a_{max_{s^{*}}}\in s^{*}$ such that $\hat{r}_{a_{max_{s^{*}}}}> \hat{r}^{*}$, so $a_{max_{s^{*}}}$ will be in $C_{g_{s_{k}}}$ in the $g_{s_{k}}$-th round. In this case for some sub-optimal arm $a_{max_{s_{k}}}\in C_{g_{s_{k}}}$, we have to bound the probability
	\begin{align*}
	&\mathbb{P}\bigg\lbrace\hat{r}_{a_{max_{s_{k}}}}+c_{g_{s_{k}}}\bigg\rbrace< \mathbb{P}\bigg\lbrace\hat{r}_{a_{max_{s^{*}}}}-c_{g_{s_{k}}}\bigg\rbrace
	\end{align*}		 
	 
	 
	 But, this probability can be no worse than case a3 since $r_{max_{s^{*}}} < r^{*}$ and all arms get pulled $n_{g_{s_{k}}}$ number of times in the $g$-th round. After summing over all arms in $A^{'}$ and bounding 
trivially by $T\Delta_{i}$ we get the same result as above we can show that the regret can be no more than,
 \begin{align*}
 &\sum_{i\in A^{'}}\bigg(\frac{2^{1+4\rho_{s}}\rho_{s}^{2\rho_{s}}T^{1-\rho_{s}}}{\psi^{\rho_{s}}\Delta_{i}^{4\rho_{s}-1}}\bigg)=\sum_{i\in A^{'}}\bigg(\frac{C_{1}(\rho_{s})T^{1-\rho_{s}}}{\Delta_{i}^{4\rho_{s}-1}}\bigg)
 \end{align*}
\end{enumerate}
Taking summation of the events mentioned above($a1$-$a4$) gives us an upper bound on the regret given that the optimal arm $a^{*}$ is still surviving, 
\begin{align*}
\sum_{i\in A^{'}}\bigg\lbrace \bigg(\frac{C_{1}(\rho_{a})T^{1-\rho_{a}}}{\Delta_{i}^{4\rho_{a}-1}}\bigg) + \bigg(\frac{2C_{1}(\rho_{s})T^{1-\rho_{s}}}{\Delta_{i}^{4\rho_{s}-1}}\bigg)\bigg\rbrace
\end{align*}

%So the regret for not eliminating a sub-optimal cluster even when $a^{*}\notin C_{g_{s_{k}}}$(but still surviving in $s^{*}$) can be no worse than,
%	 \begin{align*} 
%	 \bigg(\frac{2}{(T\epsilon_{g_{s_{k}}}^{2})^{\rho_{s}}}\bigg) 
%	 \end{align*}
%&\underbrace{\sum_{i\in A_{s^{*}}^{'}}\bigg(\dfrac{C_{1}(\rho_{a})T^{1-\rho_{a}}}{\Delta_{i}^{4\rho_{a}-1}}\bigg)}_{\text{case a1}} + \underbrace{\sum_{i\in A\setminus A_{s^{*}}^{'}}\bigg(\dfrac{C_{1}(\rho_{a})T^{1-\rho_{a}}}{\Delta_{i}^{4\rho_{a}-1}}\bigg)}_{\text{case a2}} \\
% & + \sum_{i\in A^{'}}\bigg\lbrace \underbrace{\bigg(\dfrac{2C_{1}(\rho_{s})T^{1-\rho_{s}}}{\psi^{\rho_{s}}\Delta_{i}^{4\rho_{s}-1}}\bigg)}_{\text{case a3+a4}}\bigg\rbrace \\
%& =
\subsection*{Case b: \textit{Either an arm $a_{i}$ is eliminated in round $max \lbrace m_{i},g_{s_{k}} \rbrace$ or before or else there is no optimal arm $a^{*}\in B_{m_{i}}$}} 

\subsubsection*{Case b1: \textit{If an optimal arm $a^{*}\in B_{m_{i}}$ then the maximum pull of all arms $a_{i}\in A^{'}$}} 

	For any sub-optimal arm still surviving given $m_{i}$ or $g_{s_{k}}:a_{i}\in s_{k}$ have not happened and $a^{*}\in s^{*}$ still surviving then they get pulled $n_{m_{i}}$ or $n_{g_{s_{k}}}$ number of times. Hence, we can show that till an arm or a cluster is eliminated, the maximum regret suffered due to pulling of a sub-optimal arm(or a sub-optimal cluster) is no less than,
 \begin{align*}
 &\sum_{i\in A^{'}}\bigg\lbrace\Delta_{i}\bigg\lceil\dfrac{2\log{(\psi T\epsilon_{m_{i}}^{2})}}{\epsilon_{m_{i}}}\bigg\rceil + \Delta_{i}\bigg\lceil\dfrac{2\log{(\psi T\epsilon_{g_{s_{k}}}^{2})}}{\epsilon_{g_{s_{k}}}}\bigg\rceil \bigg \rbrace\\
&\leq\sum_{i\in A^{'}}\Delta_{i}\bigg(1+\dfrac{32\rho_{a}\log{(\psi T(\dfrac{\Delta_{i}}{2\sqrt{\rho_{a}}})^{4})}}{\Delta_{i}^{2}}\bigg) \\
&+ \sum_{i\in A^{'}}\Delta_{i}\bigg(1+\dfrac{32\rho_{s}\log{(\psi T(\dfrac{\Delta_{i}}{2\sqrt{\rho_{s}}})^{4})}}{\Delta_{i}^{2}}\bigg)
\\
&\text{ since } \sqrt{\rho_{a}\epsilon_{m_{i}}}\leq\frac{\Delta_{i}}{2} \text{ \& } \sqrt{\rho_{s}\epsilon_{n_{g_{s_{k}}}}}\leq\frac{\Delta_{i}}{2}
\\
 &\leq \sum_{i\in A^{'}}\bigg\lbrace 2\Delta_{i}+\dfrac{32\rho_{a}\log{(\psi T\dfrac{\Delta_{i}^{4}}{16\rho_{a}^{2}})}}{\Delta_{i}} +\dfrac{32\rho_{s}\log{(\psi T\dfrac{\Delta_{i}^{4}}{16\rho_{s}^{2}})}}{\Delta_{i}}\bigg\rbrace 
 \end{align*}

%&\leq\Delta_{i}\bigg(1+\dfrac{32\rho_{a}\log{(\psi T\dfrac{\Delta_{i}^{4}}{16\rho_{a}^{2}})}}{\Delta_{i}^{2}}\bigg)\\
%&\leq\Delta_{i}\bigg\lceil\dfrac{2\log{(\psi T(\dfrac{\Delta_{i}}{2\sqrt{\rho_{a}})^{4})}}}{(\dfrac{\Delta_{i}}{2\sqrt{\rho_{a}}})^{2}}\bigg\rceil \\
%\text{, since } \sqrt{\rho_{a}\epsilon_{m_{i}}}\leq\dfrac{\Delta_{i}}{2}
 
\subsubsection*{Case b2: \textit{Optimal arm $a^{*}$ is eliminated by a sub-optimal arm}}
  
	Here, we take into consideration the error bound, that the optimal arm $a^{*}$ or the optimal cluster $s^{*}$ gets eliminated by any sub-optimal arm or sub-optimal cluster. This, can happen in $3$ ways,
\begin{enumerate}
\item In $s^{*}$, $a^{*}$ got eliminated by other arms surviving till $m_{*}$. Firstly, if conditions of case b1 holds then the optimal arm $a^{*}$ will not be eliminated in round $m=m_{*}$ or it will lead to the contradiction that $r_{i}>r^{*}$. In any round $m_{*}$, if the optimal arm $a^{*}$ gets eliminated then for any round from $1$ to $m_{j}$ all arms $a_{j}$ such that $\sqrt{\rho_{a}\epsilon_{m}}<\frac{\Delta_{j}}{2}$ were eliminated according to assumption in case $b1$. Let, the arms surviving till $m_{*}$ round be denoted by $A^{'}$. This leaves any arm $a_{b}$ such that $\sqrt{\rho_{a}\epsilon_{m}}\geq\frac{\Delta_{b}}{2}$ to still survive and eliminate arm $a^{*}$ in round $m_{*}$. Let, such arms that survive $a^{*}$ belong to $A^{''}$. Also maximal regret per step after eliminating $a^{*}$ is the maximal $\Delta_{j}$ among the remaining arms $a_{j}$ with $m_{j}\geq m_{*}$.  Let $m_{b}$ be the round when $\sqrt{\rho_{a}\epsilon_{m}}<\frac{\Delta_{b}}{2}$ that is $m_{b}=min\lbrace m|\sqrt{\rho_{a}\epsilon_{m}}<\frac{\Delta_{b}}{2}\rbrace$. Let $C_2(x) = \frac{2^{2x+\frac{3}{2}}x^{2x}}{\psi^{x}}$. Hence, the maximal regret after eliminating the arm $a^{*}$ is upper bounded by, 
\begin{align*}
&\sum_{m_{*}=0}^{max_{j\in A^{'}_{s^{*}}}m_{j}}\sum_{\substack{i\in A^{''}_{s^{*}}: \\ m_{i}>m_{*}}}\bigg(\dfrac{2}{(\psi  T\epsilon_{m}^{2})^{\rho_{a}}} \bigg).T\max_{\substack{j\in A^{''}_{s^{*}}: \\ m_{j}\geq m_{*}}}{\Delta}_{j}\\
&\leq\sum_{m_{*}=0}^{max_{j\in A^{'}_{s^{*}}}m_{j}}\sum_{i\in A^{''}_{s^{*}}:m_{i}>m_{*}}\bigg(\dfrac{2}{(\psi  T\epsilon_{m}^{2})^{\rho_{a}}} \bigg).T.2\sqrt{\rho_{a}\epsilon_{m}} \\
&\text{ since } \sqrt{\rho_{a}\epsilon_{m}}<\dfrac{\Delta_{i}}{2}\\
&\leq\sum_{m_{*}=0}^{max_{j\in A^{'}_{s^{*}}}m_{j}}\sum_{i\in A^{''}_{s^{*}}:m_{i}>m_{*}}4\bigg(\dfrac{T^{1-\rho_{a}}}{\psi^{\rho_{a}}\epsilon_{m}^{2\rho_{a}-\frac{1}{2}}} \bigg)\\
&\leq\sum_{i\in A^{''}_{s^{*}}:m_{i}>m_{*}}\sum_{m_{*}=0}^{\min{\lbrace m_{i},m_{b}\rbrace}}\bigg(\dfrac{4T^{1-\rho_{a}}}{\psi^{\rho_{a}}2^{-(2\rho_{a}-\frac{1}{2})m_{*}}} \bigg)\\
&\leq\sum_{i\in A^{'}_{s^{*}}}\dfrac{4T^{1-\rho_{a}}}{\psi^{\rho_{a}}2^{-(2\rho_{a}-\frac{1}{2})m_{*}}} +\sum_{i\in A^{''}_{s^{*}}\setminus A^{'}_{s^{*}}}\dfrac{4T^{1-\rho_{a}}}{\psi^{\rho_{a}}2^{-(2\rho_{a}-\frac{1}{2})m_{b}}} \\
&\leq\sum_{i\in A^{'}_{s^{*}}}\dfrac{T^{1-\rho_{a}}\rho_{a}^{2\rho_{a}}2^{2\rho_{a}+\frac{3}{2}}}{\psi^{\rho_{a}}\Delta_{i}^{4\rho_{a}-1}} +\sum_{i\in A^{''}_{s^{*}}\setminus A^{'}_{s^{*}}}\dfrac{T^{1-\rho_{a}}\rho_{a}^{2\rho_{a}}2^{2\rho_{a}+\frac{3}{2}}}{\psi^{\rho_{a}}b^{4\rho_{a}-1}} \\
& = \sum_{i\in A^{'}_{s^{*}}}\dfrac{ C_{2}(\rho_{a}) T^{1-\rho_{a}}}{\Delta_{i}^{4\rho_{a}-1}} +\sum_{i\in A^{''}_{s^{*}}\setminus A^{'}_{s^{*}}}\dfrac{C_{2(\rho_{a})}T^{1-\rho_{a}}}{b^{4\rho_{a}-1}}
\end{align*}

%&\leq\sum_{i\in A^{'}}\dfrac{4\rho_{a}^{2\rho_{a}}T^{1-\rho_{a}}*2^{2\rho_{a}-\frac{1}{2}}}{\psi^{\rho_{a}}\Delta_{i}^{4\rho_{a}-1}} +\sum_{i\in A^{''}\setminus A^{'}}\dfrac{4\rho_{a}^{2\rho_{a}}T^{1-\rho_{a}}*2^{2\rho_{a}-\frac{1}{2}}}{\psi^{\rho_{a}}b^{4\rho_{a}-1}} \\

% \begin{align*}
% &\sum_{i\in A^{'}_{s^{*}}}\bigg(\dfrac{C_{2}(\rho_{a})T^{1-\rho_{a}}}{\Delta_{i}^{4\rho_{a} -1}} \bigg)+\sum_{i\in A^{''}_{s^{*}}\setminus A^{'}_{s^{*}}}\bigg(\dfrac{C_{2}(\rho_{a})T^{1-\rho_{a}}}{b^{4\rho_{a} -1}} \bigg)
% \end{align*}
%We also see that here, we are concerned only within $s^{*}$ because of our assumption that there is only one $a^{*}\in A$ and clusters are fixed.
\item $a^{*}\in C_{g}$ and $s^{*}$ gets eliminated by some other cluster. 
%\newline
%Firstly, if conditions of case $b1$ holds then the optimal arm $a^{*}\in C_{g_{s_{k}}}$ will not be eliminated in round $g_{s_{k}}=g_{*}$ or it will lead to the contradiction that $r_{a_{max_{s_{k}}}}>r^{*}$ where $a_{max_{s_{k}}},a^{*}\in C_{g_{s_{k}}}$. In any round $g_{*}$, if the optimal arm $a^{*}$ gets eliminated then for any round from $1$ to $g_{s_{j}}$ all arms $a_{s_{j}}\in C_{g_{s_{k}}},\forall s_{j}\neq s^{*}$ such that $\sqrt{\rho_{s}\epsilon_{g_{s_{k}}}}<\dfrac{\Delta_{a_{s_{j}}}}{2}$ were eliminated according to assumption in case $b1$. Let, the arms surviving till $g_{*}$ round be denoted by $C_{g}^{'}$. This leaves any arm $a_{s_{b}}$ such that $\sqrt{\rho_{s}\epsilon_{g_{s_{b}}}}\geq\dfrac{\Delta_{a_{s_{b}}}}{2}$ to still survive and eliminate arm $a^{*}$ in round $g_{*}$. Let, such arms that survive $a^{*}$ belong to $C_{g}^{''}$. Also maximal regret per step after eliminating $a^{*}$ is the maximal $\Delta_{j}$ among the remaining arms $a_{j}\in B_{m}$ with $g_{s_{j}}\geq g_{
%*}$.  Let $g_{s_{b}}$ be the round when $\sqrt{\rho_{s}\epsilon_{g_{s_{b}}}}<\dfrac{\Delta_{s_{b}}}{2}$ that is $g_{b}=min\lbrace g|\sqrt{\rho_{s}\epsilon_{g_{s_{b}}}}<\dfrac{\Delta_{b}}{2}\rbrace$ and the cluster $s_{b}$ gets eliminated. Hence, the maximal regret after eliminating the arm $a^{*}$ is upper bounded by, 
% \begin{align*}
% &\sum_{g_{*}=0}^{max_{j\in C_{g}^{'}}g_{s_{j}}}\sum_{i\in C_{g}^{''}:g_{s_{k}}>g_{*}}\bigg(\dfrac{2}{(\psi T\epsilon_{g_{s_{k}}}^{2})^{\rho_{s}}} \bigg).T\max_{j\in C_{g}^{''}:g_{s_{j}}\geq g_{*}}{\Delta}_{a_{s_{j}}}
% \end{align*}
In any round $g_{*}$, if the optimal arm $a^{*}$ gets eliminated then for any round from $1$ to $g_{s_{j}}$ all arms $a_{s_{j}}\in C_{g_{s_{k}}},\forall s_{j}\neq s^{*}$ such that $\sqrt{\rho_{s}\epsilon_{g_{s_{k}}}}<\frac{\Delta_{a_{s_{j}}}}{2}$ were eliminated according to assumption in case $b1$. Let, the arms surviving till $g_{*}$ round be denoted by $C_{g}^{'}$. This leaves any arm $a_{s_{b}}$ such that $\sqrt{\rho_{s}\epsilon_{g_{s_{b}}}}\geq\frac{\Delta_{a_{s_{b}}}}{2}$ to still survive and eliminate arm $a^{*}$ in round $g_{*}$. Let, such arms that survive $a^{*}$ belong to $C_{g}^{''}$. But, we know that for any round $g$, elements of $C_{g}$ are the best performers in their respective clusters. So, taking that into account and $A'\supset C_{g}^{'}$ and $A''\supset C_{g}^{''}$ where $A^{'}$ is the set of all the arms across clusters surviving till $g_{*}$ round and $A^{''}$ be the set of all arms across clusters to still survive and eliminate arm $a^{*}$ in round $g_{*}$ respectively. Hence, regret after eliminating $a^{*}$ is

\begin{align*}
 & \sum_{g_{*}=0}^{max_{j\in A^{'}}g_{s_{j}}}\sum_{\substack{i\in A^{''}: \\ g_{s_{k}}>g_{*}}}\bigg(\dfrac{2}{(\psi T\epsilon_{g_{s_{k}}}^{2})^{\rho_{s}}} \bigg).T\max_{\substack{j\in A^{''}: \\ g_{s_{j}}\geq g_{*}}}{\Delta}_{a_{s_{j}}}\\
\end{align*}
Following the same way as Case $b21$, we can bound the regret as,
\begin{align*}
 &\sum_{i\in A^{'}}\dfrac{T^{1-\rho_{s}}\rho_{s}^{2\rho_{s}}2^{2\rho_{s}+\frac{3}{2}}}{\psi^{\rho_{s}}\Delta_{i}^{4\rho_{s}-1}} +\sum_{i\in A^{''}\setminus A^{'}}\dfrac{T^{1-\rho_{s}}\rho_{s}^{2\rho_{s}}2^{2\rho_{s}+\frac{3}{2}}}{\psi^{\rho_{s}}b^{4\rho_{s}-1}} \\
 & = \sum_{i\in A^{'}}\dfrac{C_{2}(\rho_{s})T^{1-\rho_{s}}}{\Delta_{i}^{4\rho_{s}-1}} +\sum_{i\in A^{''}\setminus A^{'}}\dfrac{C_{2}(\rho_{s})T^{1-\rho_{s}}}{b^{4\rho_{s}-1}} 
\end{align*}

%\\ & \text{ where } C_2(x) = \frac{2^{2x+\frac{3}{2}}x^{2x}}{\psi^{x}}
%&\leq\sum_{g_{*}=0}^{max_{j\in A^{'}}g_{s_{j}}}\sum_{i\in A^{''}:g_{s_{k}}>g_{*}}\bigg(\dfrac{2}{(\psi T\epsilon_{g_{s_{k}}}^{2})^{\rho_{s}}} \bigg).T.2\sqrt{\rho_{s}\epsilon_{g_{s_{j}}}} \text{, since }\sqrt{\rho_{s}\epsilon_{g_{s_{j}}}}\leq\dfrac{\Delta_{a_{s_{j}}}}{2}\leq  \dfrac{\Delta_{j}}{2}\text{, as }{r}_{a_{s_{j}}}>{r}_{j},\forall j\in s_{j}\\ 
%&\leq\sum_{g_{*}=0}^{max_{j\in A^{'}}g_{s_{j}}}\sum_{i\in A^{''}:g_{s_{k}}>g_{*}}\bigg(\dfrac{4T^{1-\rho_{s}}}{\psi^{\rho_{s}}\epsilon_{g_{s_{k}}}^{2\rho_{s} - \frac{1}{2}}} \bigg)\\
% &\leq\sum_{i\in A^{''}:g_{s_{k}}>g_{*}}\sum_{g_{*}=0}^{\min{\lbrace g_{s_{k}},g_{s_{b}}\rbrace}}\bigg(\dfrac{4T^{1-\rho_{s}}}{\psi^{\rho_{s}}2^{({2\rho_{s} - \frac{1}{2}})g_{*}}} \bigg) \\
% &\leq\sum_{i\in A^{'}}\bigg(\dfrac{4T^{1-\rho_{s}}}{\psi^{\rho_{s}}2^{({2\rho_{s} - \frac{1}{2}})g_{*}}} \bigg)+\sum_{i\in A^{''}\setminus A^{'}}\bigg(\dfrac{4T^{1-\rho_{s}}}{\psi^{\rho_{s}}2^{({2\rho_{s} - \frac{1}{2}})g_{s_{b}}}} \bigg)\\ 
% &\leq\sum_{i\in A^{'}}\bigg(\dfrac{4\rho_{s}^{2\rho_{s}}T^{1-\rho_{s}}*2^{2\rho_{s}-\frac{1}{2}}}{\psi^{\rho_{s}}\Delta_{i}^{4\rho_{s}-1}} \bigg)+\sum_{i\in A^{''}\setminus A^{'}}\bigg(\dfrac{4\rho_{s}^{2\rho_{s}}T^{1-\rho_{s}}*2^{2\rho_{s}-\frac{1}{2}}}{\psi^{\rho_{s}}b^{4\rho_{s}-1}} \bigg)\\

\item $a^{*}\notin C_{g}$ and $s^{*}$ gets eliminated by some other cluster. This will be the mirror case of Case $b22$ with $a^{*}\notin C_{g}$ and since $s^{*}$ gets eliminated by $a_{s_{b}}$ such that $\sqrt{\rho_{s}\epsilon_{g_{s_{b}}}}\geq\frac{\Delta_{a_{s_{b}}}}{2}$ round $g_{*}$. Also maximal regret per step after eliminating $a^{*}$ is the maximal $\Delta_{j}$ among the remaining arms $a_{j}\in B_{m}$ with $g_{s_{j}}\geq g_{*}$. Following the same way above we can bound the regret as,
%In this case we will consider that the cluster $s^{*}$ containing the optimal arm $a^{*}$ was eliminated by another sub-optimal cluster and $a^{*}\notin C_{g}$. 
\begin{align*}
&\sum_{i\in A^{'}}\dfrac{T^{1-\rho_{s}}\rho_{s}^{2\rho_{s}}2^{2\rho_{s}+\frac{3}{2}}}{\psi^{\rho_{s}}\Delta_{i}^{4\rho_{s}-1}} +\sum_{i\in A^{''}\setminus A^{'}}\dfrac{T^{1-\rho_{s}}\rho_{s}^{2\rho_{s}}2^{2\rho_{s}+\frac{3}{2}}}{\psi^{\rho_{s}}b^{4\rho_{s}-1}} \\
& = \sum_{i\in A^{'}}\dfrac{C_{2}(\rho_{s})T^{1-\rho_{s}}}{\Delta_{i}^{4\rho_{s}-1}} +\sum_{i\in A^{''}\setminus A^{'}}\dfrac{C_{2}(\rho_{s})T^{1-\rho_{s}}}{b^{4\rho_{s}-1}}
\end{align*}


\end{enumerate} 


Combining cases $b21$, $b22$ and $b23$ as mentioned above we can show,
 \begin{align*}
 &\underbrace{\sum_{i\in A^{'}_{s^{*}}}\bigg(\dfrac{C_{2}(\rho_{a})T^{1-\rho_{a}}}{\Delta_{i}^{4\rho_{a} -1}} \bigg)+\sum_{i\in A^{''}_{s^{*}}\setminus A^{'}_{s^{*}}}\bigg(\dfrac{C_{2}(\rho_{a})T^{1-\rho_{a}}}{b^{4\rho_{a} -1}} \bigg)}_{\text{case b21}} \\
 & + \underbrace{\sum_{i\in A^{'}}\bigg(\dfrac{2C_{2}(\rho_{s})T^{1-\rho_{s}}}{\Delta_{i}^{4\rho_{s}-1}} \bigg)}_{\text{case b22}}+\underbrace{\sum_{i\in A^{''}\setminus A^{'}}\bigg(\dfrac{2C_{2}(\rho_{s})T^{1-\rho_{s}}}{b^{4\rho_{s} -1}} \bigg)}_{\text{case b23}}
 \end{align*}
 

The main claim follows by summing the contributions to the expected regret from each of the cases above.

\end{proof}


\begin{proposition}
\label{proofTheorem:Prop:1}
The regret $R_T$ for ClusUCB-AE satisfies
\begin{align*}
&\E [R_{T}]\leq \sum\limits_{i\in A:\Delta_{i} > b}\bigg\lbrace\frac{C_{1}(\rho_{a})T^{1-\rho_{a}}}{\Delta_{i}^{4\rho_{a}-1}} + \Delta_{i}\\
&+\frac{32\rho_{a}\log{(\dfrac{\psi  T\Delta_{i}^{4}}{16\rho_{a}^{2}})}}{\Delta_{i}}
 +  \frac{C_{2}(\rho_{a})T^{1-\rho_{a}}}{\Delta_{i}^{4\rho_{a} -1}}  \bigg \rbrace\\
&+\sum\limits_{i\in A:0\leq\Delta_{i}\leq b}\frac{C_{2}(\rho_{a})T^{1-\rho_{a}}}{b^{4\rho_{a} -1}}  + \max_{i:\Delta_{i}\leq b}\Delta_{i}T
\end{align*}
, for all $b\geq\sqrt{\frac{e}{T}}$, where  $C_1(x) = \frac{2^{1+4x}x^{2x}}{\psi^{x}}$,  $C_2(x) = \frac{2^{2x+\frac{3}{2}}x^{2x}}{\psi^{x}}$, $\rho_{a}=\frac{1}{2}$ is the arm elimination parameter, $\psi=K^{2}T$ is the exploration regulatory factor, $p$ is the number of clusters and $T$ is the horizon.
\end{proposition}
\begin{proof}
%Follows in a similar fashion as the proof of Theorem $1$ in \cite{auer2010ucb}. For the sake of completeness, the proof is given in Appendix \ref{App:A}.
See Appendix \ref{App:A}.
\end{proof}

\begin{proposition}
\label{proofTheorem:Prop:2}
The regret $R_T$ for ClusUCB-CE satisfies,
\begin{align*}
&\E [R_{T}]\leq \sum\limits_{i\in A:\Delta_{i} > b}\bigg\lbrace\bigg(\dfrac{2C_{1}(\rho_{s})T^{1-\rho_{s}}}{\Delta_{i}^{4\rho_{s}-1}}\bigg)\\
& + \bigg(\Delta_{i}+\dfrac{32\rho_{s}\log{(\psi T\dfrac{\Delta_{i}^{4}}{16\rho_{s}^{2}})}}{\Delta_{i}}\bigg) + \bigg(\dfrac{2C_{2}(\rho_{s})T^{1-\rho_{s}}}{\Delta_{i}^{4\rho_{s} -1}} \bigg)\bigg\rbrace \\
& + \sum\limits_{i\in A:0\leq\Delta_{i}\leq b}\bigg(\dfrac{2C_{2}(\rho_{s})T^{1-\rho_{s}}}{b^{4\rho_{s} -1}} \bigg) + \max_{i:\Delta_{i}\leq b}\Delta_{i}T
\end{align*}
, for all $b\geq \sqrt{\frac{e}{T}}$, where $C_1(x) = \frac{2^{1+4x}x^{2x}}{\psi^{x}}$,  $C_2(x) = \frac{2^{2x+\frac{3}{2}}x^{2x}}{\psi^{x}}$, $\rho_{s}=\frac{1}{2} $ is the cluster elimination parameter, $\psi=K^{2}T$ is the exploration regulatory factor, $p$ is the number of clusters and $T$ is the horizon.
\end{proposition}
\begin{proof}
See Appendix \ref{App:B}.
\end{proof}