%\label{sec:proofTheorem}

%We sketch the proof for Theorem \ref{Result:Theorem:1} here. 
%The proof involves the following steps:\\
%\textbf{\textit{Step 1:}}
%We analyze ClusUCB-AE, i.e., the variant of ClusUCB that uses arm elimination condition only. In other words, we bound the probability of sub-optimal arm elimination, which in turn bounds the expected regret of ClusUCB-AE (see Proposition \ref{proofSketch:Prop:1} below). 
%
%\textbf{\textit{Step 2:}}
%We analyze ClusUCB-CE, i.e., the variant of ClusUCB that uses cluster elimination condition only and pulls the best arm within the last leftover cluster.
%Proposition \ref{proofSketch:Prop:2} presents the expected regret for ClusUCB-CE (see Proposition \ref{proofSketch:Prop:2} below). 
%
%\textbf{\textit{Step 3:}}
%Finally we combine the individual bounds in the steps above to get the regret upper bound in Theorem \ref{Result:Theorem:1}.  

%\pagebreak
%\subsection*{New proof sketch}
%
%Suppose we consider the following cases and analyze the contribution to expected regret for each of them:
%
%\textbf{Case a:} \textit{Some sub-optimal arm $a_{i}$ is not eliminated in round $\max(m_{i},g_{s_{k}})$ or before, with the optimal arm $a^{*}\in C_{\max(m_{i},g_{s_{k}})}$.}\\
%Note that in this case, we assume that optimal arm $a^*$ is in the set of cluster-heads in round $\max(m_{i},g_{s_{k}})$.
%	
%\textbf{Case a1:} \textit{In round $\max(m_{i},g_{s_{k}})$, $a_{i} \in s^{*}$.} \\
%For this case, the analysis is as in case a1 before. 
%
%\textbf{Case a2:} \textit{In round $\max(m_{i},g_{s_{k}})$, $a_{i} \in s_k$ for some $s_k \ne s^{*}$.} \\
%Since $a^*$ is in the set of cluster-heads $ C_{\max(m_{i},g_{s_{k}})}$, the analysis for this case is the same as that for case a3 earlier. In other words, $a_i$ belongs to a sub-optimal cluster $s_k$ and we bound the probability that the cluster $s_k$ does not get eliminated via the cluster elimination condition.
%
%\textbf{Case b:} \textit{For each arm $a_i$, either $a_{i}$ is eliminated in round $\max (m_{i},g_{s_{k}})$ or before or there is no optimal arm $a^{*}$ in $C_{\max(m_{i},g_{s_{k}})}$.} \\
%The case caption above is quite similar to that earlier, except that we have $C_{\max(m_{i},g_{s_{k}})}$ instead of $B_{\max(m_{i},g_{s_{k}})}$.
%
%\textbf{Case b1:} \textit{$a^{*}\in C_{\max(m_{i},g_{s_{k}})}$ and each $a_{i}\in A^{'}$ is  eliminated on or before $\max (m_{i},g_{s_{k}})$.} 
%
%\textbf{Case b2:} \textit{Optimal arm $a^{*}$ is eliminated by a sub-optimal arm.}\\
%This case should be like earlier case b2, though I have not checked the details of b22,b23 cases.
%
%\textbf{Case b3:} \textit{Optimal arm $a^{*}$ is not in $C_{\max(m_{i},g_{s_{k}})}$.}\\
%This is a new case and we should attempt to analyze the regret contribution for this.
%
%\hrulefill

\begin{proof}

%$\Delta_{i}^{'}=r_{a_{\max_{s_{k}}}} - r_{i}$ such that $a_{i}\in s_{k}$,
% m_{i}^{'}=\min{\lbrace m|\sqrt{\rho_{a}\epsilon_{m}} < \frac{\Delta_{i}^{'}}{2} \rbrace}
Let $A^{'}=\lbrace i \in A,\Delta_{i}> b\rbrace$,  $A^{''}=\lbrace i \in A, \Delta_{i} > 0\rbrace$, $A^{'}_{s_{k}}=\lbrace i \in A_{s_{k}},\Delta_{i}> b\rbrace$ and $A^{''}_{s_{k}}=\lbrace i \in A_{s_{k}}, \Delta_{i} > 0 \rbrace$. $C_{g}$ is the cluster set containing max payoff arm from each cluster in $g$-th round. The arm having the highest payoff in a cluster $s_{k}$ is denote by $a_{\max_{s_{k}}}$. Let for each sub-optimal arm ${i}\in A$, $m_{i}=\min{\lbrace m|\sqrt{\rho_{a}\epsilon_{m}} < \frac{\Delta_{i}}{2} \rbrace}$ and let for each cluster $s_{k}\in S$, $g_{s_{k}}=\min{\lbrace g|\sqrt{\rho_{s}\epsilon_{g}} < \frac{\Delta_{a_{\max_{s_{k}}}}}{2} \rbrace}$. 
Let $\check{A}=\lbrace {i}\in A^{'} | {i}\in s_{k} , \forall s_{k}\in S \rbrace$.
%and $A^{'''}_{s_{k}}=\lbrace i \in A_{s_{k}}, b < \Delta_{a_{\max_{s_{k}}}} < \Delta_{i}^{'}  \rbrace$
%\todos[inline]{define $g_{s_k}$ for each cluster $s_k \in S$}
%\todos[inline]{$a_{max_{s_{k}}}$ is never defined. In the notation of Sec 2 $r_{max_{s_{k}}}$ is defined as the best arm within the cluster $s_k$}
%\todos[inline]{Change max to the operator $\max$ everywhere}
%\todos[inline]{critical error: $m_i$ defintion should be with $\sqrt{\rho_{a}\epsilon_{m}}< \frac{\Delta_{i}}{2}$. Same for $g$} 

%be the first round when $\sqrt{\rho_{a}\epsilon_{m}}\leq \dfrac{\Delta_{i}}{2}$ and for each sub-optimal cluster arm $a_{max_{s_{k}}}\in C_{g_{s_{k}}}$,
%So, $g_{s_{k}}$ be the first round when $\sqrt{\rho_{s}\epsilon_{g_{s_{k}}}}\leq \dfrac{\Delta_{a_{max_{s_{k}}}}}{2}$ where $a_{max_{s_{k}}}\in C_{g_{s_{k}}}$ is the maximum payoff arm in cluster $s_{k}$ and then $s_{k}$ gets eliminated
%The theoretical analysis remains same as we have always bounded the values of $\rho_{a}\in (0,1]$(see Appendix \ref{App:E}).
%Also we cluster the arms based on $\epsilon_{m}$.
% One vital point we point out is that, $\epsilon_{m}$(in proposition $3$) = $\epsilon_{g}$(in proposition $4$).
The analysis proceeds by considering the contribution to the regret in each of the following cases:

\textbf{Case a:} \textit{Some sub-optimal arm ${i}$ is not eliminated in round $\max(m_{i},g_{s_{k}})$ or before, with the optimal arm ${*}\in C_{\max(m_{i},g_{s_{k}})}$.}

%\todos[inline]{The stmt ``In this case, we are looking at event of the maximum round till which atleast one of $m_{i}$ or $g_{s_{k}}$ did not happen.'' is unnecessary give the case caption above} 
%	In this case, we are looking at event of the maximum round till which atleast one of $m_{i}$ or $g_{s_{k}}$ did not happen. 
We consider an arbitrary sub-optimal arm ${i}$ and analyze the contribution to the regret when $i$ is not eliminated in the following exhaustive sub-cases:\\
%\todos[inline]{Get rid of enumerate to save space. You could just have the case labels, say case a1 and such}
\textbf{Case a1:} \textit{In round $\max(m_{i},g_{s_{k}})$, ${i} \in s^{*}$.}

Similar to case (a) of \cite{auer2010ucb}, observe that when the following two conditions hold, arm $i$ gets eliminated:
\begin{align}
\hat{r}_{i}  \le r_{i} + c_{m_{i}} \text{ and } 
 \hat{r}^{*}\geq  r^{*} - c_{m_{i}}, \label{eq:armelim-casea1}
\end{align}
where  $c_{m_{i}}=\sqrt{\frac{\rho_{a}\log (\psi T\epsilon_{m_{i}}^{2})}{2 n_{m_{i}}}}$.
The arm $i$ gets eliminated because 
  \begin{align*}
\hat{r}_{i} + c_{m_{i}}&\leq r_{i} + 2c_{m_{i}} < r_{i} + \Delta_{i} - 2c_{m_{i}} = r^{*} -2c_{m_{i}} \\
 &\leq \hat{r}^{*} - c_{m_{i}}.
  \end{align*}
%\todos[inline]{The stmt ``bound the probability of the event $\hat{r}_{i}+c_{m_{i}}\leq \hat{r}^{*}-c_{m_{i}}$'' is wrong. We bound the complementary event using Hoeffding} 
%\todos[inline]{The stmt ``$m_{i}$ does not happen'' makes no sense given that we are in round $\max(m_i,g_i)$}
%\todos[inline]{What is $c_m$?}
  %Now, $c_{m_{i}}=\sqrt{\frac{\rho_{a}\log (\psi T\epsilon_{m_{i}}^{2})}{2 n_{m_{i}}}}$.
In the above, we have used the fact that \\ $c_{m_{i}} = \sqrt{\rho_{a}\epsilon_{m_{i}+1}} < \frac{\Delta_{i}}{4}$, since $n_{m_{i}}=\frac{2\log{(\psi T\epsilon_{m_{i}}^{2})}}{\epsilon_{m_{i}}}$ and $\rho_{a}\in (0,1]$.

From the foregoing, we have to bound the events complementary to that in \eqref{eq:armelim-casea1} for an arm $i$ to not get eliminated. Considering Chernoff-Hoeffding bound this is done as follows:
%\todos[inline]{In the following, $\exists a_i$ is spurious given that we are talking about arm $a_i$ all through in this case} 
%  %Again, $\exists a_{i} \in A_{s^{*}}^{'}$ such that, 
%\todos[inline]{$r_{i} + 2c_{m_{i}} 
% < r_{i} + \Delta_{i} - 2c_{m_{i}}$}
%\todos[inline]{The final inequality above does not hold unless you assume $\hat{r}^{*}\geq r^{*} - c_{m_{i}}$ and this is never mentioned?}
  %Hence, we get that when $\sqrt{\rho_{a}\epsilon_{m_{i}}}<\frac{\Delta_{i}}{2}$, $a_{i}$ gets eliminated. 
  %Applying Chernoff-Hoeffding bound and considering independence of events,
  \begin{align*}
&\mathbb{P}\left(\hat{r}^{*}\leq r^{*} - c_{m_{i}}\right)\leq \exp(-2c_{m_{i}}^{2}n_{m_{i}})\\
&\leq \exp(-2 * \frac{\rho_{a}\log (\psi T\epsilon_{m_{i}}^{2})}{2 n_{m_{i}}} *n_{m_{i}})\\
&\leq \frac{1}{(\psi T\epsilon_{m_{i}}^{2})^{\rho_{a}}}   
  \end{align*}
Along similar lines, we have 
$\mathbb{P}\left(\hat{r}_{i}\geq r_{i} + c_{m_{i}}\right)\leq \frac{1}{(\psi  T\epsilon_{m_{i}}^{2})^{\rho_{a}}}.$
Thus, the probability that a sub-optimal arm ${i}$ is not eliminated in any round on or before $m_{i}$ is bounded above by  $\bigg(\frac{2}{(\psi T\epsilon_{m_{i}}^{2})^{\rho_{a}}}\bigg)$. 
 Summing up over all arms in $A_{s^{*}}^{'}$ in conjunction with a simple bound of $T\Delta_{i}$ for each arm, we obtain
   \begin{align*}
&\sum_{i\in A_{s^{*}}^{'}}\bigg(\dfrac{2T\Delta_{i}}{(\psi T\epsilon_{m_{i}}^{2})^{\rho_{a}}}\bigg)
\leq\sum_{i\in A_{s^{*}}^{'}}\bigg(\frac{2T\Delta_{i}}{(\psi T\dfrac{\Delta_{i}^{4}}{16\rho_{a}^{2}})^{\rho_{a}}}\bigg)\\
%&\leq \sum_{i\in A_{s^{*}}^{'}}\bigg(\frac{2^{1+4\rho_{a}}T^{1-\rho_{a}}\rho_{a}^{2\rho_{a}}\Delta_{i}}{\psi^{\rho_{a}}\Delta_{i}^{4\rho_{a}}}\bigg)\\
& =\sum_{i\in A_{s^{*}}^{'}}\bigg(\frac{C_{1}(\rho_{a})T^{1-\rho_{a}}}{\Delta_{i}^{4\rho_{a}-1}}\bigg) \text{, where } C_1(x) = \frac{2^{1+4x}x^{2x}}{\psi^{x}}
   \end{align*}

%%%%%%%%%%%%%%%%%%%%%%%%%%%%%%%%%%%%%%%%%%%%%%%%%%%%%%%%%%%%%%%%%%%%%%%%%%%%%%%%%%%%%%%%%%%%%%%   
%\textbf{Case a2:} \textit{In round $m_{i}^{'}$, $a_{i} \in s_{k}$ for some $s_k \ne s^*$} % where $r_{\max_{s_{k}}}\leq r^{*}$ 
%
%%\todos[inline]{The description in the text below for this case doesnt make sense to me. The final bound arrived at uses $\Delta_i$, while it doesnt figure here in the argument here at all. }
%%We can show that the probability of $a_{i}$ not getting eliminated cannot be worse than Case $a1$ given that $m_{i}^{'}< g_{s_{k}}$ or else $g_{s_{k}}$ will happen and the cluster $s_{k}$ will get eliminated or $a_{\max_{s_{k}}}$ will eliminate $a^{*}$ which are dealt later. 
%
%Approaching the same way as above we define $\Delta_{i}^{'}=r_{a_{\max_{s_{k}}}} - r_{i}$, for $a_{i}\in s_{k}$, $m_{i}^{'}=\min{\lbrace m|\sqrt{\rho_{a}\epsilon_{m}} < \frac{\Delta_{i}^{'}}{2} \rbrace}$.Then plugging in $\Delta^{'}_{i}$ in Case $a1$ and bounding the complementary events mentioned in \ref{eq:armelim-casea} by using $r_{i}$ and $r_{a_{\max_{s_{k}}}}$, we can show that for an arm $a_{i}\in A_{s_{k}}^{'}$ the maximum probability of not getting eliminated on or before $m_{i}^{'}$ is  $\bigg(\dfrac{2}{(\psi T\epsilon_{m_{i}^{'}}^{2})^{\rho_{a}}}\bigg)$. So bounding trivially over $T\Delta_{i}^{'}$ the regret is bounded by,
%
%\begin{align*}
%& \sum_{i\in A_{s_{k}}^{'}}\frac{C_{1}(\rho_{a})T^{1-\rho_{a}}}{\Delta_{i}^{'^{{4\rho_{a}-1}}}} 
%   \end{align*}
%   %\leq \sum_{i\in A_{s_{k}}^{'''}} \frac{C_{1}(\rho_{a})T^{1-\rho_{a}}}{\Delta_{a_{\max_{s_{k}}}}^{4\rho_{a}-1}}
%   %\leq \sum_{\substack{i\in A_{s_{k}}^{'}: \\ \Delta_{i}^{'}\geq \Delta_{a_{\max_{s_{k}}}} }}\frac{C_{1}(\rho_{a})T^{1-\rho_{a}}}{\Delta_{a_{\max_{s_{k}}}}^{4\rho_{a}-1}}
%   %and considering $ \frac{1}{4} \leq \rho_{a} \leq 1 $
%Summing over all $p-1$ clusters excluding $s^{*}$ the regret is,
%\begin{align*}
%& \sum_{k=1}^{p-1}\sum_{i\in A_{s_{k}}^{'}\setminus A_{s^{*}}^{'}} \frac{C_{1}(\rho_{a})T^{1-\rho_{a}}}{\Delta_{i}^{'^{^{4\rho_{a}-1}}}} \leq \sum_{i\in A^{'}\setminus A^{'}_{s^{*}}}\frac{C_{1}(\rho_{a})T^{1-\rho_{a}}}{\Delta_{i}^{'^{4\rho_{a}-1}}} 
%   \end{align*}
%   
%%   For any round $m_{i}^{'} > g_{s_{k}}$ and $a_{i}\neq a_{max_{s_{k}}}$ means that $\hat{r}_{i} - c_{m_{i}} > \hat{r_{a_{max_{s_{k}}}}} + c_{m_{i}}$ and also $\hat{r_{a_{max_{s_{k}}}}}  - c_{m_{i}} > \hat{r}^{*} + c_{m_{i}}$ which leads to the violation of the condition that 




%%%%%%%%%%%%%%%%%%%%%%%%%%%%%%%%%%%%%%%%%%%%%%%%%%%%%%%%%%%%%%%%%%%%%%%%%%%%%%%%%%%%%%%%%%%%%%%   
\textbf{Case a2:} \textit{In round $\max(m_{i},g_{s_{k}})$, ${i} \in s_k$ for some $s_k \ne s^{*}$.}

%\todos[inline]{Fix this case analysis to read as well as case a1. The first part until the Hoeffding bounds can be shorter than case a1, as the analysis to arrive at Hoeffding events follows using parallel arguments.} 
%then in cluster elimination condition, given the choice of confidence interval $c_{g_{s_{k}}}=\sqrt{\frac{\rho_{s} \log (\psi T\epsilon_{g_{s_{k}}}^{2})}{2 n_{g_{s_{k}}}}}$, we want to bound the probability of the event $\hat{r}_{s_{k}}+c_{g_{s_{k}}}\geq \hat{r}^{*}-c_{g_{s_{k}}}$.
%
%
%  Putting the value of $n_{g_{s_{k}}}=\frac{2\log{(\psi T\epsilon_{g_{s_{k}}}^{2})}}{\epsilon_{g_{s_{k}}}}$ in $c_{g_{s_{k}}}$, we get $c_{g_{s_{k}}} =\sqrt{\rho_{s}\epsilon_{g_{s_{k}}+1}} < \frac{\sqrt{\rho_{s}}\Delta_{a_{\max_{s_{k}}}}}{4} < \frac{\Delta_{a_{\max_{s_{k}}}}}{4}$.
%
%  
%  \begin{align*}
%  \hat{r}_{a_{\max_{s_{k}}}} + c_{g_{s_{k}}}&\leq r_{a_{\max_{s_{k}}}} + 2c_{g_{s_{k}}} = r_{a_{\max_{s_{k}}}} + 4c_{g_{k}} - 2c_{g_{s_{k}}}\\
%  &< r_{a_{\max_{s_{k}}}} + \Delta_{a_{\max_{s_{k}}}} - 2c_{g_{s_{k}}} = r^{*} -2c_{g_{s_{k}}}\\
%  &\leq \hat{r}^{*} - c_{g_{s_{k}}} \text{, as } \hat{r}^{*}\geq r^{*} - c_{g_{s_{k}}}
%  \end{align*}
%   
% 	Hence, we get that when $\sqrt{\rho_{s}\epsilon_{g_{s_{k}}}}<\frac{\Delta_{a_{\max_{s_{k}}}}}{2}$, $a_{\max_{s_{k}}}\in C_{g_{s_{k}}}$ gets eliminated leading to elimination of $s_{k}$. Applying Chernoff-Hoeffding bound and considering independence of events,
% 
% 
% \begin{align*}
% \mathbb{P}\bigg\lbrace\hat{r}^{*} &\leq r^{*} - c_{g_{s_{k}}}\bigg\rbrace \leq exp(-2c_{g_{s_{k}}}^{2}n_{g_{s_{k}}})
% \leq \dfrac{1}{(\psi T\epsilon_{g_{k}}^{2})^{\rho_{s}}}
% \end{align*}
%
%Similarly, $\mathbb{P}\bigg\lbrace\hat{r}_{a_{\max_{s_{k}}}}\geq r_{a_{\max_{s_{k}}}} + c_{g_{s_{k}}}\bigg\rbrace\leq \dfrac{1}{(\psi T\epsilon_{g_{s_{k}}}^{2})^{\rho_{s}}}$

Following a parallel argument like in Case $a1$, we have to bound the following two events of arm $a_{\max_{s_{k}}}$ not getting eliminated on or before $g_{s_{k}}$-th round,
\begin{align*}
  \hat{r}_{a_{\max_{s_{k}}}} \geq r_{a_{\max_{s_{k}}}} +c_{g_{s_{k}}} \text{ and } \hat{r}^{*} \leq r^{*} -c_{g_{s_{k}}}  
\end{align*} 

We can prove using Chernoff-Hoeffding bounds and considering independence of events mentioned above, that for $c_{g_{s_{k}}}=\sqrt{\frac{\rho_{s} \log (\psi T\epsilon_{g_{s_{k}}}^{2})}{2 n_{g_{s_{k}}}}}$ and  $n_{g_{s_{k}}}=\frac{2\log{(\psi T\epsilon_{g_{s_{k}}}^{2})}}{\epsilon_{g_{s_{k}}}}$ the probability of the above two events is bounded by $\bigg(\dfrac{2}{(\psi  T\epsilon_{g_{s_{k}}}^{2})^{\rho_{s}}}\bigg)$.
%Summing, the two up, the probability that a sub-optimal cluster arm $a_{\max_{s_{k}}}\in C_{g_{s_{k}}}$ is not eliminated
  Now, for any round $g_{s_{k}}$, all the elements of $C_{\max(m_{i},g_{s_{k}})}$ are the respective maximum payoff arms of their cluster $s_{k}, \forall s_{k}\in S$, and since all the surviving arms are pulled equally in each round and since clusters are fixed so we can bound the maximum probability that a sub-optimal arm ${i}\in A^{'}$  and ${i}\in s_{k}$ such that $a_{\max_{s_{k}}}\in C_{g_{s_{k}}}$ is not eliminated on or before the $g_{s_{k}}$-th round by the same probability as above. 

%\begin{align*}
%\bigg(\frac{2}{(\psi T\epsilon_{g_{s_{k}}}^{2})^{\rho_{s}}}\bigg)
%\end{align*}
 
%Summing up over all arms in $s_{k}$ and bounding trivially by $T\Delta_{i}$,
%\begin{align*}
%\sum_{i\in A_{s_{k}}}\bigg(\frac{2T\Delta_{i}}{(\psi T\epsilon_{g_{s_{k}}}^{2})^{\rho_{s}}}\bigg)
%\end{align*}

Summing up over all $p$ clusters and bounding the regret for each arm $i\in A_{s_{k}}^{'}$ trivially by $T\Delta_{i}$,
 \begin{align*}
 &\sum_{k=1}^{p}\sum_{i\in A_{s_{k}}^{'}}\bigg(\frac{2T\Delta_{i}}{(\psi T\frac{\Delta_{i}^{4}}{16\rho_{s}^{2}})^{\rho_{s}}}\bigg) = \sum_{i\in A^{'}}\bigg(\frac{2T\Delta_{i}}{(\psi  T\frac{\Delta_{i}^{4}}{16\rho_{s}^{2}})^{\rho_{s}}}\bigg) \\
 &\leq \sum_{i\in A^{'}}\bigg(\frac{2^{1+4\rho_{s}}\rho_{s}^{2\rho_{s}}T^{1-\rho_{s}}}{\psi^{\rho_{s}}\Delta_{i}^{4\rho_{s}-1}}\bigg) = \sum_{i\in A^{'}}\frac{C_{1}(\rho_{s})T^{1-\rho_{s}}}{\Delta_{i}^{4\rho_{s}-1}}
 \end{align*}
% &= \sum_{i\in A^{'}}\bigg(\frac{C_{1}(\rho_{s})T^{1-\rho_{s}}}{\Delta_{i}^{4\rho_{s}-1}}\bigg) \text{, where } C_1(x) = \frac{2^{1+4x}x^{2x}}{\psi^{x}}
%&\leq \sum_{i\in A}\bigg(\frac{2^{1+4\rho_{s}}T^{1-\rho_{s}}\rho_{s}^{2\rho_{s}}\Delta_{i}}{\psi^{\rho_{s}}\Delta_{i}^{4\rho_{s}}}\bigg)\\



Summing the bounds in Cases $a1-a2$ and observing that the bounds in the aforementioned cases hold for any round $C_{\max \lbrace m_i,g_{s_k}\rbrace}$, we obtain the following contribution to the expected regret from case a:
   %Taking summation of the events mentioned above($a1$-$a4$) gives us an upper bound on the regret given that the optimal arm $a^{*}$ is still surviving, 
\begin{align*}
&\sum_{i\in A_{s^*}} \frac{C_{1}(\rho_{a})T^{1-\rho_{a}}}{\Delta_{i}^{4\rho_{a}-1}} + \sum_{i\in A^{'}}\bigg(\frac{C_{1}(\rho_{s})T^{1-\rho_{s}}}{\Delta_{i}^{4\rho_{s}-1}}\bigg)
\end{align*}

%So the regret for not eliminating a sub-optimal cluster even when $a^{*}\notin C_{g_{s_{k}}}$(but still surviving in $s^{*}$) can be no worse than,
%	 \begin{align*} 
%	 \bigg(\frac{2}{(T\epsilon_{g_{s_{k}}}^{2})^{\rho_{s}}}\bigg) 
%	 \end{align*}
%&\underbrace{\sum_{i\in A_{s^{*}}^{'}}\bigg(\dfrac{C_{1}(\rho_{a})T^{1-\rho_{a}}}{\Delta_{i}^{4\rho_{a}-1}}\bigg)}_{\text{case a1}} + \underbrace{\sum_{i\in A\setminus A_{s^{*}}^{'}}\bigg(\dfrac{C_{1}(\rho_{a})T^{1-\rho_{a}}}{\Delta_{i}^{4\rho_{a}-1}}\bigg)}_{\text{case a2}} \\
% & + \sum_{i\in A^{'}}\bigg\lbrace \underbrace{\bigg(\dfrac{2C_{1}(\rho_{s})T^{1-\rho_{s}}}{\psi^{\rho_{s}}\Delta_{i}^{4\rho_{s}-1}}\bigg)}_{\text{case a3+a4}}\bigg\rbrace \\
%& =

%%%%%%%%%%%%%%%%%%%%%%%%%%%%%%%%%%%%%%%%%%%%%%%%%%%%%%%%%%%%%%%%%%%%%%%%%%%%%%%%%%%%%%%%%%%%
\textbf{Case b:} \textit{For each arm $i$, either ${i}$ is eliminated in round $\max (m_{i},g_{s_{k}})$ or before or there is no optimal arm ${*}$ in $C_{\max(m_{i},g_{s_{k}})}$.} \\

\textbf{Case b1:} \textit{${*}\in C_{\max(m_{i},g_{s_{k}})}$ for each arm $i \in A'$ and cluster $s_k \in \check A$.} 

%\todos{define $\check A$}

The condition in the case description above implies the following: \\
\begin{inparaenum}[\bfseries (i)]
\item each sub-optimal arm ${i}\in A^{'}$ is  eliminated on or before $\max (m_{i},g_{s_{k}})$ and hence  pulled not more than pulled $n_{m_{i}}$ number of times.\\
\item each sub-optimal cluster $s_k \in \check A$ is  eliminated on or before $\max (m_{i},g_{s_{k}})$ and hence  pulled not more than pulled $n_{g_{s_{k}}}$ number of times.
\end{inparaenum}

Hence, the maximum regret suffered due to pulling of a sub-optimal arm or a sub-optimal cluster is no more than the following:
 \begin{align*}
 &\sum_{i\in A^{'}}\Delta_{i}\bigg\lceil\dfrac{2\log{(\psi T\epsilon_{m_{i}}^{2})}}{\epsilon_{m_{i}}}\bigg\rceil 
\!+\! \sum_{k=1}^{p}\sum_{i\in A_{s_{k}}^{'}}\Delta_{i}\bigg\lceil\dfrac{2\log{(\psi T\epsilon_{g_{s_{k}}}^{2})}}{\epsilon_{g_{s_{k}}}}\bigg\rceil \\
&\leq\sum_{i\in A^{'}}\Delta_{i}\bigg(1+\dfrac{32\rho_{a}\log{\left(\psi T\left(\frac{\Delta_{i}}{2\sqrt{\rho_{a}}}\right)^{4}\right)}}{\Delta_{i}^{2}}\bigg) \\
&\quad+ \sum_{i\in A^{'}}\Delta_{i}\bigg(1+\dfrac{32\rho_{s}\log{\left(\psi T\left(\frac{\Delta_{i}}{2\sqrt{\rho_{s}}}\right)^{4}\right)}}{\Delta_{i}^{2}}\bigg)
\\
 &\leq \sum_{i\in A^{'}}\!\bigg[ 2\Delta_{i}+\dfrac{32(\rho_{a}\log{(\psi T\dfrac{\Delta_{i}^{4}}{16\rho_{a}^{2}})} + \rho_{s}\log{(\psi T\dfrac{\Delta_{i}^{4}}{16\rho_{s}^{2}})})}{\Delta_{i}} \bigg]
%  \\
% & \qquad \qquad +\dfrac{32\rho_{s}\log{(\psi T\dfrac{\Delta_{i}^{4}}{16\rho_{s}^{2}})}}{\Delta_{i}}\bigg\rbrace 
 \end{align*}
In the above, the first inequality follows since $\sqrt{\rho_{a}\epsilon_{m_{i}}} < \frac{\Delta_{i}}{2}$ and $\sqrt{\rho_{s}\epsilon_{n_{g_{s_{k}}}}} < \frac{\Delta_{a_{\max_{s_{k}}}}}{2}$.

%&\leq\Delta_{i}\bigg(1+\dfrac{32\rho_{a}\log{(\psi T\dfrac{\Delta_{i}^{4}}{16\rho_{a}^{2}})}}{\Delta_{i}^{2}}\bigg)\\
%&\leq\Delta_{i}\bigg\lceil\dfrac{2\log{(\psi T(\dfrac{\Delta_{i}}{2\sqrt{\rho_{a}})^{4})}}}{(\dfrac{\Delta_{i}}{2\sqrt{\rho_{a}}})^{2}}\bigg\rceil \\
%\text{, since } \sqrt{\rho_{a}\epsilon_{m_{i}}}\leq\dfrac{\Delta_{i}}{2}
 
%%%%%%%%%%%%%%%%%%%%%%%%%%%%%%%%%%%%%%%%%%%%%%%%%%%%%%%%%%%%%%%%%%%%%%%%%%%%%%%%%%%%%%%%%%%%%%%   
%\textbf{Case b2:} \textit{Optimal arm $a^{*}$ is eliminated by a sub-optimal arm.}\\
  %
	%This, can happen in $3$ ways,
%\newline
\textbf{Case b2:} \textit{${*}$ is eliminated by some sub-optimal arm in $s^*$} \\
%In this case, we are concerned with the arm elimination condition only. 
Optimal arm $a^*$ can get eliminated by some sub-optimal arm $i$ only if arm elimination condition holds, i.e., 
\begin{align*}
\hat r_{i} - c_{m_{i}} > \hat{r}^{*}+ c_{m_{i}},
\end{align*}
where, as mentioned before, $c_{m_{i}}=\sqrt{\frac{\rho_{a}\log (\psi T\epsilon_{m_{i}}^{2})}{2 n_{m_{i}}}}$.
From analysis in Case $a1$, notice that, if \eqref{eq:armelim-casea1} holds in conjunction with the above, arm $i$ gets eliminated. Also, recall from Case $a1$ that the events complementary to \eqref{eq:armelim-casea1} have low-probability and can be upper bounded by $\frac{2}{(\psi  T\epsilon_{m_{*}}^{2})^{\rho_{a}}}$. Moreover, a sub-optimal arm that eliminates $*$ has to survive until round $m_*$. In other words, 
all arms ${j}\in s^{*}$ such that $m_{j} < m_{*}$ are eliminated on or before $m_*$ (this corresponds to case b1). 
Let, the arms surviving till $m_{*}$ round be denoted by $A^{'}_{s^{*}}$. This leaves any arm $a_{b}$ such that $m_{b}\geq m_{*} $ to still survive and eliminate arm ${*}$ in round $m_{*}$. Let, such arms that survive ${*}$ belong to $A^{''}_{s^{*}}$. Also maximal regret per step after eliminating ${*}$ is the maximal $\Delta_{j}$ among the remaining arms in $A^{''}_{s^{*}}$ with $m_{j}\geq m_{*}$.  Let $m_{b}=\min\lbrace m|\sqrt{\rho_{a}\epsilon_{m}}<\frac{\Delta_{b}}{2}\rbrace$. Let $C_2(x) = \frac{2^{2x+\frac{3}{2}}x^{2x}}{\psi^{x}}$. Hence, the maximal regret after eliminating the arm ${*}$ is upper bounded by, 
\begin{align*}
&\sum_{m_{*}=0}^{max_{j\in A^{'}_{s^{*}}}m_{j}}\sum_{\substack{i\in A^{''}_{s^{*}}: \\ m_{i}\geq m_{*}}}\bigg(\dfrac{2}{(\psi  T\epsilon_{m_{*}}^{2})^{\rho_{a}}} \bigg).T\max_{\substack{j\in A^{''}_{s^{*}}: \\ m_{j}\geq m_{*}}}{\Delta}_{j}\\
&\leq\sum_{m_{*}=0}^{max_{j\in A^{'}_{s^{*}}}m_{j}}\sum_{i\in A^{''}_{s^{*}}:m_{i} \geq m_{*}}\bigg(\dfrac{2}{(\psi  T\epsilon_{m_{*}}^{2})^{\rho_{a}}} \bigg).T.2\sqrt{\rho_{a}\epsilon_{m_{*}}} \\
&\leq\sum_{m_{*}=0}^{max_{j\in A^{'}_{s^{*}}}m_{j}}\sum_{i\in A^{''}_{s^{*}}:m_{i} \geq m_{*}}4\bigg(\dfrac{T^{1-\rho_{a}}}{\psi^{\rho_{a}}\epsilon_{m_{*}}^{2\rho_{a}-\frac{1}{2}}} \bigg)\\
&\leq\sum_{i\in A^{''}_{s^{*}}:m_{i} \geq m_{*}}\sum_{m_{*}=0}^{\min{\lbrace m_{i},m_{b}\rbrace}}\bigg(\dfrac{4T^{1-\rho_{a}}}{\psi^{\rho_{a}}2^{-(2\rho_{a}-\frac{1}{2})m_{*}}} \bigg)\\
&\!\leq\!\!\sum_{i\in A^{'}_{s^{*}}}\frac{4T^{1-\rho_{a}}}{\psi^{\rho_{a}}2^{-(2\rho_{a}-\frac{1}{2})m_{*}}}\! +\!\!\!\sum_{i\in A^{''}_{s^{*}}\setminus A^{'}_{s^{*}}}\!\frac{4T^{1-\rho_{a}}}{\psi^{\rho_{a}}2^{-(2\rho_{a}-\frac{1}{2})m_{b}}} \\
&\!\leq\!\!\sum_{i\in A^{'}_{s^{*}}}\frac{T^{1-\rho_{a}}\rho_{a}^{2\rho_{a}}2^{2\rho_{a}+\frac{3}{2}}}{\psi^{\rho_{a}}\Delta_{i}^{4\rho_{a}-1}} \!+\!\!\!\sum_{i\in A^{''}_{s^{*}}\setminus A^{'}_{s^{*}}}\!\!\frac{T^{1-\rho_{a}}\rho_{a}^{2\rho_{a}}2^{2\rho_{a}+\frac{3}{2}}}{\psi^{\rho_{a}}b^{4\rho_{a}-1}} \\
& = \sum_{i\in A^{'}_{s^{*}}}\dfrac{ C_{2}(\rho_{a}) T^{1-\rho_{a}}}{\Delta_{i}^{4\rho_{a}-1}} +\sum_{i\in A^{''}_{s^{*}}\setminus A^{'}_{s^{*}}}\dfrac{C_{2(\rho_{a})}T^{1-\rho_{a}}}{b^{4\rho_{a}-1}}.
\end{align*}

%&\text{ since } \sqrt{\rho_{a}\epsilon_{m}}<\dfrac{\Delta_{i}}{2}\\
%&\leq\sum_{i\in A^{'}}\dfrac{4\rho_{a}^{2\rho_{a}}T^{1-\rho_{a}}*2^{2\rho_{a}-\frac{1}{2}}}{\psi^{\rho_{a}}\Delta_{i}^{4\rho_{a}-1}} +\sum_{i\in A^{''}\setminus A^{'}}\dfrac{4\rho_{a}^{2\rho_{a}}T^{1-\rho_{a}}*2^{2\rho_{a}-\frac{1}{2}}}{\psi^{\rho_{a}}b^{4\rho_{a}-1}} \\

% \begin{align*}
% &\sum_{i\in A^{'}_{s^{*}}}\bigg(\dfrac{C_{2}(\rho_{a})T^{1-\rho_{a}}}{\Delta_{i}^{4\rho_{a} -1}} \bigg)+\sum_{i\in A^{''}_{s^{*}}\setminus A^{'}_{s^{*}}}\bigg(\dfrac{C_{2}(\rho_{a})T^{1-\rho_{a}}}{b^{4\rho_{a} -1}} \bigg)
% \end{align*}
%We also see that here, we are concerned only within $s^{*}$ because of our assumption that there is only one $a^{*}\in A$ and clusters are fixed.


%%%%%%%%%%%%%%%%%%%%%%%%%%%%%%%%%%%%%%%%%%%%%%%%%%%%%%%%%%%%%%%%%%%%%%%%%%%%%%%%%%%%%%%%%%%%%%%   
\textbf{Case b3:} \textit{$s^{*}$ is eliminated by some sub-optimal cluster.} 
%\newline
%Firstly, if conditions of case $b1$ holds then the optimal arm $a^{*}\in C_{g_{s_{k}}}$ will not be eliminated in round $g_{s_{k}}=g_{*}$ or it will lead to the contradiction that $r_{a_{max_{s_{k}}}}>r^{*}$ where $a_{max_{s_{k}}},a^{*}\in C_{g_{s_{k}}}$. In any round $g_{*}$, if the optimal arm $a^{*}$ gets eliminated then for any round from $1$ to $g_{s_{j}}$ all arms $a_{s_{j}}\in C_{g_{s_{k}}},\forall s_{j}\neq s^{*}$ such that $\sqrt{\rho_{s}\epsilon_{g_{s_{k}}}}<\dfrac{\Delta_{a_{s_{j}}}}{2}$ were eliminated according to assumption in case $b1$. Let, the arms surviving till $g_{*}$ round be denoted by $C_{g}^{'}$. This leaves any arm $a_{s_{b}}$ such that $\sqrt{\rho_{s}\epsilon_{g_{s_{b}}}}\geq\dfrac{\Delta_{a_{s_{b}}}}{2}$ to still survive and eliminate arm $a^{*}$ in round $g_{*}$. Let, such arms that survive $a^{*}$ belong to $C_{g}^{''}$. Also maximal regret per step after eliminating $a^{*}$ is the maximal $\Delta_{j}$ among the remaining arms $a_{j}\in B_{m}$ with $g_{s_{j}}\geq g_{
%*}$.  Let $g_{s_{b}}$ be the round when $\sqrt{\rho_{s}\epsilon_{g_{s_{b}}}}<\dfrac{\Delta_{s_{b}}}{2}$ that is $g_{b}=min\lbrace g|\sqrt{\rho_{s}\epsilon_{g_{s_{b}}}}<\dfrac{\Delta_{b}}{2}\rbrace$ and the cluster $s_{b}$ gets eliminated. Hence, the maximal regret after eliminating the arm $a^{*}$ is upper bounded by, 
% \begin{align*}
% &\sum_{g_{*}=0}^{max_{j\in C_{g}^{'}}g_{s_{j}}}\sum_{i\in C_{g}^{''}:g_{s_{k}}>g_{*}}\bigg(\dfrac{2}{(\psi T\epsilon_{g_{s_{k}}}^{2})^{\rho_{s}}} \bigg).T\max_{j\in C_{g}^{''}:g_{s_{j}}\geq g_{*}}{\Delta}_{a_{s_{j}}}
% \end{align*}

Let $C_{g}^{'}=\lbrace a_{max_{s_{k}}}\in A^{'}|\forall s_{k}\in S \rbrace$ and $C_{g}^{''}=\lbrace a_{max_{s_{k}}}\in A^{''}|\forall s_{k}\in S \rbrace$. A sub-optimal cluster $s_k$ will eliminate $s^*$ in round $g_*$ only if the cluster elimination condition of Algorithm \ref{alg:clusucb} holds, which is the following when ${*}\in C_{g_{*}}$:
\begin{align}
\hat r_{a_{\max_{s_k}}} - c_{g_{*}} > \hat{r}^{*}+ c_{g_{*}}.
\label{eq:caseb3-cluselim}
\end{align}
Notice that when ${*}\notin C_{g_{*}}$, since $r_{a_{max_{s_{k}}}}>r^{*}$, the inequality in \eqref{eq:caseb3-cluselim} has to hold for cluster $s_k$ to eliminate $s^*$.
As in case $b2$, the probability that a given sub-optimal cluster $s_k$ eliminates $s^*$ is upper bounded by  $\frac{2}{(\psi T\epsilon_{g_{s^{*}}}^{2})^{\rho_{s}}}$ and all sub-optimal clusters with $g_{s_{j}}< g_{*}$ are eliminated before round $g_*$. 

This leaves any arm $a_{\max_{s_{b}}}$ such that $g_{s_{b}}\geq g_{*}$ to still survive and eliminate arm ${*}$ in round $g_{*}$. Let, such arms that survive ${*}$ belong to $C_{g}^{''}$. Hence, following the same way as case $b2$,  the maximal regret after eliminating ${*}$ is,
 \begin{align*}
 \!\!\sum_{g_{*}=0}^{\max\limits_{a_{\max_{s_{j}}}\in C_{g}^{'}}g_{s_{j}}}\!\!\!\!\!\sum_{\substack{\scriptsize a_{\max_{s_{k}}}\in C_{g}^{''}: \\ g_{s_{k}} \geq g_{*}}}\bigg(\dfrac{2}{(\psi T\epsilon_{g_{s^{*}}}^{2})^{\rho_{s}}} \bigg)T\max_{\substack{a_{\max_{s_{j}}}\in C_{g}^{''}: \\ g_{s_{j}}\geq g_{*}}}{\Delta}_{a_{\max_{s_{j}}}}
 \end{align*}
Using $A'\supset C_{g}^{'}$ and $A''\supset C_{g}^{''}$, we can bound the regret contribution from this case in a similar manner as Case $b2$ as follows:
% \begin{align*}
%  & \sum_{g_{*}=0}^{max_{j\in A^{'}}g_{s_{j}}}\sum_{\substack{i\in A^{''}: \\ g_{s_{k}}\geq g_{*}}}\bigg(\dfrac{2}{(\psi T\epsilon_{g_{s^{*}}}^{2})^{\rho_{s}}} \bigg).T\max_{\substack{j\in A^{''}: \\ g_{s_{j}}\geq g_{*}}}{\Delta}_{a_{\max_{s_{j}}}}
% \end{align*}
% Like Case $b2$, we can bound the regret as,
\begin{align*}
 &\!\!\sum_{i\in A^{'}\setminus A_{s^*}^{'}}\frac{T^{1-\rho_{s}}\rho_{s}^{2\rho_{s}}2^{2\rho_{s}+\frac{3}{2}}}{\psi^{\rho_{s}}\Delta_{i}^{4\rho_{s}-1}} 
 \!+\!\!\!\sum_{i\in A^{''}\setminus A^{'}\cup A_{s^*}^{'}}\!\!\!\!\frac{T^{1-\rho_{s}}\rho_{s}^{2\rho_{s}}2^{2\rho_{s}+\frac{3}{2}}}{\psi^{\rho_{s}}b^{4\rho_{s}-1}} \\
 & = \sum_{i\in A^{'}\setminus A_{s^*}^{'}}\frac{C_{2}(\rho_{s})T^{1-\rho_{s}}}{\Delta_{i}^{4\rho_{s}-1}} +\sum_{i\in A^{''}\setminus A^{'}\cup A_{s^*}^{'}}\frac{C_{2}(\rho_{s})T^{1-\rho_{s}}}{b^{4\rho_{s}-1}} 
\end{align*}

%where $A^{'}$ is the set of all the arms across clusters surviving till $g_{*}$ round and $A^{''}$ be the set of all arms across clusters to still survive and eliminate arm $a^{*}$ in round $g_{*}$ respectively
%But, we know that for any round $g$, elements of $C_{g}$ are the best performers in their respective clusters. 
%Let, the arms surviving till $g_{*}$ round in $C_{\max \lbrace m_{i},g_{s^{*}}\rbrace}$ be denoted by $C_{g}^{'}$. 
%\\ & \text{ where } C_2(x) = \frac{2^{2x+\frac{3}{2}}x^{2x}}{\psi^{x}}
%&\leq\sum_{g_{*}=0}^{max_{j\in A^{'}}g_{s_{j}}}\sum_{i\in A^{''}:g_{s_{k}}>g_{*}}\bigg(\dfrac{2}{(\psi T\epsilon_{g_{s_{k}}}^{2})^{\rho_{s}}} \bigg).T.2\sqrt{\rho_{s}\epsilon_{g_{s_{j}}}} \text{, since }\sqrt{\rho_{s}\epsilon_{g_{s_{j}}}}\leq\dfrac{\Delta_{a_{s_{j}}}}{2}\leq  \dfrac{\Delta_{j}}{2}\text{, as }{r}_{a_{s_{j}}}>{r}_{j},\forall j\in s_{j}\\ 
%&\leq\sum_{g_{*}=0}^{max_{j\in A^{'}}g_{s_{j}}}\sum_{i\in A^{''}:g_{s_{k}}>g_{*}}\bigg(\dfrac{4T^{1-\rho_{s}}}{\psi^{\rho_{s}}\epsilon_{g_{s_{k}}}^{2\rho_{s} - \frac{1}{2}}} \bigg)\\
% &\leq\sum_{i\in A^{''}:g_{s_{k}}>g_{*}}\sum_{g_{*}=0}^{\min{\lbrace g_{s_{k}},g_{s_{b}}\rbrace}}\bigg(\dfrac{4T^{1-\rho_{s}}}{\psi^{\rho_{s}}2^{({2\rho_{s} - \frac{1}{2}})g_{*}}} \bigg) \\
% &\leq\sum_{i\in A^{'}}\bigg(\dfrac{4T^{1-\rho_{s}}}{\psi^{\rho_{s}}2^{({2\rho_{s} - \frac{1}{2}})g_{*}}} \bigg)+\sum_{i\in A^{''}\setminus A^{'}}\bigg(\dfrac{4T^{1-\rho_{s}}}{\psi^{\rho_{s}}2^{({2\rho_{s} - \frac{1}{2}})g_{s_{b}}}} \bigg)\\ 
% &\leq\sum_{i\in A^{'}}\bigg(\dfrac{4\rho_{s}^{2\rho_{s}}T^{1-\rho_{s}}*2^{2\rho_{s}-\frac{1}{2}}}{\psi^{\rho_{s}}\Delta_{i}^{4\rho_{s}-1}} \bigg)+\sum_{i\in A^{''}\setminus A^{'}}\bigg(\dfrac{4\rho_{s}^{2\rho_{s}}T^{1-\rho_{s}}*2^{2\rho_{s}-\frac{1}{2}}}{\psi^{\rho_{s}}b^{4\rho_{s}-1}} \bigg)\\

%%%%%%%%%%%%%%%%%%%%%%%%%%%%%%%%%%%%%%%%%%%%%%%%%%%%%%%%%%%%%%%%%%%%%%%%%%%%%%%%%%%%%%%%%%%%%%%   
%\textbf{Case b23:} \textit{$a^{*}\notin C_{\max \lbrace m_{i},g_{s_{k}} \rbrace}$ and $s^{*}$ gets eliminated by another sub-optimal cluster arm} 
%
%This will be the mirror case of Case $b22$ with $a^{*}\notin C_{g}$. So, let $a_{\max_{s^{*}}}$ satisfies $\hat{r}_{a_{\max_{s^{*}}}}> \hat{r}^{*}$ in round $C_{\max \lbrace m_{i},g_{s^{*}} \rbrace}$. Following the same way as Case $a2$, we can bound the events
%\begin{align*}
  %\hat{r}_{a_{\max_{s_{k}}}} \geq r_{a_{\max_{s_{k}}}} +c_{g_{s_{k}}} \text{ and } \hat{r}_{a_{\max_{s^{*}}}} \leq r_{a_{\max_{s^{*}}}} -c_{g_{s_{k}}}
%\end{align*}
 %
%by Chernoff-Hoeffding bound and considering independence of events and show that it cannot be worse than $\bigg(\dfrac{2}{(\psi  T\epsilon_{g_{s^{*}}}^{2})^{\rho_{s}}}\bigg)$ for any $g_{s_{k}}=g_{*}$.
%%In this case for some sub-optimal arm $a_{\max_{s_{k}}}\in C_{g_{s_{k}}}$, we have to bound the events
%%	\begin{align*}
%%  \hat{r}_{a_{\max_{s_{k}}}} \geq r_{a_{\max_{s_{k}}}} +c_{g_{s_{k}}} \text{ and } \hat{r}_{a_{\max_{s^{*}}}} \leq r_{a_{\max_{s^{*}}}} -c_{g_{s_{k}}}
%%\end{align*} 
%So $s^{*}$ gets eliminated by $a_{\max_{s_{b}}}$ such that $g_{s_{b}}\geq g_{*}$. Also maximal regret per step after eliminating $a^{*}$ is the maximal $\Delta_{j}$ among the remaining arms $a_{j}\in A^{''}$ with $g_{s_{j}}\geq g_{*}$. Following the same way above we can bound the regret as,
%%In this case we will consider that the cluster $s^{*}$ containing the optimal arm $a^{*}$ was eliminated by another sub-optimal cluster and $a^{*}\notin C_{g}$. 
%\begin{align*}
%&\sum_{i\in A^{'}}\dfrac{T^{1-\rho_{s}}\rho_{s}^{2\rho_{s}}2^{2\rho_{s}+\frac{3}{2}}}{\psi^{\rho_{s}}\Delta_{i}^{4\rho_{s}-1}} +\sum_{i\in A^{''}\setminus A^{'}}\dfrac{T^{1-\rho_{s}}\rho_{s}^{2\rho_{s}}2^{2\rho_{s}+\frac{3}{2}}}{\psi^{\rho_{s}}b^{4\rho_{s}-1}} \\
%& = \sum_{i\in A^{'}}\dfrac{C_{2}(\rho_{s})T^{1-\rho_{s}}}{\Delta_{i}^{4\rho_{s}-1}} +\sum_{i\in A^{''}\setminus A^{'}}\dfrac{C_{2}(\rho_{s})T^{1-\rho_{s}}}{b^{4\rho_{s}-1}}
%\end{align*}
%\newline
%Combining Cases $b21$, $b22$ and $b23$ as mentioned above we can show,
 %\begin{align*}
 %&\underbrace{\sum_{i\in A^{'}_{s^{*}}}\bigg(\dfrac{C_{2}(\rho_{a})T^{1-\rho_{a}}}{\Delta_{i}^{4\rho_{a} -1}} \bigg)+\sum_{i\in A^{''}_{s^{*}}\setminus A^{'}_{s^{*}}}\bigg(\dfrac{C_{2}(\rho_{a})T^{1-\rho_{a}}}{b^{4\rho_{a} -1}} \bigg)}_{\text{case b21}} \\
 %& + \underbrace{\sum_{i\in A^{'}}\bigg(\dfrac{2C_{2}(\rho_{s})T^{1-\rho_{s}}}{\Delta_{i}^{4\rho_{s}-1}} \bigg)}_{\text{case b22}}+\underbrace{\sum_{i\in A^{''}\setminus A^{'}}\bigg(\dfrac{2C_{2}(\rho_{s})T^{1-\rho_{s}}}{b^{4\rho_{s} -1}} \bigg)}_{\text{case b23}}
 %\end{align*}
 
\textbf{Case b4:} \textit{${*}$ is not in $C_{\max(m_{i},g_{s_{k}})}$, but belongs to $B_{\max(m_{i},g_{s_{k}})}$.}

In this case the optimal arm ${*}\in s^{*}$ is not eliminated, also $s^{*}$ is not eliminated. So, for all sub-optimal arms $i$ in $A_{s^*}^{'}$ which gets eliminated on or before $\max \lbrace m_{i},g_{s_{k}} \rbrace$ will get pulled no less than $\bigg\lceil\dfrac{2\log{(\psi T\epsilon_{m_{i}}^{2})}}{\epsilon_{m_{i}}}\bigg\rceil$ number of times, which leads to the following bound the contribution to the expected regret, as in Case $b1$:
% 
% \begin{align*}
%  &\sum_{i\in A^{'}}\Delta_{i}\bigg\lceil\dfrac{2\log{(\psi T\epsilon_{m_{i}}^{2})}}{\epsilon_{m_{i}}}\bigg\rceil
% \end{align*}
% Since $a^{*}$ is definitely in $B_{\max(m_{i},g_{s_{k}})}$ then following the same way as Case $b1$ we can show that can be no worse than,
\begin{align*}
 &\sum_{i\in A_{s^*}^{'}}\bigg\lbrace \Delta_{i}+\dfrac{32\rho_{a}\log{(\psi T\dfrac{\Delta_{i}^{4}}{16\rho_{a}^{2}})}}{\Delta_{i}} \bigg\rbrace 
\end{align*} 

For arms $a_i \notin s^*$, the contribution to the regret cannot be greater than that in Case $b3$. So the regret is bounded by,

\begin{align*}
\sum_{i\in A^{'}\setminus A_{s^*}^{'}}\dfrac{C_{2}(\rho_{s})T^{1-\rho_{s}}}{\Delta_{i}^{4\rho_{s}-1}} +\sum_{i\in A^{''}\setminus A^{'} \cup A_{s^*}^{'}}\dfrac{C_{2}(\rho_{s})T^{1-\rho_{s}}}{b^{4\rho_{s}-1}}
\end{align*}




The main claim follows by summing the contributions to the expected regret from each of the cases above.
\end{proof}


\begin{proposition}
\label{proofTheorem:Prop:1}
The regret $R_T$ for ClusUCB-AE satisfies
\begin{align*}
&\E [R_{T}]\leq \sum\limits_{\substack{i\in A\\\Delta_{i} > b}}\bigg\lbrace\frac{C_{1}(\rho_{a})T^{1-\rho_{a}}}{\Delta_{i}^{4\rho_{a}-1}} + \Delta_{i}+\frac{32\rho_{a}\log{(\frac{\psi  T\Delta_{i}^{4}}{16\rho_{a}^{2}})}}{\Delta_{i}}\\
& +  \frac{C_{2}(\rho_{a})T^{1-\rho_{a}}}{\Delta_{i}^{4\rho_{a} -1}}  \bigg \rbrace +\sum\limits_{\substack{i\in A\\0 <\Delta_{i}\leq b}}\frac{C_{2}(\rho_{a})T^{1-\rho_{a}}}{b^{4\rho_{a} -1}}  + \max_{\substack{i\in A: \\ \Delta_{i}\leq b}}\Delta_{i}T,
\end{align*}
for all $b\geq\sqrt{\frac{e}{T}}$. In the above, $C_1, C_2$ are as defined in Theorem \ref{Result:Theorem:1}.
% (x) = \frac{2^{1+4x}x^{2x}}{\psi^{x}}$,  $C_2(x) = \frac{2^{2x+\frac{3}{2}}x^{2x}}{\psi^{x}}$, $\rho_{a}=\frac{1}{2}$ is the arm elimination parameter, $\psi=K^{2}T$ is the exploration regulatory factor, $p$ is the number of clusters and $T$ is the horizon.
\end{proposition}
\begin{proof}
%Follows in a similar fashion as the proof of Theorem $1$ in \cite{auer2010ucb}. For the sake of completeness, the proof is given in Appendix \ref{App:A}.
See Appendix \ref{App:A}.
\end{proof}

\begin{proposition}
\label{proofTheorem:Prop:2}
The regret $R_T$ for ClusUCB-CE satisfies,
\begin{align*}
&\E [R_{T}]\leq \sum\limits_{\substack{i\in A: \\ \Delta_{i} > b}}\bigg\lbrace\dfrac{C_{1}(\rho_{s})T^{1-\rho_{s}}}{\Delta_{i}^{4\rho_{s}-1}} +\dfrac{64\rho_{s}\log{(\psi T\dfrac{\Delta_{i}^{4}}{16\rho_{s}^{2}})}}{\Delta_{i}} \\  
& + 2\Delta_{i} \bigg\rbrace + \sum\limits_{\substack{i\in A\setminus A_{s^*}: \\ \Delta_{i} > b}} \dfrac{2C_{2}(\rho_{s})T^{1-\rho_{s}}}{\Delta_{i}^{4\rho_{s} -1}} \\ 
& + \sum\limits_{\substack{i\in A\setminus A_{s^*}: \\ 0 <\Delta_{i}\leq b}}\dfrac{2C_{2}(\rho_{s})T^{1-\rho_{s}}}{b^{4\rho_{s} -1}} 
 + \max_{i\in A:\Delta_{i}\leq b}\Delta_{i}T,
\end{align*}
 for all $b\geq \sqrt{\frac{e}{T}}$, with $C_1$ and $C_2$ as defined in Theorem \ref{Result:Theorem:1}.
%  (x) = \frac{2^{1+4x}x^{2x}}{\psi^{x}}$,  $C_2(x) = \frac{2^{2x+\frac{3}{2}}x^{2x}}{\psi^{x}}$, $\rho_{s}=\frac{1}{2} $ is the cluster elimination parameter, $\psi=K^{2}T$ is the exploration regulatory factor, $p$ is the number of clusters and $T$ is the horizon.
\end{proposition}
\begin{proof}
See Appendix \ref{App:B}.
\end{proof}