%\begin{remark}
%\label{Result:Rem:6}
We sketch the proof for Theorem \ref{Result:Theorem:1} here. 
The proof involves the following steps:\\
\textbf{\textit{Step 1:}}
We analyze ClusUCB-AE, i.e., the variant of ClusUCB that uses arm elimination condition only. In other words, we bound the probability of sub-optimal arm elimination, which in turn bounds the expected regret of ClusUCB-AE (see Proposition \ref{proofSketch:Prop:1} below). 

\textbf{\textit{Step 2:}}
We analyze ClusUCB-CE, i.e., the variant of ClusUCB that uses cluster elimination condition only and pulls the best arm within the last leftover cluster.
Proposition \ref{proofSketch:Prop:2} presents the expected regret for ClusUCB-CE.

\textbf{\textit{Step 3:}}
Finally we combine the individual bounds in the steps above to get the regret upper bound in Theorem \ref{Result:Theorem:1}.  
%\end{remark}
	

\begin{proposition}
\label{proofSketch:Prop:1}
The regret $R_T$ for ClusUCB-AE satisfies
\begin{align*}
&\E [R_{T}]\leq \sum\limits_{i\in A:\Delta_{i}\geq b}\bigg\lbrace\frac{C_{1}(\rho_{a})T^{1-\rho_{a}}}{\Delta_{i}^{4\rho_{a}-1}} + \Delta_{i}\\
&+\frac{32\rho_{a}\log{(\dfrac{\psi  T\Delta_{i}^{4}}{16\rho_{a}^{2}})}}{\Delta_{i}}
 +  \frac{C_{2}(\rho_{a})T^{1-\rho_{a}}}{\Delta_{i}^{4\rho_{a} -1}}  \bigg \rbrace\\
&+\sum\limits_{i\in A:0\leq\Delta_{i}\leq b}\frac{C_{2}(\rho_{a})T^{1-\rho_{a}}}{b^{4\rho_{a} -1}}  + \max_{i:\Delta_{i}\leq b}\Delta_{i}T
\end{align*}
, for all $b\geq\sqrt{\dfrac{e}{T}}$, where  $C_1(x) = \frac{2^{1+4x}x^{2x}}{\psi^{x}}$,  $C_2(x) = \frac{2^{2x+\frac{3}{2}}x^{2x}}{\psi^{x}}$, $\rho_{a}=\dfrac{1}{2}$ is the arm elimination parameter, $\psi=K^{2}T$ is the exploration regulatory factor, $p$ is the number of clusters and $T$ is the horizon.
\end{proposition}
\begin{proof}
Follows in a similar fashion as the proof of Theorem 3.1 in \cite{auer2010ucb}. For the sake of completeness, the proof is given in Appendix \ref{App:A}.
\end{proof}


%\begin{remark}
%\label{Result:Rem:7}	
%\paragraph*{} Thus, we see that the confidence interval term $c_{m}=\sqrt{\dfrac{\rho_{a}\log (\psi T\epsilon_{m}^{2})}{2 n_{m}}}$ makes the algorithm eliminate an arm $a_{i}$ as soon as $\sqrt{\rho_{a}\epsilon_{m}}<\dfrac{\Delta_{i}}{2}$. The above result is in contrast with UCB-Improved which only deletes an arm if $\tilde{\Delta}_{m}<\dfrac{\Delta_{i}}{2}$, where $\tilde{\Delta}_{m}$ is initialized at $1$ and is halved after every round. We also point out here that when $c_{m}=\sqrt{\dfrac{\rho_{a}\log (\psi T\epsilon_{m}^{2})}{2 n_{m}}}$ and $\rho_{a}$ is decreased after every round, then an arbitrary arm $a_{i}$ is removed as soon as  $\sqrt{\rho_{a}\epsilon_{m}}<\dfrac{\Delta_{i}}{2}$ which essentially makes it a faster elimination procedure than UCB-Improved if $\rho_{a}\leq \epsilon_{m}$.
%\end{remark}


\begin{corollary}
\label{proofSketch:Corollary:1}
For $\rho_{a}=1$ in the result of proposition $1$, 
\begin{align*}
& \sum\limits_{i\in A:\Delta_{i}\geq b}\bigg\lbrace\frac{C_{1}(\rho_{a})T^{1-\rho_{a}}}{\Delta_{i}^{4\rho_{a}-1}} + \Delta_{i}\\
&+\frac{32\rho_{a}\log{(\dfrac{\psi  T\Delta_{i}^{4}}{16\rho_{a}^{2}})}}{\Delta_{i}}
 +  \frac{C_{2}(\rho_{a})T^{1-\rho_{a}}}{\Delta_{i}^{4\rho_{a} -1}}  \bigg \rbrace\\
&+\sum\limits_{i\in A:0\leq\Delta_{i}\leq b}\frac{C_{2}(\rho_{a})T^{1-\rho_{a}}}{b^{4\rho_{a} -1}}  + \max_{i:\Delta_{i}\leq b}\Delta_{i}T
\end{align*}
 
 we get a regret bound of 
 \begin{align*}
 &\sum\limits_{i\in A:\Delta_{i}\geq b}\bigg(\Delta_{i} + \dfrac{44}{\psi(\Delta_{i})^{3}} + \dfrac{32\log{(\psi T\Delta_{i}^{4})}}{\Delta_{i}}\bigg)\\ 
 & + \sum\limits_{i\in A:0\leq\Delta_{i}\leq b}\dfrac{12}{\psi b^{3}}
 \end{align*}
 
 for using just arm elimination with $p=1$.
\end{corollary}


\begin{proof}
In the result of Proposition $1$ if we take $\rho_{a}=1$ then the regret bound becomes $ \sum\limits_{i\in A:\Delta_{i}\geq b}\bigg(\Delta_{i} + \dfrac{44}{\psi(\Delta_{i})^{3}} + \dfrac{32\log{(\psi T\Delta_{i}^{4})}}{\Delta_{i}}\bigg) + \sum\limits_{i\in A:0\leq\Delta_{i}\leq b}\dfrac{12}{\psi b^{3}}$. From the result we can see that for small $\Delta_{i}$ and large $K$, the terms like $ \sum\limits_{i\in A:\Delta_{i}\geq b}\bigg(\dfrac{44}{\psi(\Delta_{i})^{3}}\bigg) + \sum\limits_{i\in A:0\leq\Delta_{i}\leq b}\dfrac{12}{\psi b^{3}}$ can become the dominant term in the regret rather than $\sum\limits_{i\in A:\Delta_{i}\geq b}\dfrac{32\log{(\psi T\Delta_{i}^{4})}}{\Delta_{i}}$. Intuitively, this actually suggests that the algorithm is trying to eliminate arms with too low exploration and so the probability of elimination is low and error(risk) is high. For this essentially we introduce $\rho_{a},\rho_{s}$ and $\psi$ and by carefully defining their values enables us to eliminate arms and clusters aggressively and thereby reduce those two terms. 
\end{proof}

\begin{proposition}
\label{proofSketch:Prop:2}
Considering only the cluster elimination condition and $p>1$, the total regret till $T$ is upper bounded by 
\begin{align*}
&\E [R_{T}]\leq \sum\limits_{i\in A:\Delta_{i}\geq b}\bigg\lbrace\bigg(\dfrac{2C_{1}(\rho_{s})T^{1-\rho_{s}}}{\Delta_{i}^{4\rho_{s}-1}}\bigg)\\
& + \bigg(\Delta_{i}+\dfrac{32\rho_{s}\log{(\psi T\dfrac{\Delta_{i}^{4}}{16\rho_{s}^{2}})}}{\Delta_{i}}\bigg) + \bigg(\dfrac{2C_{2}(\rho_{s})T^{1-\rho_{s}}}{\Delta_{i}^{4\rho_{s} -1}} \bigg)\bigg\rbrace \\
& + \sum\limits_{i\in A:0\leq\Delta_{i}\leq b}\bigg(\dfrac{2C_{2}(\rho_{s})T^{1-\rho_{s}}}{b^{4\rho_{s} -1}} \bigg)
\end{align*}
, for all $b\geq \sqrt{\dfrac{e}{T}}$, where $C_1(x) = \frac{2^{1+4x}x^{2x}}{\psi^{x}}$,  $C_2(x) = \frac{2^{2x+\frac{3}{2}}x^{2x}}{\psi^{x}}$, $\rho_{s}=\dfrac{1}{2} $ is the cluster elimination parameter, $\psi=K^{2}T$ is the exploration regulatory factor, $p$ is the number of clusters and $T$ is the horizon.
\end{proposition}
\begin{proof}
See Appendix \ref{App:B}.
\end{proof}
